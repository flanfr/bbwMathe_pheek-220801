%% \author
\renewcommand{\author}{Autor mit renewcommand author in 'bbw.sty' definieren.}
%% \author
\newcommand{\grafikautor}{Grafikautor mit renewcommand author in 'bbw.sty' definieren.}
%%E-Mail des Autohrs
\newcommand{\authoremail}{Autor E-Mail mit renewcommand «authoremail» in 'bbw.sty' erstellen.}

%% \erstellungsdatum
\newcommand{\erstellungsdatum}{Datum mit renewcommand erstellungsdatum in 'bbw.sty' definieren.}
%% Version des Dokumentes
\newcommand{\docversion}{Version mit renewcommand docversion in 'bbw.sty' definieren.}
%% \doctitel
\newcommand{\doctitel}{Titel mit renewcommand doctitel in 'bbw.sty' festlegen.}

%% \fachthema
\newcommand{\fachthema}{Fachthema via renewcommand in 'bbw.sty' überschreiben.}%
%% \untertitel
%%\newcommand{\untertitel}{Untertitel via renewcommand 'untertitel' in 'bbw.sty' überschreiben.}
\newcommand{\untertitel}{\fachthema{}}

\newcommand{\rechte}{\textbf{Rechte}
 Diese Unterlagen sind Allgemeingut. Jegliches Kopieren (auch
 auszugsweise) ist gestattet und im Zusammenhang mit Ausbildung auch erw\"unscht.
 Einzig der/die Originalautor(en) und der Dokumenttitel sollten f\"ur
 R\"uckfragen und Verbesserungen bei jeder Kopie mit angegeben werden.}%



%IMPRESSUM 
\newcommand{\impressum}{

\section*{Impressum}%

\vspace{8mm}%
 
\paragraph{Präambel} Eine geschlechtsneutrale Bezeichnung von
  Personen oder eine Bezeichnung verschiedener Geschlechter wurde
  weitgehend vermieden,
  um die Lesbarkeit des vorliegenden Dokuments zu erleichtern.
  Alle Leser sind selbstverständlich gleichermaßen angesprochen.

  \vspace{8mm}

\rechte{}%

\vspace{7mm}

\copyright{} \author{}\ (Inhalt: \erstellungsdatum{}, Druck: \today{})


Version: \docversion{}

Grafik/Bilder/Fotos (sofern nichts anderes vermerkt): \grafikautor{}

Editiert durch: GNU Emacs 26.3

Verleger: \LaTeX{} pdfTeX 3.14159265-2.6-1.40.20 (TeX Live 2019/Debian)

Quellecode des Skripts: \texttt{https://github.com/pheek/bbwMathe}\\
\TRAINER{(git-push-Rechte bei Philipp verlangen!)}

\noTRAINER{\vspace{2mm}}

(Feedback an \texttt{\authoremail{}})

\noTRAINER{\vspace{14mm}}

\textbf{Über den Autor} Ph. G. Freimann studierte an der Universität Zürich
Mathematik, Informatik und Physik. Von den Lieblingsfächern im
Gymnasium waren Mathematik und Deutsch seine drei gutesten.

\newpage


}% END impressum
