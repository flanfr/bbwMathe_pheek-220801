%% Philipp G Freimann Juli 2019 für die BBW
%% Phi BBW-Vorlage für Mathematische Dokumente (LaTeX)
%% 2019 - 07 - 11
%%%%%%%%%%%%%%%%%%%%%%%%%%% M a t h e   M a k r o s %%%%%%%%%%%%%%%%%%%%%%%%%%%%%5

\usetikzlibrary{arrows.meta}

%% Kleine Symbole über anderen. Z. B. "?" über einem
%% Gleichheitszeichen:
%% Use \ueberMini{=}{?} um ein kleines Fragezeichen über ein
%% Gleichheitsszeichen zu schreiben.
\newcommand{\ueberMini}[2]{ \mathrel{\stackrel{\makebox[0pt]{\mbox{\normalfont\tiny #2}}}{#1}} }

%% Gleichungssystem mit zwei Zeilen und vier Einträgen (je zwei links
%% bzw. rechts).
\def\gleichungZZ#1#2#3#4{%%
  $$\left|
  \begin{array}{rcl}
    {#1} &=& {#2}\\
    {#3} &=& {#4}
    \end{array}\right|$$}%%

\def\gleichungDD#1#2#3#4#5#6{%%
  $$\left|
  \begin{array}{rcl}
    {#1} &=& {#2}\\
    {#3} &=& {#4}\\
    {#5} &=& {#6}\\
    \end{array}\right|$$}%%

%% Entspricht-Symbol
%%\usepackage{accents}
\newcommand{\hatset}[1]{\accentset{\wedge}{#1}}
\newcommand{\entspricht}{\,\,\hatset{=}\,\,}
\newcommand*\mittelwert[1]{\bar{#1}}
\newcommand*\mediantilde[1]{\widetilde{#1}}

%%
%% Graphiken mit tikz: BBW-Mathe-akros
%%
\tikzset{bbwgrid/.style={help lines,color=cyan!18, step=0.5cm}}

\newcommand{\bbwGridPart}[4]{
 % grid:
 \draw[bbwgrid] (#1,#3) grid (#2,#4);

 % axes
 \draw[thick] (#1,0) -- (#2,0);
 \draw[thick] (0,#3) -- (0,#4);
 \foreach \x in {#1, ..., -1}  \draw (\x cm, 2pt) -- (\x cm, -2pt)  node[anchor=north]{$\x$};
 \foreach \x in {1, ..., #2}   \draw (\x cm, 2pt) -- (\x cm, -2pt)  node[anchor=north]{$\x$};
 \foreach \y in {#3, ..., -1}  \draw (-2pt, \y cm) -- (2pt, \y cm)  node[anchor=east]{$\y\,\,$};
 \foreach \y in {1, ..., #4}   \draw (-2pt, \y cm) -- (2pt, \y cm)  node[anchor=east]{$\y\,\,$};
 \draw[->,thick] (#2,0) -- ({#2+0.5},0) node[anchor=west]{$x$};
 \draw[->,thick] (0,#4) -- (0,{#4+0.5}) node[anchor=south]{$y$};
}


%% A function within a Grid (without painting the grid)
%% #1: funciton eg 2*\x  (x has to be backquoted)
%% #2: Domain eg. -1:2.5
%% #3: colour eg red
\newcommand{\bbwFuncC}[3]{\draw[domain=#2,smooth,samples=200,variable=\x,#3] plot ({\x},{#1});
}
%% A function within a Grid (without painting the grid)
\newcommand{\bbwFunc}[2]{\bbwFuncC{#1}{#2}{blue}}

%% Declare a function-plot
%% xmin,xmax,ymin,ymax,fct,domain(x-min, x-max)
%% example: \bbwFunction{-4}{3}{-2}{5}{-\x*\x- \x + 4.5}{-3:2}
\newcommand{\bbwFunction}[6]{\begin{tikzpicture}
\bbwGridPart{#1}{#2}{#3}{#4}
 \bbwFunc{#5}{#6}
%% \draw[domain=#6,smooth,samples=200,variable=\x,blue] plot ({\x},{#5});
\end{tikzpicture}}
%% a whole graph having a coordinate-system #1-#4 and any tizpicture content #5:
\newcommand{\bbwGraph}[5]{\begin{tikzpicture}\bbwGridPart{#1}{#2}{#3}{#4}#5\end{tikzpicture}}

%% Dots and lines:
%% Dot example: \bbwDot{-1,2}{red}{east}{A}
\newcommand{\bbwDot}[4]{\filldraw[color=#2!60, fill=#2!5, thick](#1) circle(0.05) node[anchor=#3]{$#4$};}

%% Line example: \bbwLine{-1,0}{2,3}{red}
\newcommand{\bbwLine}[3]{\draw[thick,color=#3] (#1)--(#2);}

\newcommand{\bbwArrow}[3]{\draw[thick,color=#3,->] (#1)--(#2);}


%% Draw a single letter or small text
% #1: Position eg  4,4
% #2: letter eg f or blah
% #3: colour
\newcommand{\bbwLetter}[3]{\draw[color=#3](#1) node{$#2$};}

%%% ABC-Formel
%% usage \abcd{<a>}{<b>}{<c>}
%% example usage: \abcd{b}{5}{\sqrt{4}}
\newcommand{\abcd}[3]{$\frac{-(#2)\pm\sqrt{(#2)^2 - 4\cdot{}(#1)\cdot{}(#3)}}{2\cdot{}(#1)}$}



%% Trigonometrische Koordinatensysteme
%% Alle heißen "trigsysS" wobei da S einer der folgenden Sub-Systeme
%% bezeichnet:
%%  A  phi von  0 ... 360
%%     y   von -3 ...   3
%%
%%  B  phi von  0 ... 360
%%     y   von -1 ...   1
%%
%%  C  phi von  -270 ... 450
%%     y   von    -2 ...   2
%%
%%  D  phi von  -270 ... 450
%%     y   von    -1 ...   1
%%

%% coordSysBBWFlex
%% Flexibles Koordinatensystem mit Pfeilen und Pfeilbeschriftung, aber
%% noch ohne "ticks".
%% #1   : Rastergröße
%% #2-#5: Größe des Rasters in cm
%% #6   : Beschriftung in x-Richtung (in y-Richtung ist es immer y
%% #7   : Zu zeichnende Funktion
%% #8   : Ticks oder was sonst noch komplexeres in die Grafik muss
\newcommand{\coordSysBBWFlex}[8]{
\begin{tikzpicture}
\draw[step = #1,  thin, cyan!20] (#2, #4) grid (#3, #5);
\draw[thick, ->] (#2,0) -- (#3,0) node[anchor = west] {$#6$};
\draw[thick, ->] (0,#4) -- (0,#5) node[anchor = south] {$y$};
\draw[domain=#2:#3,smooth,samples=200,variable=\x,gray!30] plot ({\x},{#7});
#8;
\end{tikzpicture}
}%% end coordSysBBW

%% Koordinatensystem von 0 - 360 Grad (y -Ricthung -1 bis 1
%% Die Funktion kann mit dem 1. Parameter eingegeben werden
\newcommand{\trigsysAFct}[1]{
\coordSysBBWFlex{0.5cm}{-1}{13}{-4}{4}{\varphi}{#1}{
  \foreach \x [evaluate=\x as \degree using int(\x*30)] in {1,...,12}{ 
    \draw (\x cm, 1pt) -- (\x cm, -1pt) node[anchor = north] {$\degree^\circ$};
  }
  \foreach \y in {-3,-2,-1,1,2,3}{
   \draw (1pt, \y cm) -- (-1pt, \y cm) node[anchor = east] {$\y$};
  }
}
}%% end trigsysC

%% Leeres Koordinatensystem (fct = 0)
\newcommand{\trigsysA}{\trigsysAFct{0}}


%% Koordinatensystem von -270 bis 450 Grad. In y-Richtung von -2 bis 2
%% Funktion wird mit #1-Parameter angegeben
\newcommand{\trigsysBFct}[1]{
\coordSysBBWFlex{0.5cm}{-1}{13}{-4}{4}{\varphi}{#1}{
  \foreach \x [evaluate=\x as \degree using int(\x*30)] in {1,...,12}{ 
    \draw (\x cm, 1pt) -- (\x cm, -1pt) node[anchor = north] {$\degree^\circ$};
  }
  \foreach \y in {-1,1}{
   \draw (1pt, \y *3cm) -- (-1pt, \y *3cm) node[anchor = east] {$\y$};
  }
}
}%% end trigsysC

%% Leeres B-System
\newcommand{\trigsysB}{\trigsysBFct{0}}


%% Wie B-SYstem, jedoch in y-Richtung von -1 bis +1
\newcommand{\trigsysCFct}[1]{
\coordSysBBWFlex{0.2cm}{-6}{10}{-2.5}{2.5}{\varphi}{#1}{
  \foreach \x [evaluate=\x as \degree using int(\x*90)] in {-3,-2,-1,1,2,3,4,5}{ 
   \draw (\x *18mm, 1pt) -- (\x * 18mm, -1pt) node[anchor = north] {$\degree^\circ$};
  }
   
  \foreach \y in {-2,-1,1,2}{
    \draw (1pt, \y cm) -- (-1pt, \y cm) node[anchor = east] {$\y$};
  }
}
}%% end trigsysC

\newcommand{\trigsysC}{\trigsysCFct{0}}


\newcommand{\trigsysDFct}[1]{
\coordSysBBWFlex{0.2cm}{-6}{10}{-2.5}{2.5}{\varphi}{#1}{
 \foreach \x [evaluate=\x as \degree using int(\x*90)] in {-3,-2,-1,1,2,3,4,5}{ 
   \draw (\x *18mm, 1pt) -- (\x * 18mm, -1pt) node[anchor = north] {$\degree^\circ$};
  }   
  \foreach \y in {-1,1}
   \draw (1pt, \y *2cm) -- (-1pt, \y *2cm) node[anchor = east] {$\y$};
  }
} %% end command: trig sys D cos()


\newcommand{\trigsysDcos}{\trigsysDFct{2*cos(\x*50)}}
\newcommand{\trigsysDsin}{\trigsysDFct{2*sin(\x*50)}}
\newcommand{\trigsysD}{\trigsysDFct{0}}


