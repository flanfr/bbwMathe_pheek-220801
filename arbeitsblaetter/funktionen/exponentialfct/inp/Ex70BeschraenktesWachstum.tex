%%%%%%%%%%%%%%%%%%%%%%%%%%%%%%%%%%%%%%%%%%%%%%%%%5

\subsection{Begrenztes Wachstum}


\bbwActAufgabenNr{} \textbf{Cola}

Eine Cola wird bei $5^\circ \textrm{ C}$ aus der Kühlbox genommen. Die
Umgebungstemperatur ist $32^\circ \textrm{ C}$. Nach 3 Minuten messen
wir bereits eine Temperatur von $9^\circ \textrm{ C}$ 

Wir gehen davon aus, dass die Temperaturdifferenz der Cola zur
Umgebungstemperatur exponentiell abnimmt.

\begin{bbwAufgabenBlock}
\item Machen Sie eine Skizze im Koordinatensystem, welche die
Abhängigkeit von der Cola-Temperatur von der Zeit ($t$) aufzeichnet.

\TRAINER{Graph}

\item Geben Sie eine Funktionsgleichung $f(t)$ an, welche die
Temperatur in Minuten ($t$) nach dem herausnehmen der Cola aus der Kühlbox
angibt.

\TRAINER{$$f(t) = c - b\cdot{} a^{\frac{t}{\tau}}$$

$$f(t) = 32 - 27 \cdot \left( \frac{23}{27}\right)^{\frac{t}{3}}$$
}

\item Wie «warm» ist die Cola nach 10 Minuten?

\TRAINER{
$$f(10) = 32 -
27 \cdot \left( \frac{23}{27}\right)^{\frac{10}{3}} \approx
16.18^\circ \textrm{ C}$$
}%% end TRAINER

\item Nach wie vielen Minuten ist die Cola $18^\circ \textrm{ C}$
«warm»?

\TRAINER{
$$f(t) = 18 = 32 - 27\cdot{} \left(\frac{23}{27}\right)^\frac{t}{3}$$
$$-14 =  - 27\cdot{} \left(\frac{23}{27}\right)^\frac{t}{3}$$
$$\frac{14}{27} =  - \left(\frac{23}{27}\right)^\frac{t}{3}$$
$$t =
3\cdot{} \log_{\left(\frac{23}{27}\right)}\left(\frac{14}{27}\right) \approx
12.29 \textrm{ min.}$$
}%% END TRAINER

\end{bbwAufgabenBlock}


\platzFuerBerechnungenBisEndeSeite{}


%%%%%%%%%%%%%%%%%%%%%%%%%%%%%%%%%%%%%%%%%%%%%%%%%%%%%%%%%%%%%%%%%%%%%%%%%%%%%%%%%%%%%%%%%%%%%%%%%%


\bbwActAufgabenNr{} \textbf{«Silly Blubb»}

Das neue Waschmittel «Silly Blubb» will sich im Markt etablieren.
Dank einer tollen Fernseh- und Internetwerbung nehmen die
Verkaufszahlen rasant zu.

Es ist jedoch zu erwarten, dass nicht mehr als 20\% aller Käufer auf
«Silly Blubb» umschwenken werden.

Nach dem ersten Monat sind bereits 5\% der Waschmittelkäufer auf
«Silly Blubb» umgelenkt worden. Nach zwei weiteren Monaten sind wir
bei 8\% gelandet.


\begin{bbwAufgabenBlock}
\item Machen Sie eine Skizze im Koordinatensystem, welche die
Abhängigkeit von Monat ($x$-Achse) zu Käuferzahl in Prozent darstellt.

\TRAINER{Graph}

\item Geben Sie einen mögliche Funktionsterm $f(t)$ an, welcher die
Prozentzahl der Käufer nach Monaten angibt. Bedenken Sie, dass die
«Sättigungsgrenze» bei 20\% liegt.

\TRAINER{$$f(t) = 20 - 15 \cdot{} (\frac{12}{15})^{\frac{t-1}{2}}$$
}

\item Wie viele Prozente der Käufer sind in Monat 6 nach Verkaufsstart
  bereits «Silly Blubb» Käufer?
  
\TRAINER{
$$y = 20 - 15\cdot{} \left(\frac{12}{15}\right)^{\frac{6-1}{2}} \approx 11.41\% $$%%
}%% end TRAINER

\item Nach wie vielen Monaten sind 16\% der Käufer von «Silly Blubb»
  überzeugt worden?

\TRAINER{
$$f(t) = 16 [\%] = 20 - 15\cdot{} \left(\right)^{\frac{t-1}2}$$
$$ \frac4{15} = \left(\right)^{\frac{t-1}2}$$
$$ \frac4{15} = \left(\right)^{\frac{t-1}2}$$
$$\frac{t-1}2  = \log_{\left(\frac{12}{15}\right)}\left(\frac4{15}\right)$$
$$t = 1 + 2\cdot{}
  \log_{\left(\frac{12}{15}\right)}\left(\frac4{15}\right) \approx
  12.85 \textrm{ Monate}$$
}%% END TRAINER

\end{bbwAufgabenBlock}


\platzFuerBerechnungenBisEndeSeite{}


%%%%%%%%%%%%%%%%%%%%%%%%%%%%%%%%%%%%%%%%%%%%%%%%%%%%%%%%%%%%%%%%%%%%%%%%%%%%%%%%%%%%%%%%%%%%%%%%%%%%%%%%

\bbwActAufgabenNr{} \textbf{Ritter Nimmersatt}

\nextBbwAufgabenNummer{}

Ritter Nimmersatt ist nimmer satt. Erst bei einer Magenfülle von
5 Litern stößt ihm alles wieder auf. Richtig wohlgenährt ist er
normalerweise erst bei einem Magen, der zu 4.5 Liter gefüllt ist.

Er beginnt das Festmahl bei Kunigundes Hochzeit mit einer Magenfülle
von 2 Litern\footnote{Darunter würde er echte Hungerschübe
  leiden!}. In der ersten Minute schafft er es, 3 dl Flüssigkeit oder Nahrung
in sich regelrecht hineinzustopfen. Danach nimmt sein Futtern
«exponentiell gesättigt» ab (bis zur Sättigungsgrenze von 5 Litern, die er
hoffentlich nicht erreicht).

Nach wie vielen Minuten ist er so richtig gesättigt; m.\,a.\,W. wann
hat er besagte 4.5 Liter im Magen\footnote{Der Moment ist gekommen, wo
  die Wachen den Ritter Nimmersatt vorsorglich unter ominösem Vorwand aus der Burg schaffen sollten.}?

\TNT{16}{
  $c = 5$, $b = m_0 = 5-2=3$, $m=m_1= 5-2.3=2.7$, $\tau = 1$, $a=\frac{2.7}{3} = 0.9$ und somit:
  $$f(t) = 4.5 = 5 - 3\cdot{}\left(\frac{2.7}{3}\right)^t$$

  $$t = log_{\frac{4.5-5}{-3}}(\frac{2.7}3) \approx 17.006$$
Und so ist er nach 17 Minuten so richtig satt.\vspace{52mm}}%% END TNT





\newpage
