%%%%%%%%%%%%%%%%%%%%%%%%%%%%%%%%%%%%%%%%%%%%%%%%%5

\subsection{Begrenztes Wachstum}


\bbwActAufgabenNr{} \textbf{Cola}

Eine Cola wird bei $5^\circ \textrm{ C}$ aus der Kühlbox genommen. Die
Umgebungstemperatur ist $32^\circ \textrm{ C}$. Nach 3 Minuten messen
wir bereits eine Temperatur von $9^\circ \textrm{ C}$ 

Wir gehen davon aus, dass die Temperaturdifferenz der Cola zur
Umgebungstemperatur exponentiell abnimmt.

\begin{bbwAufgabenBlock}
\item Machen Sie eine Skizze im Koordiantensystem, welche die
Abhängigkeit von der Cola-Temperatur von der Zeit ($t$) aufzeichnet.

\TRAINER{Graph}

\item Geben Sie die Funktionsgleichung $f(t)$ an, welche die
Temperatur in Minuten ($t$) nach dem herausnehmen der Cola aus der Kühlbox
angibt.

\TRAINER{$$f(t) = c - b\cdot{} a^{\frac{t}{\tau}}$$

$$f(t) = 32 - 27 \cdot \left( \frac{23}{27}\right)^{\frac{t}{3}}$$
}

\item Wie «warm» ist die Cola nach 10 Minuten?

\TRAINER{
$$f(10) = 32 -
27 \cdot \left( \frac{23}{27}\right)^{\frac{10}{3}} \approx
16.18^\circ \textrm{ C}$$
}%% end TRAINER

\item Nach wie vielen Minuten ist die Cola $18^\circ \textrm{ C}$
«warm»?

\TRAINER{
$$f(t) = 18 = 32 - 27\cdot{} \left(\frac{23}{27}\right)^\frac{t}{3}$$
$$-14 =  - 27\cdot{} \left(\frac{23}{27}\right)^\frac{t}{3}$$
$$\frac{14}{27} =  - \left(\frac{23}{27}\right)^\frac{t}{3}$$
$$t =
3\cdot{} \log_{\left(\frac{23}{27}\right)}\left(\frac{14}{27}\right) \approx
12.29 \textrm{ min.}$$
}%% END TRAINER

\end{bbwAufgabenBlock}


\platzFuerBerechnungenBisEndeSeite{}


\newpage
