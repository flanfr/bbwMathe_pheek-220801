
\subsection{Beobachtungsintervall}
\textit{Andere Startwerte und andere Zeitintervalle als die Einheiten}

%%%%%%%%%%%%%%%%%%%%%%%%%%%%%%%%%%%%%%%%%%%%%%%%%%


\bbwActAufgabenNr{} \textbf{Sauerteig}

Eine Sauerteigkultur verdopple sich bei optimaler, stetiger «Fütterung» alle 5.5 Stunden.

Anfänglich werden 53g gemessen.

\begin{bbwAufgabenBlock}

\item Welche Masse kann nach 11 Stunden erreicht werden?
      \TRAINER{$y = b \cdot{} a^{\frac{t}{\tau}}$ mit $a = 2$, $b=53$, $\tau=5.5$, Einheit = Stunden.

$y=53 \cdot{} 2^{\frac{11}{5.5}} = 53 \cdot{} 4 = 212 \textrm{g}$
}

\item Welche Masse kann nach 24 Stunden erreicht werden?
      \TRAINER{$y = b \cdot{} a^{\frac{t}{\tau}}$ mit $a = 2$, $b=53$, $\tau=5.5$, Einheit = Stunden.

$y=53 \cdot{} 2^{\frac{24}{5.5}} = 53 \cdot{} 20.59 = 1091 \textrm{g}$
}

\item Wann ist mit einer Verzehnfachung der Masse der Kultur zu rechnen?
      \TRAINER{$y = b \cdot{} a^{\frac{t}{\tau}}$ mit $a = 2$, $b=53$, $\tau=5.5$, Einheit = Stunden.

$$530=53 \cdot{} 2^{\frac{t}{5.5}} $$
$$10= 2^{\frac{t}{5.5}} $$
$$t = 5.5 \cdot{} \log_2(10) \approx 18.27 \textrm{Stunden}$$
}


\end{bbwAufgabenBlock}
\platzFuerBerechnungenBisEndeSeite{}




%%%%%%%%%%%%%%%%%%%%%%%%%%%%%%%%%%%%%%%%%%%%%%%%%%

\bbwActAufgabenNr{} \textbf{Bakterien}
 Ohne Nahrungsmangel und ohne Platzmangel wächst eine Bakterienkultur
 in der Anzahl exponentiell.

Um 7:00 Uhr waren es 2\,500 Bakterien.
Um 12:00 Uhr waren es bereits 40\,000 Stück.

\begin{bbwAufgabenBlock}

\item Um welchen Faktor haben die Bakterien (genauer die Anzahl der Bakterien) während diesen fünf
Stunden zugenommen?

      \TRAINER{$40\,000 / 2\,500 = 16$}

\item Geben Sie eine Funktionsgleichnug an, welche die Bakterienzahl ab
7:00 in Stunden angibt. Null Stunden liefert 2\,500 Stück, fünf
      Stunden ergibt 40\,000 Stück ...
      \TRAINER{$\tau = 5: t\mapsto b \cdot{} a^{\frac{t}{\tau}} =
      2\,500\cdot{} 16^{\frac{t}{5}}$}

\item Geben Sie die Anzahl der Bakterien um 9:30 Uhr an.
     \TRAINER{$2\,500 \cdot{} 16^{\frac12} = 10\,000$}

\item In welcher Zeitspanne wird sich die Anzahl der Bakterien
verdoppelt haben?
\TRAINER{$t_{10} = 5\cdot{} log_{16}(2) \approx{} 1.25 \textrm{h} $
also nach jeweis 75 Minuten = $\frac54$ Stunden}

\item Vie viele Stunden vor 7:00 Uhr waren es 150 Bakterien?

  \TRAINER{$$150 = 2\,500 \cdot{} 16^{\frac{t}{5}}$$
$$\frac{150}{2\,500} = 16^\frac{t}{5}$$
$$t = 5\cdot{} \log_{16}\left(\frac{150}{2\,500}\right) \approx
    -5.0736 \textrm{ h}$$

  }%% END TRAINER

\end{bbwAufgabenBlock}
\platzFuerBerechnungenBisEndeSeite{}

%%%%%%%%%%%%%%%%%%%%%%%%%%%%%%%%%%%%%%%%%%%%%%%%%%%%%%%%%%%
%%%%%%%%%%%%%%%%%%%%%%%%%%%%%%%%%%%%%%%%%%%%%%%%%%

\bbwActAufgabenNr{} \textbf{Inflation}

 Ein Europäisches Land hatte während fünf Jahren eine kumulierte
 Inflation von 16.5\%. Das heißt: Durschritt der Waren, die vor fünf Jahren zu einem
 bestimmten Preis bezogen werden konnten, kosten heute 16.5\% mehr.
 
\begin{bbwAufgabenBlock}

\item Berechnen Sie die jährliche Inflationsrate und geben Sie das
  Resultat in \% auf eine Dezimale gerundet an.

      \TRAINER{$p = \sqrt[5]{1.165} \approx 1.031$ somit ist die
        jährliche Rate = 3.1\%.}

    \item Eine Schubkarre kostete vor 5 Jahren 80.- \euro{}. Welcher
      Preis ist heute zu erwarten?

    \item Gehen wir von gleichbleibender Inflationsrate während der
      letzten sieben Jahre aus: Ein Bürostuhl kostet heute \euro{}
      75.-. Wie viel hat der Stuhl vor sieben Jahren gekostet?

      
\end{bbwAufgabenBlock}
\platzFuerBerechnungenBisEndeSeite{}

%%%%%%%%%%%%%%%%%%%%%%%%%%%%%%%%%%%%%%%%%%%%%%%%%%%%%%%%%%%


\bbwActAufgabenNr{} \textbf{Luftdruck}

Der Atmosphärische Luftdruck nimmt ab, je weiter wir uns von der
Meereshöhe entfernen. Der Druck wird in $\textrm{hPa}$ = Hectopascal
gemessen. Ein $\textrm{Pa}$ ist gleich $1 \textrm{N}/\textrm{m}^2$, also ein
Pascal ist die Kraft von einem Newton pro Quadratmeter.

Auf Meereshöhe ist der Luftdruck ca. $ 1013 \textrm{ hPa}$ und nimmt
ca 13\% pro km Höhenunterschied ab.

\begin{bbwAufgabenBlock}

\item Zeichnen Sie den Funktionsgraphen, der den Luftdruck ($y$-Achse)
  in Abhängigkeit der Höhe über Meer ($x$-Achse) angibt.

  \TRAINER{Graph}

\item Wie lautet ein möglicher Funktionsterm, der die Abhängigkeit
  Luftdruck ($\textrm{hPa}$) von Meereshöhe (in Metern) angibt?

  \TRAINER{$y = 1013 \cdot{} 0.87^{\frac{x}{1000}}$}


\item Wie hoch ist der Luftdruck auf dem Matterhorn (4478 Meter über
  Meeresspiegel)?

  \TRAINER{$y = 1013 \cdot{} 0.87^{\frac{4478}{1000}} \approx 543
    \textrm{ hPa}$}

\item In welcher Höhe über Meeresspiegel ist der Luftdruck nur noch
  $500 \textrm{ hPa}$?

  \TRAINER{
    $$500 = 1013 \cdot{} 0.87^{\frac{x}{1000}}$$
    $$\frac{500}{1013} = 0.87^{\frac{x}{1000}}$$
    $$x = 1000 \cdot{} \log_{0.87}\left(\frac{500}{1013}\right)
    \approx 5070 \textrm{ m}$$
    }%% END TRAINER
  
\end{bbwAufgabenBlock}
\platzFuerBerechnungenBisEndeSeite{}

%%%%%%%%%%%%%%%%%%%%%%%%%%%%%%%%%%%%%%%%%%%%%%%%%%%%%%%%%%%

\newpage
