\subsection{Startwerte}
\textit{Andere Startwerte als eine Einheit.}


\bbwActAufgabenNr{} \textbf{Gummiball}

Ein Gummiball wird fallen gelassen. Der Ball spring jeweils 72\% der Fallhöhe wieder zurück.

Anfänglich wird der Ball aus $1.7 \textrm{m}$ Höhe fallen gelassen

\begin{bbwAufgabenBlock}

\item Wie hoch springt der Ball nach dem 3. Aufprall wieder zurück?
      \TRAINER{$1.7 \cdot{} 0.72^3 \approx 63.45 \textrm{cm}$}
\item Wie hoch springt der Ball nach dem $n$. Aufprall wieder hoch?
      \TRAINER{$1.7 \cdot{} 0.72^n$}
\item Nach dem wievielten Aufprall springt der Ball noch 10 cm hoch?
      \TRAINER{$1.7 \cdot{} 0.72^n = 0.1 \Longrightarrow n = \log_{0.72}\left(\frac{0.1}{1.7}\right) \approx 8.6$. Das heißt: Nach 8 Sprüngen ist der Ball noch höher als 10 cm; nach dem 9. Sprung hingegen weniger als 10 cm.}
\end{bbwAufgabenBlock}
\platzFuerBerechnungenBisEndeSeite{}

%%%%%%%%%%%%%%%%%%%%%%%%%%%%%%%%%%%%%%%%%%%%%%%%%%


\bbwActAufgabenNr{} \textbf{Neophytenplage}

Eine neu eingeschleppte Brombeerenart vermehrt sich im Wald exponentiell.

Anfänglich werden 82 Pflazen gezählt. Ein Jahr später sind es bereits 102 Pflanzen.

\begin{bbwAufgabenBlock}

\item Wie groß ist der jährliche Zunahmefaktor?
      \TRAINER{$a = \frac{102}{82} = 1.\overline{24390}$}
\item Wie groß ist die jährliche Zuwachsrate?
      \TRAINER{$p = \frac{102}{82} - 1 \approx 24.39\%$}
      
\item Wie viele Brombeerpflanzen sind nach fünf Jahren zu erwarten?
      \TRAINER{$82\cdot{}a^5 \approx 244$ Pflanzen}

\item Wie viele Brombeerpflanzen sind nach $n$ Jahren zu erwarten?
      \TRAINER{$82\cdot{}a^n \approx{} 82\cdot{} 1.2349^N$ Pflanzen}
\item Nach wie vielen Jahren werden 1000 Pflanzen erwartet, wenn sich die Pflanzenart weiterhin ungehindert ausbreiten kann?
\TRAINER{$82\cdot{} a^n = 1000 \Longrightarrow n=\log_a(\frac{1000}{82}) \approx 11.46$}
\end{bbwAufgabenBlock}
\platzFuerBerechnungenBisEndeSeite{}



%%%%%%%%%%%%%%%%%%%%%%%%%%%%%%%%%%%%%%%%%%%%%%%%%%


\bbwActAufgabenNr{} \textbf{Federpendel}

Ein Federpendel wird in Schwingung gebracht. Wegen diversen
physikalischen Einflüssen (Reibung, ...) nimmt die Amplitude
exponentiell ab.

Nach einer Minute beträgt die Amplitude 5.3 cm und nach 4 min nur noch
2.6 cm.

\begin{bbwAufgabenBlock}

\item Machen Sie eine Skizze.
      \TRAINER{Graph}
\item Geben Sie die Funktionsgleichung der Amplitude in Abhängikeit
der Zeit an.
      \TRAINER{... Lösung noch aussthend ...}
      
\item Wie war die Amplitude zu Beginn ($t=0$)?
      \TRAINER{... Lösung noch ausstehend ...}

\item Wie groß wird die Amplitude nach 10 Minuten noch seinWie viele Brombeerpflanzen sind nach $n$ Jahren zu erwarten?
      \TRAINER{$82\cdot{}a^n \approx{} 82\cdot{} 1.2349^N$ Pflanzen}
\item Nach wie vielen Jahren werden 1000 Pflanzen erwartet, wenn sich die Pflanzenart weiterhin ungehindert ausbreiten kann?
\TRAINER{$82\cdot{} a^n = 1000 \Longrightarrow n=\log_a(\frac{1000}{82}) \approx 11.46$}
\end{bbwAufgabenBlock}
\platzFuerBerechnungenBisEndeSeite{}



%%%%%%%%%%%%%%%%%%%%%%%%%%%%%%%%%%%%%%%%%%%%%%%%%%


\bbwActAufgabenNr{} \textbf{Licht im Wasser}

In einem Wasserbecken nimmt die Lichtintensität pro Meter auf 24\% ab.
Dies spielt keine Rolle, ob honizontal oder vertikal gemessen wird.


\begin{bbwAufgabenBlock}

\item Skizzieren Sie die Intensität im Koordinatensystem wie folgt:
   $x$-Achse = Eindringtiefe in Metern. Die Maßeinheit der
   Lichtintensität wird in Watt pro Quadratmetern gemessen ($\textrm{W}/\textrm{m}^2$); dies
   hat für die Aufgabe hier jedoch keine Relevanz.
   $y$-Achse = Lichtintensität in \%. (Gestartet bei 100\%)
         \TRAINER{graph}
\item Wie lautet eine Funktionsvorschrift von $\textrm{m}\mapsto \%$?
\TRAINER{$x\mapsto 100\% \cdot{} 0.24^x$}
\item Wie viel Prozent der ursprünglichen Lichtintensität sind nach 5
   m noch übrig? \TRAINER{$0.24^5 \approx 0.07963 \%$}

\item In wie vielen Metern ist noch 1\% der Lichtintensität übrig?
\TRAINER{$x = \log_{0.24}(0.01) \approx  3.22 [\textrm{m}]$}
\end{bbwAufgabenBlock}
\platzFuerBerechnungenBisEndeSeite{}

%%%%%%%%%%%%%%%%%%%%%%%%%%%%%%%%%%%%%%%%%%%%%%%%%%


\bbwActAufgabenNr{} \textbf{Tierpopulation}

Eine Tierpopulation von fünf Tausend Stück nimmt wegen verändernder Klima-
und Umweltbedingungen jährlich um sechs Prozent ab.


\begin{bbwAufgabenBlock}

\item Skizzieren Sie die Population über die nächsten 30 Jahre.
         \TRAINER{graph}
\item Wie lautet eine Funktionsgleichung, welche die Tierpopulation in
  Abhängigkeit vom Jahr angibt?
\TRAINER{$y = 5\,000 \cdot{} 0.94^{t}$}
\item Um wie viele Tiere hat die Population nach 15 Jahren abgenommen?

  \TRAINER{$$y = 5\,000 \cdot{} 0.94^{15} = 1976 \textrm{ Tiere sind
      noch übrig}$$
d.\,h.:
  $$5000 - 1976  = 3024 \textrm{ Tiere sind «verschwunden»}$$}%% End TRAINER 

  

\item Wann wird die Population auf eine kritische «Größe» von 100
  Idividuen geschrumpft sein?
  \TRAINER{$y = 100 = 5\,000 \cdot{} 0.94^{t}$
    $$\frac{100}{5\,000} = 0.94^t$$
$$t = \log_{0.94}\left(\frac{100}{5\,000}\right) \approx{} 63
    \textrm{ Jahre} $$
  }%% end TRAINER
\end{bbwAufgabenBlock}
\platzFuerBerechnungenBisEndeSeite{}


\newpage
