%%
%% 2020 02 03 ph. freimann
%%

\section{Manipulation}\index{Manipulation}
\sectuntertitel{Ungefähr 97.223\% aller Statistiken sind frei
  erfunden, wie \zB diese.}

\theorieGESO{389}{25}
\theorieTALS{267}{4.6}
%%%%%%%%%%%%%%%%%%%%%%%%%%%%%%%%%%%%%%%%%%%%%%%%%%%%%%%%%%%%%%%%%%%%%%%%%%%%%%%%%
\subsection*{Lernziele}

\begin{itemize}
\item Bias\index{Bias} \theorieTALS{267}{4.6 (Bias)}
\item Festlegen der «Mitte» (Durchschnitt statt Median) \theorieTALS{268}{4.6.1}
\item Weglassen (Schubladisieren/Yang ohne Yin)
\item Illusion der Präzision
  \theorieTALS{268}{4.6.2}\index{Präzision!Illusion der}\index{Illusion der Präzision}
\item Graphische Darstellung \theorieTALS{269}{4.6.3}
  \begin{itemize}
  \item Skala verschweigen\index{Skala}
  \item $y$-Achse spiegeln
  \item Verschiedene Abstände (Säulenbreiten) bei Histogrammen
  \item Farben\index{Farben}
  \item Optische Verzerrungen:
    \begin{itemize}
    \item Nullpunktverschiebung\index{Nullpunktverschiebung}
    \item Allgemeine Achsenverschiebung\index{Achsenverschiebung} bzw. -stauchung\index{Achsenstauchung}
    \item logarithmische Achse
    \item Fläche (quadratische Vergrößerung)\index{Fläche!Manipulation in der Datenverarbeitung}
    \item Kuchendiagramm ist nicht 100\%
    \item In die Zukunft projizieren mit Pfeilen
    \end{itemize}
  \end{itemize}
\end{itemize}

Bei den meisten obigen «Lügen» kann der sog. Lügenfaktor\index{Lügenfaktor} berechnet werden, indem die scheinbare Größe ins Verhältnis zu echten Größe gesetzt wird.

\newpage

\subsection*{Aufgaben}
\TALSAadB{267}{992 bis 998}
\GESOAadB{403}{11, 34}

Notizen:

\mmPapier{5.2}

\bbwCenterGraphic{16cm}{allg/daan/img/Dreun.png}

\newpage
