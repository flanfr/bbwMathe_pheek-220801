%%
%% 2019 07 04 Ph. G. Freimann
%%
\section{Datenerhebung / -gewinnung}\index{Datenerhebung}\index{Datengewinnung}
\sectuntertitel{Traue keiner Statistik, die Du nicht selbst verfälscht hast}

\theorieGESO{371}{22}
\theorieGESO{373}{23}
\theorieTALS{250}{4.2}
%%%%%%%%%%%%%%%%%%%%%%%%%%%%%%%%%%%%%%%%%%%%%%%%%%%%%%%%%%%%%%%%%%%%%%%%%%%%t%%%%%
\subsection*{Lernziele}

\begin{itemize}
\item Stichprobe (Zusammensetzung der Stichprobe)
\item Methoden der Datengewinnung
\item Fehlererkennung
\end{itemize}


\TALS{Theorie \cite{frommenwiler17alg}: S.250 Kap. 4.2}

\subsection{Merkmale}
Bei einer Datenerhebung fallen diverse Merkmale mehr oder weniger ins
Gewicht. Wenn wir Personen untersuchen, so ist jede Person ein
sog. \textbf{Merkmalsträger}. Merkmale können Gewicht, Alter, Wohnort
etc. sein.

\subsubsection{Abgrenzung}\index{Abgrenzung}
Um den Überblick zu behalten, unterscheiden wir die Merkmale in drei
Kategorien:
\begin{itemize}
\item \textbf{Untersuchungsmerkmal}\index{Untersuchungsmerkmal}: Das
Merkmal, worüber ich eine Aussage machen will.
\item \textbf{Abgrenzungsmerkmal}\index{Abgrenzungsmerkmal}: Merkmale,
die die Untersuchende Gruppe von allen anderen Abgrenzt. Wenn ich eine
Untersuchung über die Jugendlichen Schweizerinnen und Schweizer im
Jahr 2020 machen will, so sind weder Erwachsene, noch Liechtenstein,
noch Jugendliche in den 60'er Jahren zu Untersuchen
\item Alle übrigen Merkmale sind für die Untersuchung irrelevant.
\end{itemize}

Ein gutes Abgrenzungsmerkmal sollte die folgenden Eigenschaften
aufweisen:
\begin{itemize}
\item Örtlich: Nur Personen innerhalb einer geographischen Grenze
\item Zeitlich: Die Untersuchung untersucht nur einen vorgegebenen
Zeitraum
\item Sachlich: Beispiele: jugendlich, berufstätig, ...
\end{itemize}

Bei allgemeingültigen Experimenten (Physik, Chemie, Biologie, ...)
sollten hingegen Zeit und Ort keine Rolle spielen.


\subsection{Arten der Datengewinnung}
\begin{itemize}
 \item Experiment
 \item Befragung
 \item Beobachtungsstudie
 \item Datensammlung
\end{itemize}

\subsection{Vorgehen (Optional)}
Das Vorgehen von den Rohdaten bis zur Interpretation kann grob wie folgt zusammengefasst werden. Dabei wird eine Variable (ein Merkmal)
\begin{enumerate}
\item gemessen,
\item kategorisiert\footnote{nominal, ordinal, «Intervalle-bildbar»,
    «Verhältnisse-bildbar»},
\item allenfalls sortiert,
\item graphisch dargestellt und
\item interpretiert.
\end{enumerate}
\newpage


\subsection{Fehlerarten}\index{Fehlerarten!in der Datengewinnung}
\begin{itemize}

\item Übertragungsfehler (der Werte, \zB falsche Maßeinheit,
  Legasthenie, Übermüdung beim Abschreiben, ...)

\item Systematische Fehler (Es wird etwas anderes gemessen, als das,
  wofür die Aussage gerade stehen muss.)

\item Zufällige Fehler, Messungenauigkeit

\item Mutwilliger oder fahrlässiger Fehler («Traue keiner Statistik...»)

\item Bias: Stichprobenverzerrung\index{Bias}\index{Stichprobenverzerrung}
\end{itemize}
\newpage

\subsection{Grundbegriffe}
\GESO{Ab S. 373 im Buch \cite{marthaler17}\\%%
}%%

\begin{tabular}{p{5cm}|l}
  Grundgesamtheit         &  Als Grundgesamtheit bezeichnet man ....\\
  \\
  Stichprobe\index{Stichprobe}              & \\
  \\
  Stichprobenumfang       & \\
  \\
  Bias\index{Bias}                   & \\
  \\ 
  Repräsentativität       & \\
  \\
  Datensatz\index{Datensatz}               & \\
  \\
  Untersuchungseinheit\index{Untersuchungseinheit}    & \\
  \\
  Beobachtung             & \\
  \\
  Merkmal\index{Merkmal} / Variable      & \\
  \\
  Merkmalsträger\index{Merkmalsträger}          & \\
  \\
  Wert/Ausprägung         & \\
  \\
  nominal\index{nominal}                 & \\
  \\
  ordinal\index{ordinal}                 & \\
  \\
  diskret\index{diskret}                 & \\
  \\
  stetig (kontinuierlich) & \\
  \\
  Intervalle\index{Intervalle} sind bildbar & \\
  \\
  Verhältnisse\index{Verhältnisse!bildbar} (Quotienten) sind bildbar & \\
  \\
  relative Häufigkeit\index{Häufigkeit!relative}     & \\
  \\
\end{tabular}\\

\subsection*{Aufgaben}
\aufgabenfarbe{Füllen Sie die Umfrage «Eine Umfrage» aus.}
\GESOAadB{398}{2-4}
\newpage
