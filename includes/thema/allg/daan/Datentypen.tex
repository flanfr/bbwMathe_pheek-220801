%%
%% 2019 07 04 Ph. G. Freimann
%%

\section{Datentypen}\index{Datentypen}
\sectuntertitel{Mit Statistik kann man alles beweisen...}

\theorieGESO{363}{21}
\theorieTALS{284}{4.1}
%%%%%%%%%%%%%%%%%%%%%%%%%%%%%%%%%%%%%%%%%%%%%%%%%%%%%%%%%%%%%%%%%%%%%%%%%%%%%%%%%
\subsection*{Lernziele}

\begin{itemize}
  \item Grundlagen
  \item Tabellenkalkulation
  \item Datengewinnung
  \item Datenqualität
  \item Datentypen (Merkmalsausprägung)
\end{itemize}


\newpage
\subsection{Merkmalstypen}
Merkmale werden in verschiedene Kategorien und Skalen eingeteilt. Hier ein Versuch eines Überblickes:

\bbwCenterGraphic{13cm}{allg/daan/img/Datentypen.png}

\GESO{\cite{marthaler17} S. 376}\TALS{\cite{frommenwiler17alg} S. 249}

Die «Datentypen» (\zB Ganze Zahlen) werden in der deskriptiven
Statistik auch \textbf{Merkmalsausprägung}\index{Merkmalsausprägung} genannt.

\subsection*{Aufgaben}
\GESO{In Kapitel 21 (S. 379) sind dreizehn
  Beispiele aufgezeigt. Wählt Euch in Teamarbeit ein Beispiel aus, die
  ihr in max. 5 Minuten der Klasse erklärt:
  \begin{itemize}
  Welcher Datentyp
\item
  liegt zu Grunde? 
\item
  Welche Darstellungsart ist sinnvoll? 
\item
  Welche Erhebungsart wurde verwendet? 
\item
  Welche Fehler könnten sich einschleichen?
  \end{itemize}
}%% END GESO
\TALSAadB{249}{942. 943.}
\GESOAadB{398}{1}

