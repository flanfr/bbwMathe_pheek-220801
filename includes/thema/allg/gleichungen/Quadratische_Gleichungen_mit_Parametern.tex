%%
%% 2019 07 04 Ph. G. Freimann
%% Ergänzung für TALS zu Quadratischen Gleichungen I (Ohne Parameter)
%%

\newpage
\section{Quadratische Gleichungen II (mit
  Parametern)}\index{Gleichungen!quadratische mit Parametern}
\sectuntertitel{Welches ist denn nun die Gesuchte?}

\theorieTALS{93}{2.3.1}
%%%%%%%%%%%%%%%%%%%%%%%%%%%%%%%%%%%%%%%%%%%%%%%%%%%%%%%%%%%%%%%%%%%%%%%%%%%%%%%%%
\subsection*{Lernziele}

\begin{itemize}
\item Taschenrechner: Mit Parameter(n)
\item Taschenrechner: Visualisierung, Interpretation
\end{itemize}

\subsection{Vorzeigeaufgabe 262 a)}
Für welchen Wert von $a$ wird die folgende Gleichung quadratisch?
$$ax^3 + 7x^2 -6 = 5(x^3 -x)$$

\noTRAINER{%%
  \vspace{5cm}
}
\TRAINER{
  $\Rightarrow ax^3 + 7x^2 -6 = 5x^3 -5x$\\
  $\Rightarrow ax^3 - 5x^3 + 7x^2 + 5x -6 = 0$ (Grundform)\\
  $\Rightarrow (a-5)x^3 + 7x^2 + 5x - 6 = 4$\\
  Die Gleichung wird quadratisch, wenn $a-5 = 0$ ist.
}

\subsection{Aufgaben}
\TALSAadB{99}{ab 279}
