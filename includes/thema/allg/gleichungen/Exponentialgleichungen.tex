
%% 2019 07 04 Ph. G. Freimann
%%

\section{Exponentialgleichungen\TALS{ ...}}\index{Gleichungen!Exponentialgleichungen}
\sectuntertitel{$P\neq NP ?$}

\theorieGESO{199}{12}
%%%%%%%%%%%%%%%%%%%%%%%%%%%%%%%%%%%%%%%%%%%%%%%%%%%%%%%%%%%%%%%%%%%%%%%%%%%%%%%%%
\subsection*{Lernziele}

\begin{itemize}
\item Exponentialgleichungen
  \item logarithmische Gleichungen \GESO{(\cite{marthaler17} S. 203
    Kap. 12.2)}
\item Exp-Vergleich
\item Logarithmen mit $\ln()$
\TALS{\item Exponentialgleichungen mit Parametern}
\end{itemize}

\subsection{Exponentialgleichung}
Bei einer \textbf{Exponentialgleichung} kommt die gesuchte Größe im
Exponenten (von Potenzen) vor. Beispiel:

$$5^x = 32$$

Typischerweise werden diese Gleichungen gelöst, indem auf beiden
Seiten mit dem 5er Logarithmus logarithmiert wird:

\begin{beispiel}{andere Basen}{beispiel_andere_basen}
$$x=\log_5(32) \approx 2.15338$$

Es können jedoch auch Logarithmen zu anderen Basen verwendet werden:

\begin{tabular}{rccl}
  \             & $5^x$           &=&    32          \\
  $\Rightarrow$ & $\log(5^x)$     &=&  $\log(32)$     \\
  $\Rightarrow$ & $x\cdot\log(5)$ &=&  $\log(32)$     \\
  $\Rightarrow$ & $x$             &=&  $\frac{\log(32)}{\log(5)} = \log_5(32) \approx 2.15338$ 
\end{tabular}
\end{beispiel}
\newpage

\begin{rezept}{Exponentialgleichung lösen}{rezept_allgemeine_exponentialgleichung}
  $$8^{x-1}=7^{x+2}$$
  \TRAINER{
  Erst mal auf beiden Seiten logarithmieren, mit einem Logarithmus zu beliebiger Basis:
  $$\log(8^{x-1})=\log(7^{x+2})$$
  $$(x-1)\cdot{}\log(8)=(x+2)\cdot{}\log(7)$$
  ausmultiplizieren
  $$\log(8)x-\log(8)=\log(7)x+2\log(7)$$
  und $x$ auf eine Seite bringen
  $$\log(8)x-\log(7)x = 2\log(7)+\log(8)$$
  ausklammern
  $$x(\log(8)-\log(7)) = 2\log(7)+\log(8)$$
  und durch die Klammer $(\log(8)-\log(7))$ teilen
  $$x = \frac{2\log(7)+\log(8)}{\log(8)-\log(7)}\approx 44.718$$}
  \noTRAINER{\mmPapier{10.8}}
\end{rezept}


\GESO{Siehe auch \cite{marthaler17} Seite 200 im roten Kasten.}

\subsection*{Aufgaben}
\TALSAadB{???}{???}
\GESOAadB{206}{2. a) d) g) h) 3. a) b) c) d) f) 4. a) - h) 9. 10.}

