%%
%% 2019 07 04 Ph. G. Freimann
%%
\section{Textaufgaben}\index{Textaufgaben}
\sectuntertitel{Auf dem Schiff sind 26 Schafe und 10 Ziegen. Wie alt ist der Kapitän?}

\theorieGESO{124}{8.5}
%%%%%%%%%%%%%%%%%%%%%%%%%%%%%%%%%%%%%%%%%%%%%%%%%%%%%%%%%%%%%%%%%%%%%%%%%%%%%%%%%
\subsection*{Lernziele}

\begin{itemize}
\item Texte analysieren, Unbekannte bestimmen
\item Gleichung aufstellen
\item Probe, Lösungsmenge
\end{itemize}
(Beispiele: Zahlenrätsel, Geschwindigkeit, Arbeit, Leistung)
\newpage

\begin{rezept}{systematischer Lösen von Textaufgaben}{}
\index{Textaufgaben!Gleichungen}
Die folgenden Vorgehensschritte\footnote{Entliehen aus dem Skript von Michael
  Rohner (BBW)} sind sehr hilfreich, auch wenn einige davon bei
  einfacheren Aufgaben gut übersprungen werden können:

\begin{enumerate}\label{textaufgaben_verfahren_in_sieben_schritten}
\item Aufgabe analysieren und verstehen:
  \begin{itemize}
  \item zuerst ganz durchlesen
  \item Situation in eigenen Worten formulieren / Aufgabe verstehen.
  \item gegebene \& gesuchte Informationen herausschreiben
    (ev. eine Skizze / Tabelle machen)
  \end {itemize}
  
\item Unbekannte Größe(n) eindeutig einführen
  (Variable immer möglichst präzise (inkl. Einheiten) definieren!)

\item mit den gegebenen Informationen Terme bilden
  
\item mit den Termen Gleichung(en) aufstellen
  (Gesuchtes \& Gegebenes in eine Beziehung zueinander bringen.)

\item Gleichung lösen

\item Kontrolle: Ergibt mein Ergebnis Sinn? (Plausibilitätsüberlegung)

\item  Antwortsatz geben (d.h. die mathematische Lösung sprachlich
    beschreiben)
    
\end{enumerate}    
\end{rezept}

\fbox{\parbox{\textwidth}{%
\begin{center}
{\textit{«Mathe ist der einzige Ort, wo Leute 144 Eier kaufen
\\
und sich keiner fragt wozu.»}}
\end{center}}}

\subsection*{Aufgaben}
\TALSAadB{81}{223. a) b) e) 225. a) d) e) f) 227.a) b) c) 225. a) d) e) f) 229. a) 230. a) 231. d) 232. b)}
\GESOAadB{131ff}{24. - 33., 35., 39., 42., 43., 46.-51., 54., 59. 62.}
