\section{Lineare Gleichungssysteme}\index{Gleichungssysteme}\index{lineare Gleichungssysteme}

\theorieTALS{124}{2.5.1}
\theorieGESO{135}{9}
\subsection*{Lernziele}
\begin{itemize}
	\item{Lineare Gleichung mit mehreren unbekannten}
	\item{Grundform}
	\item{Lösungsmethoden}
	\begin{itemize}
		\item{Einsetzungsverfahren}
		\item{Gleichsetzungsverfahren}
		\item{Additionsverfahren}
		\TALS{		\item{Substitution}}
		\TALS{		\item{Gauss-Algorithmus}}
		\item{Graphische Methode}
		\item{Taschenrechner}
	\end{itemize}

\end{itemize}

\newpage
\subsection{Einstiegsbeispiel}
Ein Apfel koste CHF $a$, eine Birne CHF $b$. Wenn ich drei Äpfel und zwei Birnen kaufe, bezahle ich CHF 3.10. Wenn ich hingegen vier Äpfel und fünf Birnen kaufe, so muss ich CHF 6.00 bezahlen. Wie viel kostet ein Apfel, wie viel eine Birne?

Wir schreiben dies üblicherweise als sortiertes Gleichungssystem auf:

\gleichungZZ{3a + 2b}{3.10}{4a + 5b}{6.00}


\subsection{Einsetzungsverfahren}\label{einsetzungsverfahren}\index{Einsetzungsverfahren!Gleichungssysteme}
Beim Einsetzungsverfahren wird aus einer der beiden Gleichungen eine der beiden Variablen alleine gestellt. Zum Beispiel hier aus der zweiten Gleichung $b = ...$; danach wird dieser $b$-Term in die \textbf{andere} Gleichung eingesetzt, und so erhalten wir eine Gleichung mit einer Unbekannten.

\gleichungZZ{3a + 2b}{3.10}{4a + 5b}{6.00}

Aus der zweiten Gleichung erhalten wir für $b$:

\TNT{1.6}{$$b = {\color{orange}\frac{6.00-4a}{5}}.$$}

Diesen Term setzen wir nun für $b$ in die erste Gleichung ein:
$$3a+2b = 3.10$$

Dies ergibt

\TNT{2.0}{$$3a+2{\color{orange}\frac{6.00 - 4a}{5}} = 3.10.$$}

Wenn wir diese Gleichung nun nach $a$ auf"|lösen, so erhalten wir $a = 0.50$ (CHF).
Nun setzen wir a in eine der Gleichungen ein, lösen diese und erhalten $b = 0.80$ (CHF).

\GESO{\subsection*{Aufgaben}
\GESOAadB{151}{8. a) b) d) f)}
}

\newpage
\subsection{Additions- bzw. Subtraktionsverfahren}\index{Additionsverfahren!lineare Gleichnugen}\index{Subtraktionsverfahren!Gleichungssysteme}
Eine weitere Möglichkeit Gleichungen mit mehreren Unbekannten zu lösen ist das Verfahren, die linke und die rechte Seite miteinander zu addieren (bzw. voneinander zu subtrahieren), sodass eine Variable wegfällt. Dazu müssen die Gleichungen vorher so \textit{präpariert} werden, dass in einer Variablen (hier \zB $a$) die Koeffizienten übereinstimmen:

\gleichungZZ{3a + 2b}{3.10}{4a + 5b}{6.00}

  \begin{rezept}{Rechengesetz Additionsverfahren}{gesetz_additionsverfahren}
    Eine beliebige Variable wird gewählt und danach beide Gleichungen so multipliziert,
    dass die Zahl vor den Variablen gleich werden. Dies geht immer mit
    dem kleinsten gemeinsamen Vielfachen (kgV); kann aber auch mit
    einem beliebigen Vielfachen beider Zahlen erreicht werden.
  \end{rezept}

Wir wählen einfach so in obigem Beispile die Variable $a$ und multiplizieren die obere Gleichung mit 4 und die untere mit 3 und
erhalten:

\TNT{2.8}{\gleichungZZ{12a + 8b}{12.40}{12a + 15b}{18.00}}

Nun können wir die obere Gleichung von der unteren Gleichung abziehen (damit die Variable $a$ eliminiert wird)
und erhalten:

$$0a  + (15b - 8b)  = 18.00 - 12.40$$
Also
$$7b = 5.60,$$ und somit
$$b = 0.80.$$
Nun setzen wir dieses $b$ in eine der beiden obigen Gleichungen ein und erhalten nach dem Auf"|lösen $a=0.50$ (CHF).
\newpage


\GESO{\subsection*{Aufgaben}
\GESOAadB{150}{7. b) c) e) g)}
}
\newpage


\subsection{Gleichsetzungsverfahren}\label{lin_gl_gleichsetzungsverfahren}\index{Gleichsetzungsverfahren}
Beim Gleichsetzungsverfahren werden zwei Gleichungen nach der selben Variable \textit{aufgelöst} und die beiden Gleichungen der Form $a = ...$ einander gleichgesetzt.

Lösen wir also beide Gleichungen nach $a$ auf:
\gleichungZZ{3a + 2b}{3.10}{4a + 5b}{6.00}
und erhalten:

\TNT{4.4}{\gleichungZZ{a}{{\color{orange}\frac{3.10 - 2b}{3}}}{a}{{\color{green}\frac{6.00 - 5b}{4}}}}


Setzen wir nun die beiden $b$-Terme gleich, so erhalten wir die folgende Gleichung:
$${\color{orange}\frac{3.10 - 2b}{3}} = {\color{green}\frac{6.00 - 5b}{4}}$$
Nach Auf"|lösung erhalten wir $b=0.80$ (CHF) und das $a$ erhalten wir danach durch Einsetzen in eine der beiden bereits nach $a$ aufgelösten Gleichungen: $a=0.50$ (CHF).

Das Gleichsetzungsverfahren ist ein Spezialfall des
Einsetzungsverfahrens\totalref{einsetzungsverfahren}.
\newpage


\GESO{\subsubsection*{Aufgaben}
Oft kommen auch in Gleichungssystemen Bruchgleichungen vor. Tipp:
bevor Sie die folgenden Aufgaben lösen, betrachten Sie die folgende
Abkürzung, um eine Bruchgleichung zu vereinfachen...

$$\frac{\color{green}2-x}{\color{red}2x-3} = \frac{\color{red}5-y}{\color{green}y-4}$$
... wird mit «übers Kreuz multiplizieren» zu:

$$({\color{green}2-x})({\color{green}y-4}) = ({\color{red}2x-3})({\color{red}5-y})$$

Lösen Sie mit der für Sie am besten geeigneten Methode, indem Sie die
Gleichungen in der Regel zunächst in die Grundform bringen.
  \GESOAadB{151}{9. b), d), e) und f)}
}
\newpage

%%%%%%%%%%%%%%%%%%%%%%%%%%%%%%%%%%%%%%%%%%%%%%%%%%%%%%%%%%%%%%%%%%%%%%%%%%%%%%%%%%%%%

\subsection{Graphische Methode}\index{graphische Methode!Gleichungssysteme}
Das Gleichsetzugsverfahren\totalref{lin_gl_gleichsetzungsverfahren}kann auch als Schnittpunkt zweier linearer Funktionen
\totalref{lineare_funktionen}
aufgefasst werden. Betrachten wir folgende beiden Gleichungen:

\gleichungZZ{2x - 10y}{-10}{6x+15y}{60}

Lösen wir beide Gleichungen nach $y$ auf, so erhalten wir zwei Funktionsgleichungen:

\gleichungZZ{f: y}{\noTRAINER{...............}\TRAINER{0.2x + 1}}{g:y}{\noTRAINER{...............}\TRAINER{-\frac{2}{5}x + 4}}


Zeichnen Sie die beiden linearen Funktionen als Graph ins folgende
Koordinatensystem ein und lesen Sie die Lösung \TRAINER{bei $(5|2)$} ab:

\noTRAINER{
  \bbwGraph{-6}{8}{-1}{6}{}
}
\TRAINER{
  \bbwGraph{-6}{8}{-1}{6}{
    \bbwFuncC{\x * 0.2 + 1}{-5.5:8}{green}
    \bbwLetter{7.5,3}{f}{green}
    \bbwFuncC{-0.4*\x + 4}{-1:7}{blue}
    \bbwLetter{7.5,1}{g}{blue}
  }
}
\newpage


\GESO{\subsection*{Aufgabe Graphisch}
  Lösen Sie die folgende Aufgabe Graphisch. Bringen Sie die beiden Gleichungen erst in die Form $y=ax+b$ der Linearen Funktionen. Zeichnene Sie danach beide Funktionen in ein Koordinatensystem ($x$ von -6 bis 1 und $y$ von -3 bis 5). Schätzen Sie die Lösung, bevor Sie diese berechnen.
\GESOAadB{151}{8. g)}
}
\newpage


\GESO{\subsection{Taschenrechner}\index{Taschenrechner!Gleichungssysteme}
Lineare Gleichungssysteme sind so zentral, dass viele heutige Taschenrechner diese lösen kann.

\gleichungZZ{3a + 2b}{3.10}{4a + 5b}{6.00}

\GESO{Suchen Sie \tiprobutton{2nd}\tiprobutton{tan_sys-solv} und geben Sie die Zahlen 3, 2, 3.10, 4, 5 bzw. 6.00 in die entsprechenden Felder ein.}
\TALS{Definieren Sie das Gleichungssystem gls:=\{3a+2b=3.1, 4a+5b=6.0\}. Dies können Sie nun einfach mit solve(gls,\{a, b\}) auflösen lassen.}

\GESO{
\begin{bemerkung}{}{}
  Die Variable beim TI-30 PRO müssen bei 2x2-Gleichungssystemen $x$ und $y$  heißen.
\end{bemerkung}
}

\GESO{
  \subsubsection{Eingabe negativer Zahlen}
  Lösen Sie das folgende Gleichungssystem mit dem Taschenrechner.

  \gleichungZZ{-4x + (-8)y}{16}{3x-5y}{32}

  Beachten Sie die Eingabe negativer Zahlen auf dem TI 30 Pro
  MathPrint. Das negative Vorzeichen wird mit \tiprobutton{neg} eingegeben,
  wohingegen die Subtraktion mit dem einfachen \tiprobutton{minus} eingegeben wird.

  \TNT{2.4}{Lösung: $x=4$, $y=-4$\vspace{22mm}}

}

\GESO{
\aufgabenfarbe{  
  Prüfen Sie mit diesem Wissen von Seite 150ff die Resultate von Aufgabe 7. g) [$x=\frac{42}{61}$ und $y=\frac{60}{61}$], 7. b) [$x=\frac52$ und $y=-\frac{15}{2}$] 8. a) [$x=2$ und $y=6$] und 8. b) [$x=-2$ und $y=2$]
}}

\newpage
}
\newpage

\TALS{\subsection{Lineare Abhängigkeit}\index{abhängig!linear}\index{Lineare
  Abhängigkeit}

Lösen Sie das folgende Gleichungssystem vorerst mit dem
Taschenrechner:

\gleichungZZ{9x-6y}{18}{30x-20y}{60}

Die Lösung ist nicht vielsagend.

Durch die Additionsmethode erhalten wir folgendes Gleichungssystem:

\gleichungZZ{90x-60y}{180}{90x-60y}{180}
oder $$0=0$$.

Dies bedeutet, wir haben durch Äquivalenzumformungen nun zweimal die selbe
Gleichung da stehen. Wir können $x$ nur in Abhängigkeit von $y$
berechnen (mehr nicht):

$$x=\frac{6+2y}{3}$$

Oder wir können $y$ in Abhängigkeit von $x$ berechnen:
$$y=\frac{3x-6}{2}$$

Wichtig ist vor allem, dass Sie auch die Antwort des Taschenrechners verstehen!

\newpage}

\subsection{Substitution}\index{Substitution!Gleichungssysteme}\index{Lineare Gleichungssysteme!mit Substitutionsmethode}
Manchmal gibt es Situationen, in denen ein Gleichungssystem besser mit
einer Ersetzung (Substitution) als mit sturem Ausmultiplizieren gelöst
werden kann.

Betrachten Sie einmal das folgende Gleichungssystem:

\gleichungZZ{\frac{2a}{3+b} - \frac{b}{5-a}}{1}{\frac{3a}{b+3} + \frac{2b}{5-a}}{19}

Es ist offensichtlich, dass die Terme $\frac{a}{3+b}$ und
$\frac{b}{5-a}$ mehrfach vorkommen.

Hier bietet sich eine Ersetzung (Substitution) an:

$$X := \LoesungsRaum{\frac{a}{3+b}}$$

und

$$Y := \LoesungsRaum{\frac{b}{5-a}}$$

Das neue entstandene Gleichungssystem ist viel übersichtlicher und
auch einfacher zu lösen:

\TNT{2.4}{\gleichungZZ{2X - Y}{1}{3X+2Y}{19}}

Nach dem Auf"|lösen (\zB Taschenrechner) erhalten wir $X=\LoesungsRaum{3}$ bzw. $Y=\LoesungsRaum{5}$. Mit diesen Werten
können wir $a$ bzw. $b$ bestimmen.
\newpage

\textbf{Rücksubstitution}\index{Rücksubstitution}\,\\

\vspace{1mm}

\TNT{10.8}{

  \gleichungZZ{3}{\frac{a}{3+b}}{5}{\frac{b}{5-a}}

    und somit:

\gleichungZZ{9+3b}{a}{25-5a}{b}

...sortieren...

\gleichungZZ{a-3b}{9}{5a+b}{25}

Nach dem Auf"|lösen erhalten wir:

$$\mathbb{L}_{(a;b)} = \left\{ \left(\frac{21}{4} ;  -\frac{5}{4} \right)  \right\}$$
}%% END TNT


\subsection*{Aufgaben}
\TALSAadB{127ff}{390. b), 391.}
\GESOAadB{152}{12. c), 13. a)}
\olatLinkGESOKompendium{2.2.2}{14}{33. bis 34.}

\newpage

\TALS{\subsection{Gleichungssysteme mit Parametern}\index{Parameter!Gleichungssysteme}\index{Lineare Gleichungssysteme!mit Parametern}
Das folgende Gleichungssystem hat offensichtlich vier und nicht wie
üblich zwei Variable. Wenn wir das System hingegen nach $x$ und $y$
auf"|lösen, so können wir diese beiden Größen
\textbf{in Abhängigkeit} der beiden Parameter\footnote{Parameter =
  Bei- oder Nebenmaß} ($p$ und $q$) ausdrücken.

\gleichungZZ{3x-5y}{p-5q}{5x+6y}{16p+6q}

Zum Lösen können wir wieder mit der Additionsmethode vorgehen, indem
wir die erste Gleichung mit 6 und die zweite Gleichung mit 5 multiplizieren\footnote{Danke Melisa für den Tipp: Natürlich könnte man auch zuerst die erste Gleichung mit 5 und die zweite Gleichung mit 3 multiplizieren, um so das $x$ zu eliminieren. Doch mit Melisas Trick, fällt auch das $q$ direkt weg.
}:

\gleichungZZ{18x-30y}{6p-30q}{25x+30y}{80p+30q}

Nach Addition der beiden Gleichungen erhalten wir $43x=86p$, was uns zu $x=2p$ bringt.
Einsetzen in eine der beiden ursprünglichen Gleichungen liefert $y=p+q$.

\subsection*{Aufgaben}
\TALSAadBMTA{127ff}{392. a) b), 394. a)}
\GESOAadBMTA{151}{10. a) b) c) und d)}

  \newpage}

\TALS{\subsection{Drei Unbekannte}\index{Drei unbekannte!Gleichungssysteme}\index{Lineare Gleichungssysteme!mit drei Unbekannten}\index{Gleichungssysteme!lineare mit drei Unbekannten}
Das Additionsverfahren funktioniert auch mit mehr als zwei unbekannten. Betrachten wir dazu das folgende lineare Gleichungssystem:

\gleichungDD{2x -3y +4z}{33}{3x+2y-z}{-5}{5x-y-5z}{-12}

Zunächst eliminieren wir die Variable $x$. Dazu erzeugen wir die Gleichung (IV), indem wir die erste Gleichung mit 3 und die zweite Gleichung mit 2 multiplizieren und danach die Gleichungen voneinander abziehen.

(IV)

\TNT{3.6}{\gleichungZZ{6x-9y+12z}{99}{6x + 4y -2z}{-10}
  Daraus ergibt sich (IV):
  $$-13y + 14z = 109$$
}%%

Analog mit Gleichung (II)$\cdot{}5$ und (III)$\cdot{}3$:

(V)

\TNT{3.6}{\gleichungZZ{15x+10y-5z}{-25}{15x-3y-15z}{-36}
  Daraus ergibt sich (V):
  $$-13y + 10z = 11$$
}
\newpage
Gleichung (IV) und (V) enthalten nur noch zwei Variable und können nach einem gewohnten Verfahren gelöst werden:

\TNT{6.0}{\gleichungZZ{-13y + 14z}{109}{13y+10z}{11}
  und somit $24z = 120$ und schließlich $z=5$. Dieses $z$ setzen wir in Gleichung (V) ein:
  $$-13y + 10\cdot{}(5) = 11$$
  und wir erhalten $y=-3$

  Zuletzt $z=5$ und $y=-3$ einsetzen in die Gleichung (I):
  $$2\cdot{}x -3\cdot{}(-3) + 4\cdot{}(5) = 33$$
  was uns zu $x=2$ bringt.

  $$\mathbb{L}_{(x;y;z)} = \{(2; -3; 5)\}$$
}%% END TNT

\newpage}



\subsection{Textaufgaben zu linearen Gleichungssysetmen}
Typische Textaufgaben, die auf Gleichungssysteme führen werden in den folgenden Kapiteln angeschaut.


Verwenden Sie zum Lösen jeweils das \textit{Verfahren in sieben
Schritten}, das Sie bereits aus dem Kapitel zu linearn Gleichungen kennen\totalref{textaufgaben_verfahren_in_sieben_schritten}.

\subsubsection{Mischaufgaben}
\TNT{10}{Vorzeigen Aufgabe Farben mischen aus dem Kompendium}
\GESO{
  \subsubsection*{Aufgaben zum Misch-Typ}
  \GESOAadB{156}{37., 39., 38., 40., 41.}
}%% END GESO
\newpage


\subsubsection{Zinsaufgaben}
\TNT{10}{Vorzeigen Aufgabe zu Zinsen aus dem Kompendium}
\GESO{
  \subsubsection*{Aufgaben zum Zins-Typ}
  \GESOAadB{157}{44., 45.}
}%% END GESO
\newpage



\subsubsection{Ziffern-Aufgaben}

\paragraph{Rest-Aufgaben}: Eine Zahl $x$ ergibt beim Teilen durch 5 das Resultat 3 mit Rest 2.
\TNT{5.2}{Was bedeutet «Teilen mit Rest»\index{Rest!teilen mit}\index{teilen mit Rest}?
  $$x : 5 = 3 Rest 2$$
  Dies heißt eigentlich nichts anderes als:
  $$3\cdot{} 5 + 2 = x $$
  Somit ist $x=17$.
}%% end TNT


\paragraph{Ziffern-Tauschen}: Eine zweistellige Zahl $z$ ist gesucht. Die Quersumme sei 9. Ebenso erhalte ich gleich viel, wenn ich zur Zahl $z$ drei addiere, wie wenn ich der Zahl die Ziffern tausche und dann 42 subtrahiere.

\TNT{5.2}{
  Die Variable sind die Ziffern! Sinnvoll: $x$ = Zehnerziffer, $y$= Einerziffer.
  Somit ist die Quersumme = $x+y$.
  Die Zahl ist aber nicht(!) $xy$, das wäre ja $x\cdot{}y$. Die Zahl ist
  $10\cdot{}x + y$. Die Ziffern-vertauschte Zahl ist nun $10\cdot{}y + x$.

  Damit lassen sich die Gleichnugen aufstellen:

  \gleichungZZ{10x+y+3}{10y+x-24}{x+y}{9}
  Ordnen:
  \gleichungZZ{9x-y}{-45}{x+y}{9}
  Taschenrechner: $x=2$ und $y=7$. Somit ist die Zahl $z=27$.
}%% END TNT

\TALS{\subsection*{Aufgaben}

\TALSAadB{133ff}{422. 432.}
}%% END TALS

\GESO{
  \subsubsection*{Aufgaben zum Ziffern-Typ}
\GESO{\aufgabenfarbe{Kompendium S. 14 Aufg. 37.}}
\GESOAadB{155ff}{28. und 32.}
}%% END GESO
\newpage
