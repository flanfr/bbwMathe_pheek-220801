%%
%% 2019 07 04 Ph. G. Freimann
%%

\section{Terme}\index{Term}
\sectuntertitel{Römische Bäder?}

\theorieTALS{10}{1.1.3}
\theorieGESO{17}{1.3}
%%%%%%%%%%%%%%%%%%%%%%%%%%%%%%%%%%%%%%%%%%%%%%%%%%%%%%%%%%%%%%%%%%%%%%%%%%%%%%%%%
\subsection*{Lernziele}

\begin{itemize}
 \item Termanalyse (Summe, Differenz, Produkt, Quotient, Potenz)
 \item Hierarchie der Terme (Vorrangregeln)
 \item Termumformungen
 \item Ausmultiplizieren
\end{itemize}
\newpage

\subsection{Term-Definition}\index{Term}
\begin{definition}{Term}{definition_term}
  Ein \textbf{Term} ist entweder
  \begin{itemize}
  \item ein Atom (eine Zahl (\zB{} $4.86$) oder  eine Variable (\zB{}
  $a$, $x$))
  \item ein Klammerausdruck (\zB{} \fbox{({\color{green}T})}, inkl.  Wurzeln \fbox{$\sqrt{\color{green}T}$})\footnote{Dabei wird der
  horizontale Strich wie eine Klammer aufgefasst.}
%%  \item weitere \TALS{Monome\index{Monom}}\GESO{Summanden (Teile einer Summe)}:
%%    \begin{itemize}
   \item eine \textbf{Potenz}\index{Potenz}\footnote{Als Exponent darf ein beliebiger Term eingesetzt werden, wohingegen als Basis lediglich Atome oder Klammerausdrücke verwendet werden dürfen, wegen der Verwechslungsgefahr.} (\fbox{${\color{green}T_1}^{\color{green}T_2}$} \zB{} $a^3$, $(2a - 4)^{x+2}$)
    \item  ein Bruchterm\footnote{Wie bei der Wurzel, dient der Bruchstrich als Klammerpaar: $\frac{U}{V}=(U):(V)$} (\fbox{$\frac{\color{green}T_1}{\color{green}T_2}$} \zB{} $\frac{2^x}{x^2}$)
    \item ein implizites \textbf{Produkt}\footnote{Ein implizites Produkt ist mit Koeffizienten angereicherter Ausdruck \textbf{ohne} Multiplikationszeichen.} (\zB{} ${\color{red}4a}{\color{green}T}$)\footnote{Tritt eine Zahl auf, so ist diese immer ganz links zu schreiben. Zahlen rechts von Ausdrücken werden mit einem Multiplikationszeichen versehen ($\cdot$).}
    \item ein explizites \textbf{Produkt}\index{Produkt} ($\cdot$;  ${\color{green}T_1}\cdot {\color{green}T_2}$ \zB{} $a\cdot(-1)$) bzw. ein expliziter \textbf{Quotient}\index{Quotient} ($:$, $/$, $\div$; $6a\cdot3b$ bzw. ${\color{green}T_1}:{\color{green}T_2}$ \zB{} $36m^2:12m^2$)
%%    \end{itemize}
  \item eine \textbf{Summe}\index{Summe} (bzw. \textbf{Differenz}\index{Differenz}) von \TALS{Monomen}\GESO{Summanden\index{Summand}}
  (Zum Beispiel bilden die folgenden
  vier «Pakete» eine Differenz):\\
  $-4x^2 + \frac{3a+b}{x} + \sqrt{5y^2-6} - 5t:2t$

    \end{itemize}
(In obiger Aufzählung hat der am höchsten stehende Term die größte
«Bindungskraft». Beispiel «Punkt vor Strich».)
\end{definition}

\textbf{Gegenbeispiele}
Keine Terme sind \zB{}: $x \cdot{}-8$, $\sqrt{+^2}$, $4+*($, $\frac{7}{+}$, $\frac{(a+b}{-c-d)}$.

Generell werden die fünf wichtigsten Termarten in die folgenden drei Kategorien eingeteilt:
\begin{itemize}
\item Potenz
\item Produkt und Quotient
\item Summe und Differenz
\end{itemize}


\newpage

\subsection{Vorrangregeln}\index{Vorrangregeln}
\GESO{
Es gilt Punkt vor Strich. Daneben bindet ein Exponent (\zB $5^8$) noch
stärker. Am stärksten binden Klammern oder horizontale Linien
(Bruchstrich, Wurzelzeichen).
}
\TRAINER{\TALS{
Im Compilerbau (Schreiben einer Programmiersprache) werden die
folgenden Vorrangregeln verwendet:

\begin{itemize}
\item Term :== Summand \{'+'|'-' Summand\}*\\
\TRAINER{$4a^3 - 6\cdot az^{(7+b)} : \sin(30)$}

\item Summand :== ExpilziterFaktor \{'$\cdot$'|'/' ExpliziterFaktor\}*\\
\TRAINER{$4a^3$, $6 \cdot az^{(7+b)} : \sin(30)$}

\item ExpliziterFaktor :== Faktor \{Faktor\}*\\
\TRAINER{$4a^3$, $6$, $az^{(7+b)}$, $\sin(30)$}

\item Faktor :== SkalarOderKlammerausdruck \{${\,}^{Term}$\}?\\
\TRAINER{$4$, $a^3$, $6$, $a$, $z^{(7+b)}$, $\sin(30)$}

\item SkalarOderKlammerausdruck :== Zahl |
           Variable |
           $\sqrt[Term]{Term)}$ | 
           '(' Term ')' |
           $\frac{Term}{Term}$|
           \textit{Funktionsname} '(' Term ')'\\
\TRAINER{$4$, $a$, $3$, $6$, $a$, $z$, $(7+b)$, $\sin(30)$}

\item \textit{Funktionsname} := 'sin', 'cos', 'tan', 'log', 'lg', 'ln', ....
\end{itemize}
}}

\TALS{
\textbf{Achtung} Bei zusammengeschriebenen Faktoren (\zB $ab$) bindet
           die Multiplikation stärker als beim expliziten verwenden
           des Multiplikationszeichens (\zB $a\cdot{}b$). Beispiel
           $a\cdot bm = a\cdot (b\cdot m)$.

Gleich ein Beispiel, wo dies eine Rolle spielt:
$$111x : 37x = (111x) : (37x) = 3$$
Aber
$$111\cdot x : 37\cdot x = ((111 \cdot x) : 37) \cdot x = 3x^2$$
}%% END TALS


\newpage
\subsection{Terme mit Namen}
Oft gibt man Termen Namen, um sie einfacher identifizieren und
bezeichnen zu können. So könnte \zB die Oberfläche einer
Konservendose mit $A$ (Area) wie folgt bezeichnet werden, wenn $r$ den
Radius bzw. $h$ die Höhe bezeichnen:

$$A(r, h) = r^2\pi + r^2\pi + 2r\pi{}h$$

Dabei ist $A$ der Name des Terms und $r$ bzw. $h$ sind die Parameter.
\vspace{3mm}
\begin{beispiel}{Werte einsetzen}{beispiel_terme_werte_einsetzen}
  Wir betrachten den Term

  $T({\color{red}a}, {\color{blue}x}) = 5{\color{red}a}{\color{blue}x} - {\color{red}a} + 7$.

  Nun gilt, dass für jeden Parameter im Term (hier ${\color{red}a}$
  bzw. ${\color{blue}x}$) jede Zahl eingesetzt
  werden kann.

  $T({\color{red}2}, {\color{blue}3}) = \LoesungsRaumLang{5\cdot{}{\color{red}2}\cdot{\color{blue}3} - {\color{red}2} + 7}$

  Es können auch Terme anstelle der Parameter eingesetzt werden:

  $T({\color{red}2y}, {\color{blue}z-4}) = \LoesungsRaumLang{5\cdot({\color{red}2y})\cdot({\color{blue}z-4}) - {\color{red}2y} + 7}$
\end{beispiel}

\begin{bemerkung}{}{}
Achten Sie beim Ersetzen des Parameters durch das Argument auf die
Klammersetzung. Wenn nicht sicher: Immer Klammern um die Argumente
setzen, welche für die Parameter eingesetzt werden:

$$ x = z-4$$
$$ x  \rightarrow (z-4)$$
\end{bemerkung}
\newpage

\subsubsection{Übungsbeispiel}
$T(b, y) = 7y^2 - 4by$

Wir berechnen

$T(4, -2+3) = $ \noTRAINER{.......................................................}\TRAINER{$T(4, -2+3)=7\cdot (-2+3)^2 - 4\cdot 4 \cdot (-2+3) = 7-16=-9$}

und

$T(x, 2b) = $ \noTRAINER{.........................................................}\TRAINER{$=T(x, 2b) = 7(2b)^2 - 4\cdot x \cdot (2b) = 28b^2 - 8bx =
  4b (7b-2x)$}
\subsection*{Aufgaben}
\TALSAadB{11}{8,9}
\GESOAadB{24ff}{21., 22., 24., 26. a) c), 25., 27. a) 28., 29.}
\newpage
