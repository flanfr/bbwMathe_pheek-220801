\subsubsection{Negative Exponenten}
Für sinnvolle Rechengesetze (s. oben) müsste auch gelten:

\vspace{4mm}

\TRAINER{$a^5 \cdot a^{(-2)} = a^{5 + (-2)} = a^3$}%%
\noTRAINER{$a^5 \cdot a^{(-2)} = ....................................................$}%%

\vspace{4mm}

Dies gilt, wenn wir $a^{-2}$ als $\frac{1}{a^{2}}$ definieren.

\begin{definition}{}{}
$$a^{-n} := 1 : \underbrace{a : a : a : ... : a}_{n \textrm{\ Divisoren}}$$
\end{definition}

\begin{itemize}
	\item \fbox{$a^{-n} = \frac{1}{a^{+n}} = \left(\frac{1}{a}\right)^n$}   

  Begründung:

  \TNT{2.4}{
     \TALS{Wir wissen schon: $a^{m-n} = a^m : a^n = a^m\cdot{}\frac{1}{a^n}$   und es soll gelten:
       $a^{m-n} = a^m\cdot{} a^{-n}$ Beide Gleichungen gleichsetzen:
       $\Longrightarrow a^{m}\cdot{} \frac{1}{a^n} = a^m\cdot{} a^{-n} $ Beide Seiten durch $a^m$ teilen:
     $\frac{1}{a^n}=a^{-n}$}
     \GESO{$a^5\cdot{} a^{-2} = a^{5+(-2)} = a^3$ daraus folgt
       (beide Seiten durch $a^5$ teilen)\\
       $a^{-2} = \frac{a^3}{a^5} = \frac{a^3 : a^3}{a^5 : a^3} = \frac{1}{a^2}$
     }%% end GESO
}%% END TRAINER

   
   Analog:
	\item \fbox{$a^{n} = \frac{1}{a^{-n}} = \left(\frac{1}{a}\right)^{-n}$}   


  \item \fbox{$\left(\frac{a}{b}\right)^{-n} = \left(\frac{b}{a}\right)^{+n}$}
    
    \TALS{ Begründung

      \TNT{2.4}{$\left( \frac{b}{a} \right)^n  =
       \left(b \cdot{} \frac{1}{a} \right)^n =
       b^n \cdot \left(\frac{1}{a}\right)^n =
       \left(\frac{1}{b}\right)^{-n} \cdot{} a^{-n} =
       \left(\frac{1}{b}\cdot{}a\right)^{-n} =
       \left(\frac{a}{b}\right)^{-n} 
       $}} %% END TALS


\end{itemize}

Nun gilt:

\TRAINER{$a^5 \cdot a^{(-2)} = a^5 \cdot \frac{1}{a^2} =\frac{a\cdot a\cdot a\cdot a\cdot a}{ a\cdot a}$}

\noTRAINER{$a^5 \cdot a^{(-2)} = ......................................................$}

%%$$\left(\frac{1}{a}\right)^{-n} = \frac{1}{\left(\frac{1}{a}\right)^n} = a^n$$
\newpage



\subsubsection{Null}

Ebenso müsste $a^5 \cdot a^0 = a^{5+0} = a^5$ gelten. Dies ist aber
nur möglich, wenn wir $a^0 := 1$ definieren ($a\ne 0$).

\begin{definition}{Exponent Null}{} Für alle Basen $a \in \mathbb{R}\backslash\{0\}$ gilt:
\begin{center}
\fbox{$a^0 := 1$}
\end{center}
\end{definition}

Zweite Begründung wenn die Rechengesetze gelten sollten: $a^0 = a^{1-1} = \frac{a^1}{a^1} = 1$

%\textbf{Rechengesetze zusammengefasst:}

%\begin{itemize}
%\item  $\frac{a^m}{a^n} = a^{m-n}$ (Dies gilt auch wenn $n > m$.)
 
%\item $a^{-n} := a^{0-n}=\frac{a^0}{a^n} = \frac{1}{a^n} = \left(\frac{1}{a}\right)^n$

%\item
%$\left(\frac{1}{a}\right)^{-n} = \frac{1}{\left(\frac{1}{a}\right)^n} = a^n$


%\item $\left(\frac{a}{b}\right)^{-n} = \left(\frac{b}{a}\right)^{+n}$ gilt daher auch. 
%\end{itemize}

\subsection{Aufgaben}

\TALS{Potenzen:}\TALSAadB{32ff}{Von Hand: 79. a) 82. a) 83. b) 86.b)
91. a) l) 94. c) f) 96. h) 101. a)\\
Von Hand \textbf{und} Taschenrechner (TR): 103. a)\\
Von Hand: 103. b)\\
TR: 103. c) 106. h)}

\GESOAadB{67ff}{15., 18. c), 19. b), 20. h), 25. f), 26. b), 31. b),
  34. b), 38. c) e), 41. e), 44. i)}

Optional: \GESOAadB{72ff}{51. (Koch)}
\newpage
