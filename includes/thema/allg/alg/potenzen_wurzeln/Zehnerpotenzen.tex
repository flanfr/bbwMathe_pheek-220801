\subsection{Zehnerpotenzen}\index{Zehnerpotenzen}
\theorieTALS{32}{1.5.5}

\bbwCenterGraphic{8cm}{allg/alg/potenzen_wurzeln/img/one_in_a_mellon.jpg}

Neben den im Buch (\GESO{\cite{marthaler17}
  S. 62}\TALS{\cite{frommenwiler17alg} S. 40}) angegebenen SI-Einheiten (Kilo, Mega, ...) sind die
Namen der positiven Zehnerpotenzen im Englischen anders als im
Deutschen. Hier zur Vollstänigkeit:

\begin{tabular}{lclll}
Exponent   & SI-Kürzel & SI-Vorsätze & Deutsch & Englisch\\
\hline\\
$10^{2}$   & h         & Hekto       & Hundert   & hundred\\
$10^{3}$   & k         & Kilo        & Tausend   & thousand\\
$10^{6}$   & M         & Mega        & Million   & million\\
$10^{9}$   & G         & Giga        & Milliarde & billion\\
$10^{12}$  & T         & Tera        & Billion   & trillion\\
$10^{15}$  & P         & Peta        & Billiarde & quadrillion\\
$10^{18}$  & E         & Exa         & Trillion  & quintillion\\
\end{tabular}

\begin{tabular}{lclll}
Exponent   & SI-Kürzel & SI-Vorsätze & Deutsch\\
\hline\\
$10^{-1}$  & d         & Dezi        & Zehntel\\
$10^{-2}$  & c         & Centi       & Hundertstel\\
$10^{-3}$  & m         & Milli       & Tausendstel\\
$10^{-6}$  & $\mu$     & Mikro       & Millionstel\\
$10^{-9}$  & n         & Nano        & Milliardstel\\
\end{tabular}

\GESO{Mit dem Taschenrechner können große Zehnerpotenzen mit der \tiprobutton{EE}-Taste eingegben werden: 0.37 Milliarden: \tiprobutton{0}\tiprobutton{dot}\tiprobutton{3}\tiprobutton{7}\tiprobutton{EE}\tiprobutton{9}}%% END GESO
\newpage


\subsubsection{Ausklammern von Zehnerpotenzen}
Gegeben ist die folgende Summe. Leider etwas mühsam zum Lesen wegen der vielen Nullen. Klammern Sie Tausend (= $10^3$) aus:

$$5\,000 + 40\,000 + 6\,000\,000 + 70$$
\TNT{2.4}{$=10^3\cdot{} (5 + 40 + 6\,000 + 0.07 = 6054.07)$ \vspace{12mm}}


\paragraph{Ausklammern negativer Zehnerpotenzen:}
\,\\

Genauso, wie man positive Zehnerpotenzen ausklammern kann, kann man auch negative Zehnerpotenzen ausklammern. Dies ist insofern praktisch, um sich einen Überblick zu verschaffen, bei sehr kleinen positiven Größen:

Klammern Sie einen Millionstel (=$10^{-6}$) aus:


$$a\cdot{}10^{-6} + b\cdot{}10^{-2} + c\cdot{}10^{-5} + d\cdot{}10^{-1} = \LoesungsRaumLang{10^{-6} (a + b\cdot{}10^4 + 10c + d\cdot{}10^{5})}$$

\TALS{Zehnerpotenzen:}\TALSAadB{41ff}{101. a) b) c) d) 111. c) f) 112. b) d) e) f) k) m)}
\GESO{
  \subsection*{Aufgaben}
  \GESOAadB{65ff}{3. b) c), 4. c), 12., 15. und  16. a) und c)}
  \GESOAadB{68ff}{24. c), 28. c)}
  \GESOAadB{74ff}{52. a) c) d) 53. a) b) d) h) i) 56. a) b) 57. a) b) e) 60. 63. 64.}

}
\newpage
