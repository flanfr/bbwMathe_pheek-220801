%%
%% 2019 07 11 Ph. G. Freimann
%%

\section{Zehnerlogarithmus}\index{Logarithmus!Basis 10}\index{Zehnerlogarithmus}
\sectuntertitel{\textit{logos}: Das Verhältnis, \textit{arithmos}: Die Zahl}

\theorieGESO{97}{6.1}

%%%%%%%%%%%%%%%%%%%%%%%%%%%%%%%%%%%%%%%%%%%%%%%%%%%%%%%%%%%%%%%%%%%%%%%%%%%%%%%%%
\subsection*{Lernziele}

\begin{itemize}
\item Zehnerpotenzen
\item wissenschaftliche Notation (Repetition)
\item Zehnerlogarithmus
\item Notation
\item Logarithmengesetze I
\item Beispiele: Richterskala (Erdbeben), dB (Lautstärke), pH («Säuregehalt»)
\end{itemize}

\ifisALLINONE{
\subsubsection*{wissenschaftliche Notation}
Die wissenschaftliche Notation wurde bereits
eingeführt. Zur Erinnerung:\totalref{wissenschaftlicheNotation}.
}\fi{}%% END ALL IN ONE
\newpage

\subsection{Definition Zehnerlogarithmus}\index{Logarithmus!Definition Zehnerlogarithmus}
$${\color{blue}10}^{\color{red}3} = {\color{green}1000}$$

$$\sqrt[{\color{red}3}]{{\color{green}1000}} = {\color{blue}10}$$

$${\color{blue}\lg}({\color{green}1000}) = {\color{red}3}$$

Allgemein für ${\color{red}z}\in \mathbb{Z}$:

\begin{center}
  \fbox{\fbox{{$\lg({\color{green} 10^z}) = {\color{red} z}$}}}
\end{center}


Definition für $r\in\mathbb{R}, p \in \mathbb{R}^{+}\backslash\{0\}$:
\begin{definition}{}{}
  \begin{center}
    ${\color{red}r} = {\color{blue}\lg}({\color{green}p})$
    $:\Leftrightarrow$
    ${\color{blue}10}^{\color{red}r}={\color{green}p}$
    \end{center}
\end{definition}

Der Logarithmus ist der Exponent des Potenzwertes.

(Die Wurzel ist die Basis des Potenzwertes.)

\begin{beispiel}{Logarithmus der
    Wurzel}{beispiel_logarithmus_der_wurzel}
  Es gilt wegen den Gesetzen zu rationalen Exponenten:\\
$10^{\frac12} = \sqrt{10} \approx 3.1627$

  Und daraus folgt direkt $\lg(\sqrt{10}) = 0.5$.
\end{beispiel}

\subsubsection{Notation}
Der Zehnerlogarithmus $\lg()$ wird oft auch explizit mit der Basis 10
angegeben; dann wird anstelle von $\lg()$ $\log_{10}()$ geschrieben:

\begin{center}
\fbox{$\lg = \log_{10}$ }
\end{center}

Auf vielen Taschenrechnern steht einfach $\log$ anstelle $\lg$ oder $\log_{10}$.
\newpage


\subsection*{Aufgaben}
\GESOAadB{102ff}{2. 3. a) c) e) 4. a) 5. a)} 
\GESOAadB{107ff}{36. 37. 38. und 39.}

\TALSAadB{???}{???}
\newpage


\subsection{Rechengesetze}
Per Definition gilt:

$\lg(10) = 1$ und $\lg(1) = 0$.

Es gilt zudem $\lg(r\cdot s) = \lg(r) + \lg(s)$, denn

\TNT{6.0}{
$10^n * 10^m = 10^{n+m}.$
\\
Zahlenbeispiel:
$$\log(10^2\cdot{}10^4) = \log(10^{2+4}) = 2+4
= \log(10^2) + \log(10^4) $$

Oder allgemein (für $r = 10^n$ und $s = 10^m$)\footnote{\TRAINER{Wobei hier $n$ und $m$ nicht notwendigerweise in $\mathbb{N}$}}:
$$\lg(r \cdot s) = \lg(10^n \cdot 10^m) = \lg(10^{(n+m)}) = n+m = \lg(10^n) + \lg(10^m) = \lg(r) + \lg(s).$$
\vspace{3cm}
}%% END TNT

Beispiel: \TRAINER{Beim Vorzeigen mit den beiden 8ern von links und
  rechts beginnen.}

$${\color{gray}8=\lg(10^8)=\lg(10^{3+5})}=\lg(10^3\cdot{}10^5) = \lg(10^3)+\lg(10^5){\color{gray}=3+5=8}$$
