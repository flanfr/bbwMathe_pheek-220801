%%
%% 2019 07 04 Ph. G. Freimann
%%

\newpage
\section{Textaufgaben/Zinsrechnung}\index{Textaufgaben zur Zinsrechnung}
\sectuntertitel{``Klar hab' ich Probleme --- ich bin Mathelehrer.''}
%%%%%%%%%%%%%%%%%%%%%%%%%%%%%%%%%%%%%%%%%%%%%%%%%%%%%%%%%%%%%%%%%%%%%%%%%%%%%%%%%
\subsection*{Lernziele}

\begin{itemize}
  \item Der Zins ist eine  Multiplikation, der Zinseszins ist eine Potenz
\item Textaufgaben mit Zins und Zinseszins
\end{itemize}

\subsection{Einstiegsaufgabe}
Einstiegsaufgabe:
Ein Händler gibt Ihnen auf eine Ware von 234.50 CHF einen Jugend-Rabatt von
5\%, wenn Sie ihren BMS-Ausweis zeigen. Wenn Sie in bar bezahlen, erhalten Sie
einen weiteren Rabatt von 3\%. Ist es für Sie nun schlauer, zuerst den
Jugend-Rabatt (5\%) und danach den Bar-Rabatt(\%) einzufordern oder
ist die umgekehrte Reihenfolge schlauer? \TRAINER{Lsg: 216.10}
\newpage



\subsection{Zins als Faktor}\index{Zins}
Beispiel $2\%$ Zins auf Kapital CHF $1\,000.-$:\\

\begin{tabular}{l|l|l}
  \textit{Variable}  &   \textit{Bedeutung}   & \textit{Beispiel}\\%%
\hline%%
 $K_0$  & Startkapital & 1\,000.-  [CHF]\\\hline

 $p$    & Zinsfuß       & 2  [\%]\\\hline

        & Zins          & $\frac{K_0}{100[\%]}\cdot{}2[\%] =
                           K_0 \cdot{} \frac2{100} = $\\
        &               & $K_0\cdot 0.02$ \\\hline

$K_1$   & neues Kapital &   $K_0+$ Zins $=$ \\
        &               &   $K_0+K_0\cdot{}0.02 = $ \\
        &               &   $K_0\cdot(1+0.02) = $\\
        &               &   $K_0\cdot{}1.02$\\
\end{tabular}

\begin{bemerkung}{Faktor}{}

\textbf{100\% = 1} und  \textbf{2\% = 0.02}. Somit gilt \textbf{100\% + 2\% = 1.02}

\end{bemerkung}
\newpage


%Um von einer Ausgangsgröße 100\% zu berechnen, kann einfach der Faktor
%1.0 genommen werden. Ebenso kann der Faktor 0.03 genommen werden, um
%3\% der Größe zu berechnen. Ein Anwachsen eines Kapitals um 3\% ist
%also nichts anderes als das multiplizieren mit dem Faktor 1.03.

\subsection{Zinseszins}\index{Zinseszins}

\TRAINER{EinstiegsComics Donald Duck im OLAT}

Beim Zinseszins, wird bei jeder weiteren Verzinsung der bereits
erhaltene Zins weiterverzinst. Beispiel 2\%:

CHF 20.- $\stackrel{\cdot{}1.02}{\longrightarrow}$ 20.40 $\stackrel{\cdot{}1.02}{\longrightarrow}$ 20.808

$$K_2 = {\color{ForestGreen}K_1}\cdot{}1.02 = ({\color{ForestGreen}K_0\cdot{}1.02})\cdot{}1.02 = K_0\cdot{}1.02^2$$

Bei 2\% kann also jedesmal mit einem \textbf{Verzinsungsfaktor}\index{Verzinsungsfaktor} von
1.02 multipliziert werden.



\TRAINER{
Nach 100 jahren wachsen die 20 Taler von Tick Trick und Track auf
$20\cdot{}1.02^100\approx 144.89$ an.


Nach 1000 Jahren wachsen unsere CHF 20.-- also auf $20 \cdot{}
1.02^{1000}$ an ($\approx 8.0 Mia. Taler$)\footnote{Dies entspricht einem Faktor von fast 400 Millionen}. 
}

\subsubsection{Zinsformel}\index{Zins und Zinseszins}
\TRAINER{Hinweis an die Lehrperson: Insbesondere BM2 gut behandeln, denn der Stoff ist ev. in der Sekundarschule nicht behandelt worden (Sek B) oder es liegt generell zu weit zurück.}

Bei gegebenem Startkapital $K_0$ und gegebenem Zinsfuß $p$ (in \%) kann das Endkapital $K_n$ nach $n$ Jahren wie folgt berechnet werden:

\begin{center}\fbox{$K_n = K_0 \cdot{} a^n$}\end{center}

mit \textbf{Verzinsungsfaktor}

\begin{center}\fbox{$a := 1 + \frac{p}{100}.$}\end{center}


\TRAINER{Bemerkung: Die Zinseszinsformel beschreibt ein exponentielles Wachstum.}
\newpage

\subsubsection{Zinsbeispiele}

Berechnen Sie:

\begin{itemize}
  \item Berechnen Sie das Endkapital nach 20 Jahren bei einem
  Startkapital von CHF 15\,000.- und einem Zins von jährlich
  2.5\%.\\%%

\TNT{2.4}{Endkapital = $15\,000\cdot{} 1.025^{20}\approx 24\,579.25$ CHF\vspace{2cm}}%%
\item In einem Wald werden 200 Fichten für Weihnachten
  gepflanzt. Wegen der hohen Nachfrage wird der Bestand jedes Jahr um
  3\% ausgeweitet (vergrößert).
  Wie viele Fichten werden nach fünf Jahren gepfanzt?


  \TNT{2.4}{Endkapital = $200\cdot{} (1.03)^{5} \approx 231$ Fichten.\vspace{2cm}}%%
\item Ein Auto hat einen Neupreis von CHF 40\,000.-. Jedes Jahr müssen wegen Abnutzung ein Wertverlust von 3\% in Kauf
  genommen werden. Wie viel Wert hat das Auto noch nach 15 Jahren? (Achtung, hier ist der Zins negativ!)

\TNT{2.4}{Endkapital = $40\,000\cdot{} (1-0.03)^{15} = 40\,000\cdot{} 0.97^{15}\approx 25\, 330.-$ CHF\vspace{2cm}}
\end{itemize}
\newpage

\GESO{\subsection*{Aufgaben}}
\TALSAadB{???}{???}

\GESOAadB{207}{10. a)} %% die anderen setzen Lograithmen voraus! und 12.}

\GESOAadB{207}{9. a) und 11. a)}
\GESOAadB{355}{13.}
\newpage

\subsubsection{Momentanverzinsung (optional)}\index{Momentanverzinsung}\index{stetige Verzinsung}\index{Verzinsung!stetige}
(auch stetige Verzinsung)\index{Verzinsung!stetige}
Wird ein Kapital zu 100\% verzinst, so wächst unser Kapital auf 200\%
an. Wenn wir das Kapital jedoch zweimal zu 50\%
verzinsen\footnote{Sprich: Wir lassen uns nach 6 Monaten den Zins
auszahlen, heben das Konto auf und bezahlen sofort wieder mit Zins in
ein neues Konto ein.}, erhalten wir den folgenden, besseren Verzinsungsfaktor:

$1.50^2  = (1 + \frac12)^2 = 2.25$

Bei 4-maliger Verzinsung steigt der Faktor weiter an:
$1.25^4 = (1 + \frac14)^4 \approx 2.44 $

Füllen Sie die folgende Tabelle aus:

\begin{tabular}{c|c|c|c} 
  Anzahl Teile  & Faktor                      & Formel          & Endkapital \\ \hline
  $1$           & $2^1$                      & $(1+1)^1$ & $= K_0 \cdot{} 2 $ \\ \hline
  $2$           & $1.5^2$                      & $(1+\frac12)^2$ & $= K_0 \cdot{} 2.25 $ \\ \hline
  $4$           & $1.25^4$                  & $(1+\frac14)^4$ & $\approx K_0 \cdot{} 2.4414 $ \\ \hline
  $5$           & $\LoesungsRaum{1.2^5}$  & $\LoesungsRaum{(1+\frac14)^2}$ & $\LoesungsRaum{= K_0 \cdot{} 2.48832} $ \\ \hline
  $10$           & $\LoesungsRaum{1.1^{10}}$  & $\LoesungsRaum{(1+\frac{1}{10})^{10}}$ & $\LoesungsRaum{\approx K_0 \cdot{} 2.5937} $ \\ \hline
  $100$           & $\LoesungsRaum{1.01^{100}}$  & $\LoesungsRaum{(1+\frac{1}{100})^{100}}$ & $\LoesungsRaum{\approx K_0 \cdot{} 2.7048 }$ \\ \hline
  $1000$           & $\LoesungsRaum{1.001^{1000}}$  & $\LoesungsRaum{(1+\frac{1}{1000})^{1000}}$ & $\LoesungsRaum{\approx K_0 \cdot{} 2.7169 }$ \\ \hline
  Großes $n$           & ****  & $\LoesungsRaum{(1+\frac{1}{n})^{n}}$ & $\LoesungsRaum{\approx K_0 \cdot{} e }$ \\ \hline
\end{tabular} 

Dieses maximal erreichbare Kapital entspricht etwa dem Faktor 2.71828 und
wird als Eulersche Konstante\index{Eulersche Zahl} bezeichnet.

\bbwCenterGraphic{140mm}{allg/img/euler_banknote.jpg}
Bildlegende: Leonhard Euler (1707-1783) auf der Schweizer Zehnernote.

\begin{definition}{Eulersche Zahl}{}
$$e \approx 2.71828182746$$
\end{definition}
\newpage
