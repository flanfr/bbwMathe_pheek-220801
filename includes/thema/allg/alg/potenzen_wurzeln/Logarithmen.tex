%%
%% 2019 07 11 Ph. G. Freimann
%%

\section{Logarithmen}\index{Logarithmus}
\sectuntertitel{Wer hat noch andere Basen?}
%%%%%%%%%%%%%%%%%%%%%%%%%%%%%%%%%%%%%%%%%%%%%%%%%%%%%%%%%%%%%%%%%%%%%%%%%%%%%%%%%
\subsection*{Lernziele}

\begin{itemize}
\item andere Basis
\item Basiswechsel
\item Logarithmengesetze II
\end{itemize}
\newpage

\subsection{Andere Basis als Basis 10}\index{Logarithmus!beliebige Basis}

Wir können bereits $$x^5=32$$ lösen, indem wir beidseitig die 5. Wurzel ziehen: $$x = \sqrt[5]{x^5} = \sqrt[5]{32} = 2$$

Ebenso können wir Probleme mit der Unbekannten im Exponenten lösen, wenn die Basis 10 beträgt: Das Problem $10^x = 1000$ können wir mit Logarithmen lösen: $$\lg(1000) = \lg(10^3) = 3$$

Aber wie lösen wir so ein Problem:

$$2^x = 128$$

Die Lösung ist ein Logarithmus mit anderer Basis als 10, sodass «Zweierlogarithmus» von $(2^x) = x$.

Und den gibt es:
$$\log_2(2^x) = x$$
Somit können wir das Problem $2^x=128$ lösen (mit Logarithmentabellen oder heute mit dem Taschenrechner):

$$x = \log_2(2^x) = \log_2(128) = \LoesungsRaum{7}$$

Es gilt allgemein

\begin{center}
  \fbox{$\log_a(a^x) = x .$}
\end{center}
\newpage

\subsubsection{Rechengesetze}

\GESO{Weitere Rechengesetze im Buch \cite{marthaler17} ab Seite 97:}
Allgemeine Basis $a$:


\begin{definition}{Logarithmus zu allgemeiner Basis}{}
  
$$ a^b=c \Leftrightarrow: b = \log{}_a(c) $$
\end{definition}

\TNT{2.4}{Beispiel $2^4 = 16$, d.\,h. $\log_2(16)=4 $.\vspace{1cm}}


\begin{bemerkung}{}{}
$$\log_a(a)=1$$
\end{bemerkung}
\TNT{2.4}{Denn: $\log_a(a) = \log_a(a^1)$ Bsp:  $\lg(10) = \lg(10^1)$.\vspace{1cm}}


\begin{bemerkung}{}{}
$$x=\log{}_a(a^x)$$
\end{bemerkung}
\TNT{2.4}{Beispiel $2^4 = 16$, d.\,h. $\log_2(16)=4 $.\vspace{1cm}}

\TALS{
\begin{bemerkung}{}{}
$$y=a^{\log{}_a(y)}$$
\end{bemerkung}
\TNT{2.4}{Beispiel $128=2^{\log_2(128)} = 2^{\log_2(2^7)} = 2^7$.\vspace{1cm}}
}%% END TALS

\begin{bemerkung}{}{}
  
$$\log{}_a(1)=0$$
\end{bemerkung}
\TNT{2.4}{Bsp.: $\log_{10}(1) = \lg(1) = \lg(10^0) = 0$.\vspace{1cm}}
\GESO{\newpage}


\begin{bemerkung}{}{}  
$$\log_a(u\cdot{}v)=\log_a(u) + \log_a(v)$$
\TALS{$$\log_a\left(\frac{u}{v}\right)=\log_a(u) - \log_a(v)$$}
\end{bemerkung}

\TNT{2.4}{Bsp.:  $\log_2(8\cdot 32) = \log_2(2^3 \cdot 2^5) = \log_2(2^{3+5}) = 3+5 = \log_2(2^3) + \log_2(2^5) $.\vspace{1cm}}


Aus obigem Gesetz zur Addition folgt unmittelbar die 
\begin{gesetz}{Potenzregel}{}
$$\log_a(u^k)=k\cdot{}\log_a(u)$$
\end{gesetz}
\TNT{2.4}{
  $\log(5^3) = \log(5\cdot 5\cdot 5) = \log(5)+\log(5)+\log(5) =
  3\cdot \log(5)$
}%% END TNT

\TALS{
\begin{bemerkung}{}{}
$$\log_a\left(\frac{1}{v}\right)=-\log_a(v)$$
\end{bemerkung}

\TNT{2.4}{Bsp.:  $\log_2(8^2) = \log_2((2^3)^2) = \log_2(2^6) = 6 = 2
  \cdot  3 = 2 \cdot \log_2(2^3) = 2 \cdot \log_2(8)$.\vspace{1cm}}%% END TNT
}%% END TALS

\TALS{
\begin{bemerkung}{}{}
$$a^x = b^{x\cdot{}\log_b(a)}$$
\end{bemerkung}
\TNT{2.4}{Wir ersetzen im 2. Gesetz $y$ durch $a^x$ und $a$ durch $b$:
  $$y = b^{\log_b(y)} \longrightarrow  a^x = b^{\log_b(a^x)} =
  b^{x\cdot{}\log_b(a)}$$} %% END TNT
}%% END TALS
\newpage



\subsubsection{Spezielle Basen}
Der Zweierlogarithmus wird vor allem in der Informatik benutzt. Wie viel Bit brauche ich, um 200 Zustände abzubilden?

$2^x = 200$

Lösung:


\TNT{2.4}{
$\log_2(200) \approx 7.64$, d.\,h. ich benötige 8 Bit, um 200 Zustände abzubilden.
}


\subsubsection{Basiswechsel}\index{Basiswechsel!Logarithmen}
\begin{gesetz}{Basiswechsel}{}
  Für $a\in\mathbb{R}^{+}\backslash\{0,1\}$ und $x>0$ gilt:
  $$\log_a(x) = \frac{\lg(x)}{\lg(a)} = \frac{\log_b(x)}{\log_b(a)}$$
\end{gesetz}

Begründung:

\TNT{2.4}{
\GESO{$\log_5(32)=x \Longleftrightarrow 5^x=32$\\
    Daher: $\log(5^x) = \log(32) \Longleftrightarrow
    x\log(5)=\log(32) \Longleftrightarrow x=\frac{\log(32)}{\log(5)} =
    \log_5(32)$} %% END GSEO
\TALS{Beweis: $\log_a(b)=x \Longleftrightarrow a^x=b$\\
    Daher: $\log(a^x) = \log(b) \Longleftrightarrow
    x\log(a)=\log(b) \Longleftrightarrow x=\frac{\log(b)}{\log(a)} =
    \log_a(b)$} %% END TALS
}%% END TNT
\newpage


\subsubsection{Die Eulersche Konstante}\index{$e$, die Eulersche Konstante}

Die Basis $e$ ($e$ = Eulersche\footnote{Leonhard Euler (1707-1783)} Konstante $e\approx 2.7182817246$) wird vor allem bei Wachstums- und Zerfallsprozessen verwendet.

Die Zahl $e$ ist in mehrerer Hinsicht spannend:

\paragraph{Logarithmus Naturalis:}\index{Logarithmus Naturalis} Da die
Zahl $e$ bei exponentiellen Prozessen eine sehr große Rolle spielt,
darf die Eulersche Zahl $e$, $e^x$ und $\log_e(x)$ auch auf keinem Rechner fehlen. Der $\log_e()$ hat dabei sogar einen eigenen Namen erhalten:
$\ln()$ steht für «Logarithmus Naturalis».

\begin{center}
  \fbox{$\ln() := \log_e()$ = \textit{\textbf{Logarithmus Naturalis}}}
\end{center}

\paragraph{Logarithmentabellen}
Wenn ich auf einem Taschenrechner lediglich $\ln()$ und $e^x$ zur Verfügung habe\footnote{Oder ich habe, wie früher, nur eine Logarithmentabelle.} und dennoch $\log_a(c)$ oder $a^x$ berechnen will, so kann ich dies einfach mit einer Transformation zur Basis $e$ vollbringen:

$$a^x = e^{x\cdot{}\ln(a)}$$
$$\log_a(c) = \frac{\ln(c)}{\ln(a)}$$

\newpage



%%\TALS{(\cite{frommenwiler17alg} S.??? (Kap. ???))}
%%\GESO{(\cite{marthaler17}       S.??? (Kap. ???))}

\subsection*{Aufgaben}
\TALSAadB{???}{???}

Andere Basis:

\GESOAadB{103ff}{8. a) b) c) f), 9., 10. a), 13. a) f), 14. a) g),
  15. a) c) d) e) f), 16. a) c) d) e), 17. a) b) c) }

Gesetze:

\GESOAadB{105ff}{21. a) c) e), 22. a) b), 24. a), 25. a) }

Taschenrechner:

\GESOAadB{106}{29.}
