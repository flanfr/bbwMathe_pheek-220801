\subsection{Aus alten Maturaprüfungen}

\subsubsection{Serie 2017 Aufgabe 10}

Gegeben ist folgende Funktionsgleichung der Geraden $g$:

$$g: y = -3.75x + \frac{1}{4}$$

a) Bestimmen Sie die Schnittpunkte von $g$ mit den Koordinatenachsen.
Geben Sie die exakten Werte an.

\TNT{7.2}{Schnittpunkt mit der $y$-Achse heißt $x=0$ also $y=-3.75\cdot{}0 + \frac{1}{4} = \frac{1}{4}$. Somit ist der Schnittpunkt mit der $y$-Achse $(0 | \frac{1}{4})$.

Schnittpunkt mit der $x$-Achse heißt $y=0$ also $0=-3.75\cdot{}x + \frac{1}{4}$. Gleichung Auf"|lösen: Auf beiden Seiten $- \frac{1}{4}$ ergibt

$$-\frac{1}{4} = -3.75 x$$
Nun beide Seiten dividieren durch $-\frac{1}{4}$ ergibt:

$$1 = 15x$$

Und somit ist $x=\frac{1}{15}$, und der gesuchte Punkt auf der $x$-Achse lautet $(\frac{1}{15} | 0)$.


}%% END 'noTRAINER'
\newpage

b) Nennen Sie zwei verschiedene Punkte $A(x_A | y_A)$ und $B(x_B | y_B)$, die
auf $g$ liegen und folgende Vorgabe erfüllen:

$x_A$, $y_A$, $x_B$ und $y_B$ müssen ganzzahlig sein. \TRAINER{\zB Wertetabelle im TR mit x= -10 step 1}

\TNT{16}{
  \bbwGraph{-4}{4}{-13}{6}{
    \bbwFunc{-3.75*\x+0.25}{-2:3.5}
 \bbwDot{3,-11}{blue}{west}{B}
 \bbwDot{-1,4}{blue}{east}{A}
}%% END BBW Graph
}%% END TRAINER
\newpage



\subsubsection{2016 Aufgabe 7}

Carunternehmen A bietet folgenden Tarif an:
Pro Tag CHF 300.-- Grundtaxe und für jeden gefahrenen Kilometer CHF 1.50.

Bei Unternehmen B bezahlte eine Reisegruppe für Grund- und Kilometertaxe für einen
Tagesausflug von 600 km den Betrag von CHF 1320.--.

Eine andere Gruppe, die ebenfalls mit Unternehmen B fuhr, bezahlte mit demselben
Tarif für einen Tagesausflug von 750 km den Betrag von CHF 1590.-- .

a) Bestimmen Sie die Kostenfunktionen $y = f(x)$ für die Unternehmen A und B;
$x$ = Anzahl km gefahrene Strecke, $y$ = Anzahl CHF Gesamtkosten.

\TNT{5.6}{A: $y=1.5x + 300$

B: Steigungsdreieck: $a=\frac{1590 - 1320}{750-600} = \frac{270}{150} = 1.8$ daraus folgt $y = 1.8x + b$. Nun können wir die erste Reisegruppe einsetzen: $1320 = 1.8\cdot{}600 + b$ und somit ist $b= 240$. Die Funktionsgleichung lautet also

B: $y = 1.8x + 240$
\vspace{30mm}
}%% END TRAINER

b)  Bei welcher Fahrstrecke sind beide Firmen\footnote{Müsste in der Aufgabenstellung wohl «Unternehmen» lauten; wird aber vorausgesetzt, dass verstanden wird, was gemeint ist.} gleich teuer?

\TNT{5.6}{
$y$ = Kosten gleichsetzen: $1.5x + 300 = 1.8x + 240$. Gleichung auf"|lösen: $60 = 0.3x$ und somit $x=200$. Ergo: Bei 200 km sind beide Unternehmen (=Firmen) gleich teuer.
}%% end TRAINER
\newpage



c) 
Zwei weitere Firmen, C und D, haben folgende Kostenfunktion:

Firma C: $y = 1.8x + 250$ , Firma D: $y = 2.5x + 150$ ;

$x$ = Anzahl km gefahrene Strecke, $y$ = Anzahl CHF Gesamtkosten.

Eine Reisegruppe wählt für einen Tagesausflug von 300 km Länge Firma D, da diese
den ansprechenderen Internetauftritt hat. Um welchen Betrag wird diese Reise gegen-
über dem Angebot von Firma C teurer?

\TNT{4}{300 km in beide Funktionsgleichungen als $x$-Wert eintragen: C kostet $1.8\cdot{}300 + 250 = 790$; D kostet $2.5\cdot{}300 + 150 = 900$. Somit kostet D CHF 110.-- mehr als C.

}%% end TRAINER
\newpage


\subsubsection{2018 Serie 1 Aufgabe 9}

Gegeben ist die Gerade $g$ mit der Funktionsgleichung $y = \frac{2}{3}x-4$.

a) Berechnen Sie den Schnittpunkt von $g$ mit der $x$-Achse.

\TNT{4}{Schnittpunkt mit der $x$-Achse heißt $y=0$. Somit $0 = \frac{2}{3}x - 4$ nach $x$ auf"|lösen:
$4=\frac{2}{3}x$ $\longrightarrow$ $6 = x$. Der Schnittpunkt ist folglich $(6|0)$.

  \vspace{30mm}
}%% end TRAINER

b) Welche der folgenden Punkte $A$, $B$, $C$ liegen \textit{oberhalb}, welche
\textit{unterhalb} und welche \textit{auf} der Geraden $g$?

$$A\left(5\middle|-\frac{3}{4}\right), B(2.1|-2.6), C(50|30)$$

\textbf{Antworten:}

\begin{tabular}{lcl}
$A$ liegt & \TRAINER{unterhalb}\noTRAINER{....................................................} & der Geraden $g$.\\
$B$ liegt & \TRAINER{auf}\noTRAINER{....................................................} & der Geraden $g$.\\
$C$ liegt & \TRAINER{oberhalb}\noTRAINER{....................................................} & der Geraden $g$.\\

 \end{tabular}{

\TNT{5.2}{Jeweils $x$ in die Gleichungen einsetzen und mit $y$ vergleichen.}%% End TNT
