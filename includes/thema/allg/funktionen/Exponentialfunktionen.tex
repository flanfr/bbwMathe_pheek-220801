%%
%% 2019 07 04 Ph. G. Freimann
%%

\section{Exponentialfunktionen}\index{Funktion!Exponentialfunktion}\index{Exponentialfunktionen}
\sectuntertitel{Go viral!}
%%%%%%%%%%%%%%%%%%%%%%%%%%%%%%%%%%%%%%%%%%%%%%%%%%%%%%%%%%%%%%%%%%%%%%%%%%%%%%%%%
\subsection*{Lernziele}

\begin{itemize}
\item Definition Exponentialfunktion
\item Koeffizienten interpretieren
\item Graph: Symmetrien, Polstellen, Asymptoten, Schnittpunkte mit Achsen
\item Basiswechsel
\end{itemize}

\TALS{(\cite{frommenwiler17alg} S.215 (Kap. 3.10))}
\GESO{(\cite{marthaler17}       S.322 (Kap. 19))}



\TRAINER{EinstiegsComics Donald Duck im OLAT}
%  Besser bei der Zins und Zinseszinsformel
%\subsection{Einstiegsbeispiel}
%
%(Einstiegscomics «Donald Duck» im OLAT)
%\newpage



\begin{definition}{Exponentialfunktion}{}\index{Exponentialfunktion!Definition}
  Eine Funktion der Form $$f(x): x \mapsto a^x$$
  bzw. $$y = a^x$$
  heißt \textbf{Exponentialfunktion}.
\end{definition}

Der Parameter $a$ gibt den Wachstumsfaktor pro Zeiteinheit $e_x$ an.
 Oft wird die Zeitachse auch mit $t$ statt $x$ bezeichnet $t$ steht für \textit{time}.
\newpage

\subsubsection{Allgemeine Form der Exponentialfunktion}

Obige Exponentialfunktion hat für den Wert $x=0$ immer den $y$-Wert = 1. Da dies in der Praxis aber meist nicht der Fall ist, gib es die folgende \textbf{allgemeine Form der Exponentialfunktion}:

\begin{center}\fbox{$f: y = c\cdot{}a^x$}\end{center}

Dabei ist $c$ der Startwert zum Zeitpunkt $x$ = 0.

\bbwCenterGraphic{8cm}{allg/funktionen/img/exp/c_faktor.png}
\newpage



\subsubsection{Verschiebung und Streckung}

Eine Verschiebung der Exponentialfunktion in der Zeit ($x$-Richtung) kann auch in Form einer Veränderung der Startfaktors $c$ umgeschrieben werden.

Verschieben wir \zB $$y=2^x$$ um fünf Einheiten nach rechts, so liest sich die neue Funktionsgleichung wie folgt:
$$y=2^{x-5}.$$

(Zeichnen Sie in \texttt{geogebra.org} a) $y=2^x$ und b) $y=2^{x-5}$.)

Dies kann jedoch auch umgeschrieben werden:

$$2^{x-5} = 2^x \cdot{} 2^{-5} = 2^{-5} \cdot{} 2^x = \frac{1}{2^5} \cdot{} 2^x =
\frac{1}{32}\cdot{}2^x$$

\bbwCenterGraphic{8cm}{allg/funktionen/img/exp/verschiebung_gleich_streckung.png}
Bildlegende: Eine Verschiebung ($x$-Richtung) der Exponentialfunktion entspricht einer Stauchung ($y$-Richtung) der selben Exponentialfunktion.

\TALS{Es gilt hier $$a^{x-b}=\frac{a^x}{k}$$ mit
$k=a^b$ und mit $b=\log_a(k)$.}


\subsection*{Aufgaben}
\TALSAadB{215ff}{809-830}
\GESOAadB{335ff}{11. a) b), 15. a) b) [Zeichnen mit \texttt{geogebra.org}], 17. a) b) e), 22. a) b), 23. d), 25. a) b)}
\newpage
