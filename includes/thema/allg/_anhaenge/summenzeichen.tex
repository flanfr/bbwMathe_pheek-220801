%
% Das Summenzeichken SIGMA
%

\subsection{Das Summenzeichen}\label{Summenzeichen}\index{Sigma}\index{Sigma@$\Sigma$}\index{$\Sigma$ s. Sigma}\index{Summe}
«Der Mathematiker zählt lieber, als dass er glaubt.»
\vspace{4mm}

%% Summenzeichen für Mittelwert bei Datenreihen.

\subsubsection{Notation}
Um Summen mit vielen Summanden abzukürzen wird das mathematische Summenzeichen (ein
griechisches Sigma) $\sum{}$ benutzt:

\begin{definition}{}{}
  $$1 + 2 + 3 + 4 + 5 =: \sum_{z=1}^{5}{z}$$
\end{definition}

  Sprich «Summe über alle $z$; von
$z$ gleich eins bis fünf».

\subsubsection{Laufvariable}
Es spielt keine Rolle, mit welchem Variablennamen der Laufindex abgekürzt wird. Üblich sind Buchstaben wie $i$, $j$, $k$, $m$, $n$:

$$\sum_{n=1}^{5}{n} = \sum_{i=1}^5{i} = \sum_{k=1}^5{k} = 15$$

Die Laufvariable erhöht ihren Wert für jeden Summanden um 1 (eins).

\begin{beispiel}{Summenzeichen}{beispiele_summenzeichen}
  $${\color{green}\sum_{{\color{orange}n}{{\color{green}=\color{blue}3}}}^{\color{blue}5\color{green}}}  {\color{orange}n}^7 = {\color{blue}3}^7 {\color{green}+} {\color{blue}4}^7 {\color{green}+} {\color{blue}5}^7$$
\end{beispiel}

\begin{beispiel}{}{}
$${\color{green}\sum_{{\color{orange}s}={\color{blue}5}}^{{\color{blue}8}}{\color{black}(3{\color{orange}s}-4)}} = (3 \cdot{\color{blue}5} - 4) {\color{green}+} (3 \cdot{\color{blue}6} - 4) {\color{green}+} (3 \cdot {\color{blue}7} - 4) {\color{green}+} (3 \cdot {\color{blue}8} - 4)$$
\end{beispiel}

\textbf{Aber:}

\begin{beispiel}{}{}
  $$\sum_{s=5}^{8}{3s-4} = (3\cdot 5) + (3\cdot 6) + (3\cdot 7) + (3\cdot 8) - 4$$  
\end{beispiel}


\subsubsection{Übungsaufgaben}
Berechnen Sie
$$\sum_{x=3}^5{x^2} = \LoesungsRaum{50}$$
$$\sum_{x=2}^4{(x+1)^2} = \LoesungsRaum{50}$$
$$\sum_{n=-40}^{60}{(n-10)} = \LoesungsRaum{0}$$
$$\sum_{1=1}^{10}{\frac{1}{i}} = \LoesungsRaum{\frac{7381}{2520}
  \approx 2.9290}$$

Für Spezialisten:
$$\sum_{3}^{10}\left(i^6 - ( i-1 )^6\right) = \LoesungsRaum{1\,000\,000 - 2^6 = 9999936}$$

Tipp: Schreiben Sie die ersten drei und die letzten beiden Summanden
explizit hin.

$$\sum_{i=0}^{n-1}x^i(x-1)=\LoesungsRaum{x^n-1}$$
\newpage


%%\subsubsection{Verdeutlichung}
%%Die folgende Schreibweise rechts verdeutlicht, von wo bis wo der Wert
%%der Variable $z$ wandert, wenn auch die rechte Schreibweise eher unüblich ist:
%%$$\sum_{z=1}^{5}{z} = \sum_{z=1}^{z=5}{z}$$

\subsubsection{Indizes, Mittelwert}
Im Zusammenhang Indizes ($i$) (\zB bei Datenreihen) wird meist diese
Notation benutzt.

Seien also $x_1=5$, $x_2=8$, $x_3=6$ und $x_4=5$ vier
Messwerte. Betrachten wir die Summe über
alle $x_i$ von $i$ gleich eins bis vier:

$$5+8+6+5 = x_1 + x_2 + x_3 + x_4 =: \sum_{i=1}^4{x_i}=24$$


Dies wird \zB beim Mittelwert (arithmetisches Mittel)
verwendet. Seien wieder $x_1=5$, $x_2=8$, $x_3=6$ und $x_4=5$. So ist
$\mittelwert{x}$ wie folgt berechnet:

$$\mittelwert{x} = \frac{1}{4}\cdot{}\sum_{i=1}^{4}{x_i}$$

Dabei ist $$\frac{1}{4}\cdot{}\sum_{i=1}^{4}{x_i}=\frac{1}{4}\cdot{}\left( \sum_{i=1}^{4}{x_i}\right) =\frac{1}{4}\cdot{}(x_1 + x_2 + x_3 + x_4)$$

oder ganz allgemein für $n$ Datensätze. Der Mittelwert der $n$
Datensätze ist der $n$-te Teil der Summe über alle $x_i$ für $i$
gleich eins bis $n$:

$$\mittelwert{x} = \frac{1}{n}\cdot{}\sum_{i=1}^{n}{x_i}$$

Hierbei ist:
$$\frac{1}{n}\cdot{}\sum_{i=1}^{n}{x_i} = \frac{1}{n}\cdot{}(x_1 + x_2 + x_3 + \dots + x_n)$$

\newpage
