%%
%% Wahrscheinlichkeit
%% 2021 - 03 - 08 φ
%%

\section{Wahrscheinlichkeit}\index{Wahrscheinlichkeit}
\begin{definition}{Wahrscheinlichkeit}{}
Mit $$P(E)$$ bezeichnen wir die Wahrscheinichkeit, dass ein Ereignis $E$ eintritt.

$P(E)$ ist eine Zahl im Intervall $[0,1]$.
\end{definition}
Es gelten nach A. Kolmogorofs-Axiomen\footnote{Andrey Nikolaevich Kolmogorov 1903-1987} die folgenden Gesetze:
\begin{gesetz}{}{}
\begin{itemize}
\item $P(\Omega) = 1$: Die Wahrscheinlichkeit, dass irgend etwas eintritt, ist immer 100\%, also 1.
\item $P(\{\}) = 0$: Die Wahrscheinlichkeit, dass kein Ergebnis eintritt, ist 0.
\item Falls $A \cap B = \{\}$ gilt: $P(A\cup B) = P(A) + P(B)$. Mit anderen Worten: Nur bei Ereignissen, die keine gemeinsamen Ergebnisse aufweisen, gilt diese Summenregel.
\item $\overline{E}$ bezeichnet die Komplementärmenge von $E$, also alle Elemente aus $\Omega$, die nicht in $E$ liegen. Es gilt $P(\overline{E}) = 1-P(E)$. Mit anderen Worten: Die Wahrscheinlichkeit, dass ein Eregins \textbf{nicht} eintritt ist gleich 100\% \textbf{minus} die Warhscheinlichkeit, dass das Ereignis eintritt.
\end{itemize}
\end{gesetz}

\newpage

\begin{beispiel}{}{}\textbf{Ziehen einer Kugel aus einer Urne}

Aus einer Urne mit 4 roten (r), 3 schwarzen (s) und 2 weissen (w) Kugeln wird eine Kugel gezogen.

Ergebnisraum: $\Omega = \{r, s, w\}$



Wahrscheinlichkeiten für die Elementarereignisse:

$P(\{r\}) = \frac49$

$P(\{s\}) = \frac39 = \frac13$

$P(\{w\}) = \frac29$

Wie gross ist nun die Wahrscheinlichkeit eine rote oder eine weisse Kugel zu ziehen?
Wir definieren $E_1$ als das günstige Ereignis, dass eine rote oder eine weiße Kugel gezogen wird.

$E_1 = \{r\} \cup \{w\} = \{r,w\}$

$P(E_1) = \frac49 + \frac29 = \frac23$



Wie gross wäre dabei die Wahrscheinlichkeit weder eine rote noch eine weisse Kugel zu ziehen?

$E_2 := \{s\}$

$P(E_2) = \frac13$, denn $E_2 = \overline{E_1}$ und somit

$P(E_2) = P(\overline{E_1}) = 1 - P(E_1) = 1 - \frac23 = \frac13$.

\end{beispiel}

\subsection*{Aufgaben}
\aufgabenfarbe{Kompendium: Kap. 5.3 Aufg. 5.-8.}

\newpage


