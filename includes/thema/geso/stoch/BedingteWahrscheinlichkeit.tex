%%
%% Bedingte Wahrscheinlichkeit
%% 2021 - 03 - 15 φ
%%

\section{Bedingte Wahrscheinlichkeit}\index{Wahrscheinlichkeit! Bedingte}

Betrachten wir kurz nochmals die obige (ausgefüllte) Kontingenztafel (Vier-Feld-Tafel).

Wie groß ist die Wahrscheinlichkeit eine

a) Braunhaarige männliche Person zu finden? \LoesungsRaum{1/31}

b) Unter den Männern eine braunhaarige Person zu finden? \LoesungsRaum{1/13}

c) Unter den Braunhaarigen eine männliche Person zu finden? \LoesungsRaum{1/3}

Sie sehen: Je nach Fragestellung verändert sich die Wahrscheinlichkeit.

Wir nennen dies eine

\begin{definition}{Bedingte Wahrscheinlichkeit}{}
Die \textbf{Bedingte Wahrscheinlichkeit} oder \textit{konditionale
  Wahrscheinlichkeit} ist die Wahrscheinlichkeit eines Ereignisses
\textbf{unter der Bedingung}, dass ein anderes Ereignis bereits
eingetreten ist.
  \end{definition}

\newpage
