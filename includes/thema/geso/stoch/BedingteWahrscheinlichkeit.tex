%%
%% Bedingte Wahrscheinlichkeit
%% 2021 - 03 - 15 φ
%%

\section{Bedingte Wahrscheinlichkeit}\index{Wahrscheinlichkeit!
  Bedingte}\index{bedingte Wahrscheinlichkeit}

Betrachten wir kurz nochmals die obige (ausgefüllte) Kontingenztafel (Vier-Feld-Tafel).

\TRAINER{Auf Beispiel von Kontingenztafeln anpassen, sobald dort abgeändertes Beispiel...}

Wie groß ist die Wahrscheinlichkeit eine

a) Braunhaarige männliche Person zu finden? \LoesungsRaum{1/32}

b) Unter den Männern eine braunhaarige Person zu finden? \LoesungsRaum{1/14}

c) Unter den Braunhaarigen eine männliche Person zu finden? \LoesungsRaum{1/3}

Sie sehen: Je nach Fragestellung verändert sich die Wahrscheinlichkeit.

Wir nennen dies eine

\begin{definition}{Bedingte Wahrscheinlichkeit}{}
Die \textbf{Bedingte Wahrscheinlichkeit} oder \textit{konditionale
  Wahrscheinlichkeit} ist die Wahrscheinlichkeit eines Ereignisses
\textbf{unter der Bedingung}, dass ein anderes Ereignis bereits
eingetreten ist.
  \end{definition}

\begin{definition}{}{}
  Mit
  $$P(A | B)$$
  bezeichnen wir die Wahrscheinlichkeit, dass das Ereigins $A$
  eintritt \textbf{unter der Bedingung}, dass das Ereignis $B$ bereits
  eingetroffen ist.
\end{definition}
\newpage


\subsection{Beispiel und Referenzaufgabe}
Betrachten wir die folgende Kontingenztafel:

  \begin{tabular}{|l|r|r|r|r|}\hline
                 & Test positiv (P) & Test negativ (N)& Total & relativ (\%) \\\hline
    krank (K)    & 95               & 2               & 97    & 78.9 \%      \\\hline    
    gesund (G)   & 4                & 22              & 26    & 21.1 \%      \\\hline
    Total        & 99               & 24              & 123   &  --- \%      \\\hline
    relativ (\%) & 80.5\%           &19.5\%           & ---   &   100\%      \\\hline

  \end{tabular}
  
Dabei wurden 24 Personen von 123 negativ getestet: $P(N) =
\frac{24}{123}\approx 19.5\%$.
Wenn ich die negativen Tests nun aber nur bei den kranken Personen
(total 97) betrachte, so erhalte ich
$$P(N|K) = \frac{P(N\cap K)}{P(K)} = \frac{2}{97} \approx 2.06\%.$$

\begin{definition}{}{}
$$A\cap B$$ ist dasjenige Ereignis, bei dem $A$ \textbf{und} $B$
gleichzeitig zutrifft.
\end{definition}

In obigen Beispiel: $N\cap K$ sind alle Personen, die sowohl negativ
getestet (N), sowie auch krank (K) sind (also 2 Personen).

\begin{gesetz}{bedingte Wahrscheinlichkeit}{}
  Es gilt
  $$P(A|B) = \frac{P(A\cap B)}{P(B)}$$
\end{gesetz}

Berechnen Sie für die obige Tafel die folgenden Wahrscheinlichkeiten:

\begin{tabular}{rr}
$P(K) = \LoesungsRaum{\frac{97}{123} \approx{} 78.9\%}$&
$P(G) = \LoesungsRaum{\frac{26}{123} \approx{} 21.1\%}$\\
$P(P) = \LoesungsRaum{\frac{99}{123} \approx{} 80.5\%}$&
$P(N) = \LoesungsRaum{\frac{24}{123} \approx{} 19.5\%}$\\


$P(P\cap K) = \LoesungsRaum{\frac{95}{123} \approx{} 77.2\%}$&
$P(P\cap G) = \LoesungsRaum{\frac{4}{123}  \approx{} 3.25\%}$\\
$P(N\cap K) = \LoesungsRaum{\frac{2}{123}  \approx{} 1.63\%}$&
$P(N\cap G) = \LoesungsRaum{\frac{22}{123} \approx{} 17.9\%}$\\

$P(P|K) = \LoesungsRaum{\frac{95}{97} \approx{} 97.9\%}$&
$P(K|P) = \LoesungsRaum{\frac{95}{99} \approx{} 96.0\%}$\\

$P(P|G) = \LoesungsRaum{\frac{4}{26} \approx{} 15.4\%}$&
$P(G|P) = \LoesungsRaum{\frac{4}{99} \approx{} 4.04\%}$\\

$P(N|K) = \LoesungsRaum{\frac{2}{97} \approx{} 0.0208\%}$&
$P(K|N) = \LoesungsRaum{\frac{2}{24} \approx{} 8.33\%}$\\

$P(N|G) = \LoesungsRaum{\frac{22}{26} \approx{}  84.6\%}$&
$P(G|N) = \LoesungsRaum{\frac{22}{24} \approx{} 91.7\%}$\\
\end{tabular}
\newpage


\subsection*{Aufgaben}
\GESOAadB{Kompendium}{Kap. 5.6 Aufg 30., 32., 34., 35. und Maturaprüfungen: 2018 (GESO) Serie 4. Aufg. 14. (Krankheit, Früherkennung)}
\newpage
