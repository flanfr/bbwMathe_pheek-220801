%%
%% 2020 - 08 -03 φ
%%
\newpage
\section{Kombinatorik}
Die Kombinatorik befasst sich damit, wie viele Möglichkeiten für
verschiedene Konstellationen zur Wahl stehen.

Als Einstiegsbeispiel dient die folgende Wanderung:

\vspace{5mm}

\bbwGraphic{12cm}{geso/stoch/img/wanderung.png}

Auf wie viele Arten kann der Wanderer von Wesen (im Westen) nach Obertal (im Osten) gelangen, wenn er ausschließlich von West nach Ost wandern will?

Die Antwort kann durch Abzählen (oder eine Kombination von Abzählen und Multiplizieren) gefunden werden:

\TNT{2.4}{$3\cdot{}(1+(2\cdot3)) + 3 = 24$ Möglichkeiten. Mögliche
  Erklärung: Bis «L» sind 3 Wege offen. Bis «R» sind total 3*2 + 1,
  also sieben Wege möglich. Um nach «O» zu gelangen sind nun 7*3 + die
  drei Wege ab Lisibach und dann direkt, also total 24 Wege möglich. Parallele Wege werden addiert;
  serielle Wegmöglichkeiten multiplizieren sich.}

\newpage
\subsection{Variation mit Wiederholung (Produktregel)}
Ein Zahlenschloss hat vier Ringe und auf jedem Ring sind die Zahlen
von 0 - 5 einstellbar. Also pro Ring sechs Möglichkeiten.
Wie viele Variationen gibt es im ganzen für dieses Zahlenschloss?

\bbwCenterGraphic{5cm}{geso/stoch/img/zahlenschloss.jpg}

\TNT{4.8}{
Jeder Ring hat 6 Möglichkeiten. Somit habe ich am 1. Ring sechs
Varianten. Für jede dieser Varianten habe ich für den 2. Ring auch
sechs Varianten. Insgesamt also

$n$ = Anzahl Elemente der Auswahl (hier 6 Ziffern)

$k$ = Anzahl geordnete auszuwählende Elemente (hier 4 Ringe)

$N$ = Anzahl Variationen

$$N = n^k = 6^4$$
}

Dieses Experiment entspricht im Urnenmodell dem...
\begin{gesetz}{...Ziehen mit zurücklegen}{}
Beim Urnenmodell \textbf{«Ziehen mit Zurücklegen»} kann jeder Zug unabhängig vom
vorangehenden wieder alle Werte annehmen. Die \textbf{Reihenfolge} der gewählten Kugeln ist hier \textbf{wesentlich}. Für $k$ Züge aus genau $n$
Kugeln, die alle verschieden sind, gibt es $N$ Möglichkeiten:
$$N = n^k$$
\end{gesetz}

\begin{definition}{Variation}{}\index{Variation}
Eine \textbf{Variation} ist eine geordnete Stichprobe.
\end{definition}

\subsection*{Aufgaben}
\aufgabenfarbe{Kompendium S. 46 Kap. 5.5.1 Aufg. 16. a)}
\newpage


\subsection{Permutationen (Fakultät)}\index{Permutation}\index{Fakultäten}
Lateinisch «permutare» = vertauschen

Hanna und Igor fahren Bus. Die beiden für sie reservierten Plätze sind
nebeneinander, doch nur einer davon ist ein Fensterplatz. Auf wie
viele Arten können sich die beiden Personen auf die beiden Plätze
verteilen?

\TNT{2.4}{Genau auf zwei Arten: Entweder ist
Hanna am Fenster, oder Igor.\vspace{12mm}}

Das Problem ist etwas komplizierter, wenn nun Hanna, Igor mit Jana
eine Flugreise machen. Die drei reservierten Plätze sind wieder
nebeneinander. Ein Platz ist am Fenster, einer zum Gang und der dritte
Platz ist zwischen den beiden anderen. Auf wie viele Arten können nun
Hanna, Igor und Jana ihre Plätze wählen?

\TNT{3.6}{6 Varianten. Jede der drei Personen
am Fenster ergibt drei Hauptvarianten, dann jede der verbleibenden
beiden zum Gang hin; Ergo: $3\cdot{}2$}

Machen Sie die selbe Überlegung noch mit vier Personen\footnote{Ach ja: Die vierte Person ist Karl.} und vier
Plätzen...

\TNT{2.8}{24 = 4!\vspace{28mm}}

... und mit fünf Plätzen und fünf Personen\footnote{Die fünfte Person heißt übrigens Lena, auch wenn es für die Berechnung keine Rolle spielt.}.

\TNT{3.2}{120 = 5!\vspace{30mm}}
\newpage

Ein alter Bekannter:

\begin{bemerkung}{Summenzeichen}{}
Erinnern Sie sich an das Summenzeichen? Berechnen
Sie $$\sum_{n=0}^{15}\frac{1}{n!} = \frac{1}{0!} + \frac{1}{1!}
+ \frac{1}{2!} + \frac{1}{3!} + \frac{1}{4!} + ...$$ Das Summenzeichen finden Sie unter
der Taste \tiprobutton{math}\tiprobutton{5}.\footnote{Gönnen Sie sich
ein Glas süßen Sirup, während Ihr Taschenrechner diese Summe für Sie berechnet.}
\end{bemerkung}

Erinnern Sie sich auch an dieses Resultat? \TRAINER{$= e \approx 2.718281828459045$}


\newpage

\subsubsection{Fakultät als Operation}\index{Fakultät}
Ganz allgemein gilt: Bei $n$ Personen auf $n$ Plätzen gibt es
$n\cdot{} (n-1) \cdot{} (n-2) \cdot{} (n-3) \cdot{} (n-4) \cdot{}
... \cdot{} 2 \cdot{} 1$ Möglichkeiten.

Diese Rechnung ist im Taschenrechner unter der Operation ``Fakultät''
bekannt und wird üblicherweise mit der Taste \fbox{n!} bezeichnet. Auf
Ihrem Rechner ist es die Taste \tiprobutton{ncrnpr}.

Berechnen Sie die Fakultät von 3, 4, 5 und 20 mit dem Taschenrechner.

\TNT{3.2}{\vspace{32mm}}

Auf wie viele Arten können Sie sich als Klasse in Ihre Bänke
verteilen? Machen Sie die Überlegung so, wie es aussehen würde, wenn es keine leeren Plätze gäbe.

\TNT{5.2}{Wenn es keine leeren Plätze hat, entspricht die Anzahl der
Möglichkeiten der Fakultät. $n$ = Anzahl Schüler = Anzahl Plätze, dann
gilt $n!$ = Anzahl Mögliche Sitzordnungen.
\vspace{32mm}
}

\begin{gesetz}{}{}
Im Urnenmodell entspricht die Fakultät einer Urne mit $n$ Kugeln, die
alle verschieden sind. Alle $n$ Kugeln werden gezogen. Wie viele
Reihenfolgen sind möglich?
$$N = n\cdot{} (n-1) \cdot{} (n-2) \cdot{} (n-3) \cdot{}
(n-4) \cdot{} ... \cdot{} 3\cdot{} 2 \cdot{} 1 = n!$$
\end{gesetz}

\subsection*{Aufgaben}
\aufgabenfarbe{Kompendium Kap. 5.5.1 Aufg. 15}
\newpage


\subsection{Variation ohne Zurücklegen}\index{Variation!ohne Zurücklegen}
Stellen wir uns vor, wir könnten sechs Kunstbände (Bücher) in ein schmales
Regal stellen. Mehr geht nicht, weniger wollen wir nicht.
Nun haben wir 10 Bücher zur Auswahl. Auf wie viele Arten können wir nun sechs Bücher in unser Regal einordnen?

\TNT{6}{
Für das erste Buch haben wir zehn Möglichkeiten. Wenn das Buch da mal steht, haben wir für das zweit Buch nur noch neun Möglichkeiten.
Schlussendlich bleiben $N$-Möglichkeiten:
$$N=10\cdot{}9\cdot 8\cdot 7\cdot 6\cdot 5 = \frac{10\cdot 9\cdot 8\cdot 7\cdot 6\cdot 5\cdot 4\cdot 3\cdot 2\cdot 1}{4\cdot 3\cdot 2\cdot 1} = \frac{10!}{4!}$$
\vspace{20mm}
}%% END TNT

Oder als Formel:
\begin{gesetz}{Ohne Zurücklegen / Reihenfolge Wesentlich}{}
Es seien
$n = $ Anzahl Elemente der Auswahl (oben die Anzahl der Bücher)\\
$k = $ Anzahl auszuwählende geordnete Elemente (die Anzahl Plätze im Regal)\\
$N = $ Anzahl der Möglichkeiten\\

$$N =\frac{n!}{(n-k)!}$$
\end{gesetz}

Veranschaulichung (2 aus 5):

\begin{tabular}{c|c|c|c|c}
12 & 21 & 31 & 41 & 51\\
13 & 23 & 32 & 42 & 52\\
14 & 24 & 34 & 43 & 53\\
15 & 25 & 35 & 45 & 54\\
\end{tabular}

Das sind fünf Blöcke mit je vier Elementen = $5\cdot{}4 = \frac{5!}{(5-2)!}$.
\newpage

\begin{bemerkung}{}{}
Wenn wir alle $n$ Elemente geordnet auswählen, so erhalten wir
$\frac{n!}{(n-n)!} = n!$, was dem bereits bekannten Spezialfall der \textbf{Permutation}
entspricht.
\end{bemerkung}

\begin{bemerkung}{}{}
Wenn wir 3 Plätze für 5 Personen haben, so sind die 5 Personen die $n$
Elemente zu Auswahl und die 3 (geordneten) Plätze ist unser $k$.

Wenn wir jedoch 5 Plätze haben und 3 Personen wählen darauf ihre
Position, so sind die 5 Plätze unsere $n$ Elemente aus denen jede
Person ($k=3$) eine Platznummer wählen kann.
\end{bemerkung}

\GESO{%%
\begin{bemerkung}{}{}
Die Variation ohne zurücklegen kann mit dem Taschenrechner mit «nPr» erreicht werden.
Eine Rangliste von 3 Skifahrern (1., 2. und 3. Platz) aus 20
Mitfahrenden ist also $$\frac{20!}{(20-3)!}\TRAINER{=6840}$$ und kann auf dem
Taschenrechner mit der Taste «nPr» \tiprobutton{ncrnpr} erreicht
werden\footnote{Die Taste muss dabei drei Mal gedrückt werden!}.
\end{bemerkung}
}%%% end GESO
\newpage

\begin{gesetz}{Variation ohne Wiederholung}{}
Bei $n$ Elementen zur Auswahl und $k$ ausgewählten geordneten
Elementen, ist immer
$$n > k.$$
\end{gesetz}


Wie viele Möglichkeiten hat eine Klasse mit $n$ Lernenden, sich auf
$n+1$ Plätze zu verteilen?

\TNT{2.4}{Hier stellt man sich einfach einen ``Phantom-Schüler'' vor,
der Unsichtbar auf den leeren Bank sitzt. Somit gilt auch hier: Die
Anzahl der Möglichen Sitzordnungen = Fakultät der leeren Plätze = $(n+1)!$.}

Wie viele Sitzordnungen sind möglich, wenn es zwei oder mehr freie Plätze gibt?

\TNT{5.2}{Hier gibt es grundsätzlich zwei Betrachtungsweisen.

a) Wir stellen uns vor, dass wir $p$ Plätze haben. Davon bleiben $f$
frei. Nun zählen wir alle Schüler und die ``Phantomschüler'' zusammen, das ergibt
gerade die $p$ Plätze. Hier gibt es also $p!$ Varianten. Die $f$
Phantomschüler setzen sich aber auch für alle Variationen mit realen
Schülern auf $f!$ verschiedene Varianten hin. Somit müssen wir diese
Varianten wieder wegdividieren. Es bleiben total $\frac{p!}{f!}$ Varianten.

b) Wir stellen uns die Sitzbänke nummeriert vor und legen für jeden der
$n$ Nummern eine Kugel in eine Urne. Die $k$ Schüler stellen wir der Reihe
nach hin, und jede/r darf der Reihe nach eine Kugel (somit seine
Platznummer) auswählen. Somit verbleiben $n-k$ Kugeln in der Urne und
hier gilt das \textbf{Ziehen ohne Zurücklegen mit wesentlicher
Reihenfolge}:

Varianten = $\frac{n!}{(n-k)!}$.

}


\subsection*{Aufgaben}
\aufgabenfarbe{Kompendium: S. 46 Aufg. 16. b)}

\newpage



%%%%%%%%%%%%%%%%%%%%%%%%%%%%%%%%%%%%%%%%%%%%%%%%%%%%%%%55

\subsection{Kombinationen}\index{Kombination}
Im Folgenden betrachten wir die Familie G. aus W.:
\begin{itemize}
\item Mutter, klein, dunkelhaarig
\item Vater, groß, blond
\item 1. Kind: Tochter, groß, blond, Teenager
\item 2. Kind: Sohn, klein, blond, Teenager
\item 3. Kind: Tochter, klein, dunkelhaarig (noch kein Teenager)
\end{itemize}

Die Familie hat gemerkt, dass es meist nicht nötig ist, alle Namen
aufzuzählen, wenn eine Teilmenge der Familie angesprochen werden soll:

\begin{itemize}
\item «Heute kochen die blonden»
\item «Die großen dürfen heute ausnahmsweise länger aufbleiben»
\item ...
\end{itemize}

\newpage
Beantworten Sie die folgenden Fragestellungen:
\begin{itemize}
\item Wie viele solcher ``Teilmengen'', solcher \textbf{Kombinationen}, bestehend aus genau \textbf{drei} Familienmitgliedern sind möglich?

\TNT{1.6}{10\vspace{10mm}}

\item Suchen Sie zu zweit für jede «Dreiergruppe» der Familie G. eine
treffende Bezeichnung.

\TNT{4.4}{
  die drei ältesten,
  Mutter und Männer (M\&Ms),
  keine Teenager,
  Mutter und Teenager,
  die «Damen»,
  die blonden,
  Vater und Töchter,
  Vater und beide Jüngsten,
  die Kinder,
  Vater und Teenager
  \vspace{30mm}
}% END TNT

\item Wie viele solcher ``Teilmengen'' gibt es in der
fünfköpfigen Familie im ganzen?
\item Wie viele Teilmengen wären möglich in einer sechsköpfigen,
bzw. siebenköpfigen Familie?
\end{itemize}

Die Anzahl der Möglichkeiten, drei Elemente aus einer Menge mit total
fünf Elementen auszuwählen wird in der Mathematik mit dem
Binomialkoeffizienten angegeben:

$${5\choose 3} = 10$$

Berechnet wird dies mit der Fakultät (Anzahl der Permutationen), indem
alle fünf Elemente permutiert werden. Wir erhalten so zu viele
Möglichkeiten. Wir dividieren die Zahl durch die Anzahl aller Permutationen der
gewählten Personen, aber auch durch die Anzahl aller Permutationen der
nicht gewählten Personen:

$${5\choose 3} = \frac{5!}{3!\cdot{} 2!}$$
\newpage

\begin{definition}{Kombination}{}\index{Kombination!Definition}
Eine \textbf{Kombination} ist eine \textbf{un}geordnete Stichprobe.
\end{definition}

Wenn die Reihenfolge der Elemente keine Rolle spielt, gilt ganz
allgemein:

Die Anzahl der Möglichkeiten $k$ Elemente aus einer Grundgesamtheit
von $n$ Elementen auszuwählen ist gleich

\begin{definition}{Binomialkoeffizient}{}\index{Binomialkoeffizient!Definition}
$${n\choose k} = \frac{n!}{k!(n-k)!}$$
\end{definition}

Diese Zahl wird Binomialkoeffizient genannt und kann mit dem
Taschenrechner einfach
mittels \tiprobutton{ncrnpr}\footnote{Zweimaliges Drücken der Taste:
Wählen Sie für den Binomialkoeffizienten ``nCr'', nicht ``nPr''.}
berechnet werden.

Berechnen Sie gleich mit dem Taschenrechner
\begin{itemize}
\item $5\choose 3$\TRAINER{ = 10}
\item $5\choose 2$ (begründe) \TRAINER{  = 10, denn 3 Auswählen ist gleich
wie zwei nicht wählen!}
\item Swiss LOTTO ${42\choose 6} \cdot{} {6 \choose 1}$ \TRAINER{$ = 31\,474\,716$}
\end{itemize}

\begin{bemerkung}{}{}
Zur Begründung: Stellen Sie die $n$ Elemente in eine Reihe. Dazu gibt
es $n!$ Möglichkeiten. Nun dividieren wir die Vertauschungen der
gewählten Elemente ($k!$) und die Vertauschungen der nicht gewählten
Elemente $(n-k)!$ davon weg, es bleibt $\frac{n!}{k!(n-k)!}$.
\end{bemerkung}

\begin{gesetz}{}{}
Im Urnenmodell entspricht der Binomialkoeffizient dem Ziehen von $k$ Kugeln aus einer
Urne mit $n$ verschiedenen Kugeln. Die Reihenfolge der gewählten
Kugeln ist hier nicht relevant.
\end{gesetz}

\subsection*{Aufgaben}
%%\aufgabenfarbe{Kompendium: Kap. 5.2 Aufg. 1 - 4}

\aufgabenfarbe{Kompendium: Kap. 5.5.1 Aufg. 11., 12., 13., 14.*}
\newpage
