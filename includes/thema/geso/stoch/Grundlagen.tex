%%
%% Stochastik Grundlagen
%% 2020 - 07 - 24 φ
%%

\section{Stochastik Grundlagen}\index{Stochastik}\index{Wahrscheinlichkeitsrechnung}
\sectuntertitel{Mit an Sicherheit grenzender Wahrscheinlichkeit...}

Als \textbf{Stochastik} bezeichnet man den Oberbegriff über
Wahrscheinlichkeitsrechnung, Kombinatorik und Statistik. Wir befassen
uns vor allem mit...

\subsection{Lernziele}
\begin{itemize}
	\item ...elementarer Wahrscheinlichkeitsrechnung,
	\item Zufallsexperimenten und
	\item{statistischem Schließen.}
\end{itemize}


\subsection{Begriffe}
Am einfachsten beginnen wir mit einem Münzwurf-Experiment.

Werfen Sie eine Münze dreimal hintereinander (oder werfen Sie drei
Münzen, eine erste, eine zweite und eine dritte). Jede Münze hat die
Möglichkeit auf Kopf oder Zahl (alles andere heißt: nochmals werfen).

Notieren Sie den Ausgang Ihres Wurfexperimentes

\TNT{2.4}{%
\zB Kopf, Kopf, Zahl}%%

\TRAINER{\vspace{24mm}}

Wiederholen Sie das Experiment nun noch weitere drei Male:

\noTRAINER{\mmPapier{1.2}}

\noTRAINER{\mmPapier{1.2}}

\noTRAINER{\mmPapier{1.2}}
\newpage



\subsubsection{Ergebnis}\index{Ergebnis}

\TRAINER{Mögliche \textbf{Ergebnisse} wären zum Beispiel

  \begin{itemize}
  \item Kopf Kopf Zahl
  \item Zahl Zahl Kopf
  \item Zahl Zahl Zahl
  \item Kopf Kopf Zahl
  \end{itemize}
}

Diese insgesamt vier Resultate oder Ausgänge unseres Experimentes nennen wir \textbf{Ergebnisse}, womit wir bereits
bei den ersten Begriffen ankommen:

\newpage
Ein \textbf{Ergebnis}\index{Ergebnis} ist ein möglicher Ausgang eines
Zufallsexperiments. Ob beim Werfen «Kopf»-«Kopf»-«Kopf» nun
vorgekommen ist oder nicht, spielt hier keine Rolle. Die Menge \textbf{aller}
möglichen Ergebnisse nennen wir die

\subsubsection{Ergebnismenge}\index{Ergebnismenge}

und bezeichnen diese als $\Omega$\index{Omega!Ergebnismenge}\index{$\Omega$ s. Omega}.

Zeichnen Sie drei schöne {\huge $\Omega$}-Symbole

\TNT{1.6}{\vspace{16mm}}

Notieren Sie die Ergebnismenge $\Omega$ zu obigem
Zufallsexperiment. Notiert wird dies üblich in der Mengennotation:

$$\Omega = \{ (Kopf-Kopf-Kopf), (Kopf-Kopf-Zahl), ...\}$$

$\Omega = ...$\TRAINER{$\{(Kp.-Kp.-Kp.); (Kp.-Kp.-Zl.); (Kp.-Zl.-Kp);
  (Kp.-Zl.-Zl.);$
  
  $(Zl.-Kp.-Kp.); (Zl.-Kp.-Zl.); (Zl.-Zl.-Kp); (Zl.-Zl.-Zl.)\}$}

\TNT{3.6}{\vspace{30mm}}


\subsubsection{Ein zweites Beispiel}
Notieren Sie die Ergebnismenge $\Omega$ zum Wurf nacheinander mit zwei
Spielwürfeln.\footnote{Achtung: Es gibt 36 mögliche Ergebnisse!}

$\Omega=...$

\TRAINER{$...= \{$

  $  (1,1); (1,2); (1,3); (1, 4); (1,5); (1,6);$

  $  (2,1); (2,2); (2,3); (2, 4); (2,5); (2,6);$

  $  (3,1); (3,2); (3,3); (3, 4); (3,5); (3,6);$

  $  (4,1); (4,2); (4,3); (4, 4); (4,5); (4,6);$

  $  (5,1); (5,2); (5,3); (5, 4); (5,5); (5,6);$

  $  (6,1); (6,2); (6,3); (6, 4); (6,5); (6,6);$

  
  $\}$}

\noTRAINER{\mmPapier{4.8}}


\subsubsection{Ereignis}\index{Ereignis}
\begin{definition}{Ergebnis}{}
Der Begriff \textbf{Ereignis} bezeichnet eine Menge möglicher
Ergebnisse.
\end{definition}

Da die Begriffe nahe beieinander liegen schauen wir
nochmals das Münzwurf-Experiment an. Wir haben acht mögliche
Ergebnisse. 

Das \textbf{Ereignis} «Es wurde genau zwei mal Kopf geworfen» besteht
aus den Ergebnissen

\begin{itemize}
\item Kopf-Kopf-Zahl
\item Kopf-Zahl-Kopf
\item Zahl-Kopf-Kopf
\end{itemize}

\begin{bemerkung}{}{}
  Ein Ereignis kann natürlich auch aus genau einem Ergebnis bestehen. So ist bei einem einmaligen Wurf mit einem Würfel die Zahl 6 sowohl ein Ergebnis: ``6'', wie auch ein Ereignis $\{6\}$, diesmal jedoch als Menge bestehend aus einem einzigen Element. 
  \end{bemerkung}

\begin{definition}{Elementarereignis}{}
Besteht ein Ereignis genau aus einem Ergebnis, so sprechen wir von einem \textbf{Elementarereignis}\index{Elementarereignis}.
\end{definition}

Notieren Sie für obigen dreimaligen Münzwurf das Ereignis $E$ in Mengennotation «Es wurde mindestens
zwei mal Zahl geworfen»:

$E = \{ ...$

\TNT{4.8}{Es gibt vier mögliche Ergebnisse mit besagter Eigenschaft
  wobei das Ergebnis drei mal Zahl beim Ereignis mit dabei sein
  muss.
\vspace{1cm}
}
\newpage
\subsection{Gegenereignis}\index{Gegenereignis}
Das Gegenereignis zu einem bestimmten Ereignis $E$ sind alle möglichen Ergebnisse,
die im Ereignis $E$ nicht vorkommen.
So ist das Gegenereignis zu «\textit{Es wurde mindestens zwei mal Zahl
geworfen}» wäre folglich «\textit{Es wurde maximal einmal Zahl
  geworfen}».

Notieren Sie das Gegenereignis von «\textit{Der erste der drei Würfe
  ist Zahl.}» in der Mengenschreibweise, indem Sie alle
möglichen Ergebnisse in Mengenklammern angeben:

\noTRAINER{$$\{(..... - ...... - ......);  (..... - ...... - ......); (..... - ...... - ......); (..... - ...... - ......); \}$$}
\TRAINER{$$\{(Kopf - Kopf - Kopf);  (Kopf - Kopf - Zahl); $$ $$(Kopf - Zahl - Kopf); (Kopf - Zahl - Zahl); \}$$}

Das Gegenereignis wird mit einem Strich über dem Ereignis angegeben:

$$\bar{E} = \Omega \backslash E$$
Sprich: Das Gegenereignis $E$-Strich ist gleich Omega($\Omega$) ohne die Elemente
aus $E$.

Gehen wir zurück zum Würfelexperiment, bei dem zwei Spielwürfel (mit
eins bis sechs Augen) nacheinander geworfen werden:

Notieren Sie das Gegenereignis in der Mengennotation zu folgendem
Würfelereignis:

$E=$ «Bei mindestens einem der beiden Würfel liegen
mehr als zwei Augen.».

$\bar{E} = ...$

\TNT{2.4}{$...= \{(1,1); (1,2); (2,1); (2,2)\}$}


\subsection{Sicheres und unmögliches Ereignis}\index{Ereignis!unmögliches}\index{Ereignis!sicheres}
Unter dem sicheren Ereignis verstehen wir nichts anderes als die Menge
$\Omega$ selbst. Ein Ergebnis muss ja zutreffen, und da in $\Omega$
\textbf{alle} möglichen Ergebnisse zusammengefasst sind, bezeichnen
wir mit $\Omega$ das \textbf{sichere Ereignis}.

\subsubsection{unmögliches Ereignis}
Da ein Zufallsexperiment mindestens ein Ergebnis (also einen Ausgang
oder ein Resultat) aufweisen wird, so bezeichnen wir mit der leeren
Menge $\{\}$ das unmögliche Ereignis. Mit anderen Worten:

Es ist nicht möglich, dass das Experiment keinen Ausgang hat.


\begin{bemerkung}{}{}
$\Omega$ ist selbst eine Teilmenge von $\Omega$. Eine
  \textbf{Teilmenge} $T$ von $\Omega$ muss die Eigenschaft haben, dass
  jedes Element in $T$ auch in Omega vorkommt. Der Begriff
  «\textbf{Teil}» wird in der Mengenlehre etwas
  \textit{überstrapaziert}; in dem Sinne, dass jede Menge auch
  Teilmenge ihrer selbst ist.
\end{bemerkung}

\begin{definition}{}{}
Mit Betragsstrichen $|.|$\index{Betragsstriche}\index{$\mid\cdot \mid$ s. Betrag(-striche)} wird die Anzahl Elemente in einer Menge angegeben.
\end{definition}
\newpage
