%%
%% Stochastik Grundlagen
%% 2020 - 08 - 03 φ
%%

\subsection{Zufallsexperimente}\index{Experiment}\index{Zufallsexperimente}

\subsubsection{Laplace-Experiment}\index{Experiment!Laplace}\index{Laplace-Experiment}
Nach Pierre-Simon Laplace (1749-1827).

Ein fairer Spielwürfel wird geworfen. Jede Seite hat genau die selbe
Wahrscheinlichkeit. Wie groß ist nun die Wahrscheinlichkeit, mit einem
Wurf eine Augensumme von mehr als vier zu werfen?

\TNT{2.4}{Glückliche Ereignisse: 5 und 6 (\textbf{zwei} Stück). Alle möglichen Ergebnisse: 1 bis 6 (\textbf{sechs} Stück). Wahrscheinlichkeit = $\frac{2}{6}=\frac13$}

Dabei handelt es sich um ein klassisches Laplace-Experiment:
\begin{itemize}
\item Jeder einzelne Ausgang (Ergebnis) hat die selbe Wahrscheinlichkeit
      (hier $p = \frac{1}{6}$).
\item Die Wahrscheinlichkeit, dass ein Ereignis (hier $E = \epsdice{5}$
oder $\epsdice{6}$) eintritt, wird berechnet mit $P(E)
= \frac{\textrm{Anzahl gewünschte Ergebnisse}}{\textrm{Anzahl
mögliche Ergebnisse}}$.

Hier $P(\epsdice{5}\cup\epsdice{6}) = \frac{2}{6}$.
\end{itemize}

Kein Laplace-Experiment ist \zB das Werfen eines gezinkten Würfels,
bei dem die Augenzahl 6 häufiger auftritt, als die anderen Augenzahlen.

\subsection*{Aufgaben}
\aufgabenfarbe{Kompendium: Kap. 5.2 Aufg. 1. - 3.}

\newpage


%%%%%%%%%%%%%%%%%%%%%%%%%%%%%%%%%%%%%%%%%%%%%%%%%%%%%%%%%%
\subsection{Baumdiagramme}\index{Baumdiagramm!Stochastik}
(Einstufige und Mehrstufige Experimente)

\TRAINER{(S. Youtube Videso dazu: Einfach-Mathe!)}

\subsubsection{Einstiegsbeispiel: Glücksrad}
\bbwCenterGraphic{4cm}{geso/stoch/img/gluecksrad75deg}
An einem Glücksrad wird gedreht. $75\degre$ erzielen sofort einen Gewinn. Die anderen $285\degre$ geben einen Verlust an. Zum Glück darf ich am Rad drei mal drehen. Das Spiel endet entweder beim ersten Gewinn oder aber spätestens nach dreimaligem Drehen.

Wie groß ist die Wahrscheinlichkeit, bei diesem Spiel zu gewinnen?

Zeichnen Sie dazu das Baumdiagramm, das bei jedem ``Sieg'' sofort endet und
maximal drei Stufen tief geht.

\TNT{8}{Einseitiger entarteter Baum mit Sieg = 75/360 und Verlust = 285/360. Drei Mal Verlust bei $\left(\frac{285}{360}\right)^3 \approx 49.6\%$}


\newpage

Um Wahrscheinlichkeiten aufzuzeichnen bietet sich das Baumdiagramm an.
Dabei zeichnet man einen «Baum» von links-nach-rechts oder wie in der Informatik üblich von oben nach unten.

Zu den Ästen schreibt man deren Wahrscheinlichkeiten.

Man beginnt bei der Wurzel und für jeden möglichen Ausgang zeichnet man einen Ast. Dabei gelten die folgenden Gesetze:



\begin{gesetz}{Produktregel/Pfadregel}{}\index{Produktregel!Baumdiagramm}\index{Pfadregel!Baumdiagramm}
Alle Wahrscheinlichkeiten von der Wurzel bis zum Endknoten werden aufmultipliziert, um die Endwahrscheinlichkeit (am Endknoten) zu erhalten.
\end{gesetz}

\begin{gesetz}{Summenregel I}{}\index{Summenregel!Baumdiagramm}
Alle von einem Knoten ausgehenden Äste haben in der Summe die Wahrscheinlichkeit 1 (= 100\%).
\end{gesetz}

\begin{gesetz}{Summenregel II}{}
Alle Endwahrscheinlichkeiten (bei den Endknoten\footnote{Endknoten werden auch als Blätter des Baumes bezeichnet.}) zusammen addiert ergeben in der Summe die Wahrscheinlichkeit 1 (= 100\%).
\end{gesetz}

\newpage


\subsubsection{Referenzaufgaben}

\paragraph{Mindestens zwei Sechser} Ein Würfel wird dreimal hintereinander geworfen.

a) Wie groß ist die Wahrscheinlichkeit, dass mindestens zwei Sechser dabei sind?

b) Wie groß ist die Wahrscheinlichkeit, dass genau zwei Sechser dabei sind?

Zeichnen Sie das zugehörige Baumdiagramm:

\TNT{14}{Lösung mit Baum oder s. Binomialvertelung.

  a) mindestens 2 Sechser: $\frac2{27}$

  b) genau zwei Secheser: $\frac{5}{72}$
\vspace{10cm}}
\newpage


\paragraph{Urne ohne zurücklegen} In einer Urne liegen drei grüne, zwei blaue und eine rote Kugel. Ich ziehe blind (Urne) zwei Kugeln hintereinander, ohne diese wieder zurückzulegen.

Wie groß ist die Wahrscheinilchkeit, dass genau eine blaue Kugel dabei ist?

Zeichnen Sie das zugehörige Baumdiagramm:

\TNT{14}{Lösung: $P(E)=\frac8{15}$\vspace{13cm}}
  
\newpage

\paragraph{Ziege}\index{Ziege} Heidi und Peter spielen mit einem Würfel um eine Ziege.

Sie vereinbaren folgende Regeln:

\begin{enumerate}
\item Die Ziege erhält sofort, wer eine 5 oder eine 6 würfelt.
\item Zuerst würfelt Heidi, dann (falls Heidi noch nicht gewonnen hat) Peter, dann allenfalls noch einmal Heidi.
\item Ist nach drei Würfen noch nichts entschieden, dann erhält Peter die Ziege.
\end{enumerate}


Zeichnen Sie das zugehörige Baumdiagramm:

\TNT{14}{\vspace{14cm}}

\subsection*{Aufgaben}

Mehrstufige Zufallsexperimente:

\aufgabenfarbe{Kompendium: Kap. 5.2 Aufg. 4. und Kap. 5.5.2 Aufg 17. - 23.}
\newpage

%%%%%%%%%%%%%%%%%%%%%%%%%%%%%%%%%%%%%%%%%%%%%%%%%%%%%%%%%%%%%%%%%%%%%%%%

\subsection{Binomialverteilung}

\TRAINER{Einstiegsvideo: ``EinfachMathe!'' (S. Wiki-Lernvideos)}


\subsubsection{Bernoulli-Experiment}\index{Experiment!Bernoulli}\index{Bernoulli-Experiment}
Ein mehrfaches Drehen am Glücksrad ist ein typisches
Bernoulli\footnote{Jakob I Bernoulli, Schweizer Mathematiker,
  1654-1705} Experiment, d. h.:

\begin{itemize}
\item Das Experiment wird $n$-mal durchgeführt.
\item Das Einzelexperiment hat genau zwei Ausgänge: Erfolg/Misserfolg.
\item Jedes Einzelexperiment hat für die erfolgreichen Ausgänge die gleiche
      Wahrscheinlichkeit $p$ (somit ist die Wahrscheinlichkeit $q$ für den
      Misserfolg gleich $1-p$).
\item Die Einzelexperimente sind voneinander unabhängig.
\end{itemize}

Typischerweise sind wir bei Bernoulli-Experimenten an der
Zufallsvariable $X$ interessiert, welche angibt, wie oft ein Erfolg
eingetreten ist: $X$ hat also die Werte 0, 1, 2, 3, 4, ..., $n$.

\textbf{Kein} Bernoulli-Experiment ist beispielsweise das Ziehen von zwei Bällen aus
einer Urne mit drei gelben und sechs orangefarbenen Bällen ohne diese
jeweils zurückzulegen; denn dabei verändern sich die
Wahrscheinlichkeiten der verbleibenden Bälle nach jedem Herausnehmen.

\paragraph{Beispiele von Bernoulli-Experimenten}

\begin{itemize}
\item
In einer Urne liegen zehn Bälle: vier blaue und sechs grüne. Ein
Einzelexperiment besteht darin, genau eine Kugel zu ziehen, die Farbe
zu notieren und die Kugel sogleich wieder zurückzulegen.
Das Gesamtexperiment besteht darin, fünf Einzelexperimente
durchzuführen. Jedesmal ist die Wahrscheinlichkeit einen blauen Ball
zu erwischen  gleich groß und wir sprechen von einem Bernoulli-Experiment.
\end{itemize}

Es müssen aber nicht immer Würfel, Münzen oder Kugeln in Urnen sein:

\begin{itemize}
\item
Ein Basketballspieler trifft mit einer Wahrscheinlichkeit von 70\%
($p=0.7$). Der Spieler wirft vier Mal. Somit ist $q=0.3$ und $n=4$.
\end{itemize}
\newpage


\subsubsection{Formel zur Binomialverteilung}
Bei der Binomialverteilung handelt es sich um ein
Bernoulli-Experimetes
in einem mehrstufigen Baum. Dabei interessiert
uns, wie groß die Wahrscheinlichkeit ist, dass von den beiden
möglichen Ausgängen einer davon $k$ mal auftritt.

Wie groß ist die Wahrscheinlichkeit mit einem Würfel bei fünfmaligem
Werfen drei Mal die Zahlen 1 oder 2 zu erreichen.

Dabei legen wir fest:
\begin{itemize}
\item
  $n$ = Anzahl mögliche Durchführungen; hier 5
\item
  $k$ = Anzahl der gewüntschten ``Treffer''; hier 3, denn wir wollen 3
mal eine 1 oder eine 2 werfen
\item
  $p$ = Wahrscheinlichkeit, dass wir eine 1 oder eine 2 werfen, hier
  also wie beim Laplace-Experiment: $p=\frac62=\frac13$
\end{itemize}

Nun gilt

\begin{gesetz}{Bernouilli-Formel}{}
  $$P(X=k) = {{n}\choose {k}}\cdot{}p^k\cdot{}(1-p)^{n-k}$$
\end{gesetz}

In obigem Beispiel ergibt sich:
$$P(X=3) =
{{5}\choose {3}} \cdot{}\left(\frac13\right)^3\cdot{}\left(1-\frac13\right)^{5-3}
\approx{} 0.1646 = 16.46\%$$

Dies kann mit dem Taschenrechner gelöst werden:

\tiprobutton{data_stat-reg-distr} : \texttt{DISTR} :  4:Binomialpdf :
\texttt{SINGLE} : dann $n$ bein $n$, $p$ bei $p$ und $k$ bei $x$ eintragen.
\newpage


\subsection*{Aufgaben}
Bernoulli-Ketten:

\aufgabenfarbe{Kompendium: 5.5.4 Aufg. 28. und 29.}

Minimale und maximale Ergebnisse bei Bernoulli-Ketten:

\aufgabenfarbe{Kompendium: Kap. 5.7. Aufg. 36. und 37.}
\newpage



\subsection{Hypergeometrische Verteilung}
In einer Urne liegen sieben Kugeln. Drei davon sind Treffer (grün) und vier davon sind Nieten (schwarz).
Wir dürfen zwei Kuglen blind herausnehmen (ohne diese wieder zurückzulegen).

\bbwCenterGraphic{5cm}{geso/stoch/img/Urne3T4N.png}

Wie groß ist nun die Wahrscheinlichkeit
\begin{itemize}
\item keinen Treffer zu erzielen ($k=0$),
\item genau einen Treffer zu erzielen ($k=1$) oder
\item beide Treffer zu erzielen ($k=2$)?
\end{itemize}

Diese Wahrscheinlichkeiten können wir nun einerseits durch ein Baumdiagramm lösen, indem wir die Treffer mit \textbf{T} und die Nieten mit \textbf{N} bezeichnen:

\noTRAINER{\mmPapier{7.2}}
\TRAINER{
\bbwCenterGraphic{7cm}{geso/stoch/img/Baum2aus3T4N.png}
}

Somit erhalten wir für zwei Treffer ($k=2$) eine Wahrscheinlichkeit von $\frac{6}{42}$; für einen Treffer ($k=1$) die Summe aus $\frac{12}{42}$ und nochmals $\frac{12}{42}$ also total $\frac{24}{42}$ und schlussendlich für keinen Treffer ($k=0$) die Wahrscheinlichkeit $\frac{12}{42}$.
Die Summe aller Wahrscheinlichkeiten muss dann immer gleich 1.0 (100\%) sein.
\newpage



Andererseits gibt es dazu auch eine Formel:

$$h(k|N;M;n) := P(X=k) = \frac{ {M\choose k} \cdot{} {{N-M}\choose {n-k}}}{{N \choose n}}$$

Dabei bezeichnen
\begin{itemize}
\item $N$ die Anzahl Elemente in der Urne (hier $N = 7$)
\item $M$ die Anzahl der möglichen Treffer in der Urne. Hier die Anzahl der grünen Kugeln (also $M = 3$)
\item $n$ die Anzahl der herausgezogenen Kugeln (hier $n = 2$)
\item $k$ die Anzahl der gewünschten bzw. zu erreichenden Treffer. Hier \zB null, einen oder zwei.
\end{itemize}
\newpage


Daraus werden die folgenden Werte berechnet:

\begin{itemize}
\item $N-M$ die Anzahl der möglichen Nieten in der Urne. Hier die Anzahl der schwarzen Kugeln (also $N-M = 4$)
\item $n-k$ die Anzahl der Nieten, mit anderen Worten, der nicht gewünschten Kugeln.
\item $M\choose k$ die Anzahl der Möglichkeiten $k$ Elemente aus $M$ Trefferkugeln zu erzielen. Ist \zB $k=1$ und $M=3$, so haben wir drei Varianten, eine Trefferkugel zu erzielen.
\item ${N-M}\choose {n-k}$ Die Anzahl Möglichkeiten $n-k$ (erhaltene Nietenanzahl) aus den möglichen Nieten zu wählen. Ist \zB $k=1$, so haben wir vier Varianten, eine Niete gezogen zu haben.
\item ${M\choose k} \cdot{} {{N-M}\choose {n-k}}$ die Anzahl der Möglichkeiten $k$ Treffer und $n-k$ Nieten zu ziehen. Ist \zB $k=1$, so haben wir hier $3\cdot{}4$, also zwölf Möglichkeiten, genau einen Treffer und genau eine Niete zu erhalten.
\item $N \choose n$ die Anzahl der Möglichkeiten zwei ($n$) Elemente
  aus sieben ($N$) Elementen auszuwählen. Dies ist der übliche
  Binomialkoeffizient\footnote{Taschenrechner: \fbox{$nCr$} (nCr, nicht nPr wählen)}.
\item $P(X=k)$ ist die Wahrscheinlichkeit, genau $k$ Treffer zu erzielen.
\item $h(k|N;M;n)$ ist eine andere Notation der hypergeometrischen Verteilung, indem $k$, $N$, $M$ und $n$ explizit angegeben werden. Hier \zB für genau einen gewünschten Treffer aus obiger Urne (7 Kugeln drei Treffer) mit dem Ziehen von zwei Kugeln: $h(1|7;3;2)=\frac{{3 \choose 1} \cdot{} {{7-3}\choose {2-1}  }   }{{ 7 \choose 2 }}=\frac{12}{21}$
\end{itemize}

\subsection*{Aufgaben}

\aufgabenfarbe{Kompendium: Kapitel 5.5.3 Aufgaben 24 - 27}
\newpage

