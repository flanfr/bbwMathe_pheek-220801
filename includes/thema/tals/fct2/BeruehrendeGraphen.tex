\section{Berührende Graphen}\index{Graphen!berührende}\index{berührende Graphen}

\textbf{Einführungsbeispiel}


Gegeben ist die Parabel $f: y=\frac{1}{2}x^2 -2x +3$ und von einer
Geraden $g$ ist die Steigung $a =\frac12$ gegeben.

Gesucht ist von der Geraden $g$ der $y$-Achsenabschnitt so, dass die
Gerade die Parabel tangiert; also genau in einem Punkt berührt.

Wo (in welchem Punkt $P=(x_P|y_P)$) tangiert also die Gerade $g$ die Parabel $f$?

In der folgenden Skizze sind drei mögliche Geraden mit Steigung
$\frac{1}{2}$ gezeichnet. Nur eine \textit{tangiert} die Parabel.

\bbwGraph{-2}{6}{-2}{6}{
  \draw[thick,color=blue,variable=\x,domain=-1:5] plot ({\x}, {0.5*\x*\x - 2*\x + 3}); 
  \draw[color=red,variable=\x,domain=-1:5] plot ({\x},{0.5*\x + 1});
  \draw[color=red,variable=\x,domain=-1:5] plot ({\x},{0.5*\x - 1.5});
  \draw[color=cyan,thick,variable=\x,domain=-1:6] plot ({\x},{0.5*\x  -0.125});
  \bbwDot{2.5, 1.125}{green}{north}{P}
}

Lösungsidee: Wähle als Parameter den Achsenabschnitt $b$ (der Geraden)
und berechne die Schnittpunkte mit $f(x) = g(x)$. Nur für ein $b$ gibt
es genau eine Lösung (Diskriminante = 0). Den $x$-Wert dieses
Schnittpunkts nennen wir $x_P$:

\TNT{3.2}{
\begin{tabular}{rclr}
$f(x_P)$                   & $=$ & $g(x_P)$                            & \\
$\frac{1}{2} x_P^2 - 2x_P + 3$ & $=$ & $\frac{1}{2}x_P + b$                  & \\
$0$                        & $=$ & $\frac{1}{2}x_P^2 - 2.5x_P + 3 - b$ & (I)\\
\end{tabular}
}%% END TNT

\TRAINER{Ev besseres Beispiel:   $f: y=qx^2-2x+3$ und $g: \frac12x + \frac12$, doch noch nachrechnen. Das obige Besipiel hatte sich nicht sehr bewährt, da es eine Abkürzung gibt oder da es nicht den allgemeinen Fall abdeckt.}
\newpage

Die Diskriminante $D=B^2 - 4AC$ muss gleich 0 sein; mit

\TNT{5.2}{
$A = \frac{1}{2}$, $B = -2.5$ und $C = 3-b$.

$$D=0=B^2-4AC = 6.25 -4\cdot{}\frac{1}{2}(3-b)$$
$$\Longrightarrow 6.25 = 6-2b$$
$$\Longrightarrow b=-\frac{1}{8}$$

$$\Longrightarrow g: y=\frac{1}{2}x - \frac{1}{8}$$
}%% END TNT
  
Für den Schnittpunkt müssen wir nun nur doch das gefundene $b$ ($b =
-\frac{1}{8}$) in die Gleichung (I) einsetzen:

$$0 = \frac{1}{2}x_P^2 - 2.5x_P +3 - \frac{1}{8}$$

Da die Gleichung in $x_P$ die Diskriminante 0 hat, ist die Lösung
einfach $x_P = \frac{-B}{2A}$ mit $A = \frac{1}{2}$ und
$B=-2.5$. Somit erhalten wir sofort $x_P = 2.5$. Für das $y_P$
brauchen wir das $x_P$ nur in $f$ oder in $g$ einzusetzen und
erhalten $y_P =\frac{9}{8}$. Der gesuchte Berührungspunkt $S$ lautet
also:

$$P = \left(\frac{5}{2}\middle|\frac{9}{8}\right)$$

\TRAINER{(Je nach Zeit wäre 699. a) noch eine Vorzeigeaufgabe.)}

\subsection{Aufgaben}
\TRAINER{Achtung bei 702., dass nicht das Gleichungssystem bereits
  nach dem Gleichsetzen der Kurven aufgestellt wird. Erst nach dem
  Null-Setzen der Diskriminante erhalten wir eine gültige Gleichung
  für das Gleichungssystem.}
\TALSAadB{190}{698., 702., 703., 705.}
\newpage


\subsection{Grenzwerte und Steigungsfunktion (Optional)}

Wie macht das ein CAS, dass es den tiefsten bzw. den höchsten Punkt
einer Funktion bestimmen kann? Hier ein Erklärungsversuch am Beispiel
der quadratischen Funktion.
Betrachten wir die allgemeine quadratische Funktion $$p: y=ax^2 + bx +
c$$
Mit dem selben $a$ und dem selben $b$ kann ich eine Gerade $s$
definieren, die ich die \textbf{Steigungsfunktion} nenne:
$$s: y= 2ax+b$$

Diese Steigungsfunktion $s$ gibt in jedem Punkt $x$ die Steigung der
Parabel $p$ an.

\begin{beispiel}{Parabel}{}
  Gegeben ist die Parabel $p: y=2.5x^2 - 3x + 6.5$. Wo hat diese Parabel
  ihren Tiefpunkt? Im Scheitelpunkt $x_S =
  \frac{-b}{2a} = \frac{-(-3)}{2\cdot{}2.5}$. Dies ist gleichzeitig die Nullstelle der
  Steigungsfunktion $s: y= 2\cdot{}2.5x - 3$.

  Wir können damit aber auch die Steigung in einem ganz anderen Punkt
  \zB für $x=10$ berechnen. Die Parabel $p$ hat an der Stelle $x=10$
  die Steigung, die durch die Steigungsfunktion im Punkt $10$
  ermittelt wird: $s(10) = 2\cdot{}10\cdot{2.5} - 3 = 47$.
  
\end{beispiel}

\begin{beispiel}{Gerade gesucht}{}
Gegeben ist die Parabel $p: y=4x^2 -6x + 3$. Gesucht ist die Gerade
$g: y=ax+b$, sodass die Gerade die Parabel bei $x=7$ berührt.

\begin{enumerate}
\item Die Steigungsfunktion $s$ lautet: $s: y=2\cdot{}4x - 6$
\item Für $x=7$ hat die Steigungsfunktion den Wert 50, und somit hat
  die Parabel bei $x=7$ die Steigung 50.
\item Um den Berührungspunkt $B$ zu finden, setzen wir 7 diesen in $p$
  ein: $p(7)= 4\cdot{}49-6\cdot{}7+3 = 157$; und somit $B=(\LoesungsRaum{7}|\LoesungsRaum{157})$.
\item Die gesuchte Gerade $g: y=ax+b$ hat also auch die Steigung 50
  und verläuft durch den Berührungspunkt $B=(7|157)$. Also
  $157=50\cdot{}7+b$. Somit ist das gesuchte $b = 157-50\cdot{}7=-193$
  und die gesuchte Funktionsgleichung der Geraden lautet: $y = 50x-193$
  \end{enumerate}
\end{beispiel}
\newpage



\subsubsection{Beweis der Steigungsfunktion}

Betrachten wir auf auf der $x$-Achse zwei benachbarte Punkte $x$ und
$x+\Delta$. Die Funktionswerte lauten $f(x)$ und $f(x+\Delta)$. Die
\textbf{Steigung} kann nun für eine Gerade bestimmt werden durch:

\bbwCenterGraphic{7cm}{tals/fct3/img/Steigungsfunktion.jpg}

$$\frac{f(x+\Delta) - f(x)}{\Delta}$$

Dies gilt für eine Parabel $p: y=ax^2+bx+c$ näherungsweise auch:
$$\frac{f(x+\Delta)-f(x)}{\Delta} = \frac{(a(x+\Delta)^2 + b(x+\Delta)
  + c) - (ax^2 + bx +c)}{\Delta}=2ax+a\Delta+b$$

Wenn wir nun das $\Delta$ gegen Null gehen lassen, also immer kleinere
Werte einsetzen, so verschwindet der Term $a\Delta$ fast und unsere
Formel stimmt annähernd --- jedoch präzise genug, um dies als Beweis
gelten zu lassen.

Ein anderer Beweis wäre die Tangente effektiv einzusetzen und diese so
zu wählen, dass es genau einen Schnittpunkt gibt, was wieder darauf
zurückführt, dass die Diskriminante = Null gesetzt werden muss. Dann
sind wir exakt, aber der Beweis ist ungleich aufwändiger.
\newpage

