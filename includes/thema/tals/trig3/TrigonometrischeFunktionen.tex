%% Trigonometrie III
%% Trigonometrische Funktionen
%% 2020 - 12 - 21 φ@bbw.ch

\section{Trigonometrische
  Funktionen}\index{Funktionen!trigonometrische}\index{Trigonometrische
  Funktionen}

Lernziele:
\begin{itemize}
\item Aufzeichnen von $\sin()$-, $\cos()-$ und $\tan()$-Funktionen
\item Sie können trigonometrische Funktionen strecken, stauchen und
  verschieben (Amplitude $\hat{y}$, Frequenz $\omega$, Phase
  $\varphi$): $y=\hat{y}\cdot{}\sin(\omega\cdot{}(x+\varphi))$
\end{itemize}


\subsection{Periodische Vorgänge}
Machen Sie Beispiele von periodischen Vorgängen
\begin{itemize}
  \item \TRAINER{Schallwellen}
    \noTRAINER{......................................}
  \item \TRAINER{Tageslänge, Jahresrythmus}
    \noTRAINER{......................................}
  \item \TRAINER{Puls}
    \noTRAINER{......................................}
  \item \TRAINER{Wechselstrom}
    \noTRAINER{......................................}
\end{itemize}
\newpage

\subsection{Sinus-, Cosinusfunktionen}
Sinus, Cosinus und Tangens können nicht nur als Verhältnisse angesehen
werden, sondern auch als Funktionen abgebildet werden.
Zeichnen Sie zu jedem Winkel von $0\degre - 360\degre$ den zugehörigen
\textbf{Sinus}-Wert ins folgende Koordinatensystem:

\bbwCenterGraphic{10cm}{tals/trig3/img/coordSystem0-450.png}


Zeichnen Sie zu jedem Winkel von $0\degre - 360\degre$ den zugehörigen
\textbf{Cosinus}-Wert ins folgende Koordinatensystem:

\bbwCenterGraphic{10cm}{tals/trig3/img/coordSystem0-450.png}
\newpage

\subsection{Tangensfunktion}
Zeichnen Sie zu jedem Winkel von $0\degre - 360\degre$ den zugehörigen
\textbf{Tangens}-Wert ins folgende Koordinatensystem:

\bbwCenterGraphic{10cm}{tals/trig3/img/coordSystem0-450.png}

\subsection{Definitions- und Wertebereiche}

Geben Sie von Sin-, Cos- und Tangensfunktion die Definitions- und
Wertebereiche an:

\begin{tabular}{|l|c|c|c|} \hline
  Funktion & Definitionsbereich $\mathbb{D}$ & Wertebereich
  $\mathbb{W}$ & Nullstellen\\ \hline
  $\sin()$ & \TRAINER{$\mathbb{R}$} & \TRAINER{$[-1,1]$} &
  \TRAINER{$0\pm  n\cdot{}180\degre: n\in\mathbb{N}$} \\ \hline
  $\cos()$ & \TRAINER{$\mathbb{R}$} & \TRAINER{$[-1,1]$}& \TRAINER{$90\pm  n\cdot{}180\degre: n\in\mathbb{N}$}\\ \hline
  $\tan()$ & \TRAINER{$\mathbb{R}\backslash\{90\degre \pm
    n\cdot{}180\degre: n\in\mathbb{N} \}$} & \TRAINER{$\mathbb{R}$} & \TRAINER{$0\pm  n\cdot{}180\degre: n\in\mathbb{N}$}\\ \hline
\end{tabular}
\newpage


\subsection{Die allgemeine Sin- bzw. Cosinusfunktion}

Sie möchten die Sinusfunktion um 50\% in $y$-Richtung strecken und um
$30\degre$ nach rechts verschieben.

a) Wie sieht die Funktion aus?


\bbwCenterGraphic{10cm}{tals/trig3/img/coordSystem0-450.png}


b) Wie lautet die Funktionsgleichung

$$y = \LoesungsRaum{1.5 \cdot{} sin(x - 30\degre)}$$
\newpage

Sie möchten die Cosinusfunktion so verändern, dass sich die
Nullstellen nicht mehr bei $-90\degre$, $90\degre$, $270\degre$
etc. befinden, sonedern bei $60\degre$, $-60\degre$, $180\degre$
etc. also bei $60\degre \pm$ Vielfachen von
$120\degre$. Nullstellenmenge = $\{60\degre\pm n\cdot{}120\degre:
n\in\mathbb{N}\}$

Machen sie eine Skizze

\bbwCenterGraphic{10cm}{tals/trig3/img/coordSystem0-450.png}

Wie lautet die Funktionsgleichnug?

$$y = \LoesungsRaum{cos(x\cdot\frac{90\degre}{60\degre})}$$
\newpage




\subsubsection{Allgemeine Sinusfunktion}

Die allgemeine Sinusfunktion (ohne Verschiebung in $y$-Richtung und
ohne Phasenverschiebung) lautet:
$$y=\hat{y}\cdot{}\sin(\omega\cdot{}x)$$

Dabei sind:
\begin{itemize}

\item
  Amplitude $\hat{y}$: Ausschlag in $y$-Richtung.
\item
  Frequenz $\omega$: Je größer $\omega$, umso schneller wechselt die
  Funktion die Werte von positiv nach negativ und umgekehrt.
\end{itemize}

Wird zusätzlich eine Phasenverschiebung in $x$-Richtung vorgenommen, so lautet die Gleichung:

\begin{center}
  \fbox{$y=\hat{y}\cdot{}\sin(\omega\cdot{}(x-\varphi))$}
  \end{center}

Hier gilt:
\begin{itemize}

\item
  Amlitude $\hat{y}$
\item
  Frequenz $\omega$
\item
  Phasenverschiebung\footnote{Dabei wird der Funktionsgraph in positiver $x$-Richtung um
    $\varphi$ verschoben.} $\varphi$
\end{itemize}


\subsection*{Aufgaben}
\TALSAadB{232ff}{880. - 884.}
