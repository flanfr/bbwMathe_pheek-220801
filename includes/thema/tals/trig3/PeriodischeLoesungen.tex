%% Trigonometrie III
%% Periodische Lösungen
%% 2020 - 12 - 21 φ@bbw.ch

\section{Periodische Lösungen}\index{periodische Lösungen}

\subsection*{Lernziele}

Periodische Lösungen bei trigonometrischen Gleichungen.

\begin{itemize}
\item Periodische Lösungen bei trigonometrischen Gleichungen
\item Arkusfunktionen interpretieren und periodische Lösungen angeben
\end{itemize}



\subsection{Einstiegsbeispiel}
Das folgende Beispiel stammt 1:1 aus dem Lehrplan:

Lösen Sie die folgende Gleichung im Definitionsbereich $x \in \mathbb{D} = [ -\pi, 3\pi[$ und finden Sie die Lösungsmenge für $x$:

$$\cos\left(2x+\frac{\pi}{9}\right) = 0.5$$

Die Einstiegsaufgabe liest sich im Gradmaß also folgendermaßen:

$$\cos\left(2x+20\degre\right) = 0.5$$

Die Gleichung hat offensichtlich periodische Lösungen, welche wir entweder am Einheitskreis, mittels der Cosinus-Funktion oder mit dem Taschenrechner herausfinden und schlussendlich auf den vorgegebenen Definitionsbereich einschränken müssen.


\begin{bemerkung}{}{}
  Ist eine Gleichung mit den Winkelfunktionen $\cos$, $\sin$ oder $\tan$ gesucht, so müssen wir immer zuerst überlegen, ob wir die Größen im Grad- oder im Bogenmaß vor uns haben. Eine Aufgabe im Stil $$0.75 = \sin(x + 2\pi + 40\degre + 1.5)$$ \textbf{wird} hier \textbf{nicht behandelt}, da bei $1.5$ nicht klar ist, ob es sich nun um eine Zahl im Grad- oder im Bogenmaß handelt.
\end{bemerkung}%%
\newpage


\textbf{Lösungsidee:} Substitution:
Wir setzen $y := 2x + \frac{\pi}{9}$ und erhalten somit:
$$\cos(y) = 0.5.$$

Aus dem Einheitskreis lesen wir die Lösung für $y$ bei $y_1=60\degre = \frac{\pi}{3}$ und bei $y_2=300\degre = \frac{5\pi}{3}$ ab. Nun sind aber alle periodischen Erweiterungen dieser beiden Lösungen auch Lösungen der Gleichung:

$$y \in \left\{\frac{\pi}{3} + z\cdot{}2\pi: z\in\mathbb{Z}\right\} \cup \left\{\frac{5\pi}{3}+ z\cdot{}2\pi: z\in\mathbb{Z}\right\}$$

Beachten Sie, dass $n$ in $\mathbb{Z}$ liegt; also auch negative Lösungen sind zugelassen.

Bei einer Substitution dürfen wir die Rücksubstitution nicht vergessen! Wie lautet nun unser $x$, wenn wir das $y$ 

Für $y_1$ erhalten wir: $y_1= 2x_1 + \frac{\pi}{9}$ und somit $\frac{\pi}{3} + z\cdot{}2\pi = 2x_1 + \frac{\pi}{9}$. Durch Abziehen von $\frac{\pi}{9}$ und Teilen durch 2 erhalten wir (mit $z\in\mathbb{Z}$)

$$x_1 = \frac{\pi}{9} + z\cdot{}\pi.$$

Analog für $x_2$: $y_2 = 2x_2 + \frac{\pi}{9}$ und somit $\frac{5\pi}{3} + z\cdot{}2\pi = 2x_2 + \frac{\pi}{9}$. Ebenfalls durch Abziehen von $\frac{\pi}{9}$ und Halbieren ergibt sich

$$x_2 = \frac{7\pi}{9} + z\cdot{}\pi.$$

Als allerletztes gilt es noch durch «Abzählen» herauszufinden, welche der Lösungen im vorgegebenen Definitionsbereich $\mathbb{D} = [-\pi, 3\pi[$ liegen:

    Für $x_1$ sind dies $\frac{\pi}{9} - \pi$, $\frac{\pi}{9}$, $\frac{\pi}{9} + \pi$ und $\frac{\pi}{9} + 2\pi$.

    Für $x_2$ sind dies $\frac{7\pi}{9} - \pi$, $\frac{7\pi}{9}$, $\frac{7\pi}{9} + \pi$ und $\frac{7\pi}{9} + 2\pi$.
   
    Zusammengefasst haben wie die folgenden Schritte verwendet:
    \begin{rezept}{}{}
\begin{itemize}
    \item Substitution
    \item Lösen, Ablesen
    \item periodische Lösungen angeben
    \item Rücksubstitution
    \item Lösung durch «Abzählen» auf Definitionsbereich einschränken
\end{itemize}
\end{rezept}%%
\newpage

\subsection{Taschenrechner}
Freilich kann mit dem Taschenrechner die Gleichung aus dem
Einstiegsbeispiel $$\cos(2x+\frac{\pi}{9})=0.5$$ relativ einfach gelöst werden:

Mit der Eingabe des Definitionsbereichs (domain) «dom» ...
$$\mathbb{D} =  [ -\pi, 3\pi[$$
... und der Gleichung «gls» ...
$$\cos\left(2x+\frac{\pi}{9}\right) = 0.5$$

... erhalten wir die selben acht Lösungen.
    
\bbwCenterGraphic{8cm}{tals/trig4/img/nspire1.png}%%
\newpage
