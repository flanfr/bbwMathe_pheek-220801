%% Trigonometrie III
%% Grad und Bogenmaß
%% 2020 - 12 - 21 φ@bbw.ch

\section{Grad- und Bogenmaß}\index{Bogenmaß}

Anwendungen der trigonometrischen Funktionen im Grad- und Bogenmaß

Wir erinnern uns, dass ein Winkel auf verschiedene Weise dargestellt
werden kann. Das uns geläufige System ist die Einteilung in Grad von
$0\degre$ bis $360\degre$.

Es ist aber auch üblich, Anstelle von Grad die Länge des Bogens im
Einheitskreis (Maßeinheit \texttt{rad}) anzugeben. Dabei gilt:

\begin{gesetz}{Bogenmaß}{}
  
  $$90\degre=\frac{\pi}2 \textrm{ [rad]}$$
  $$180\degre=\pi\textrm{ [rad]}$$
  $$360\degre=2\cdot{}\pi \textrm{ [rad]}$$

  $$\frac{180\degre}{\pi}  =1 \textrm{ [rad]} \approx{} 57.30\degre$$
  
\end{gesetz}

\newpage
