%% Trigonometrie III
%% Taschenrechner
%% 2020 - 12 - 21 φ@bbw.ch

\section{Taschenrechner}\index{Taschenrechner}

\subsubsection*{Lernziele}

\begin{itemize}
\item Schnittpunkte
\item Graphen im Grad- und Bogenmaß
\end{itemize}

\subsection{Periodische Lösungen einschränken}\index{periodische Lösungen}
Freilich kann mit dem Taschenrechner die Gleichung aus dem
Einstiegsbeispiel $$\cos\left(2x+\frac{\pi}{9}\right)=0.5$$ relativ einfach gelöst werden:

Mit der Eingabe eines Definitionsbereichs (domain) «dom»\footnote{Die Einschränkung auf den Definitionsbereich kann der Taschenrechner rasch ausführen; daher ist dieser hier evtl. bewusst etwas größer gewählt, als im Einstiegsbeispiel.}  ...
$$\mathbb{D} =  [ -\pi , 3\pi]$$
... und der Gleichung «gls» ...
$$\cos\left(2x+\frac{\pi}{9}\right) = 0.5$$

... erhalten wir die selben acht Lösungen.
    
\bbwCenterGraphic{8cm}{tals/trig4/img/nspire1.png}%%

\newpage

\subsection{Referenzaufgabe Tageslänge}\index{Tageslänge}
(Aus Maturitätsprüfung 2019 Serie 1 Teil 2 Aufgabe 6.)

Die Zeitspanne zwischen Sonnenauf\index{Sonnenaufgang}- und Sonnenuntergang\index{Sonnenuntergang} verändert sich im Laufe eines Jahres.
In München kann die Zeitspanne $T(x)$ in Stunden am $x$-ten Tag des Jahres näherungsweise durch die Funktion

$$T(x) = a\cdot{}\cos\left( 2\pi\cdot{}\frac{x-172}{365}\right) + c\textrm{, Cosinusfunktion im Bogenmass}$$

modelliert werden.

Am 81. Tag misst die Zeitspanne 12 Stunden und am 249. Tag 13 Stunden.

a) Berechnen Sie die Koeffizienten $a$ und $c$ der Funktionsgleichung.

b)  Berechnen Sie $x$ so, dass die Zeitspanne 14 Stunden misst.

c)  Erstellen Sie eine qualitative Skizze für 1 Jahr à 365 Tage.

\TNT{14.8}{
  Vorbereitung im Taschenrechner: Definiere $T(x)$:
  $$T(x) := a\cdot{}\cos\left(2\pi\cdot{}\frac{x-172}{365}\right) + c$$

  a) Löse das Gleichungssystem mittels \texttt{solve(gls, \{a, c\})}:

  \gleichungZZ{T(81)}{12}{T(249)}{13}
  
   Dies liefert $a\approx{}4.192$ und $c\approx{}11.98$

  b) Füge dem Gleichungssystem $T(x)=14$ als Gleichung hinzu und schränke den Definitionsbereich mit $|$ ein:
    $$\textrm{\texttt{solve}}(\{T(81)=12;T(249)=13;T(x)=14\},\{a,c,x\})|x>0 \textrm{ AND } x<365$$
    Dies liefert für $x\approx{110.0}$ oder $x=234.1$.

    c) $x$-Achse von 0 bis 365 und $y$-Achse = $T(x)$; hier die vier bekannten Punkte eintragen.
    Lösung s. Maturaprüfung 2019 Serie 1 Teil 2 Aufgabe 6.
}%% END TNT
\newpage

\subsection{Referenzaufgabe alte Maturaprüfung}
(Aus der Maturaprüfung Grundlagenfach 2020 Januar Teil 2 Aufgabe 1.)

Der Graph der Funktion $f: y=3\cdot{}\sin(2x+5)$ ($x$ im Bogenmass)  wird von der Geraden $g$ in den beiden Punkten $A=(-2|...)$ und $B=(2|...)$ geschnitten.

a) Berechnen Sie die $y$-Koordinaten der Schnittpunkte $A$ und $B$.

b)Berechnen Sie den Steigungswinkel der Geraden $g$.


c) $f$ hat im Intervall $[0; \pi]$ die beiden Nullstellen $x_1$ und $x_2$. Berechnen Sie die Funktionsgleichung der Parabel $p$ mit den Nullstellen $x_1$ und $x_2$, die die $y$-Achse bei $y=10$ schneidet. Geben Sie $p$ in der Nullstellenform an.

\TNT{16}{
  Vorbereitung neues «Problem» im Taschenrechner

  $$f(x) := 3\cdot{} \sin(2x+5)$$
  
  \textbf{a)} (Braucht die Gerade noch nicht)
  Berechne (im Bogenmaß) $f(-2) = 3\cdot{}\sin(1) \approx 2.524$ und $f(2) = 3\cdot{} \sin(9) \approx 1.236$

  \textbf{b)} Definiere
  $$g(x) := a\cdot{}x+b$$
  $$gls:=\{ f(-2)=g(-2)  ; f(2) = g(2)\}$$
  $$ \textrm{solve} (gls;\{a, b\})$$

  Dies liefert die Steigung $a$ = 

  $a = \frac{3\cdot{}\sin(9) -\sin(1)}{4} (\approx{} -0.3220)$ (Winkel im Bogenmaß)!

  Somit ist der \textbf{Steigungswinkel} $\varphi = \arctan(a)$:

  rad: $\varphi = \arctan\left(\frac{3\cdot{}(\sin(9)-\sin(1))}4\right) \approx -0.3115 \textrm{ rad }$

  Grad: $\varphi = \arctan\left(\frac{3\cdot{}\left(\sin\left(\frac{9\cdot{}180}{\pi}\right)-\sin\left(\frac{1\cdot{}180}{\pi}\right)\right)}4\right) \approx -17.85\degre$


  \textbf{c)} Die Nullstellen werden typischerweise mit $\textrm{\texttt{solve}} (f(x)=0, x)$ gelöst:

  $$x_1\approx{} 0.6416 \textrm{ und } x_2\approx{} 2.212$$
  Somit lautet die Funktionsgleichung in Nullstellenform:

  $$f(x) = a(x-x_1)\cdot{}(x-x_2)\approx{} a(x-0.6416)\cdot{}(x-2.212)$$
  Das $a$ kann mit folgender Gleichung gefunden werden:
  $$p(x) := a\cdot{}(x-0.6416)\cdot{}(x-2.212); \textrm{\texttt{solve}} (p(0)=10; a); a\approx{}7.045$$
}%% END TNT
\newpage
