%% Trigonometrie III
%% Taschenrechner
%% 2020 - 12 - 21 φ@bbw.ch

\section{Taschenrechner}\index{Taschenrechner}

\subsubsection*{Lernziele}

\begin{itemize}
\item Schnittpunkte
\item Graphen im Grad- und Bogenmaß
\end{itemize}

\subsection{Periodische Lösungen einschränken}\index{periodische Lösungen}
Freilich kann mit dem Taschenrechner die Gleichung aus dem
Einstiegsbeispiel $$\cos\left(2x+\frac{\pi}{9}\right)=0.5$$ relativ einfach gelöst werden:

Mit der Eingabe des Definitionsbereichs (domain) «dom» ...
$$\mathbb{D} =  [ -\pi, 3\pi[$$
... und der Gleichung «gls» ...
$$\cos\left(2x+\frac{\pi}{9}\right) = 0.5$$

... erhalten wir die selben acht Lösungen.
    
\bbwCenterGraphic{8cm}{tals/trig4/img/nspire1.png}%%

\newpage


\subsection{Referenzaufgabe}
(Aus der Maturaprüfung Grundlagenfach 2020 Teil 2)

Der Graph der Funktion $f: y=3\cdot{}\sin(2x+5)$ ($x$ im Bogenmass)  wird von der Geraden $g$ in den beiden Punkten $A=(-2|...)$ und $B=(2|...)$ geschnitten.

a) Berechnen Sie die $y$-Koordinaten der Schnittpunkte $A$ und $B$.

b)Berechnen Sie den Steigungswinkel der Geraden $g$.


c) $f$ hat im Intervall $[0; \pi]$ die beiden Nullstellen $x_1$ und $x_2$2x. Berechnen Sie die Funktionsgleichung der Parabel $p$ mit den Nullstellen $x_1$ und $x_2$, die die $y$-Achse bei $y=10$ schneidet. Geben Sie $p$ in der Nullstellenform an.

\TNT{10}{
a)
  
}
