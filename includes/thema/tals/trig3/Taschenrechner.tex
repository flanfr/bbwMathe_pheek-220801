%% Trigonometrie III
%% Taschenrechner
%% 2020 - 12 - 21 φ@bbw.ch

\section{Taschenrechner}\index{Taschenrechner}

\subsubsection*{Lernziele}

\begin{itemize}
\item Schnittpunkte
\item Graphen im Grad- und Bogenmaß
\end{itemize}

\subsection{Periodische Lösungen einschränken}\index{periodische Lösungen}
Freilich kann mit dem Taschenrechner die Gleichung aus dem
Einstiegsbeispiel $$\cos(2x+\frac{\pi}{9})=0.5$$ relativ einfach gelöst werden:

Mit der Eingabe des Definitionsbereichs (domain) «dom» ...
$$\mathbb{D} =  [ -\pi, 3\pi[$$
... und der Gleichung «gls» ...
$$\cos\left(2x+\frac{\pi}{9}\right) = 0.5$$

... erhalten wir die selben acht Lösungen.
    
\bbwCenterGraphic{8cm}{tals/trig4/img/nspire1.png}%%

\newpage
