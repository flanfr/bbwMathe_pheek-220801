%%
%% 2019 07 04 Ph. G. Freimann
%%
\section{Extremwertaufgaben}\index{Extremwertaufgaben}
\sectuntertitel{Was für einen maximal ist, ist für den andern minimal}
%%%%%%%%%%%%%%%%%%%%%%%%%%%%%%%%%%%%%%%%%%%%%%%%%%%%%%%%%%%%%%%%%%%%%%%%%%%%%%%%%
\theorieTALSGeom{173}{3.5}

\subsection*{Lernziele}


\begin{itemize}
\item Minima und Maxima von Parabeln mittels Scheitelpunkt
\item Taschenrechner 
\end{itemize}

 \cite{frommenwiler17alg}

 Parabeln: S. 192ff (Kap. 3.4.8)

  Und Kap. 3.13. 3 (S. 242ff) ab Aufg. 914.

\newpage
\subsection{Einstiegsbeispiel Parabel}
Die Funktion $f: y = 4x + 3x^2 - 6$ hat ein Extremum (Minimum oder
Maximum). Wo liegt dieses? Handelt es sich um ein Minimum oder ein Maximum?

\noTRAINER{\platzFuerBerechnungen{5.2}}
\TRAINER{
  \begin{enumerate}
  \item Quadratische Funktion in die Grundform $y=ax^2 + bx + c$
    bringen:
    $$y = 3x^2 + 4x -6$$
  \item $x$-Koordinate des Scheitelpunktes bestimmen $x_S =
    \frac{-b}{2a}$:
    $$x_s = \frac{-b}{2a}= \frac{-4}{2\cdot{}3} = -\frac{2}{3}$$
  \item Den zugehörigen Funktionswert $y = f(x)$ durch Einsetzen
    bestimmen:
    $$y_s = 3x_S^2 + 4x_S - 6 = 3\left(-\frac{2}{3}\right)^2 +
    4\left(-\frac{2}{3}\right) - 6 = - \frac{22}{3}$$
  \end{enumerate}
  Es handelt sich um ein Minimum, denn der Parameter $a$ (vor dem
  $x^2$) ist positiv; somit ist die Parabel nach oben geöffnet.
}

\subsection*{Aufgaben}
Ohne Taschenrechner:

\TALSAadB{192ff}{715., 716., 717., 178.*, 720., 721. und 722.}
\newpage


\subsection{Einstigsbeispiel Taschenrechner}
Gegeben ist die Funktion
$$f: y = 0.1 x^3 + 0.07 x^2 - 0.8 x + 2$$

Wir suchen diejenige positive $x$-Position, wo die Funktion am kleinsten ist.

Vorgehen:
\begin{enumerate}
\item Neues «Graph» Fenster mit dem TI-$n$spire CX II-T erstellen
\item $f1()$ definieren : $f1(x) := 0.1 x^3 + 0.07 x^2 - 0.8 x + 2$
\item Funktion zeichnen lassen
\item Menu -> Analyze Graph -> Minimum suchen
\item Untere Schranke bei ca. 0 und obere Schranke bei ca. 2-3 anwählen.
\item Lesen Sie das Resultat bei $x\approx 1.42$ und $y\approx 1.29$ ab.
  \item Mehr Ziffern gefällig? Unter Menu: [9] Einstellungen: Angezeigte Ziffern auf \zB «Fließ 5» umstellen.
\end{enumerate}

\bbwCenterGraphic{6.5cm}{tals/fct2/img/minimumNSpire.png}

\subsubsection{«von Hand»}\index{fMin()!nSpire}\index{nSpire!fMin()}
Das Funktionsminimum kann auch im «notes» berechnet werden. Vorteil: Hier können Zwischenresultate weiterverwendet werden. Nachteil, man sieht von vornherein nicht, wo die Maxima und Minima in etwa sind:

\bbwCenterGraphic{7cm}{tals/fct2/img/fMin.png}

\newpage

\subsection{Blechdose}
In eine Konservendose (Zylinderform) passt genau ein Liter Zwiebelsuppe. Gesucht ist das Verhältnis von Radius $r$ zur Höhe $h$, sodass die Oberfläche (Blechmenge) möglichst klein wird.

Das Volumen $V$ berechnet sich aus

$$V = r^2\pi\cdot{}h$$
Da das Volumen V = 1 (Liter) ist, gilt für die Höhe:
$$h = \frac{1}{r^2\pi}$$
Die Oberfläche $A$ berechnet sich aus Deckel+Boden+Mantel:
$$A=r^2\pi + r^2\pi + 2r\pi\cdot{}h$$
Wenn wir die Höhe $h$ aus dem Volumen hier einsetzen, erhalten wir für die Oberfläche:
$$A=2r\pi\left(r+\frac{1}{r^2\pi}\right)$$

Betrachten Sie $A$ in Abhängigkeit von $r$ und finden Sie mit einem
Computeraglebra-System erst
\begin{itemize}
\item
  den optimalen Radius $r=\LoesungsRaum{0.542}\textrm{dm}$ und dann
\item
  die minimale Oberfläche $A= \LoesungsRaum{5.54}\textrm{dm}^2$.
\end{itemize}

\subsection*{Aufgaben}

\TALSAadB{242}{915. a) b) c) [ohne Wertebereiche], 927., 916., 917., 918., 919., 922., 923., 925., 926., 928., ...}
