\section{Dreidimensionale Objekte}\index{Objekte!dreidimensionale}\index{dreidimensionale Objekte}
\sectuntertitel{Endlich technisch möglich: 3D in Farbe: {\color{red}D}{\color{blue}D}{\color{green}D}} 

\theorieTALSGeom{133}{3}

\subsection*{Lernziele}
\begin{itemize}
\item allgemeiner Zylinder
  \begin{itemize}
  \item Prisma
  \item Kreiszylinder
    \end{itemize}
\item Konus (verallgemeinerter Kegel)
  \begin{itemize}
  \item Pyramide
  \item Kreiskegel
  \end{itemize}
  \item stumpfe Körper
  \item Kugel
\end{itemize}
\newpage
\subsection{Überblick über die betrachteten Körper}\index{Prisma}\index{Kreiszylinder}\index{Pyramide}\index{Kreiskegel}\index{Kugel}\index{Kouns}

\bbwCenterGraphic{16cm}{tals/stereo/img/Ueberblick.png}

Bemerkungen: Auch wenn es beliebige allgemeine Zylider oder Kegel
gibt, betrachten wir nur solche, welche als Grundfläche ein Polygon
(Vieleck) oder einen Kreis aufweisen.

Die folgenden Begriffe werden für die Berechnungen benötigt:

\begin{tabular}{|l|l|p{12cm}|}
  \hline
  Kürzel & Begriff& Bemerkung\\\hline
  $r$  & Radius & (bei Kreiszylinder, Kreiskegel und Kugel)\\\hline
  $G$  & Grundfläche & (nicht vorhhanden bei der Kugel)\\\hline
  $h$  & Höhe & Die Höhe steht senkrecht zur Grundfläche und reicht bis zur Deckfläche oder bis zur Spitze\\\hline
  $D$  & Deckfläche & verschwindet beim Konus\\\hline
  $M$  & Mantelfläche & Summe aller Seitenflächen ohne Grund- und Deckfläche\\\hline
  $S$  & Oberfläche & Summe von Grund-, Deck- und Mantelfläche\\\hline
  $V$  & Volumen & \\\hline
  \end{tabular} 

\newpage
\subsection{Allgemeiner Zylinder (Prisma und Zylinder)}\index{Zylinder}\index{Prisma}
Spezialfälle: Quader und Würfel

\begin{gesetz}{Volumen Zylinder (inkl. Prisma)}{}\index{Volumen!Zylinder ink. Prisma}

  $V$ = Volumen\\
  $G$ = Grundfläche\\
  $h$ = Höhe\\
  \begin{center}\fbox{$V = G\cdot{}h$}\end{center}
\end{gesetz}

\begin{gesetz}{Oberfläche Zylinder (inkl. Prisma)}{}\index{Oberfläche!Zylinder inkl. Prisma}

  $S$ = Oberfläche (manchmal als $O$ bezeichnet)\\
  $G$ = Grundfläche\\
  $D$ = Deckfläche (Es gilt: $D=G$)\\
  $h$ = Höhe\\
  $M$ = Mantelfläche = $h \cdot{} \textrm{Umfang der Grundfläche}$\\
  \begin{center}\fbox{$S = G + D + M  = 2\cdot{}G + M$}\end{center}
\end{gesetz}

Prisma: \theorieTALS{141}{3.2.1}\\
Kreiszylinder: \theorieTALS{158}{3.3.1}

\subsection*{Aufgaben}
Prisma:

\TALSAadB{142ff}{32., 33., 57. und 60.}

Kreiszylinder:
\TALSAadB{158ff}{136}
\newpage
\subsection{Quader und Würfel}
Die beiden speziellsten Prismen sind der Quader und der Würfel.

Für ein Quader mit den Seitenlängen $a$, $b$ und $c$ gilt:

\begin{gesetz}{}{}\\
  Volumen $V = G\cdot{} h = a\cdot{}b\cdot{}c$\\
  Oberfläche $S=2\cdot{}G + M = 2\cdot{}ab + ac + bc + ac + bc = 2\cdot{}(ab + ac + bc)$\\
  Raumdiagonale $d = \sqrt{a^2 + b^2 + c^2}$
\end{gesetz}

Im Spezialfall «Würfel» gilt $a=b=c$:
\begin{gesetz}{}{}\\
  Volumen $V = G\cdot{} h = a^3$\\
  Oberfläche $S=2\cdot{}G + M = 6\cdot{}a^2$\\
  Raumdiagonale $d = \sqrt{3\cdot{}a^2} = a\cdot{}\sqrt{3}$
\end{gesetz}


\newpage

\subsection{Konus (Pyramide und Kreiskegel)}\index{Kegel}\index{Kreiskegel}\index{Pyramide}\index{Konus}

\TRAINER{Einstiegsvideo: Beweis der 1/3 Formel auf Youtube (s. Wiki).}


\begin{gesetz}{Volumen Konus (Pyramide und Kreiskegel)}{}
  
  $V$ = Volumen
  
  $G$ = Grundfläche

  $h$ = Höhe
  $$V = \frac13\cdot{}G\cdot{}h$$
\end{gesetz}

\begin{gesetz}{Oberfläche}{}

  $S$ = Oberfläche

  $G$ = Grundfläche

  $M$ = Mantelfläche

  $$S = G + M$$
  \end{gesetz}
\newpage

\subsubsection{Spezialfall: Pyramide}\index{Pyramide}

\bbwCenterGraphic{10cm}{tals/stereo/img/Pharao.png}

\begin{center}
{\textit{«Keine Fenster? Keine sanitären Anlagen? Kein W-LAN? Ist ja schon
jetzt eine Ruine. Da kann ich mich gleich darin begraben
lassen.»}}\footnote{Die rund 100m hohe \textbf{rote Pyramide} ist das
  erste in echter Pyramidenform erbaute Wahrzeichen im «Alten
  Ägyptischen Reich». Sie wurde um 2500 v.\,u.\,Z. von König Snofru in
Dahschur erbaut.}
\end{center}



\begin{bemerkung}{Höhe}{}
  Wir unterscheiden

  $h$ = Höhe der Pyramide

  $h_s$ = Höhe einer Seitenfläche (Dreieckshöhe)
  \end{bemerkung}


\subsection*{Aufgaben}
Pyramide:
\TALSAadB{146ff}{86.}
\newpage

\subsubsection{Spezialfall: Kreiskegel}\index{Kreiskegel}

\begin{gesetz}{Oberfläche eines Kreiskegels}{}

  $S$ = Oberfläche (Surface)

  $r$ = Radius der Grundfläche
  
  $G$ = Grundfläche = $r^2\pi$

  $h$ = Höhe

  $m$ = Mantellinie = $\sqrt{h^2 + r^2}$
  
  $M$ = Mantelfläche = $\frac{2r\pi \cdot{} m}2$
  
  $$S = G + M$$
\end{gesetz}

\begin{bemerkung}{}{}
  Wir unterscheiden beim Kreiskegel die Winkel

  $\alpha$ = Öffnungswinkel des Kegels an der Spitze

  $\beta$ = Böschungswinkel (Neigung der Mantellinie = $90\degre - \frac{\alpha}{2}$

  $\varphi$ = Sektorwinkel des abgewickelten Mantels = $360\degre \cdot \sin\left(\frac{\alpha}2\right)$

\end{bemerkung}


\subsection*{Aufgaben}

Kreiskegel:
\TALSAadB{158ff}{216.}
\newpage

