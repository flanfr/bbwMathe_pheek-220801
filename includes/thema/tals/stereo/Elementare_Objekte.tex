\section{Dreidimensionale Objekte}\index{Objekte!dreidimensionale}\index{dreidimensionale Objekte}
\sectuntertitel{Endlich technisch möglich: 3D in Farbe: {\color{red}D}{\color{blue}D}{\color{green}D}} 

\theorieTALSGeom{133}{3}

\subsection*{Lernziele}
\begin{itemize}
\item Zylinder
  \begin{itemize}
  \item Prisma
  \item Kreiszylinder
    \end{itemize}
\item Kegel
  \begin{itemize}
  \item Pyramide
  \item Kreiskegel
  \end{itemize}
  \item stumpfe Körper
  \item Kugel
\end{itemize}
\newpage

\subsection{Prisma und Zylinder}
Spezialfälle: Quader und Würfel

\begin{gesetz}{Volumen Zylinder (inkl. Prisma)}{}
  
  $V$ = Volumen
  
  $G$ = Grundfläche

  $h$ = Höhe
  $$V = G\cdot{}h$$
  \end{gesetz}

\theorieTALS{141}{????}

\subsection*{Aufgaben}
Prisma:

\TALSAadB{142ff}{32., 33., 34., 57. und 60.}

Kreiszylinder:
\TALSAadB{158ff}{136}
\newpage

\subsection{Pyramide und Kegel}
\bbwCenterGraphic{10cm}{tals/stereo/img/Pharao.png}

\begin{center}
{\textit{«Keine Fenster? Keine sanitären Anlagen? Kein W-LAN? Ist ja schon
jetzt eine Ruine. Da kann ich mich gleich darin begraben
lassen.»}}\footnote{Die rund 100m hohe \textbf{rote Pyramide} ist das
  erste in echter Pyramidenform erbaute Wahrzeichen im «Alten
  Ägyptischen Reich». Sie wurde um 2500 v.\,u.\,Z. von König Snofru in
Dahschur erbaut.}
\end{center}


\begin{gesetz}{Volumen Kegel (inkl. Pyramide)}{}
\TRAINER{Einstiegsvideo: Beweis der 1/3 Formel auf Youtube (s. Wiki).}
  
  $V$ = Volumen
  
  $G$ = Grundfläche

  $h$ = Höhe
  $$V = \frac13\cdot{}G\cdot{}h$$
  \end{gesetz}

\subsection*{Aufgaben}
Pyramide:
\TALSAadB{146ff}{80. und 86.}
Kegel:
\TALSAadB{158ff}{216.}
\newpage

\subsection{Pyramiden- und Kegelstumpf}
\TRAINER{Optional?}
coming soon...
\subsection*{Aufgaben}
\TALSAadB{146ff}{105}
\newpage
