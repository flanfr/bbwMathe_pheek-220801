\section{Konus (Pyramide und Kreiskegel)}\index{Kegel}\index{Kreiskegel}\index{Pyramide}\index{Konus}

\TRAINER{Einstiegsvideo: Beweis der 1/3 Formel auf Youtube (s. Wiki).}

\bbwCenterGraphic{5cm}{tals/stereo/img/AllgemeinerKonus.png}

\begin{gesetz}{Volumen Konus (Pyramide und Kreiskegel)}{}
  
  $V$ = Volumen
  
  $G$ = Grundfläche

  $h$ = Höhe
  $$V = \frac13\cdot{}G\cdot{}h$$
\end{gesetz}

\begin{gesetz}{Oberfläche}{}

  $S$ = Oberfläche

  $G$ = Grundfläche

  $M$ = Mantelfläche

  $$S = G + M$$
  \end{gesetz}
\newpage

\subsection{Spezialfall: Pyramide}\index{Pyramide}

\bbwCenterGraphic{10cm}{tals/stereo/img/Pharao.png}

\begin{center}
{\textit{«Keine Fenster? Keine sanitären Anlagen? Kein W-LAN? Ist ja schon
jetzt eine Ruine. Da kann ich mich gleich darin begraben
lassen.»}}\footnote{Die rund 100m hohe \textbf{rote Pyramide} ist das
  erste in echter Pyramidenform erbaute Wahrzeichen im «Alten
  Ägyptischen Reich». Sie wurde um 2500 v.\,u.\,Z. von König Snofru in
Dahschur erbaut.}
\end{center}

\bbwCenterGraphic{5cm}{tals/stereo/img/Pyramide.png}

\begin{bemerkung}{Höhe}{}
  Wir unterscheiden

  $h$ = Höhe der Pyramide

  $h_s$ = Höhe einer Seitenfläche (Dreieckshöhe)
\end{bemerkung}
\begin{gesetz}{}{}
  Es gilt
  $$h_S \ge h$$
  \end{gesetz}
\newpage

\subsection*{Aufgaben}
Pyramide:
\TALSAadB{146ff}{86. (Trigonometrie)}
\newpage

\subsection{Spezialfall: Kreiskegel}\index{Kreiskegel}

\bbwCenterGraphic{5cm}{tals/stereo/img/Kreiskegel.png}

\begin{gesetz}{Oberfläche eines Kreiskegels}{}

  $S$ = Oberfläche (Surface)

  $r$ = Radius der Grundfläche
  
  $G$ = Grundfläche = $r^2\pi$

  $h$ = Höhe

  $m$ = Mantellinie = $\sqrt{h^2 + r^2}$
  
  $M$ = Mantelfläche = $\frac{2r\pi \cdot{} m}2 = r\pi\cdot{}m$

  
  $$S = G + M$$
\end{gesetz}


\begin{gesetz}{Volumen eines Kreiskegels}{}

  $V$ = Volumen

  $r$ = Radius der Grundfläche
  
  $G$ = Grundfläche = $r^2\pi$

  $h$ = Höhe

  $$V = \frac13\cdot{} G\cdot{}h = \frac13 \cdot{} \pi r^2 \cdot{} h$$
\end{gesetz}



\begin{bemerkung}{}{}
  Wir unterscheiden beim Kreiskegel die Winkel

  $\alpha$ = Öffnungswinkel des Kegels an der Spitze

  $\beta$ = Böschungswinkel (Neigung der Mantellinie = $90\degre - \frac{\alpha}{2}$

  $\varphi$ = Sektorwinkel des abgewickelten Mantels = $360\degre \cdot \sin\left(\frac{\alpha}2\right)$

\end{bemerkung}
\newpage

\subsection*{Aufgaben}


\TALSAadB{158ff}{157. (Netzabw.), 158. (Pyramide in Kegel), 163. (zwei Kegel im Zylinder)}
\newpage

