\section{Schiefe und stumpfe Körper}\index{schief!Körper}\index{stumpf!Körper}

\theorieTALSGeom{??}{??}

\subsection*{Lernziele}
\begin{itemize}
\item Prinzip von Cavalieri
\item Stumpf = Ersatzkörper minus Spitze
\end{itemize}

\bbwCenterGraphic{8cm}{tals/stereo/img/SchiefUndStumpf.png}

\newpage


\subsection{Schiefe Körper: Cavalieri}\index{schiefe Körper}\index{Cavalieri}
Bonaventura Francesco Cavalieri (1598 – 1647)

Das \textbf{Prinzip von Cavalieri} sagt aus, dass zwei Körper das
selbe Volumen haben, genau dann, wenn sie in jeder horizontalen
Schnittebene die selbe Fläche aufweisen.

\bbwCenterGraphic{18cm}{tals/stereo/img/Cavalieri.png}
Bildquelle: \texttt{www.slideshaer.net} (Mai 2021)
\vspace{9mm}

\begin{gesetz}{schiefe Körper}{}
  Schiefe Körper haben das selbe Volumen, wie der entsprechende gerade Körper, dabei wird die Höhe $h$ auch als Abstand zwischen Grund- und Deckfläche verwendet.

  $$V = G\cdot{} h$$
\end{gesetz}

\newpage


\subsection{Kreiskegelstumpf / Pyramidenstumpf}
\TRAINER{Optional?}
coming soon...


\subsection*{Aufgaben}
\TALSAadB{146ff}{105}
\newpage



