\section{Ähnlichkeit im Raum}\index{Ähnlichkeit!im Raum}
\sectuntertitel{}

\theorieTALSGeom{171}{3.4}

\subsection*{Lernziele}
\begin{itemize}
\item Streckungsfaktor (analog 2d)
\end{itemize}


\subsection{Streckungsfaktor}

Sei $K$ der Originalkörper und $K'$ der gestreckte, ähnliche Körper, dann gilt
für die Volumen ($V$ und $V'$), die Flächen ($A$ und $A'$) und
die Strecken ($a$ und $a'$) folgendes Gesetz:

\begin{gesetz}{}{}
  Der (lineare) Streckungsfaktor sei $k$.

  $$a \cdot{} k^1 = a'$$
  $$A \cdot{} k^2 = A'$$
  $$V \cdot{} k^3 = V'$$
\end{gesetz}

\begin{bemerkung}{Volumina}{}
Der Exponent beim Streckungsfaktor $k$ entspricht der Dimension der
betrachteten «Volumina». Dabei ist das \textbf{zweidimensionale Volumen} die
Fläche während die Streckenlänge das \textbf{eindimensionale Volumen} darstellt.
\end{bemerkung}

\begin{bemerkung}{Verhältnisse}{}
  $$k = \frac{a'}{a} = \sqrt{\frac{A'}{A}} = \sqrt[3\,\,\,]{\frac{V'}{V}}$$
\end{bemerkung}

  
\newpage
