%% 2020 - 12 - 25 ph. g. freimann @bbw.ch
%% Vektoren im Zweidimensionalen

\section{Zweidimensionale Vektoren}\index{Vektoren!zweidimensionale}\index{zweidimensionale Vektoren}
\sectuntertitel{Kommt ein Nullvektor zum Psychiater: ``... ich bin so orientierungslos!''}

\theorieTALSGeom{175}{4}

\subsection*{Lernziele}
\begin{itemize}
\item Komponenten / Zerlegung
\item elementare Operationen
\item Addition, Subtraktion (Gegenvektor)
\end{itemize}
\newpage

\subsection{Einführung}

Betrachten Sie die beiden folgenden (freien) Vektoren ${\color{blue} \vec{a}}$ und
${\color{red}\vec{b}}$, welche beide durch mehrere Repräsentanten\index{Repräsentant!Vector}
(Vertreter\index{Vertreter!Vector}) eingezeichnet sind:

\bbwGraph{-4}{7}{-3}{3}{
\bbwLetter{3.5,3}{\vec{a}}{blue}
\draw [->,blue] (1,1) -- (4,2);
\draw [->,blue] (2,2) -- (5,3);
\draw [->,blue] (-3.5,-1) -- (-0.5,0);
\draw [->,blue] (2,0.5) -- (5,1.5);
\bbwLetter{-1,3}{\vec{b}}{red}
\draw [->,red] (-1,1) --(-2,3);
\draw [->,red] (-1,-2) --(-2,0);
\draw [->,red] (2.5,-0.5) --(1.5,1.5);
\draw [->,red] (5,-3) --(4,-1);
\draw [->,red] (7,0.5) --(6,2.5);
}%% END bbwGraph

Tragen Sie die fehlenden Werte in die Tabelle ein\footnote{Der
  mathematisch positive Winkel wird ab der $x$-Achse im
  Gegenuhrzeigersinn gemessen.}:

\begin{tabular}{|c|c|c|}\hline
                 & ${\color{blue}\vec{a}}$   & ${\color{red}\vec{b}}$   \\\hline
  $x$-Komponente & \LoesungsRaumLang{3}      & \LoesungsRaumLang{-1}    \\\hline
  $y$-Komponente & \TRAINER{1}               & \TRAINER{2}              \\\hline
  Betrag\index{Betrag!eines Vektors} (=Länge) & \TRAINER{$\sqrt{10}$}     & \TRAINER{$\sqrt{5}$}     \\\hline
  math. pos. Winkel  & \TRAINER{$\arctan{}\left(\frac13\right)\approx
    18.43\degre$} & \TRAINER{$90\degre +
    \arctan{}\left(\frac12\right)\approx 116.6\degre$}               \\\hline
\end{tabular}
\platzFuerBerechnungen{3.2}
\newpage
\subsection{Vektorbegriff}
\begin{definition}{Vektor}{}
  Ein \textbf{Vektor} besteht aus einer Länge und einer Richtung.
\end{definition}

\subsubsection{Notation}\index{Notation!Vector}\index{Vector!Notation}

Ein Vektor wird entweder mit einem Pfeil über dem Buchstaben
${\color{blue}\vec{a}}$ oder aber mit einem Pfeil über der zwei
Punkebezeichnungen ${\color{red}\overrightarrow{{PQ}}}$ angegeben:

\bbwGraph{-3}{6}{-1}{3}{
\bbwLetter{2,2}{\vec{a}}{blue}
\draw [->,blue] (1,1) -- (4,2);

\bbwLetter{-1.5,1}{P}{red}
\bbwLetter{-2.5,3}{Q}{red}
\draw [->,red] (-1,1) --(-2,3);
}%% END bbwGraph
\newpage

\subsubsection{Betrag (Länge) von Vektoren}
\theorieTALSGeom{176}{4.1}
Die Länge der Vectoren wird mittels «Pythagoras» berechnet

Sei $\vec{a} = \begin{pmatrix} 3\\ 1\end{pmatrix}
  = \begin{pmatrix}x_a\\y_a\end{pmatrix}$. Somit ist die Länge von
    $\vec{a}$ wie folgt zu berechnen:

    \begin{gesetz}{Betrag (Länge)}{}
      Länge von $\vec{a}$ =

      $$|\vec{a}| = \sqrt{x_a^2 + y_a^2}$$
      \end{gesetz}
    Notationen:

    \begin{definition}{Länge}{}
      Ein Vector $\vec{a}$ von Punkt $A$ nach $B$ hat den Betrag (= Länge)

      $$a = |\vec{a}| = \overline{AB} = \left|\overrightarrow{AB}\right|$$
    \end{definition}

    \TALSGeomAadB{178}{2., 5., 6.}
\newpage
    
\subsection{Elementare Vektoroperationen}
\theorieTALSGeom{178}{4.2}

\subsubsection{Addition}

\subsubsection{Subtraktion}
\newpage

\subsubsection{Skalarmultiplikation}
\theorieTALSGeom{178}{4.2}
\begin{definition}{Skalar}{}
  Vektoren haben eine Länge und eine Richtung. Eine Größe ohne
  Richtung aber mit einem Betrag nennen wir \textbf{Skalar}. 
\end{definition}
\newpage
