%% TALS Polarkoordinaten
%% 2020 - 12 - 25 ph. g. freimann @bbw.ch

\section{Polarkoordinaten}\index{Polarkoordinaten}
\sectuntertitel{Opposite of a polarbear? A rectangular bear!}

Anstelle der $x$- und der $y$-Komponente können wir mit dem selben
Informtaionsgehalt auch den Winkel
(in mathematisch positiver Richtung) und die Länge eines Vektors
angeben.

Dieser Winkel wird im mathematisch positiven Sinne ab der $x$-Achse
angegeben. Der Vector
$\begin{pmatrix}0\\1\end{pmatrix}$ hat somit den Winkel $90\degre$.

Rechnen Sie um:

... coming soon ...


\subsection{Polarkoordinaten im Taschenrechner}

... coming soon ...
