%%
%% 2019 07 04 Ph. G. Freimann
%%

\subsection{Grad- und Bogenmaß}\index{Gradmaß}\index{Bogenmaß}
\theorieTALSGeom{46}{1.3.2}

Das Bogenmaß ist eine alternative Einteilung des Kreises zum
klassischen $360\degre$-Gradmaß. Dabei wird der volle Kreis nicht in
$360\degre$ Grad sondern im Verhältnis zum Radius dargestellt. Mit
anderen Worten: Bei gegebenem Winkel $\alpha$ ist das Bogenmaß nichts
anderes als die Länge des Bogens zum Winkel $\alpha$ gemessen im Einheitskreis\index{Einheitskreis}.

\bbwGraphic{5cm}{tals/trig2/img/bogenmass.png}

Es gilt:
$$360\degre \entspricht 2\cdot\pi \textrm{\,\,rad}$$

Der Winkel im Bogenmaß wird nicht in der Maßeinheit Grad (${}\degre$)
sondern in $\textrm{rad}$ angegeben. Zum Beispiel entspricht
$\frac{3}{4}\pi\,\textrm{rad}$ unseren bekannten $135\degre$ in der
(bereits sumerischen/babylonischen) $360\degre$-Einteilung.

\subsection*{Aufgaben}
\TALSGeomAadB{46}{175. a) b) d) 176. a) f) i)}
