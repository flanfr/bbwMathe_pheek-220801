%% 2020 12 25 ph. g. Freimann
%%
\section{Operationen ausführen}\index{Operationen!auf Vektoren}\index{Vektoroperationen}

\subsection*{Lernziele}
\begin{itemize}
\item Addition, Subtraktion
\item Skalarmultiplikation
\item Skalarprodukt
\end{itemize}


Coming soon...
\newpage


\subsection{Skalarprodukt}

\theorieTALSGeom{201}{4.5}

\bbwCenterGraphic{5cm}{tals/vecg2/img/skalarprodukt.png}

Herleitung:

\TNT{12}{
  $\vec{b} + \vec{c} = \vec{a}$

  $\Longrightarrow   \vec{c} = \vec{a} - \vec{b}$

  $\Longrightarrow   c_x = a_x - b_x \textrm{ und } c_y = a_y - b_y$

  $a = |\vec{a}|, b=|\vec{b}| \textrm{ und } c = |\vec{c}|$

  Pythagoras:
  $c = \sqrt{c_x^2 + c_y^2}$ und somit $c^2 = c_x^2 + c_y^2$

  $a_x-b_x$ anstelle von $c_x$ einsetzen (analog $y$):

  $c^2 = (a_x-b_x)^2 + (a_y - b_y)^2$

  Wir erinnern an den Cosinussatz: $c^2 = a^2 + b^2 -
  2ab\cdot{}\cos(\gamma)$

  Setzen wir obiges $c^2$ ein, so erhalten wir:

  $(a_x-b_x)^2 + (a_y-b_y)^2 = \overbrace{a_x^2 + a_y^2}^{|\vec{a}|^2=a^2}
  + \overbrace{b_x^2 + b_y^2}^{b^2} - 2ab\cdot{}\cos(\gamma)$

  links Ausmultiplizieren
  
  $a_x^2 - 2a_xb_x +b_x^2 + a_y^2 - 2a_yb_y + b_y^2 = a_x^2 + a_y^2 +
  b_x^2 + b_y^2 - 2ab\cdot{}\cos(\gamma)$

  nun beginnte der Spaß mit dem Wegstreichen:

  $-2a_xb_x - 2a_yb_y = -2ab\cdot{}\cos(\gamma)$
}

\begin{definition}{Skalarprodukt}{}
  $$\vec{a\vphantom{b}}\circ\vec{b} := a_x\cdot{}b_x + a_y \cdot{} b_y$$
\end{definition}

\newpage


\begin{gesetz}{Skalarprodukt}{}
  $$\vec{a\vphantom{b}} \circ \vec{b} = |\vec{a\vphantom{b}}| \cdot{} |\vec{b}| \cdot{} \cos(\gamma)$$
bzw. ($a=|\vec{a}|$)
  $$\vec{a\vphantom{b}} \circ \vec{b} = a \cdot{} b \cos(\gamma)$$
\end{gesetz}

\begin{gesetz}{Skalarprodukt}{}
  $$\cos(\gamma) = \frac{\vec{a\vphantom{b}}\circ\vec{b}}{a\cdot b}$$
  bzw.:
  $$\gamma = \arccos{} \left(\frac{\vec{a\vphantom{b}}\circ\vec{b}}{a\cdot b}\right)$$
  
\end{gesetz}

\begin{bemerkung}{}{}
  Genau dann, wenn zwei Vektoren $\vec{a}$ und $\vec{b}$ senkrecht
  zueinander stehen, ist das Skalarprodukt null.

  $$\gamma=90\degre   \Longleftrightarrow  \vec{a\vphantom{b}}\circ\vec{b}=0$$
  \end{bemerkung}


\subsection*{Aufgaben}
\TALSGeomAadB{202}{102. a) d) e), 103. a) c), 104. a) c), 106. a),
  107. a), 108., 109., 110. a), 113.}

\newpage
