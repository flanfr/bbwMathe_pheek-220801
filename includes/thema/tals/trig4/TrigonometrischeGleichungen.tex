%% Trigonometrie IV
%% Trigonometrische Gleichungen
%% 2020 - 12 - 21 φ@bbw.ch

\section{Trigonometrische Gleichungen}\index{trigonometrische Gleichungen}\index{Gleichungen!trigonometrische}

\subsection{Trigonometrische Beziehungen}
Es gelten diverse Beziehungen zwischen $\sin$, $\cos$ und $\tan$. Die
wichtigste kennen wir bereits aus von den Definitionen:

\begin{gesetz}{Tangens}{}
  Für alle Winkel $\varphi$, deren $\cos$ nicht Null ergibt, gilt:

  $$\tan(\varphi) = \frac{\sin(\varphi)}{\cos(\varphi)}$$
  \end{gesetz}
Zur Herleitung dient einfach die Definition des Tangens mit
anschließendem Erweitern:

$$\tan(\varphi) = \frac{\textrm{Gegenkathete von
  }\varphi}{\textrm{Ankathete von }\varphi} =
\frac{\frac{\textrm{Gegenkathete von }
    \varphi}{\textrm{Hypotenuse}}}{\frac{\textrm{Ankathete von }\varphi}{\textrm{Hypotenuse}}}
=\frac{\sin(\varphi)}{\cos(\varphi)}$$
\newpage


Ebenso kann mit dem Satz des Pythagoras im Einheitskreis leicht
gezeigt werden, dass gilt:

\begin{gesetz}{}{}
  $$\sin(\varphi)\cdot \sin(\varphi) + \cos(\varphi) \cdot
  \cos(\varphi) = 1$$
\end{gesetz}

\begin{bemerkung}{}{}
  Es sind die folgenden Notationen üblich:
  $$\sin^2(\varphi) = \sin(\varphi)\cdot\sin(\varphi)$$
  $$\cos^2(\varphi) = \cos(\varphi)\cdot\cos(\varphi)$$
\end{bemerkung}
Daher merken wir uns:
\begin{bemerkung}{}{}
  $\sin^2+\cos^2 = 1$
\end{bemerkung}

\begin{bemerkung}{}{}

  \begin{tabular}{p{6cm}p{2cm}p{6cm}}
    \fbox{$\sin^2(\varphi) = \sin(\varphi)\cdot\sin(\varphi)$} & Aber: & \fbox{$\sin(\varphi^2) = \sin(\varphi\cdot\varphi)$}\\
    \end{tabular}
  
\end{bemerkung}


Des weiteren gelten die folgenden Beziehungen:
\begin{gesetz}{Sinus}{}
  $$sin(\alpha) = - sin(-\alpha)$$
  \end{gesetz}
\begin{gesetz}{Cosinus}{}
  $$cos(\alpha) = cos(-\alpha)$$
  \end{gesetz}
\begin{gesetz}{Tangens}{}
  Für $\cos(\alpha)\ne 0$ gilt:
  $$tan(\alpha) = tan(\alpha + z\cdot{}180\degre) \textrm{ für alle } z\in\mathbb{Z}$$
  \end{gesetz}
\newpage



\subsection*{Aufgaben}
\aufgabenfarbe{Strukturaufgaben S. 3 Aufg 3 $k$-$x$}
\TALSGeomAadB{101ff}{Diesmal an den Winkelfunktionen (Skizzen): 85. [$\sin$] a) c) e) f) g) und
  h), 86. [$\cos$] b) c) e) und h) 87. [$\tan$] a) und c)}
\TALSGeomAadB{124}{182., 183., 185., 189.}
\TALSAadB{234ff}{885.ff}
