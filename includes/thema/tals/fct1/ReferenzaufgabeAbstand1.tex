\subsubsection{Referenzaufgabe: Abstand}\index{Abstand!zwischen Punkt und Gerade}

Gegeben ist ein Punkt $P=(2|3)$ und eine Gerade $f: y=\frac{1}{4}x-1$. Gesucht ist der Abstand des Punktes $P$ zu $f$.

\newcommand{\vorgehensTitelchen}[1]{\textbf{\color{brown}#1
\\
}}

Wie ist der Abstand zwischen einem Punkt und einer Geraden wohl definiert?

\vorgehensTitelchen{1. Skizze}
\bbwGraph{-1}{6}{-2}{4}{
 \bbwDot{2,3}{blue}{west}{P}
      \bbwLine{-1,-1.25}{5,0.25}{green}
    \TRAINER{
    \bbwLine{2,3}{3,-1}{red}}
  }
Suchen Sie in obiger Skizze den Abstand und messen Sie ihn.


\vorgehensTitelchen{2. Senkrechte ($g\perp f$)}

\TNT{1.2}{
  $g: y = ax + b$ mit $a = -4$ (negativer Kehrwert aus $f$). Ergo:
  $g: y = -4x + b$
}

\vorgehensTitelchen{3. $g$ durch $P$ ($P\in g$)}

\TNT{2.8}{Kernidee: Punkt in Funktionsgleichung einsetzen:

  $P=(2|3)$ in $y=-4x+b$ einsetzen: $3 = -4\cdot(2) + b$

  Daraus errechnen wir $b = 11$. Und es folgt
$$g: y= -4x+11$$}

\newcommand\mussgleich{\mathrel{\stackrel{\makebox[0pt]{\mbox{\normalfont{\tiny{!}}}}}{=}}}

\vorgehensTitelchen{4. Schnittpunkt S ($S := g\cap f$)}

\TNT{3.2}{Idee: Beide Funktionsterme gleichsetzen, denn $S(x_s|y_s)$ muss auf beiden Geraden liegen:
  $$g(x_s) = y_s = f(x_s)$$
  $$-4x_s + 11 \mussgleich{} \frac{1}{4}x_s - 1$$
  ausrechnen lassen
$\Rightarrow x_s = \frac{48}{17}\approx 2.82 \Rightarrow y_s=-\frac{5}{17}\approx -0.294 \Rightarrow S(\frac{48}{17}|\frac{-5}{17}) $}
\noTRAINER{\newpage}

\vorgehensTitelchen{5. Abstand mit Pythagoras ($\left|\overline{PS}\right|$)}

\TNT{3.6}{$P=(2|3), S(\frac{48}{17}|\frac{-5}{17}) \Rightarrow \sqrt{(2-\frac{48}{17})^2 + (3-\frac{-5}{17})^2}=\frac{4\sqrt{17}}{17}\approx 3.3955 $}
\newpage

\TALS{
\subsubsection{TI nSpire}
Wie sieht das mit dem CAS-fähigen Taschenrechner TI nSpire aus? Hier
ein Beispiel einer Geraden $f1: y=\frac{4}{3}-2$ und dem Punkt
$P=(-1|5)$.

\leserluft{}

\bbwCenterGraphic{17cm}{allg/funktionen/img/AbstandGeradeZuPunkt.png}
\newpage
}%% END TALS
