%%
%% 2019 07 04 Ph. G. Freimann
%%

\section{Potenz- und Wurzelgleichungen\TALS{ ...}}\index{Gleichungen!mit Potenzen}\index{Potenzgleichungen}\index{Wurzelgleichungen}\index{Gleichungen!mit Wurzeln}

\theorieGESO{191}{11}
%%\theorieTALS{}{}
%%%%%%%%%%%%%%%%%%%%%%%%%%%%%%%%%%%%%%%%%%%%%%%%%%%%%%%%%%%%%%%%%%%%%%%%%%%%%%%%%
\subsection*{Lernziele}

\begin{itemize}
\item Potenzgleichungen
\item Wurzelgleichungen
\end{itemize}
\newpage

%%
%% 2020 05 07 Ph. G. Freimann
%%

%% Überblick über die Begriffe
%% Potenz, Potenzgleichung, Exponentialgleichung, Wurzelgleichung

\subsection{Überblick über Gleichungen mit Potenzen}

\begin{tabular}{|p{52mm}|p{52mm}|p{52mm}|}
  \hline
  Potenz gesucht           & Basis gesucht                       &  Exponent gesucht          \\
  \hline
  $10^3=x$                 & $x^3=1000$                           &  $10^x=1000$               \\
  \hline
  $x=10\cdot{}10\cdot{}10$ & $x=\sqrt[3]{1000}$                   & $x =  \log_{10}(1000)$     \\
  $x=1000$                 & $x=10$                               & $x =  3$                   \\
  \hline
  \multicolumn{3}{c}{\,}\\ %% Generiere etwas Abstand
  \multicolumn{3}{c}{Erinnerung --- Wurzeln sind rationale Exponenten:}\\
  \hline
  $\sqrt[3]{1000}=x$        & $\sqrt[3]{x}=10$                    &  $\sqrt[x]{1000}=10$               \\
  \hline
  $1000^{\frac{1}{3}}=x$     & $x^{\frac{1}{3}}=10$                   &  $1000^{\frac{1}{x}}=10$               \\
  $x=\sqrt[3]{1000}$       & $x=10^3$                             & $\frac{1}{x} =  \log_{1000}(10)$      \\
  $x=10$                   & $x=1000$                             & $\frac{1}{x} =  \frac{1}{3}$         \\
                           &                                      & $x = 3$                      \\
  \hline
  \multicolumn{3}{c}{\,}\\ %% Generiere etwas Abstand
  \multicolumn{3}{c}{Beispiel Zweierpotenzen}\\
  \hline
  Potenz                   & Potenzgleichung bzw. Wurzelgleichung &  Exponentialgleichung     \\
  \hline 
  $2^5=x$                  & $x^5=32$                             &  $2^x=32$                  \\
  \hline
  $x=2\cdot{}2\cdot{}2\cdot{}2\cdot{2}$ & $x=\sqrt[5]{32}$        & $x =  \log_{2}(32)$        \\
  $x=32$                   & $x=2$                                & $x  =  3$                  \\
  \hline
\end{tabular}

\GESO{Bem. zum Logarithmus zur Basis 2 ($\log_{2}(32)$): Dazu müssen
  Sie die

  \tiprobutton{ln_log}-Taste
 auf Ihrem TI-30 Taschenrechner 3x drücken.}

\newpage


\subsection{Potenzgleichung}\index{Potenzgleichungen}
Eine Gleichung, bei der die gesuchte Größe $x$ als Basis in einer Potenz
vorkommt, nennen wir
\textbf{Potenzgleichung}\footnote{Potenzgleichungen sind nicht zu
  verwechseln mit Exponentialgleichungen, bei denen das $x$ im
  Exponenten steht: $5^{2x-1}=7$}. Im folgenden Beispiel kommt $x$ in
der 5-ten Potenz vor:

$$x^5 = 1024$$

Typischerweise löst man diese Gleichungen mit der $n$-ten Wurzel.

\begin{tabular}{rccl}
  \             & $x^5$           &=& 1024             \\
  $\Rightarrow$ & $\sqrt[5]{x^5}$ &=& $\sqrt[5]{1024}$ \\
  $\Rightarrow$ & $x$             &=& $4$ 
\end{tabular}

Was ist die Lösungsmenge der folgenden Potenzgleichung?

$$x^2 = 100$$

\TNT{2.4}{%%
$$x^2 = 100 \textrm{ somit } x=\pm\sqrt{100} = \pm 10.$$}%% END TNT

Bestimmen Sie auch die Lösungsmenge der folgenden Gleichung:
$$x^4 = 81$$
\TNT{2.4}{%%
$x^4 = 81$ heißt, $x=\pm \sqrt[4]{81} = \pm 3$
}%% end TNT

\newpage
\subsubsection{Vorzeichen}
Vorsicht ist geboten bei negativen Zahlen.


\begin{gesetz}{}{}
  Gerade Exponenten (2, 4, 6,
8, ...) zwingen die Potenz immer dazu, positiv zu werden. Beispiel:

$$x^6 \ge 0$$


Wohingegen ungerade Exponenten das Vorzeichen der Basis
beibehalten. Beispiel
$$(-3)^5 = \LoesungsRaumLang{((-1)\cdot{}(3))^5} = \LoesungsRaum{(-1)^5 \cdot 3^5} = - (3^5)$$
\end{gesetz}


Beachten Sie die folgenden typischen Fälle:

\begin{tabular}{|c|l|}
  \hline
  $x^2 = 4$& \TRAINER{$\mathbb{L}_x=\{-2; +2\}$} \\
  \hline
  $x^3 =  8$& \TRAINER{$\mathbb{L}_x=\{2\}$} \\
  \hline
  $x^3 = -8$& \TRAINER{$\mathbb{L}_x=\{-2\}$} \\
  \hline
  $x^6 = -5$& \TRAINER{$\mathbb{L}_x=\{\,\}$} \\
  \hline
  $x^7 = -5$& \TRAINER{$\mathbb{L}_x=\{-\sqrt[7]{5}\}$}\noTRAINER{\,\,\,\,\,\,\,\,\,\,\,\,\,\,\,\,\,\,\,\,} \\
  \hline
  \end{tabular} 



\begin{gesetz}{}{}

\begin{center}\fbox{Gegeben $x^n=a$}\end{center}

$n$ gerade:


$$a \textrm{ positiv: zwei Lösungen: } x = \pm \sqrt[n]{a}$$
$$a \textrm{ negativ: \textbf{keine Lösung!}}$$

$n$ ungerade:

$$a \textrm{ positiv:} x = + \sqrt[n]{a}$$
$$a \textrm{ negativ:} x = - \sqrt[n]{-a}$$

\end{gesetz}



\subsection*{Aufgaben}
\aufgabenfarbe{Aufgaben 1. und 2. vom Blatt zu «Potenzgleichungen» im
  OLAT (M. Rohner). [Tipp bei Aufg. 2: Substitution.]}

\newpage

\subsection{Wurzelgleichungen}\index{Wurzelgleichung}
Kommt in der Gleichung die Gesuchte unter der Wurzel vor, so sprechen
wir von einer \textbf{Wurzelgleichung}. Einführungsbeispiel


$$\sqrt[5]{x}=6$$


Dies löst man indem man beide Seiten der Gleichung potenziert:

\begin{tabular}{rccl}
  \             & $\sqrt[5]{x}$   &=&     6      \\
  $\Rightarrow$ & $(\sqrt[5]x)^5$ &=&  $6^5$     \\
  $\Rightarrow$ & $x$             &=& $7\,776$ 
\end{tabular}


Lösen Sie das folgende Musterbeispiel:
$$\sqrt{x}+1=2x$$
\TRAINER{Das obige Beispiel von M. Rohner zeigt viele Sackgassen und Dinge, auf die man noch achten muss.}
\platzFuerBerechnungen{8.4}
\TRAINER{
  $$\sqrt{x}=2x-1$$ quadrieren:
  $$x=4x^2-4x+1$$
  $$0=4x^2-5x+1$$
  $$x_{1,2}=\frac{+5 \pm \sqrt{25-16}}{8}$$
  $$\mathbb{L}_x={1}$$, denn $\frac{1}{4}$ ist eine durchs Quadrieren erschienene Scheinlösung.}


\GESO{Lösungsverfahren im Buch \cite{marthaler17} Seite 192 im roten
  Kasten.}

\begin{bemerkung}{Scheinlösung}{}
  
\textbf{Achtung}: Beim \textbf{Radizieren} auf beiden Seiten
einer Gleichung können \textbf{Lösungen} \textbf{verschwinden}.


Beim \textbf{Potenzieren} können (Schein)\textbf{Lösungen}
\textbf{hinzukommen} \GESO{Bsp. S.191 und roter Kasten S. 192
  im \cite{marthaler17}}.
\end{bemerkung}
\newpage


\subsection{Rationale Exponenten}\index{Exponenten!rationale}\index{Wurzeln}
 Eine Wurzelgleichung ist nichts anderes als eine
Potenzgleichung mit rationalen Exponenten.

\TRAINER{Anstelle von $\sqrt[n]{x}$ hatten wir auch  $x^{\frac{1}{n}}$
  geschrieben.}

Beispiel: $\sqrt[4]{x^3} = 125$

\TNT{5.2}{

\begin{tabular}{rccl}
%%                & $\sqrt[4]{x^3}$               &=&   125                \\
  $\Rightarrow$ & $x^{\frac{3}{4}}$               &=&   125                \\
  $\Rightarrow$ & $(x^{\frac{3}{4}})^\frac{4}{3}$  &=& $(125)^\frac{4}{3}$  \\
  $\Rightarrow$ & $x^{\frac{3}{4}\cdot\frac{4}{3}}$  &=& $(\sqrt[3]{125})^4$  \\
  $\Rightarrow$ & $x^1$                          &=& $5^4$  \\
  $\Rightarrow$ & $x$                            &=& $625$  
\end{tabular}

}%% END TNT
\newpage


\subsection*{Aufgaben}
\TALSAadB{???}{???}
\GESO{\aufgabenfarbe{Aufgaben 3. und 4. vom Blatt im OLAT zu
    «Potenzgleichungen» (M. Rohner)}}
\GESOAadB{196 (Wurzelgleichungen)}{2., 3. a)-f),  4. a) c) d) f),
  5. a) c), 6. b),  7. d), 8. c) d) und Textaufgabe 10. a) und b)}
