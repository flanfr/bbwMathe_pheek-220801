%%
%% 2020 05 07 Ph. G. Freimann
%%

%% Überblick über die Begriffe
%% Potenz, Potenzgleichung, Exponentialgleichung, Wurzelgleichung

\subsection{Überblick über Gleichungen mit Potenzen}

\begin{tabular}{|p{52mm}|p{52mm}|p{52mm}|}
  \hline
  Potenzwert gesucht       & Basis gesucht                       &  Exponent gesucht          \\
  \hline
  $10^3=x$                 & $x^3=1000$                           &  $10^x=1000$               \\
  \hline
  $x=10\cdot{}10\cdot{}10$ & $x=\sqrt[3]{1000}$                   & $x =  \log_{10}(1000)$     \\
  $x=1000$                 & $x=10$                               & $x =  3$                   \\
                           & Potenzgleichung                      & Exponentialgleichung       \\



  \hline
  \multicolumn{3}{c}{\,}\\ %% Generiere etwas Abstand
  \multicolumn{3}{c}{Beispiel Zweierpotenzen}\\
  \hline
  Potenz                   & Potenzgleichung bzw. Wurzelgleichung &  Exponentialgleichung     \\
  \hline 
  $2^5=x$                  & $x^5=32$                             &  $2^x=32$                  \\
  \hline
  $x=2\cdot{}2\cdot{}2\cdot{}2\cdot{2}$ & $x=\sqrt[5]{32}$        & $x =  \log_{2}(32)$        \\
  $x=32$                   & $x=2$                                & $x  =  3$                  \\
  \hline


  \hline
  \multicolumn{3}{c}{\,}\\ %% Generiere etwas Abstand
  \multicolumn{3}{c}{\GESO{(Optional)}\TALS{Erinnerung} --- Wurzeln sind rationale Exponenten:}\\
  \hline
  Wurzelwert gesucht        & Radikand gesucht                   &  Wurzelexponent gesucht            \\
  \hline
  $\sqrt[3]{1000}=x$        & $\sqrt[3]{x}=10$                    &  $\sqrt[x]{1000}=10$               \\
  \hline
  $1000^{\frac{1}{3}}=x$     & $x^{\frac{1}{3}}=10$                   &  $1000^{\frac{1}{x}}=10$               \\
  $x=\sqrt[3]{1000}$       & $x=10^3$                             & $\frac{1}{x} =  \log_{1000}(10)$      \\
  $x=10$                   & $x=1000$                             & $\frac{1}{x} =  \frac{1}{3}$         \\
                           &                                      & $x = 3$                      \\\hline
\end{tabular}

\TALS{
  \begin{center}Logarithmische Gleichungen\end{center}
    Logarithmische Gleichungen sind Gleichungen, bei denen die
    gesuchte Größe ($x$) im Argument des Logarithmus vorkommen.

    Beispiel: $\log_2(x-7) = 5$

    Meist können logarithmische Gleichungen (wie obige) in die
    Potenzschreibeweise umgeschrieben werden und sind dann einfacher
    lösbar:

    \TNT{2}{$$\log_2(x-7) = 5 \Longleftrightarrow 2^5 = x-7
      \Longleftrightarrow x=39$$}%% END TNT

}%% END TALS

\GESO{Bem. zum Logarithmus zur Basis 2 ($\log_{2}(32)$): Dazu müssen
  Sie die

  \tiprobutton{ln_log}-Taste
 auf Ihrem TI-30 Taschenrechner 3x drücken.}
