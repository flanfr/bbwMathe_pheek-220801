
%% 2019 07 04 Ph. G. Freimann
%%

\section{Exponentialgleichungen}\index{Gleichungen!Exponentialgleichungen}
\sectuntertitel{$P\neq NP ?$}

\theorieGESO{199}{12}
%%%%%%%%%%%%%%%%%%%%%%%%%%%%%%%%%%%%%%%%%%%%%%%%%%%%%%%%%%%%%%%%%%%%%%%%%%%%%%%%%
\subsection*{Lernziele}

\begin{itemize}
\item Exponentialgleichungen ...
  \begin{itemize}
  \item ... durch Exponentenverglech
  \item ... durch Logarithmieren
    \end{itemize}
\item logarithmische Gleichungen \GESO{(\cite{marthaler21} S. 203
    Kap. 12.2)}
\TALS{\item Exponentialgleichungen mit Parametern}
\end{itemize}

\TALS{Theorie: (\cite{frommenwiler17alg} S.117 (Kap. 2.4.4))}

\bbwCenterGraphic{12cm}{allg/gleichungen/img/ritter.jpg}

\newpage


\subsection{Exponentialgleichung\TALS{...}}\index{Exponentialgleichungen}
Bei einer \textbf{Exponentialgleichung} kommt die gesuchte Größe im
Exponenten (von Potenzen) vor. Beispiel:

$$5^{x+1} = 34$$

Typischerweise werden diese Gleichungen gelöst, indem die Definition
des Logarithmus angewendet wird.

\textbf{Typ I:} Exponentenvergleich\\

\begin{rezept}{Exponentenvergleich}{}
Bei Exponentialgleichungen der Form $$5^{x+1} = 5^{2x-1}$$ können bei
gleicher Basis einfach die Exponenten verglichen werden:
$$\LoesungsRaumLang{\Longleftrightarrow x+1=2x-1}$$

Es entsteht eine lineare Gleichung mit der Lösung $\LoesungsRaum{\lx = \{2\}}$
\end{rezept}

\textbf{Typ II:} Lösen durch Logarithmieren\\

\begin{rezept}{Logarithmieren}{}
$$5^x=32 \Longleftrightarrow x=\log_5(32) \Longrightarrow x \approx 2.15338$$

Die Lösung kann dann auch mit Logarithmen zur Basis 10 angegeben
werden:
$$\LoesungsRaumLang{x = \log_5(32) = \frac{\lg(32)}{\lg(5)}}$$
\end{rezept}
\newpage

\textbf{Typ III:} Allgemeiner Fall\\

\begin{rezept}{Exponentialgleichung lösen}{rezept_allgemeine_exponentialgleichung}
  $$8^{x-1}=7^{x+2}$$
\end{rezept}

\GESO{\TNT{14}{
      1. Zuerst die Summen in den Exponenten wegbringen:
      $$8^x\cdot{}8^{-1} = 7^x\cdot{}7^2$$
      2. Jetzt alle Potenzen mit $x$ auf eine Seite bringen:
      $$\frac{8^x}{7^x} = \frac{7^2}{8^{-1}}$$
      3. Vereinfachen und als Potenz in $x$ schreiben:
      $$\left(\frac87\right)^x = 7^2\cdot{}8$$
      4. Definition Logarithmus anwenden ($a^x=b \Leftrightarrow x=\log_a(b)$):
      $$x = \log_{\frac87}(7^2\cdot{}8) \approx{} 44.718$$
      (5.) Dies kann (falls gefragt) mit Zehnerlogarithmen geschrieben
      werden:
      $$x = \frac{\lg(7^2\cdot{}8)}{\lg(\frac87)} \approx 44.718$$
      }%% END TNT
    }%% END GESO
\GESO{\newpage}


\GESO{\begin{rezept}{Optional 2. Lösungsweg}{}
  $$8^{x-1}=7^{x+2}$$
\end{rezept}}%% END GESO

\TNT{14}{
  Erst mal auf beiden Seiten logarithmieren, mit einem Logarithmus zu beliebiger Basis:
  $$\log(8^{x-1})=\log(7^{x+2})$$
  Log Gesetz:
  $$(x-1)\cdot{}\log(8)=(x+2)\cdot{}\log(7)$$
  ausmultiplizieren
  $$\log(8)x-\log(8)=\log(7)x+2\log(7)$$
  und $x$ auf eine Seite bringen
  $$\log(8)x-\log(7)x = 2\log(7)+\log(8)$$
  ausklammern
  $$x(\log(8)-\log(7)) = 2\log(7)+\log(8)$$
  und durch die Klammer $(\log(8)-\log(7))$ teilen
  $$x = \frac{2\log(7)+\log(8)}{\log(8)-\log(7)}\approx 44.718$$
}%% END TNT

\GESO{Siehe auch \cite{marthaler21} Seite 200 im roten Kasten.}
\newpage

\subsection*{Aufgaben}
\TALSAadB{118}{359. a) d), 360. a) d), 361. a) d), 362. a) b) c), 363. a) b) d)}
\GESOAadB{71 (Exponentenvergleich)}{44}
\GESOAadB{206}{2. a) d) g) h), 3. a) b) c) d) f), 4. a) d) e) g) h),
  9. und 11.}

