%%
%% 2019 07 04 Ph. G. Freimann
%%
\section{Bruchgleichungen II ...}\index{Gleichungen!Bruchgleichungen, quadratische}\index{quadratische Bruchgleichungen}
%%\sectuntertitel{...}

\subsection{... quadratische}

\theorieGESO{118}{8.3}

\theorieGESO{121}{8.4}
\theorieTALS{105}{2.4.1}
%%%%%%%%%%%%%%%%%%%%%%%%%%%%%%%%%%%%%%%%%%%%%%%%%%%%%%%%%%%%%%%%%%%%%%%%%%%%%%%%%
\subsection*{Lernziele}

\begin{itemize}
\item Bruchgleichungen, die in quadratischen Gleichungen münden
\item Definitionsbereich
\TALS{  \item \textit{Bruch\textbf{un}gleichungen (halbgrafische Methode)}}
\end{itemize}

Bruchgleichungen können beim Lösen auch auf quadratische Gleichungen führen:
\begin{beispiel}{quadratische Bruchgleichung}{beispiel_quadratische_bruchgleichung}
$$\frac{6x-24}{3-x} + x - 2 = \frac{6}{x-3}$$
\end{beispiel}

\newpage
Vorzeigebeispiel:

$$\frac{6x-24}{3-x} + x - 2 = \frac{6}{x-3}$$
\TRAINER{
Definitionsbereich:
  $\mathbb{D} = \mathbb{Q}\backslash \left\{3\right\}$\\

Gleichnamig:

$$\frac{6x-24}{3-x} + \frac{(x-2)(3-x)}{3-x} = \frac{-6}{+(3-x)}$$

Mit Hauptnenner multiplizieren, denn der kann ja nicht Null sein, denn
die $3$ wurde aus dem Definitionsbereich ausgeschlossen:

$$6x-24 + (x-2)(3-x) = -6$$

Ausmultiplizieren ...

$$6x - 24 + 3x -x^2 -6 +2x = -6$$

... und zusammenfassen:

$$0=x^2-11x+24 = (x-3)(x-8)$$

Nur Definitionsbereich in der Lösungsmenge:

$$\mathbb{L} = \{8\}$$
}
\noTRAINER{\mmPapier{16.4}}

\subsection*{Aufgaben}
\TALSAadB{106}{329 ff.}

\GESO{Bruchgleichungen, die auf quadratische Gleichungen führen:}
\GESOAadB{182}{12.}
\newpage


\GESO{\subsubsection*{Aufgaben aus alten Maturaprüfungen}
Bestimmen Sie den Definitionsbereich des folgenden Terms bezüglich der Grundmenge $\mathbb{R}$\TRAINER{($\mathbb{D}_x=\mathbb{R}\backslash\{-3;12\}$)}.
$$\frac{5x}{x^2 - 9x - 36}$$

Bestimmen Sie den Definitionsbereich und die Lösungsmenge der Gleichung.
Die Gleichung ist auf \textbf{Grundform}
$ax^2 + bx + c = 0$ zu bringen und kann dann
mit dem entsprechenden Taschenrechnermodus gelöst werden\TRAINER{($\mathbb{L}_x=\{5\}$)}.
$$\frac{x^2-10x}{x-4} + 1 = \frac{24}{4-x}$$

%% Das folgende, nun auskommentierte,  war die Einstiegsaufgabe oben, daher raus:
%% Bestimmen Sie den Definitionsbereich und die Lösungsmenge der Gleichung.
%% Die Gleichung ist auf \textbf{Grundform} $ax^2 + bx + c = 0$ zu bringen und
%% kann dann mit dem entsprechenden Taschenrechnermodus gelöst werden.
%% $$\frac{6x-24}{3-x} +x -2 = \frac{6}{x-3}$$

Bestimmen Sie den Definitionsbereich und die Lösungsmenge der Gleichung.
Die Gleichung ist auf \textbf{Grundform} $ax^2 + bx + c = 0$ zu bringen und kann dann mit
dem entsprechenden Taschenrechnermodus gelöst werden\TRAINER{($\mathbb{L}_x=\{12\}$)}.
$$\frac{x^2-16x}{x-3} + 1 = \frac{39}{3-x}$$

Bestimmen Sie den Definitionsbereich und die Lösungsmenge der Gleichung.
Die Gleichung soll auf die \textbf{Grundform} $ax^2 + bx + c = 0$ gebracht werden und
kann dann mit dem entsprechenden Taschenrechnermodus gelöst werden\TRAINER{($\mathbb{L}_x=\{1\}$)}.
$$\frac{x^2}{x-2} + \frac{4}{2-x} = 3$$


Bestimmen Sie die Lösung(en) der Gleichung in der Grundmenge $\mathbb{R}$.
\TRAINER{\\$\mathbb{D}=\mathbb{R}\backslash{}\{0,3\}; \mathbb{L}_x=\{-5, 2\}  $}
$$\frac{x+1}{x-3} = \frac{10-2x}{x^2-3x}$$
\newpage
} %% END GESO

