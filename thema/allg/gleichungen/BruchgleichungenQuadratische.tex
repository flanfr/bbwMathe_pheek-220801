%%
%% 2019 07 04 Ph. G. Freimann
%%
\section{Bruchgleichungen II ...}\index{Gleichungen!Bruchgleichungen, quadratische}\index{quadratische Bruchgleichungen}
%%\sectuntertitel{...}

\subsection{... quadratische}

\theorieGESO{118}{8.3}

\theorieGESO{121}{8.4}
\theorieTALS{105}{2.4.1}
%%%%%%%%%%%%%%%%%%%%%%%%%%%%%%%%%%%%%%%%%%%%%%%%%%%%%%%%%%%%%%%%%%%%%%%%%%%%%%%%%
\subsection*{Lernziele}

\begin{itemize}
\item Bruchgleichungen, die in quadratischen Gleichungen münden
\item Definitionsbereich
\TALS{  \item \textit{Bruch\textbf{un}gleichungen (halbgrafische Methode)}}
\end{itemize}

Bruchgleichungen können beim Lösen auch auf quadratische Gleichungen führen:
\begin{beispiel}{quadratische Bruchgleichung}{beispiel_quadratische_bruchgleichung}
$$\frac{6x-24}{3-x} + x - 2 = \frac{6}{x-3}$$
\end{beispiel}

\newpage
Vorzeigebeispiel:

$$\frac{6x-24}{3-x} + x - 2 = \frac{6}{x-3}$$
\TNT{15.6}{
Definitionsbereich:
  $\mathbb{D} = \mathbb{Q}\backslash \left\{3\right\}$\\

Gleichnamig:

$$\frac{6x-24}{3-x} + \frac{(x-2)(3-x)}{3-x} = \frac{-6}{+(3-x)}$$

Mit Hauptnenner multiplizieren, denn der kann ja nicht Null sein, denn
die $3$ wurde aus dem Definitionsbereich ausgeschlossen:

$$6x-24 + (x-2)(3-x) = -6$$

Ausmultiplizieren ...

$$6x - 24 + 3x -x^2 -6 +2x = -6$$

... und zusammenfassen:

$$0=x^2-11x+24 = (x-3)(x-8)$$

Nur Definitionsbereich in der Lösungsmenge:

$$\mathbb{L} = \{8\}$$
\vspace{20mm}
}%% END TNT
\newpage

\subsection*{Aufgaben}
\TALSAadB{106}{329 ff.}

\GESO{Bruchgleichungen, die auf quadratische Gleichungen führen:}
\GESOAadB{182}{12.}

\GESO{
\aufgabenfarbe{Lösen Sie auch das Blatt im OLAT zu
  Bruch\textbf{gleichungen} aus alten Maturaaufgaben.}
} %% END GESO

\newpage
