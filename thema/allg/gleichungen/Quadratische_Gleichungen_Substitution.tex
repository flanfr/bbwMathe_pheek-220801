%%
%% 2020 02 fp@bbw.ch
%%

%% Quadratische Gleichungen: Substitution
\subsection{Substitution}\index{Substitution!quadratische Gleichungen}

\TRAINER{
Zur Erinnerung beim Faktorisieren:\\
$a(2s-r) + b(2s-r) {\textrm{Subst. }T:=(2s-r) \over =} aT + bT = (a+b)T {\textrm{Rücksusbst. }2s-r=T\over = } (a+b)(2s-r)$
}

Oft können kompliziertere Gleichungen mittels einer geeigneten \textbf{Substitution} (= Ersetzung\footnote{lat. \textbf{substitu\=o} = an die Stelle \textit{jmds. o. einer Sache} setzen.
})
in eine quadratische oder lineare Gleichung verwandelt werden. Wie \zB hier:

\newcommand\tmpPart{\left(\frac{1}{3}(5x+0.5)\right)}

$$-6\tmpPart{} + 8.75 = -{\tmpPart{}}^2$$

%%
\TRAINER{«Mit Farben geht alles besser.»\\}%%
\TNT{2.4}{$$-6{\color{green}\tmpPart{}} + 8.75 = -{\color{green}{\tmpPart{}}^{\color{black}2}}$$}

\textbf{Substitution}

\TNT{1.6}{Wir ersetzne: $y := {\color{green}\tmpPart{}}$}

und die ursprüngliche Gleichung ist nun äquivalent zu
\TNT{1.6}{$$-6{\color{green}y} + 8.75 = -{\color{green}y}^2.$$}

Substituierte Gleichung lösen:
\TNT{2.4}{
  Erst in Grundform bringen $$y^2 - 6y + 8.75 = 0$$
  und danach mit der $a$-$b$-$c$ -- Formel auflösen ($a=1, b=-6, c=8.75$).
}

$$\mathbb{L}_y=\LoesungsRaumLang{\{2.5; 3.5\}}$$


\TRAINER{Substitutionsbasis: $-6\tmpPart{} + 8.75 = -\tmpPart{}^2$\\
  Substituendum: $\tmpPart{}$\\
  Substituens: $y$
}

\newpage


\textbf{Rücksubstitution}\index{Rücksubstitution}\\
Nun setzen wir die Lösungen anstelle der substituierten Variable ein:


\TNT{10}{
$$1.: y_1 = 2.5 = \tmpPart{} \Rightarrow 7.5 = 5x+0.5 \Rightarrow x = \frac{7}{5} = 1.4$$

$$2.: y_2 = 3.5 = \tmpPart{} \Rightarrow 10.5=5x+0.5 \Rightarrow x = 2$$

$$\lx=\{1.4; 2\}$$}

\newpage


\subsection*{Aufgaben}

\GESO{
  \subsubsection{Nullserie}
  Die folgende Aufgabe stammt aus der \textit{Nullserie}-Maturaprüfung:

  Lösen Sie mit Hilfe einer geeigneten Substitution \TRAINER{Tipp: $Z^{10} = \left(Z^5\right)^2$ }:
  $$\left(\frac{x}{3}-3.5 \right)^{10} -2\left( \frac{x}{3} - 3.5 \right)^5 =-1$$
  
  \TNT{12}{Substituiere $z = \left(\frac{x}{3} - 3.5\right)^5$. Somit lautet die Gleichung
$$z^2 - 2z +1 = 0$$
    Was uns zu $z = 1$ bringt. Damit ist $1 = \left(\frac{x}{3} - 3.5 \right)^5$ und somit auch $1 = \frac{x}{3} - 3.5$. Auf beiden Seiten 3.5 addieren und danach mit 3 multiplizieren liefert $x = 13.5$.
 }%% End TNT

\newpage
\aufgabenfarbe{Berufsmaturitätsprüfung 2020: Aufgabe 4 von Serie 2:
  $$(x-5)^4 - \frac{(x-5)^2}{3} = 8$$}%% END Aufgabenfarbe
\TNT{5.6}{Lösung der Substituierten $y=(x-5)^2)$ ist $y=3$ bzw. $y=-\frac83$. Die zweite Lösung kommt jedoch nur als Scheinlösung vor.\vspace{45mm}}%% END TNT
    $$\lx = \LoesungsRaumLang{\left\{5-\sqrt{3}; 5+\sqrt{3}\right\}}$$
  
\GESOAadB{184}{28. a) d) 29. a) c) 30. a) b)}
}%% END GESO

\TALSAadB{99ff (Substitution)}{283. a) b) f) 282. a) f) 284. a)}
\newpage
