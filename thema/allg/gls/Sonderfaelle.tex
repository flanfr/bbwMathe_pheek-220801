\subsection{Sonderfälle (Fallunterscheidung)}\index{Fallunterscheidung!Gleichungssysteme}\index{Sonderfälle!Gleichungssysteme}\index{Lineare Gleichungssysteme!Sonderfälle, Fallunterscheidung}

Berechnen Sie die Lösung für $(x,y)$ in Abhängigkeit von $m$ und $n$:

\gleichungZZ{ax+ay}{1}{x-ay}{-1}

\TNT{5.2}{
  Additionsverfahren:
  
  $$ax+x=0$$
  $$x(a+1)=0 | : (a+1) \textrm{ falls } a+1\ne 0$$
  $$\Longrightarrow x=0, y=\frac1a$$
  Lösung für $a+1=0$: $x$ ist beliebig ($x=\lambda$) und somit $y=-\lambda - 1$.

  Lösung für $a=0$: $0=1$ und somit $\mathbb{L}_{(x,y)} = \{\}$

}%% END TNT


\subsection*{Aufgabe}
Berechnen Sie die Lösung für $(x,y)$ in Abhängigkeit von $m$ und $n$:

\gleichungZZ{2mx+y}{3}{-x-y}{n}

\TNT{5.2}{
  Nach dem Additionsverfahren steht da:
  $$2mx-x=3+n$$
  Lösung: 
  $$(x,y) = \left(\frac{n+3}{2m-1} ,\frac{3+2nm}{1-2m} \right)$$
  für $2m-1\ne 0$!
  Wäre $2m-1 = 0$, so gäbe es mit $n=-3$ unendlich viele Lösungen.
}%% END TNT


\subsection*{Aufgaben}
\TALSAadB{128}{398. a) b) c) d)}
