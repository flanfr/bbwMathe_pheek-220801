\subsection{Nicht-Lineare Gleichungssysteme}\index{Gleichungssysteme!nicht lineare}\index{nicht lineare Gleichungssysteme}

\subsubsection*{Lernziele}
\begin{itemize}
  \item{Taschenrechner}
\end{itemize}

\subsection{Einstiegsbeispiel}
Lösen Sie die Wurzelgleichung:

$$\sqrt{x+1} = \frac{x-3}{2}$$

\TNT{12.0}{Definitionsbereich: x> -1: Mal zwei:
  $$2\sqrt{x+1} = x-3$$ Danach quadrieren (es können Scheinlösungen auftreten):
  $$4(x+1) = (x-3)^2$$
  Quadratische Gleichung aufstellen:
  $$4x+4 = x^2 - 6x + 9$$
  $$0 = x^2 -10x + 5$$
  $$x_{1,2} = \frac{10 \pm \sqrt{100 - 20}}{2}$$
  $$\mathbb{L}_x = \{0.52786..., 9.472\}$$
  Probe: Die Lösung 0.5 ist eine Scheinlösung! Nur 9.472 ist Lösung.
}
\newpage


Interpretieren Sie die obige Wurzelgleichung einmal anders, indem Sie den Term links und den Term rechts je als Funktion darstellen:

$f: y= ...................$\TRAINER{$ f: y=\sqrt{x+1}$}

und

$g: y= ...................$\TRAINER{$ g: y=\frac{x-3}{2}$}.

\bbwGraph{-2}{12}{-1}{5}{
  \TRAINER{
    \bbwFunc{\x / 2 - 1.5}{-1:11}
    \bbwFunc{sqrt(1+\x)}{-1:11}
  }%% end TRAINER
}%% end BBW Graph

Lösen Sie die Aufgabe nun mit dem Taschenrechner a) algebraisch b) graphisch.

\subsection*{Aufgaben}
\TALSAadB{136}{Mit Taschenrechner: 434 a) b), 435. a) b)}
