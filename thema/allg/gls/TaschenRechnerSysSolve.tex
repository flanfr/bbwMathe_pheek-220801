\subsection{Taschenrechner}\index{Taschenrechner!Gleichungssysteme}
Lineare Gleichungssysteme sind so zentral, dass viele heutige Taschenrechner diese lösen kann.

\gleichungZZ{3a + 2b}{3.10}{4a + 5b}{6.00}

\GESO{Suchen Sie \tiprobutton{2nd}\tiprobutton{tan_sys-solv} und geben Sie die Zahlen 3, 2, 3.10, 4, 5 bzw. 6.00 in die entsprechenden Felder ein.}
\TALS{Definieren Sie das Gleichungssystem gls:=\{3a+2b=3.1, 4a+5b=6.0\}. Dies können Sie nun einfach mit solve(gls,\{a, b\}) auflösen lassen.}

\GESO{
\begin{bemerkung}{}{}
  Die Variable beim TI-30 PRO müssen bei 2x2-Gleichungssystemen $x$ und $y$  heißen.
\end{bemerkung}
}

\GESO{
  \subsubsection{Eingabe negativer Zahlen}
  Lösen Sie das folgende Gleichungssystem mit dem Taschenrechner.

  \gleichungZZ{-4x + (-8)y}{16}{3x-5y}{32}

  Beachten Sie die Eingabe negativer Zahlen auf dem TI 30 Pro
  MathPrint. Das negative Vorzeichen wird mit \tiprobutton{neg} eingegeben,
  wohingegen die Subtraktion mit dem einfachen \tiprobutton{minus} eingegeben wird.

  \TNT{2.4}{Lösung: $x=4$, $y=-4$\vspace{22mm}}

}

\GESO{
\aufgabenfarbe{  
  Prüfen Sie mit diesem Wissen von Seite 150ff die Resultate von Aufgabe 7. g) [$x=\frac{42}{61}$ und $y=\frac{60}{61}$], 7. b) [$x=\frac52$ und $y=-\frac{15}{2}$] 8. a) [$x=2$ und $y=6$] und 8. b) [$x=-2$ und $y=2$]
}}

\newpage
