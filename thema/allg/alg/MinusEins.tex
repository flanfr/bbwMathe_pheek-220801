%%
%% 2019 11 14 ph. freimann
%%

\newpage
\subsection{Minus Eins}\index{Minus Eins}\index{$-1$}

\subsubsection*{Was wir mit der $(-1)$ tun dürfen}

\begin{tabular}{p{5cm}|rcl}
  \hline\\
  Notation          & $(-1)\cdot(...)$                &$=$& $-(...)$                      \\
  \\
  \hline\\
  Ausmultiplizieren & $(-1)\cdot(7x-4y+3-2b)$              &$=$& $-7x + 4y -3 + 2b$            \\
                    & $-(-4a + 3v -(x+2))$            &$=$& $-(-4a +3v -x-2)$             \\
                    &                                 &$=$& $+4a -3v +x+2$                \\
  \\
  \hline\\
  Ausklammern       & $-7x +4y -3 +2b$                &$=$& $(-1)\cdot (+7x -4y + 3 -2b)$ \\
                    & $4a -3v + x +2$                 &$=$& $-(-4a + 3v -x -2)$           \\
  \\
  \hline\\
  Erweitern         & $\frac{-7x+3b -8}{3x - 4y -16}$ &$=$& $\frac{{\color{green}(-1)}\cdot(-7x+3b -8)}{{\color{green}(-1)}\cdot(3x - 4y -16)}$\\
                    & $\frac{a-b}{b-a}$               &$=$& $\frac{{\color{green}(-1)}\cdot(a-b)}{{\color{green}(-1)}\cdot(b-a)}$\\
  \\
  \hline\\                      
  Erweitern und Ausmultiplizieren  & $\frac{-7x+3b -8}{3x - 4y -16}$ &$=$& $\frac{(-1)\cdot(-7x+3b -8)}{(-1)\cdot(3x - 4y -16)}$\\
                    &                                 &$=$& $\frac{7x-3b+8}{-3x+4y+16}$\\
  \\
  \hline\\                    
  Gemischtes
  Beispiel          & $\frac{a-b}{b-a}$               &$=$& $\frac{a-b}{(-1)\cdot(-b+a)}$ \TRAINER{-1 auskl.}\\
                    &                                 &$=$& $\frac{(a-b)}{(-1)(a-b)}$ \TRAINER{kommutativ}\\
                    &                                 &$=$& $\frac{1}{(-1)}$ \TRAINER{kürzen}\\
                    &                                 &$=$& $\frac{(-1)\cdot 1}{(-1)\cdot(-1)}$\TRAINER{(-1) erw.}\\
                    &                                 &$=$& $\frac{(-1)\cdot 1}{1}$ \TRAINER{(-1)(-1)=(+1)}\\
                    &                                 &$=$& $-1$ \TRAINER{kürzen}\\
\end{tabular}

\paragraph{Gleichungen} Bei Gleichungen dürfen beide Seiten des Gleichheitszeichens mit (-1) multipliziert werden (sofern man wirklich \textbf{beide} Seiten multipliziert). 

\newpage
\subsubsection*{Was wir mit der Minus Eins nicht tun!}
Wir multiplizieren nicht einfach einen Term mit Minus Eins.
Gegenbeispiel:

Konto Hr. Ph. Freimann Stand 1. Jan. 2019: CHF {\color{green}10\,377.--}.

Multiplikation am 2. Jan. 2019 mit $(-1)$.

Konto Hr. Ph. Freimann Stand 3. Jan. 2019: CHF {\color{red} -10\,377.--}.
