%%
%% 2019 07 04 Ph. G. Freimann
%%

\subsection{Betrag}\index{Betrag}\index{Absolutbetrag}
%%\sectuntertitel{Absolut? Abstand?}
\sectuntertitel{Wo ist negativ positiv? Beim Alkohol-Test!}
\TRAINER{Einstiegsvideos: Daniel Jung und Mathe-Mann}

\weblink{Mathe Mann/Mathe Frau}{https://www.youtube.com/watch?v=yiJTCL9I-aU}!

\theorieTALS{9}{1.1.2}
\theorieGESO{15}{1.2}%
%%%%%%%%%%%%%%%%%%%%%%%%%%%%%%%%%%%%%%%%%%%%%%%%%%%%%%%%%%%%%%%%%%%%%%%%%%%%%%%%%
\subsection*{Lernziele}

\begin{itemize}
  \item Symbol
  \item Bedeutung als Abstand
  \TALS{\item Gleichungen mit Betrag lösen}
\end{itemize}


\BLENDED{\matheNinjaLink{Betrag}{https://olat.bbw.ch/auth/RepositoryEntry/572162163/CourseNode/105951761084932}}

\newpage
\begin{definition}{Betrag}{definition_betrag}\index{Betrag}\index{Absolutbetrag}
  Unter dem \textbf{Betrag} oder \textbf{Absolutbetrag} einer Zahl versteht man deren (positiven)
  \textbf{Abstand}\index{Abstand} zum Nullpunkt. Das Symbol zum Betrag sind zwei
  senkrechte Striche:\index{$\mid\cdot \mid$ s. Betrag(-striche)}\\
  $|a| := a$, wenn $a$ positiv\\
  bzw.\\
  $|a| := -a$ falls $a$ negativ.
\end{definition}

\begin{bemerkung}{}{}
Mit $|a - b|$ wird der Abstand der Zahlen $a$ und $b$ berechnet. Ist
nämlich $b > a$, so ist die Differenz negativ und wird mit dem
$| \cdot{} |$-Symbol ins Positive gekehrt.
\end{bemerkung}

\begin{bemerkung}{}{}
  Einfach zum Merken: Dem Abstand zwischen zwei Punkten ist egal, in
  welche Richtung er gemessen wird. So liegt Zürich genauso weit von
  Bern entfernt, wie Bern von Zürich entfernt ist. Somit gilt $|a - b| = |b - a|$.
\end{bemerkung}

\GESO{%% TR GESO
  \begin{bemerkung}{}{}
  Auf dem Taschenrechner kann der Absolutbetrag mit \tiprobutton{math}
  «NUM» «abs(...» eingegeben werden.\\
  Tippen Sie:

  \tiprobutton{math} (Pfeil nach rechts) \tiprobutton{enter} 10 - 22 \tiprobutton{enter}

  Sie erhalten $|10-22| = 12$
  \end{bemerkung}
  }%% END GESO
\newpage


\subsubsection{Beispiele}


\TALS{
Für welche Zahlen $x \in \mathbb{R}$ gilt

$|4| = x$ \TRAINER{$x=4$}%

$|4| = -x$ \TRAINER{$x=-4$}%

$|-4| = x$ \TRAINER{$x=4$}

$|-4| = -x$ \TRAINER{$x=-4$}


$|x| = 4$ \TRAINER{$\lx=\{4, -4\}$}%

$|x| = -4$ \TRAINER{$\lx=\{\}$}%

$|-x| = 4$ \TRAINER{$\lx=\{4, -4\}$}%

$|-x| = -4$ \TRAINER{$\lx=\{\}$}%
}

\GESO{
$|4| = \LoesungsRaum{4}$%

$|-4| = \LoesungsRaum{4}$%

  aber:
  
$-|4| = \LoesungsRaum{-4}$%

$-|-4| = \LoesungsRaum{-4}$%

$|8-5| = \LoesungsRaum{3}$%

$|5 -8| = \LoesungsRaum{3}$%

  Achtung:

  $|-5-8| = \LoesungsRaum{13}$

  $|5+8| = \LoesungsRaum{13}$

  
${\big|}|6| - |-10|{\big|} = \LoesungsRaum{4}$%

}

Theorieaufgabe:
$$\big\vert 7 - \left\vert -3 \right\vert \big\vert - |-7-3|$$

\TNT{2.4}{%%
  $ = | 7 - (3) | - | (-7 - 3)|$ \\
  $ =| 4 |       - | -10 |$ \\
  $ = 4 - (+10)$ \\
  $= -6$
}%% END TNT

\TALS{%
  2. Beispiel:

  Für welche $x$ gilt folgendes:
  $$|x - 3| = 8$$

\TNT{2.4}{%%
  Erste Lösung: Welche Zahlen haben von 3 den Abstand 8? Lsg.:
  11 und -5. Formal: 1. Fall $x-3 > 0$, dann ist $x-3=8$ und somit
  $x_1=11$; 2. Fall $x-3 <=0$, dann ist $-(x-3)=8$ und somit $x_2=-5$
}% END TNT
}% 2. Beispiel Theorieaufgabe TALS


\subsection*{Aufgaben}
\GESO{\olatLinkArbeitsblatt{Betrag [A1B]}{https://olat.bbw.ch/auth/RepositoryEntry/572162163/CourseNode/105796974541159}{1. a) bis e) und 2. a) bis g)}}%% END olatLinkArbeitsblatt
\TALS{\olatLinkArbeitsblatt{Betrag [A1B]}{https://olat.bbw.ch/auth/RepositoryEntry/572162090/CourseNode/105796974602793}{1. a) bis e) und 2. a) bis g)}}%% END olatLinkArbeitsblatt



%%Weitere Aufgabe im Buch:
%%\TALSAadB{9}{5. - 7., 14., 16.-18.}
%%\GESOAadB{23}{14. a) b) c) f), 15. a), 17. a) e) und 18. a)}

\newpage

\subsubsection{Kontrolle}
Berechnen Sie:
$$\left| |4-8| - 11 \right| = \LoesungsRaumLang{7}$$

Challenge: Lösen Sie die folgenden Gleichungen nach $x$ auf:

$$|x| = 11.4 \Longrightarrow \lx = \LoesungsRaumLang{\{-11.4; 11.4\}}$$
$$|x-3| = 7 \Longrightarrow \lx = \LoesungsRaumLang{\{-4; 10\}}$$

Für welche Zahlen $x\in\mathbb{R}$ gilt:
$$|-x| = -x$$
$$\mathbb{L}_x = \LoesungsRaumLang{\mathbb{R}^-_0}$$
\weblink{Mathe Mann/Mathe Frau}{https://www.youtube.com/watch?v=yiJTCL9I-aU}!
