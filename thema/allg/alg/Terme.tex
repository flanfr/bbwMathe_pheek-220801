%%
%% 2019 07 04 Ph. G. Freimann
%%

\section{Terme}\index{Term}
\sectuntertitel{Römische Bäder?}

\theorieTALS{10}{1.1.3}
\theorieGESO{17}{1.3}
%%%%%%%%%%%%%%%%%%%%%%%%%%%%%%%%%%%%%%%%%%%%%%%%%%%%%%%%%%%%%%%%%%%%%%%%%%%%%%%%%
\subsection*{Lernziele}

\begin{itemize}
 \item Termanalyse (Summe, Differenz, Produkt, Quotient, Potenz)
 \item Hierarchie der Terme (Vorrangregeln)
 \item Termumformungen
 \item Ausmultiplizieren
\end{itemize}
\newpage

\subsection{Term-Definition}\index{Term}
\begin{definition}{Term}{definition_term}
  Ein \textbf{Term} ist entweder
  \begin{itemize}
  \item ein Atom (eine Zahl (\zB{} $4.86$) oder  eine Variable (\zB{}
  $a$, $x$))
  \item ein Klammerausdruck (\zB{} \fbox{({\color{green}T})}, inkl.  Wurzeln \fbox{$\sqrt{\color{green}T}$})\footnote{Dabei wird der
  horizontale Strich wie eine Klammer aufgefasst.}
%%  \item weitere \TALS{Monome\index{Monom}}\GESO{Summanden (Teile einer Summe)}:
%%    \begin{itemize}
   \item eine \textbf{Potenz}\index{Potenz}\footnote{Als Exponent darf
     ein beliebiger Term eingesetzt werden, wohingegen als Basis
     lediglich Atome, Klammerausdrücke und Wurzelterme verwendet
     werden dürfen, wegen der Verwechslungsgefahr. Bei
     ${\frac{a}{b}}^c$ ist nämlich nicht klar, wohin das $c$ denn gehört.} (\fbox{${\color{green}T_1}^{\color{green}T_2}$} \zB{} $a^3$, $(2a - 4)^{x+2}$)
    \item  ein Bruchterm\footnote{Wie bei der Wurzel, dient der Bruchstrich als Klammerpaar: $\frac{U}{V}=(U):(V)$} (\fbox{$\frac{\color{green}T_1}{\color{green}T_2}$} \zB{} $\frac{2^x}{x^2}$)
    \item ein implizites \textbf{Produkt}\footnote{Ein implizites
      Produkt ist mit Koeffizienten angereicherter Ausdruck
      \textbf{ohne} Multiplikationszeichen.} (\zB{}
      ${\color{red}4a}{\color{green}T}$)\footnote{Tritt eine Zahl auf,
      so ist diese immer ganz links zu schreiben. Zahlen rechts von
      Ausdrücken werden mit einem Multiplikationszeichen ($\cdot$) versehen: $5x$, aber $x\cdot{}5$.}
    \item ein explizites \textbf{Produkt}\index{Produkt} ($\cdot$;  ${\color{green}T_1}\cdot {\color{green}T_2}$ \zB{} $a\cdot(-1)$) bzw. ein expliziter \textbf{Quotient}\index{Quotient} ($:$, $/$, $\div$; $6a\cdot3b$ bzw. ${\color{green}T_1}:{\color{green}T_2}$ \zB{} $36m^2:12m^2$)
%%    \end{itemize}
  \item eine \textbf{Summe}\index{Summe} (bzw. \textbf{Differenz}\index{Differenz}) von \TALS{Monomen}\GESO{Summanden\index{Summand}}
  (Zum Beispiel bilden die folgenden
  vier «Pakete» eine Differenz):\\
  $-4x^2 + \frac{3a+b}{x} + \sqrt{5y^2-6} - 5t:2t$

    \end{itemize}
(In obiger Aufzählung hat der am höchsten stehende Term die größte
«Bindungskraft». Beispiel «Punkt vor Strich».)
\end{definition}

\textbf{Gegenbeispiele}
Keine Terme sind \zB{}: $x \cdot{}-8$, $\sqrt{+^2}$, $4+*($, $\frac{7}{+}$, $\frac{(a+b}{-c-d)}$.

Generell werden die fünf wichtigsten Termarten in die folgenden drei Kategorien eingeteilt:
\begin{itemize}
\item Potenz
\item Produkt und Quotient
\item Summe und Differenz
\end{itemize}


\newpage

\subsection{Vorrangregeln}\index{Vorrangregeln}

Es gilt Punkt vor Strich. Daneben bindet ein Exponent (\zB $5^8$) noch
stärker. Am stärksten binden Klammern oder horizontale Linien
(Bruchstrich, Wurzelzeichen).

\bbwCenterGraphic{8cm}{allg/alg/img/Klapopustri.png}
\begin{center}
  Das \textit{Klapopustri}\index{Klapopustri} meint dazu:

  \textbf{Klammern} vor \textbf{Potenzen} vor \textbf{Punkt} vor
  \textbf{Strich}
  
\end{center}




Beispiel:
$$-10^4 = \LoesungsRaumLang{-(10^4) = -
  (10\cdot{}10\cdot{}10\cdot{}10) = -10\,000}$$
\newpage

\subsubsection{Terme benennen}
Nicht jeder Term, der ein Pluszeichen enthält, ist automatisch eine
Summe.

Teilen Sie die folgenden Terme in die Kategorie «Summe/Differenz»,
«Produkt/Quotient» und «Potenz/Wurzel» ein. Tipp: Setzen Sie vorab
«unnötige» Klammern und Multiplikationspunkte:


\renewcommand{\arraystretch}{2}
\begin{tabular}{c|c|c}
  Term                       & mit Klammern                        & Zuordnung\\\hline
  $3x^{4-a} - y^{b+2}$ & $\left(3\cdot{}\left(x^{(4-a)}\right)\right) - \left(y^{(b+2)}\right)$  &  Differenz \\\hline
  $ax + 2b$ & \TRAINER{$(a\cdot{}x) + (2\cdot{}b)$} & \TRAINER{Summe}\\\hline
  $\frac{5+x}{5x}$ & \TRAINER{$\frac{(5+x)}{(5\cdot{}x)}$} & \TRAINER{Quotient}\\\hline
  $\sqrt{2x^3+5}$ & \TRAINER{$\sqrt{((2\cdot{}(x^3)) + 5)}$} & \TRAINER{Wurzelterm}\\\hline
  $(a+b)^{c+d}$ & \TRAINER{$(a+b)^{(c+d)}$} & \TRAINER{Potenz}\\\hline
  $(5-3y)c^8$ & \TRAINER{$(5-3\cdot{}y)\cdot{}(c^8)$} & \TRAINER{Produkt}\\\hline
  $(\sqrt{x-3}+\sqrt{8-b})^2$ & \TRAINER{$((\sqrt{(x-3)})+(\sqrt{(8-b)}))^2$} & \TRAINER{Potenz}\\\hline
\end{tabular}

\renewcommand{\arraystretch}{2}



\TRAINER{\TALS{
Im Compilerbau (Schreiben einer Programmiersprache) werden die
folgenden Vorrangregeln verwendet:

\begin{itemize}
\item Term :== Summand \{'+'|'-' Summand\}*\\
\TRAINER{$4a^3 - 6\cdot az^{(7+b)} : \sin(30)$}

\item Summand :== ExpilziterFaktor \{'$\cdot$'|'/' ExpliziterFaktor\}*\\
\TRAINER{$4a^3$, $6 \cdot az^{(7+b)} : \sin(30)$}

\item ExpliziterFaktor :== Faktor \{Faktor\}*\\
\TRAINER{$4a^3$, $6$, $az^{(7+b)}$, $\sin(30)$}

\item Faktor :== SkalarOderKlammerausdruck \{${\,}^{Term}$\}?\\
\TRAINER{$4$, $a^3$, $6$, $a$, $z^{(7+b)}$, $\sin(30)$}

\item SkalarOderKlammerausdruck :== Zahl |
           Variable |
           $\sqrt[Term]{Term)}$ | 
           '(' Term ')' |
           $\frac{Term}{Term}$|
           \textit{Funktionsname} '(' Term ')'\\
\TRAINER{$4$, $a$, $3$, $6$, $a$, $z$, $(7+b)$, $\sin(30)$}

\item \textit{Funktionsname} := 'sin', 'cos', 'tan', 'log', 'lg', 'ln', ....
\end{itemize}
}}

\TALS{\newpage
\textbf{Achtung} Bei zusammengeschriebenen Faktoren (\zB $ab$) bindet
           die Multiplikation stärker als beim expliziten verwenden
           des Multiplikationszeichens (\zB $a\cdot{}b$). Beispiel
           $a\cdot bm = a\cdot (b\cdot m)$.

Gleich ein Beispiel, wo dies eine Rolle spielt:
$$111x : 37x = (111x) : (37x) = 3$$
Aber
$$111\cdot x : 37\cdot x = ((111 \cdot x) : 37) \cdot x = 3x^2$$
}%% END TALS

\subsection*{Aufgaben}
\GESOAadB{23ff}{21., 22. und 24.}

\newpage
\subsection{Terme mit Namen}
Oft gibt man Termen Namen, um sie einfacher identifizieren und
bezeichnen zu können. So könnte \zB die Oberfläche einer
Konservendose mit $A$ (Area) wie folgt bezeichnet werden, wenn $r$ den
Radius bzw. $h$ die Höhe bezeichnen:

$$A(r, h) = r^2\pi + r^2\pi + 2r\pi{}h$$

Dabei ist $A$ der Name des Terms und $r$ bzw. $h$ sind die Parameter.
\vspace{3mm}
\begin{beispiel}{Werte einsetzen}{beispiel_terme_werte_einsetzen}
  Wir betrachten den Term

  $T({\color{red}a}, {\color{blue}x}) = 5{\color{red}a}{\color{blue}x} - {\color{red}a} + 7$.

  Nun gilt, dass für jeden Parameter im Term (hier ${\color{red}a}$
  bzw. ${\color{blue}x}$) jede Zahl eingesetzt
  werden kann.\leserluft{}

  $T({\color{red}2}, {\color{blue}-3}) = \LoesungsRaumLang{5\cdot{}{\color{red}2}\cdot{\color{blue}(-3)} - {\color{red}2} + 7}$

  Es können auch Terme anstelle der Parameter eingesetzt
  werden\footnote{Beachten Sie, dass beim Einsetzen von Termen in der Regel
  Klammern gesetzt werden müssen!}:\leserluft{}

  $T({\color{red}z-4}, {\color{blue}2y}) =
  \LoesungsRaumLang{5 \cdot{} {\color{red}(z-4)} \cdot {\color{blue}(2y)} - {\color{red}(z-4)} + 7}$
\end{beispiel}

\begin{bemerkung}{}{}
Achten Sie beim Ersetzen des Parameters durch das Argument auf die
Klammersetzung. Wenn nicht sicher: Immer Klammern um die Argumente
setzen, welche für die Parameter eingesetzt werden:

$$ a = z-4$$
$$ a  \rightarrow (z-4)$$
\end{bemerkung}

\begin{gesetz}{Einsetzen}{}
  Beim Einsetzen eines Term in eine Variable sind \textbf{immer} Klammern zu setzen!
\end{gesetz}
\newpage

\subsubsection{Übungsbeispiel}
$$T(b, y) = 7y^2 - 4by$$

Wir berechnen


$T(\underbrace{{\color{blue}s}}_b, \underbrace{{\color{green}-t}}_y) = $%%
\noTRAINER{.......................................................}%%
\TRAINER{$7\cdot (\underbrace{{\color{green}-t}}_{y})^2 - 4\cdot (\underbrace{{\color{blue}s}}_b) \cdot (\underbrace{{\color{green}-t}}_y) = 7t^2+4ts$}

und

$T(x, 2b) = $ \noTRAINER{.........................................................}\TRAINER{$7(2b)^2 - 4\cdot (x) \cdot (2b) = 28b^2 - 8bx =
  4b (7b-2x)$}
\subsection*{Aufgaben}



\aufgabenfarbe{Berechnen Sie die Werte der folgenden Terme und kontrollieren Sie anschließend Ihre Resultate mit dem Taschenrechner:}

\begin{tabular}{|c|c|}
  $-10^4$ & $\LoesungsRaum{-10\,000}$\\
  $(-10)^5$ & $\LoesungsRaum{-100\,000}$\\
  $(-100)^2$ & $\LoesungsRaum{+10\,000}$\\
  $x^6\textrm{, für } x=-1$ & $x^6= \LoesungsRaum{+1}$\\
  $-x^5\textrm{, für } x=-10$ & $-x^5= \LoesungsRaum{+100\,000}$\\
  $(-x)^3\textrm{, für } x=-2$ & $(-x)^3= \LoesungsRaum{+8}$\\
\end{tabular}


\TALSAadB{11}{8,9}
\GESOAadB{24ff}{26. a) c), 25., 27. a) 28., 29.}
\newpage
