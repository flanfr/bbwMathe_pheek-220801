%%
%% 2019 07 04 Ph. G. Freimann
%%
\newpage
\section{Wurzeln}\index{Wurzeln}
\sectuntertitel{Was soll's denn sein: Radi, Radieschen, Rande, Rübe, Räbe, Rettich, ...?}

\theorieGESO{5}{78}
%%%%%%%%%%%%%%%%%%%%%%%%%%%%%%%%%%%%%%%%%%%%%%%%%%%%%%%%%%%%%%%%%%%%%%%%%%%%%%%%%
\subsection*{Lernziele}

\begin{itemize}
\item Quadratwurzeln, Kubikwurzeln
\item allgemeine $n$-te Wurzeln
\item Rechengesetze
\item Rationale (gebrochene) Exponenten
\end{itemize}
\newpage




\subsection{Quadratwurzeln}\index{Quadratwurzel}%%

\begin{definition}{Quadratwurzel}{definition_quadratwurzel}
Mit $\sqrt{\vphantom{b} \,\,}$ bezeichnen wir die
Quadratwurzel\index{Quadratwurzel}.
\end{definition}


\begin{tabular}{p{11cm}c}
%% vphantom{b}: Höhe des Buchstabens «b»
Das Symbol $\sqrt{\vphantom{b} \,\,}$
bezeichnet ein stilisiertes {\textit{kleines}} {\huge{r}} und steht für
  (lat.) \textit{radix}, die Wurzel.

& \bbwGraphicRaise{-22mm}{3.5cm}{allg/alg/potenzen_wurzeln/img/rasiidli.png}\\

 \end{tabular}
 
\begin{gesetz}{}{}
Für $a$ in $\mathbb{R}_0^{+}$ gilt
$$\sqrt{a^2} = a$$
$$(\sqrt{a})^2 = a$$
\end{gesetz}

\GESO{
\begin{beispiel}{}{}
$$\sqrt{9^2}=9$$
$$\sqrt{9}^2 = 9$$
\end{beispiel}
}

\TALS{
Für $a$ in $\mathbb{R}$ gilt

$$\sqrt{a\cdot a} = |a|$$
$$\left(\sqrt{|a|}\right)^2 = |a|$$
}

\newpage
\subsection{Multiplikation, Division}

\begin{center}
\fbox{\parbox{5cm}{
\begin{tabular}{rcl}
$\sqrt{\vphantom{b}a}\cdot{\sqrt{b}}$ & $=$ & $\sqrt{\vphantom{b}a\cdot{b}}$\\


$\frac{\sqrt{\vphantom{b}a}}{\sqrt{b}}$ & $=$ & $\sqrt{\frac{\vphantom{b}a}{b}}$\\
\end{tabular}
}%% END parbox%%
}%% END fBox
\end{center}


\noTRAINER{$\sqrt{\vphantom{b}a}\cdot\sqrt{b} =$
\\
\mmPapier{3.2}}%% END noTRAINER
\TRAINER{${\color{blue}\sqrt{\mathstrut a}\cdot\sqrt{b}} = \sqrt{({\color{blue}\sqrt{a}\cdot\sqrt{b}})^2} =
  \sqrt{\sqrt{a}\cdot\sqrt{b}\cdot\sqrt{a}\cdot\sqrt{b}} =
  \sqrt{\sqrt{a}\cdot\sqrt{a}\cdot\sqrt{b}\cdot\sqrt{b}} =
  \sqrt{a \cdot b}$
\\
Die erste Gleichung gilt, wenn ich den ganzen Ausdruck links quadriere
und danach davon die Wurzel ziehe.
\vspace{1.5cm}
}%% end TRAINER


\begin{beispiel}{Partielles
Radizieren\index{radizieren!partielles}\index{partielles Radizieren}}{}
$$\sqrt{12\cdot{} a^3} = \sqrt{3\cdot{} 4 \cdot{} a \cdot{} a^2}
= \sqrt{3} \cdot{} \sqrt{4} \cdot{} \sqrt{a} \cdot{} \sqrt{a^2} = 2a\cdot{}\sqrt{3a}$$
\end{beispiel}

\subsection*{Aufgaben}
\GESOAadB{84ff}{1. g), 2. a) d), 3. a) c) e) f) g) h), 4. c) b) d) und
6. d)}

\GESO{Beachten Sie im Buch die Tipps auf Seite 79.}

\newpage



\subsection{$n$-te Wurzel}\index{n-te@$n$-te Wurzel}

\begin{verse}
\textit{«En-te Wurzel»}:
\bbwCenterGraphic{35mm}{allg/alg/potenzen_wurzeln/img/Ente_Wurzel}
\end{verse}

Wir wissen bereits, dass
$$3^2 = 9 \textrm{ heißt } \sqrt{9} = 3.$$



Doch wenn wir nun
$$x^3 = 1000$$
vor uns haben? Wie kommen wir auf das $x$?

\TNT{1.6}{Dazu gibt es die dritte Wurzel:
$$\sqrt[3]{1000} = 10$$
}


Was bedeutet $\sqrt[3]{8}$? Schätzen Sie und prüfen Sie nach (allenfalls auch mit dem Taschenrechner):

\TNT{1.6}{Verwenden Sie die n-te Wurzel Funktion des Taschenrechners.}



\newpage
\begin{definition}{$n$-te Wurzel}{definition_n_te_wurzel}
Mit $\sqrt[n]{a}$ bezeichnen wir die $n$\textbf{-te Wurzel} aus $a$; das heißt
$$\sqrt[n]{a} = x \Rightarrow x^n = a$$
\end{definition}

\begin{beispiel}{Dritte Wurzel}{}
$$\sqrt[3]{1000} = 10 \textrm{ , denn } 10^3 = 1000$$
\end{beispiel}

\begin{gesetz}{}{}
$$\sqrt[n]{a^n} = \left(\sqrt[n]a\right)^n = a$$
\end{gesetz}


\begin{definition}{}{}
Den Wurzelexponenten $n=2$ lassen wir üblicherweise weg:

$$\sqrt{a} := \sqrt[2]{a}$$
\end{definition}

\subsection*{Aufgaben}
\GESOAadB{85}{8. f) und 10. g)}

\GESO{\aufgabenfarbe{Aufgaben aus dem «Potenzen
Arbeitsblatt» \texttt{olat.bbw.ch} Kap. 1.5 Wurzeln}} 

\newpage


\subsection{Rationale Exponenten}\index{rationale Exponenten}

Wir repetieren:

\TALS{
$$(x^m)^n = x^{m\cdot n}\ \ \textrm{, }\ \ \left(\sqrt[n]{a}\right)^n
= a \textrm{ und } a^{m+n} =a^m\cdot{}a^n.$$
}
\GESO{
$$a^m\cdot{}a^n = a^{m+n}$$
}


\begin{beispiel}{}{}
Es gilt
$$\sqrt[3]{a} \cdot{}\sqrt[3]{a} \cdot{}\sqrt[3]{a} = a = a^1 =
a^{\frac13 +\frac13 +\frac13} = a^\frac13 \cdot{} a^\frac13 \cdot
a^\frac13$$
Somit ist
$$\sqrt[3]{a} = a^\frac13$$
\end{beispiel}


Für $a > 0$ gilt somit allgemein:

\begin{definition}{}{}
  $$a^{\frac{1}{n}} := \sqrt[n]{a}$$
\end{definition}

\TALS{
Denn: $\sqrt[n]{a} = \left(\sqrt[n]{a}\right)^1 =
\left(\sqrt[n]{a}\right)^{n\cdot{}\frac{1}{n}} =
\left(\left(\sqrt[n]{a}\right)^n\right)^\frac{1}{n} = a^\frac{1}{n}$
}

Es gelten die üblichen Potenzgesetze\GESO{\footnote{S. 81 \cite{marthaler17}}} nun auch für die $n$-ten
Wurzeln.


\begin{gesetz}{}{}
  $$\sqrt[m]{\sqrt[n]{a}} = \sqrt[m\cdot n]{a}=a^\frac1{mn} = \sqrt[n]{\sqrt[m]{a}}$$
\end{gesetz}


$$\sqrt[m]{\sqrt[n]{a}} =
\sqrt[m]{a^\frac{1}{n}}
= (a^{\frac{1}{n}})^{\frac{1}{m}} =
a^{\frac{1}{n}\cdot\frac{1}{m}} = a^{\frac{1}{m\cdot n}} =
\sqrt[m\cdot n]{a}$$
\newpage



\begin{beispiel}{}{}
$\sqrt[3]{\sqrt[2]{64}} =
    \sqrt[3]{8} = 2 = \sqrt[6]{64} = \sqrt[3\cdot2]{64}$
\end{beispiel}

\GESO{Die Rechengesetze für $n$-te Wurzeln sind im Buch
  \cite{marthaler17} auf Seite 82 zu finden.}

Rechenbeispiel. Schreiben Sie unter eine Wurzel:
$$ab^\frac{-3}4$$

\TNT{10}{
$$ab^\frac{-3}4 = a\cdot{}b^\frac{-3}4 =
a\cdot{} \left(\frac1b\right)^\frac34
=a\cdot{} \left(\left(\frac1b\right)^3\right)^\frac14 =
a\cdot{}\sqrt[4\,\,]{\left(\frac1b\right)^3}
= \sqrt[4\,\,]{a^4 \left(\frac1b\right)^3} =
\sqrt[4\,\,]{\frac{a^4}{b^3}}$$
}%% END TNT


\newpage

\subsection{Aufgaben}
\TALSAadB{???}{???}
\GESOAadB{85 ff}{[Wenn möglich immer mit Taschenrechner] 9. g),
11. a) e)
  g) i), 12. a) b) d), 13. a) b), 16. a) c) d), 17. a) b),
  19. a) d) i), 20. a) c), 21. a) c), 22. a) c) e), 24. a)}
\aufgabenfarbe{  ... und Aufgabenblatt Potenzgesetze
  im \texttt{olat.bbw.ch}, dort Kap. 1.6 rationale Exponenten}
\newpage
