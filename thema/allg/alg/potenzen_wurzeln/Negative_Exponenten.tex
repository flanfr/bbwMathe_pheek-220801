\subsubsection{Negative Exponenten}
\sectuntertitel{Nicht für alle ist die Potenzrechnung \textit{positiv}.}
Wir kennen bereits das Rechengesetz für positive Exponenten:

$$a^5\cdot{}a^2 = a^{5+2} = a^7$$

Sinnvoll wäre die folgende Erweiterung auf negative
Exponenten:\\\TRAINER{Damit die Rechengesetze weiterhin gelten.}
$$a^5\cdot{}a^{-2} = a^{5+(-2)} = a^3$$

Dividieren wir obige Gleichung beidseitig durch $a^5$, so erhalten wir
folgende sinnvolle Definition:

\TNT{3.2}{
Wir wollen erreichen, dass gilt: $a^5\cdot{}a^{-2} \stackrel{!}{=}a^3$

Wir dividieren durch $a^5$ beidseitig: $a^{-2} = \frac{a^3}{a^5} = \frac{1}{a^2}$

Mit dieser Herleitung von $a^{-2}= \frac1{a^2}$ gilt: $a^5\cdot{}a^{-2} = a^5\cdot{}\frac1{a^2} = a^3$
}%% END TNT

\begin{definition}{}{}
$$a^{-n} := \frac{1}{a^n}$$
\end{definition}

\begin{bemerkung}{}{}
$$\frac{1}{a^n}= 1 : \underbrace{a : a : a : ... : a}_{n \textrm{\ Divisoren}}$$
\end{bemerkung}

\begin{gesetz}{}{}
$a^{-n} = \left(\frac1a\right)^n$
\end{gesetz}
Begründung:

\TNT{4.4}{$a^{-n} = \frac1{a^n}
= \frac1{\underbrace{a\cdot{}a\cdot{}a\cdot{}...\cdot{}a}_{n \textrm{\ Faktoren}}}
= \underbrace{\frac1a\cdot{}\frac1a\cdot{}\frac1a\cdot{}...\cdot{}\frac1a}_{n \textrm{\ Faktoren}}
= \left(\frac1a\right)^n$
}%% END TNT
\newpage
Ganz analog gilt:

\begin{gesetz}{}{}
$\frac{1}{a^{-n}} =a^n$
\end{gesetz}
Begründung:
\TNT{8}{


Beispiel:

$$\frac1{10^{-3}} = \frac1{0.001} = 1000 = 10^3$$

\TALS{TALS:}
Beweis: Definition hinschreiben und auf beiden Seiten den Kehrwert bilden:

$$a^{-n} = \frac1{a^n}$$

$$\frac1{a^{-n}} = a^n$$

\vspace{2cm}
}%% END TNT



\begin{gesetz}{}{}
$\left(\frac{1}{a}\right)^{-n}=a^n$
\end{gesetz}
Begründung:
\TNT{8}{
Zahlenbeispiel (erster Schritt nach Definition):
$$\left(\frac1{10}\right)^{-3}  = \frac1{\left(\frac1{10}\right)^3}  = \frac1{0.001} = 1000 = 10^3$$

\TALS{TALS:}

Beweis: Nach Definition gilt für alle $n$:

$$x^{-n} = \frac1{x^n}$$

Somit gilt es auch, wenn wir anstelle von $x$ den Term $\frac1a$ einsetzen:

$$\left(\frac1a\right)^{-n} = \frac1{\frac1{a^n}} = a^n$$


\vspace{2cm}

}%% END TNT




\newpage
Rechenbeispiel:

Wurm «Wurli» schaft 3 cm pro Sekunde (= $3 \cdot{} 10^{-2} $ m pro Sekunde). Wie lange braucht «Wurli» für 12 m?
\TNT{2.4}{
$$t = \frac{s}v = \frac{12[ \textrm{m}]}{3 \frac{[\textrm{cm}]}{[\textrm{s}]}} =   \frac{12[ \textrm{m}]}{3\cdot{}10^{-2}\frac{[\textrm{m}]}{[\textrm{s}]}} = 4\cdot{} 10^2 [\textrm{s}] = 400 [\textrm{s}]\approx 6-7 Min.$$
}%% END TNT




Und ebenso für beliebige Brüche:

\begin{gesetz}{}{}
$\left(\frac{a}{b}\right)^{-n} = \left(\frac{b}{a}\right)^{+n}$
\end{gesetz}

    
    \TALS{ Begründung

      \TNT{2.4}{$\left( \frac{b}{a} \right)^n  =
       \left(b \cdot{} \frac{1}{a} \right)^n =
       b^n \cdot \left(\frac{1}{a}\right)^n =
       \left(\frac{1}{b}\right)^{-n} \cdot{} a^{-n} =
       \left(\frac{1}{b}\cdot{}a\right)^{-n} =
       \left(\frac{a}{b}\right)^{-n} 
       $}} %% END TNT END TALS

    \GESO{ Begründung

      \TNT{2.4}{$\left( \frac{5}{2} \right)^3  =
       \left(5 \cdot{} \frac{1}{2} \right)^3 =
       5^3 \cdot \left(\frac{1}{2}\right)^3 =
       \left(\frac{1}{5}\right)^{-3} \cdot{} 2^{-3} =
       \left(\frac{1}{5}\cdot{}2\right)^{-3} =
       \left(\frac{2}{5}\right)^{-3} 
       $}} %% END TNT END GESO

%%$$\left(\frac{1}{a}\right)^{-n} = \frac{1}{\left(\frac{1}{a}\right)^n} = a^n$$
\newpage



\subsubsection{Null}

Ebenso müsste $a^5 \cdot a^0 = a^{5+0} = a^5$ gelten. Dies ist aber
nur möglich, wenn wir $a^0 := 1$ definieren ($a\ne 0$).

\begin{definition}{Exponent Null}{} Für alle Basen
$a \in \mathbb{R}\backslash\{0\}$ definieren wir:
\begin{center}
\fbox{$a^0 := 1$}
\end{center}
\end{definition}

Zweite Begründung wenn die Rechengesetze gelten sollten: $a^0 = a^{1-1} = \frac{a^1}{a^1} = 1$

%\textbf{Rechengesetze zusammengefasst:}

%\begin{itemize}
%\item  $\frac{a^m}{a^n} = a^{m-n}$ (Dies gilt auch wenn $n > m$.)
 
%\item $a^{-n} := a^{0-n}=\frac{a^0}{a^n} = \frac{1}{a^n} = \left(\frac{1}{a}\right)^n$

%\item
%$\left(\frac{1}{a}\right)^{-n} = \frac{1}{\left(\frac{1}{a}\right)^n} = a^n$


%\item $\left(\frac{a}{b}\right)^{-n} = \left(\frac{b}{a}\right)^{+n}$ gilt daher auch. 
%\end{itemize}

\begin{rezept*}{«Kielholen»}{}{}
Exponenten vertauschen ihr Vorzeichen beim Übertreten des Bruchstrichs:
$$\frac{a^{-3}b^2}{c^5d^{-6}} = \frac{b^2d^6}{a^3c^5}$$
\end{rezept*}



\subsection{Aufgaben}

\TALS{Potenzen:}\TALSAadB{32ff}{Von Hand: 79. a) 82. a) 83. b) 86.b)
91. a) l) 94. c) f) 96. h) 101. a)\\
Von Hand \textbf{und} Taschenrechner (TR): 103. a)\\
Von Hand: 103. b)\\
TR: 103. c) 106. h)}

\GESOAadB{67ff}{15., 18. c), 19. b), 20. h), 26. b), 31. b),
  38. c) e), 41. e), 43. c), 44. d) e) f) h) i), 48. a) b), 49. a) c)}

Optional: \GESOAadB{72ff}{51. (Koch)}

\GESO{\aufgabenfarbe{Blatt zu Potenzgesetzen im \texttt{olat.bbw.ch},
dort Kap. 1.3 und 1.4}}

\newpage
