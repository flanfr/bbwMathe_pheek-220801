\newcommand{\aaaa}{a\cdot a \cdot a \cdot{} ... \cdot a}
\newcommand{\bbbb}{b\cdot b \cdot b \cdot{} ... \cdot b}

\newpage
\subsection{Potenzen, Definitionen und Gesetze}\index{Potenzen}

\subsubsection{Einführungsbeispiele}
Die sieben verschiedenen schweizer Münzen werden nacheinander geworfen
und es wird notiert, welcher Wurf Zahl oder «Kopf» aufweist. Dieses
Experiment wird einige Male wiederholt, bis man sich die Frage stellt:
Wie viele mögliche Ausgänge hat das Experiment?

\TNT{4.0}{$$2^7 = 128$$}

Berechen Sie
$$5^3$$

\TNT{2.8}{125}

Vereinfachen Sie:

$$a^5\cdot(ab)^3\cdot{}(2c)^{2+3}$$

\TNT{3.2}{$$32a^8b^3c^5$$}

\newpage

\begin{definition}{Potenz}{}\index{Potenz}
Unter der $n$-ten \textbf{Potenz} verstehen wir eine $n$-malige Multiplikaiton mit demselben Faktor:
$$a^2 = a\cdot a$$
$$a^n = \underbrace{\aaaa}_{n\, \textrm{Faktoren}}$$
\end{definition}
Dabei bezeichnen


\bbwCenterGraphic{9cm}{allg/alg/potenzen_wurzeln/img/Potenzbegriff.png}

%\begin{itemize}
% \item Potenz\index{Potenz}: $a^n$
% \item Exponent\index{Exponent}: $n$
% \item Basis\index{Basis}: $a$
%\end{itemize}


\subsubsection{gleiche Basis}\index{Potenzen!Gesetze}

\begin{itemize}
 \item \fbox{$a^m\cdot a^n = a^{m+n}$}

   Begründung:

   \TNT{2.4}{
   \TALS{$a^m\cdot a^n =
    \underbrace{{\underbrace{\aaaa}_{m\textrm{-mal}}}\cdot{\underbrace{\aaaa}_{n\textrm{-mal}}}}_{m+n\textrm{-mal}}
    = a^{m+n}$}
   \GESO{$7^3\cdot{} 7^5 = (7\cdot{}7\cdot{}7)\cdot{}(7\cdot{}7\cdot{}7\cdot{}7\cdot{}7)=7^8$}%%
}%% END TNT

   
   \item \fbox{$a^m : a^n = \frac{a^m}{a^n}= a^{m-n}$}
     
     Begründung:

     \TNT{2.4}{%
     \TALS{$a^m :  a^n =
    \underbrace{{\underbrace{\aaaa}_{m\textrm{-mal}}} : {\underbrace{\aaaa}_{n\textrm{-mal}}}}_{m+n\textrm{-mal}}
    = a^{m-n}$}
     \GESO{
  Kürzen: 
   $7^5 :  7^3 = \frac{7^5}{7^3} =
  \frac{7\cdot{}7\cdot{}7\cdot{}7\cdot{}7}{7\cdot{}7\cdot{}7} = 7^2 =
  7^{5-3}$}
     }%% END TNT

  
\item \fbox{$(a^n)^m = a^{n\cdot{}m} = (a^m)^n$}

  Begründung:

  \TNT{2.4}{Beispiel $(a^4)^3$ vorrechnen \vspace{12mm}}%%
  
\end{itemize}
\newpage

%%%%%%%%%%%%%%%%%%%%%%%%%%%%%%%%%%%%%%%%%%%%%%%%%%%%%%%%%%%

\subsubsection{gleiche Exponenten}

\begin{itemize}
\item \fbox{$a^n \cdot{} b^n  = (ab)^n$}
  
  Begründung

  \TNT{2.4}{
    \TALS{$a^n\cdot b^n
= \underbrace{\aaaa}_{n\textrm{-mal}}\underbrace{\bbbb}_{n\textrm{-mal}}
= \underbrace{ab\cdot ab\cdot ... \cdot ab}_{n\textrm{-mal}} =
(ab)^n$}%%
   \GESO{\zB $7^3 \cdot 5^3 = (7\cdot{}7\cdot{}7) \cdot{}
     (5\cdot{}5\cdot{}5) = (7\cdot{}5)\cdot{} (7\cdot{}5)\cdot{} (7\cdot{}5) =(7\cdot{}5)^3$}
}%% end TRAINER
   
\item \fbox{$a^n : b^n = (a:b)^n$} bzw. \fbox{$\frac{a^n}{b^n} = \left(\frac{a}{b}\right)^n$} (Gilt ganz analog.)
\end{itemize}

%%%%%%%%%%%%%%%%%%%%%%%%%%%%%%%%%%%%%%%%%%%%%%%%%%%%%%%%%%%%%%

\textbf{Aber Vorsicht}

$$a^3 \cdot{}  b^4 \ne ???$$
$$a^2 +        b^2 \ne ???$$
\newpage

\subsubsection{Theorieaufgaben}

Lösen Sie:

%\TRAINER{
%$a^4 \cdot a^5 = (a\cdot a\cdot a\cdot a) \cdot (a\cdot a\cdot a\cdot a\cdot a) = a^9 = a^{4+5}$}%%

 $a^4\cdot a^5 =\LoesungsRaum{(a\cdot a\cdot a\cdot a) \cdot (a\cdot a\cdot a\cdot a\cdot a) = a^9 = a^{4+5}}$

 $r^6 : r^4 =\LoesungsRaum{r^{6-4}=r^2}$

 $a^4\cdot b^4 =\LoesungsRaum{(a\cdot a\cdot a\cdot a) \cdot (b\cdot b\cdot b\cdot b) = (a\cdot{}b)^4 = (ab)^4}$

$p^5 : q^5 = \LoesungsRaum{(p:q)^5=\left(\frac{p}{q}\right)^5}$

$(s^4)^2 = \LoesungsRaum{s^{4\cdot{}2} = s^8}$

Exponentenvergleich: Finde $x$:
$(r^x)^{10}\cdot{}r^{22} = r^{72}$, $x=\LoesungsRaum{5}$



\subsection*{Aufgaben}
\GESOAadB{66ff}{5. a) c) und 6. a) b)}


\GESO{Gleiche Basis:}

\GESOAadB{69ff}{25. b) f), 29. b) f), 30. b) c) d) e), 33. e) f)}


\GESO{Potenzen von Potenzen}

\GESOAadB{70ff}{39. b) d) e)}


\GESO{Gleiche Exponenten:}

\GESOAadB{70ff}{40. a) c) f), 42. f) g) h) i)}


\GESO{Erste Exponentialgleichungen}

\GESOAadB{71}{44. a) b) c)}

\GESO{\aufgabenfarbe{Aufgabenblatt \texttt{olat.bbw.ch} Potenzen
    Arbeitsblatt: Kap. 1.1. und 1.2}}

\TALS{Aufgaben noch ohne negative Exponenten.}
\TALSAadB{32ff}{79. a), 90. d) l), 91. l), 92. a) d), 93. j), 94. c) f), 95. a),
  101. a), 103. a) b) c) und 105. a) g) h)}
\newpage
