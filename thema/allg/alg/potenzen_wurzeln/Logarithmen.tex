%%
%% 2019 07 11 Ph. G. Freimann
%%

\section{\GESO{allgemeine }Logarithmen}\index{Logarithmus}
\sectuntertitel{Wer hat noch andere Basen?}
%%%%%%%%%%%%%%%%%%%%%%%%%%%%%%%%%%%%%%%%%%%%%%%%%%%%%%%%%%%%%%%%%%%%%%%%%%%%%%%%%
\subsection*{Lernziele}

\begin{itemize}
\item andere Basis
\item Basiswechsel
\item Logarithmengesetze II
\end{itemize}
\newpage

\subsection{Andere Basis als Basis 10}\index{Logarithmus!beliebige Basis}

Wir können bereits $$x^5=32$$ lösen, indem wir beidseitig die
5. Wurzel ziehen:

$$x = \LoesungsRaum{\sqrt[5]{32}} = \LoesungsRaum{\sqrt[5]{2^5}} = 2$$

Ebenso können wir Probleme mit der Unbekannten im Exponenten lösen,
wenn die Basis 10 beträgt: Das Problem $10^x = 1000$ können wir mit
Logarithmen lösen:

$$\lg(1000) = \LoesungsRaumLang{\lg(10^3) = 3}$$

Zuletzt wissen wir bereits, dass

$$10^{\lg(1000)} = \LoesungsRaum{1000} \textrm{ oder allgemein }$$

\begin{center}
\fbox{$10^{\lg(p)} = p$}
\end{center}
\newpage


Aber wie lösen wir so ein Problem:

$$2^x = 128$$

\textbf{Die harte Tour}\footnote{Müssen Sie nicht herleiten können,
  docch daraus ergibt sich ein brauchbares Gesetz.}:\\

Nach obigem Gesetz können wir $2$ und $128$ anders schreiben:

$$2= 10^{\lg(2)} \textrm{ und } 128 = 10^{\lg(128)}$$

Setzen wir dies nun in $2^x=128$ ein, so erhalten wir

$$\left(10^{\lg(2)}\right)^x = 10^{\lg(128)}.$$


Wegen den Gesetzen zu Potenzen gilt:

$$10^{\lg(2)\cdot{}x} = 10^{\lg(128)}$$

  Durch Exponetenvergleich erhalten wir:

  $$\lg(2)\cdot{}x = \lg(128)$$

  Und damit erhalten wir:

  $$x = \frac{\lg(128)}{\lg(2)} = \LoesungsRaum{7}$$


  \textbf{... die sanfte Tour}:\\

 Das Problem $$2^x=128$$ unterscheidet sich vom bisher bekannten nur
 dadurch, dass die Basis nicht 10, sondern 2 ist. Dies kann der
 Taschenrechner auch: 

 $$2^x = 128 \Longleftrightarrow  \log_2(128) = x = \LoesungsRaum{7}$$
  
\begin{gesetz}{Basiswechsel}{}\index{Basiswechsel!Logarithmen}
  Für $a\in\mathbb{R}^{+}\backslash\{0,1\}$ und $p>0$ gilt:
  $$\log_a(p) = \frac{\lg(p)}{\lg(a)}\TALS{ = \frac{\log_b(p)}{\log_b(a)}}$$
\end{gesetz}

%Begründung:
%
%\TNT{2.4}{
%\GESO{$\log_5(32)=x \Longleftrightarrow 5^x=32$\\
%    Daher: $\log(5^x) = \log(32) \Longleftrightarrow
%    x\log(5)=\log(32) \Longleftrightarrow x=\frac{\log(32)}{\log(5)} =
%    \log_5(32)$} %% END GSEO
%\TALS{Beweis: $\log_a(b)=x \Longleftrightarrow a^x=b$\\
%    Daher: $\log(a^x) = \log(b) \Longleftrightarrow
%    x\log(a)=\log(b) \Longleftrightarrow x=\frac{\log(b)}{\log(a)} =
%    \log_a(b)$} %% END TALS
%}%% END TNT

%\GESO{
%  Drücken Sie $\log_3(17)$ durch den Zehnerlogarithmus $\lg{}$ aus:
%  \TNT{2}{$$\log_3(17) = \frac{\lg(17)}{\lg(3)}$$}%% END TNT
%}%% END GESO
\newpage


\subsubsection{Weitere Logarithmengesetze}\index{Gesetze!Logarithmen}

 \begin{definition}{Logarithmus zu allgemeiner Basis}{}
   $$a^b=c \Longleftrightarrow \log_a(c) = b$$
   \end{definition}

 
 \begin{bemerkung}{}{}
   $$\log_a(a^7) = \LoesungsRaum{7}$$
\end{bemerkung}

 \begin{bemerkung}{}{}
   $$\log_a(a) = \LoesungsRaum{1} \textrm{ , denn } a^1 = a$$
\end{bemerkung}

 \begin{bemerkung}{}{}
   $$\log_a(1) = \LoesungsRaum{0} \textrm{ , denn } a^0 = 1$$
\end{bemerkung}

\begin{bemerkung}{Logarithmus zu beliebiger Basis}{}
  $$\log_a(a^x) = x$$
\end{bemerkung}


\TNT{2.4}{Beispiel $2^4 = 16$, d.\,h. $\log_2(16)=4 $.\vspace{1cm}}


\GESO{Weitere Rechengesetze im Buch \cite{marthaler17} ab Seite 97:}

 \newpage

 
 \subsubsection{Potenzregel}

 Wir wissen bereits, dass

 $$\left(10^m\right)^n = 10^{m\cdot{}n}$$

 und

 $$\lg(10^x) = x$$

 Daraus erhalten wir


\TNT{3.2}{
  Voraussetzung:
  \begin{itemize}
  \item $(I)\,\,\,\,\,\,\,\, \, (10^m)^n = 10^{m\cdot{}n}$
  \item $(II)\,\,\,\,\,\,\, \lg(10^x) = x$
  \end{itemize}
  Beweis am Beispiel:
  $\lg(100^7) = \lg((10^2)^7) {\stackrel{(I)}{=}} \lg(10^{2\cdot{}7}) =
  \lg(10^{(7\cdot{} 2)}) {\stackrel{(II)}{=}} 7\cdot{} 2 {\stackrel{(II)}{=}} 7\cdot{}
  \lg(10^2) = 7\cdot{}\lg(100)$
}%% END TNT

 
 Daraus erhalten wir sofort die Potenzregel:

 \begin{gesetz}{Potenzregel}{}
   Für alle positiven $a$ und $u$ gilt:

   $$\log_a(u^k) = k\cdot{} \log_a(u)$$
   \end{gesetz}
\newpage




\TALS{\
  \begin{gesetz}{}{} 
  $$\log_a(u\cdot{}v)=\log_a(u) + \log_a(v)$$
  $$\log_a\left(\frac{u}{v}\right)=\log_a(u) - \log_a(v)$$
  \end{gesetz}%%
  \TNT{3.2}{Beweis s. Video ph. freimann in den Wikis}
} %% END TALS

\newpage


\GESO{
  \begin{bemerkung}{}{} 
  $$\log_a(u\cdot{}v)=\log_a(u) + \log_a(v)$$
  \end{bemerkung}%%
  \TNT{2.4}{Bsp.:  $\log_2(8\cdot 32) = \log_2(2^3 \cdot 2^5) =
    \log_2(2^{3+5}) = 3+5 = \log_2(2^3) + \log_2(2^5) = \log_2(8) + \log_2(32)$.\vspace{1cm}}
} %% END GESO





\TALS{
\begin{bemerkung}{}{}
$$\log_a\left(\frac{1}{v}\right)=-\log_a(v)$$
\end{bemerkung}

\TNT{2.4}{Bsp.:  $\log_2(8^2) = \log_2((2^3)^2) = \log_2(2^6) = 6 = 2
  \cdot  3 = 2 \cdot \log_2(2^3) = 2 \cdot \log_2(8)$.\vspace{1cm}}%% END TNT
}%% END TALS

\TALS{
\begin{bemerkung}{}{}
$$a^x = b^{x\cdot{}\log_b(a)}$$
\end{bemerkung}
\TNT{2.4}{Wir ersetzen im 2. Gesetz $y$ durch $a^x$ und $a$ durch $b$:
  $$y = b^{\log_b(y)} \longrightarrow  a^x = b^{\log_b(a^x)} =
  b^{x\cdot{}\log_b(a)}$$} %% END TNT
}%% END TALS
\newpage



%% Zweierlog ist nur TALS
\TALS{\subsubsection{Spezielle Basen}
Der Zweierlogarithmus wird vor allem in der Informatik benutzt. Wie viel Bit brauche ich, um 200 Zustände abzubilden?

$2^x = 200$

Lösung:


\TNT{2.4}{
$\log_2(200) \approx 7.64$, d.\,h. ich benötige 8 Bit, um 200 Zustände abzubilden.
}%%
}%% END TALS


\newpage


\subsubsection{Die Eulersche Konstante}\index{$e$!Eulersche Konstante}

Die Basis $e$ ($e$ = Eulersche\footnote{Leonhard Euler (1707-1783)} Konstante $e\approx 2.7182817246$) wird vor allem bei Wachstums- und Zerfallsprozessen verwendet.

Die Zahl $e$ ist in mehrerer Hinsicht spannend:

\paragraph{Logarithmus Naturalis:}\index{Logarithmus Naturalis} Da die
Zahl $e$ bei exponentiellen Prozessen eine sehr große Rolle spielt,
darf die Eulersche Zahl $e$, $e^x$ und $\log_e(x)$ auch auf keinem Rechner fehlen. Der $\log_e()$ hat dabei sogar einen eigenen Namen erhalten:
$\ln()$ steht für «Logarithmus Naturalis».

\begin{center}
  \fbox{$\ln() := \log_e()$ = \textit{\textbf{Logarithmus Naturalis}}}
\end{center}

\paragraph{Logarithmentabellen}
Wenn ich auf einem Taschenrechner lediglich $\ln()$ und $e^x$ zur Verfügung habe\footnote{Oder ich habe, wie früher, nur eine Logarithmentabelle.} und dennoch $\log_a(c)$ oder $a^x$ berechnen will, so kann ich dies einfach mit einer Transformation zur Basis $e$ vollbringen:

$$a^x = e^{x\cdot{}\ln(a)}$$
$$\log_a(c) = \frac{\ln(c)}{\ln(a)}$$

\newpage



%%\TALS{(\cite{frommenwiler17alg} S.??? (Kap. ???))}
%%\GESO{(\cite{marthaler17}       S.??? (Kap. ???))}

\subsection*{Aufgaben}
\TALSAadB{???}{???}


\GESO{Andere Basis (Vieles ist mit dem Taschenrechner lösbar):}

\GESOAadB{103ff}{8. a) b) c) f), 9., 10. a), 13. a) f), 14. a) g),
  15. a) c) d) e) f), 16. a) c) d) e) und 17. a) b)}

\GESOAadB{106}{33.}

\GESO{Logarithmus naturalis:}
\GESOAadB{104}{17. c) und 18. g) h) i)}
