%%
%% 2019 07 11 Ph. G. Freimann
%%
\newpage
\section{Grundoperationen}\index{Grundoperationen}
\sectuntertitel{Im Grunde ganz einfach?}

\theorieTALS{15}{1.2}
\theorieGESO{28}{2}
%%%%%%%%%%%%%%%%%%%%%%%%%%%%%%%%%%%%%%%%%%%%%%%%%%%%%%%%%%%%%%%%%%%%%%%%%%%%%%%%%
\subsection*{Lernziele}
\begin{itemize}
\item Addition\index{Addition}, Subtraktion\index{Subtraktion}
\item Multiplikation\index{Multiplikation}
\item Distributivgesetz\index{Distributivgesetz},
  Assoziativgesetz\index{Assoziativgesetz},
  Kommutativgesetz\index{Kommutativgesetz}
\end{itemize}
\newpage

\subsection{Addition und Subtraktion}\index{Addition}\index{Subtraktion}
\begin{beispiel}{}{}
  $$x^2-((x^3-x^2)-(-(-x+x^2)-(x^2-x^3)))$$
\end{beispiel}

\TNT{10}{\bbwCenterGraphic{14cm}{allg/alg/img/KlammernLoesen.png}}%% END TNT


Regeln:
\begin{itemize}
\item Klammern werden von innen nach außen aufgelöst.
\item Gleichwertige Operationen\footnote{Zum Beispiel alles Minus und Plus oder aber zum Beispiel alles \textit{Punkt}-Operationen (Produkt/Quotient).} werden von links nach rechts \textit{geklammert}:
  $$10-4+5 = (10-4) + 5 \ne 10-(4+5)$$
\item Negative Vorzeichen vor Klammern wechseln die Vorzeichen der
      Summanden innerhalb der Klammer.
\item Es können nur gleiche Variable (bzw. Produkte von Faktoren)
      addiert (bzw. subtrahiert) werden.
\end{itemize}

\newpage

\subsection*{Aufgaben}
\GESOAadB{34}{1. a) c) e) g), 2. a) e), 3. b), 5. a) f) und 6. d) h)}


\newpage
\subsection{Multiplikation}\index{Multiplikation}
\TALS{Theorie im Buch \cite{frommenwiler17alg} S. 16 Kap. 1.3}

Für die Addition und die Multiplikation gelten die drei folgenden
Gesetze:



\begin{gesetz}{}{}
\begin{itemize}
\item  Assoziativgesetz:
  $a+(b+c) = (a+b) +c$ und $a\cdot(b\cdot{}c) = (a\cdot b)\cdot c$
\item Kommutativgesetz:
  $a+b = b+a$ und $a\cdot b = b \cdot a$
\item Distributivgesetz\footnote{lat. \textit{\textbf{distribuere}} = verteilen}:\\
  $a\cdot (b+c) = a\cdot b + a\cdot c$\\
  $a\cdot (b-c) = a\cdot b - a\cdot c$\\
  $(a+b)\cdot c = ac + bc$\\
  $(a-b)\cdot c = ac - bc$\\
  $(a+b):c = a:c + b:c$\\
  $(a-b):c = a:c - b:c$\\
  
  \end{itemize}
\end{gesetz}

\newpage

\subsection{Ausmultiplizieren}\index{ausmultiplizieren}
Beim Ausmultiplizieren wird jeder Summand in der Klammer mit dem
Faktor vor (bzw. nach) der Klammer multipliziert.
\begin{beispiel}{}{}
  $4\cdot (x + 5) = 4\cdot x + 4\cdot 5 = 4x + 20$
\end{beispiel}

\begin{beispiel}{}{}
  $$(x + 7)\cdot(8-y) = x\cdot(8-y) + 7\cdot(8-y) = 8x-xy+56-7y$$
  oder
  $$(x + 7)\cdot(8-y) = (x+7)\cdot 8 - (x+7)\cdot y = 8x+56-xy-7y$$
\end{beispiel}
\newpage

\subsection{Achtung}
Auch wenn die beiden folgenden Ausdrücke sehr ähnlich aussehen, so
handelt es sich beim ersten um eine \textbf{Differenz} und beim zweiten um
ein \textbf{Produkt}!

$$10-(x-4) =\LoesungsRaum{14-x}$$

$$-10(x-4) = \LoesungsRaum{-10x + 40}$$

Dieser Unterschied wird auf dem Taschenrechner besonders gut deutlich:

\tiprobutton{7}\tiprobutton{minus}\tiprobutton{3}  \tiprobutton{enter}  $ 7 - 3 = 4$

\vspace{3mm}
\tiprobutton{7}\tiprobutton{neg}\tiprobutton{3} \tiprobutton{enter}  $ 7\cdot{}(-3) = -21$

\subsection*{Aufgaben}

\TALSAadB{19ff}{33a b) c) h) l)
34a) b)
35a) b) d) k)
41c)
43b)
44}%% end TALS

\GESOAadB{34ff}{8. a) d)
f) g) 9. c) 12. a) c) e) 13. c) d) 14. g) 16. b) 17. a) 18. d)}

\newpage
\subsection{Minus mal Minus (optional)}
Wir können uns vorstellen, dass $3 \cdot (-4)$ dasselbe ist wie $(-4) + (-4) + (-4)$. Daher gilt
\begin{itemize}
\item $3 \cdot 4 = 12$
\item $3 \cdot (-4) = (-12)$
\item $(-3) \cdot 4 = (-12)$
\end{itemize}
Warum soll aber $(-3)\cdot(-4)$ gleich $+12$ sein?

Hier einige Erklärungsversuche:


\paragraph{Negativer Krankheits-Befund}
Wer negativ auf einen schlimmen Virus- oder Bakterienbefall getestet wurde, kann
die sich doch in einer positiven Situation sehen.

\paragraph{Vorzeichen:} Sehen wir die Zahl $(-4)$ als Gegenzahl von $4$, so können wir auch die Gegenzahl der Gegenzahl betrachten:
$4 = -(-4) = -(1\cdot(-4)) = (-1)\cdot(-4)$. Somit ist $(-1)\cdot(-4) = +4$.

\paragraph{Rechengesetze einhalten:} Wir versuchen den Rechengesetzen, die wir von den positiven Zahlen her kennen, Allgemeingültigkeit zu verleihen, dann müssen sie auch für die negativen Zahlen gelten.
Somit ist
$$(-4)          \cdot 0 = 0$$
Null anders schreiben:
$$(-4)    \cdot (3-3) = 0$$
Gegenzahl addieren:
$$(-4)\cdot(3 + (-3)) = 0$$
Distributivgesetz:
$$(-4)\cdot3 + (-4)\cdot (-3) = 0$$
Term $(-4)\cdot3$ ausrechnen:
$$-12 + (-4)\cdot (-3)= 0$$
Der Gleichung links und rechts 12 hinzufügen (addieren):
$$(-4)\cdot (-3) = 12$$

\paragraph{Schuldscheine abgeben:}
Eine anschauliche, aus dem Leben gegriffene, Analogie ist das «Verschenken von Schuldscheinen». Bezeichnen wir Banknoten als Kreditscheine, so besitzt eine 50er Note einen Wert von $+50$. So hat ein Schuldschein von $50.-$ Franken (oder Euros) den Wert $(-50)$.

\begin{tabular}{r@{}l|rl|r@{}l}
Geldwert & Gewinn/Verlust & Effekt\\
\hline\\
 + 50&.--   (Banknote)     &  3&  (erhalten)  & + 150&.-- (Gewinn)   \\
 - 50&.--   (Schuldschein) &  3&  (erhalten)  & - 150&.-- (Verlust)  \\
 + 50&.--   (Banknote)     & -3&  (abgeben)   & - 150&.-- (Verlust)  \\
 - 50&.--   (Schuldschein) & -3&  (abgeben)   & + 150&.-- (Gewinn)   \\
\end{tabular}
