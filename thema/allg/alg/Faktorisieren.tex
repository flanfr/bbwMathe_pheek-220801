%%
%% 2019 07 04 Ph. G. Freimann
%%
\newpage
\section{Faktorisieren}\index{Faktorisieren}

\theorieTALS{18}{1.3.2}
\theorieGESO{32}{2.2.3}

%%%%%%%%%%%%%%%%%%%%%%%%%%%%%%%%%%%%%%%%%%%%%%%%%%%%%%%%%%%%%%%%%%%%%%%%%%%%%%%%%
\subsection*{Lernziele}

\begin{itemize}
\item ausklammern
 \begin{itemize}
  \item gemeinsame Faktoren ausklammern
  \item Klammerausdrücke ausklammern
   \begin{itemize}
   \item mehrmaliges Ausklammern (Teilsummen)
  \item -1 ausklammern
  \end{itemize}
\end{itemize}
\item Binomische Formeln
\item Zweiklammeransatz
\TRAINER{\item \textit{(Polynomdivision: kein Lernziel)}}
\item Gemischte Anwendungen
\end{itemize}
\newpage


Beim Faktorisieren werden Summen/Differenzen in Faktoren
zerlegt.

\bbwCenterGraphic{11cm}{allg/alg/img/faktorisieren.png}\index{faktorisieren}\index{ausmultiplizieren}

\subsection*{Fernziel und Nutzen des Faktorisierens}
Betrachten wir die beiden folgenden Bruchterme:
 $$\frac{3x(a^2p - pb^2) - 7pa^2 + 7b^2p}{3a^2xy  + 3yb^2x + 6abyx - 7y(a^2 +b^2) - 14ayb }$$
und
$$\frac{p(3x-7)(a+b)(a-b)}{y(3x-7)(a+b)(a+b)}$$

Beide Terme können durch Termumformungen ineinander übergeführt werden. Mit anderen Worten: Die Terme sind identisch. Doch nur dem zweiten können wir auf einen Blick ansehen, dass er durch Kürzen stark vereinfacht werden kann:

$$\frac{p(a-b)}{y(a+b)}$$

%%%%%%%%%%%%%%%%%%%%%%%%%%%%%%%%%%%%%%%%%%%%%%%%%%%%%%%%%%%%%%%%%%%%%%%%%%%%%%%%%%%%%%%%%%%%%%%%%%%%%
\newpage

\subsection{Ausklammern}\index{ausklammern}
Die gängigste Methode des Faktorisierens ist das Ausklammern.

Das \textbf{Ausklammern}\footnote{Ausklammern wird auch «Vorklammern»\index{Vorklammern!Ausklammern} genannt.} ist die einfachste Umkehrung des Ausmultiplizierens.
Dabei werden gemeinsame Faktoren gesucht und vor (bzw. hinter) eine
neu hinzugefügte Klammer geschrieben.

\subsubsection{gemeinsame Faktoren ausklammern}
Suche gemeinsame Faktoren (Variable, Zahlen) in allen Summanden:

\begin{beispiel}{}{}
  $$4a^3 + 2ab -6a^2x$$
  $$=\noTRAINER{\hspace{7em}}\TRAINER{ {\color{green}2a}\cdot{\color{green}(}2a^2+b -3ax{\color{green})}}$$
\end{beispiel}

\begin{beispiel}{Woher kommt die Eins?}{}

$$4a^3 + 2a^2$$
$$=\noTRAINER{\hspace{7em}}\TRAINER{2a^2(2a + 1)}$$
Wenn wir die Faktoren zurück ausmultiplizieren, sehen wir, dass die
Eins (1) nicht fehlen darf. \textbf{Tipp}: Zur Probe immer zurück ausmultiplizieren.
\end{beispiel}

\GESO{\aufgabenfarbe{S. 38: Aufg. 37. 38. a) e) }}
\newpage


\subsubsection{Minus Eins ausklammern I}

\textbf{Vertauschte Differenz}\\

$$8-a = \LoesungsRaumLang{\mathbf{(-1)\cdot{}} (-8 + a) = -(a-8)}$$

Anwendung:
$$\frac{a-3}{3-a} = \frac{(-1)\cdot{}(-a+3)}{3-a} = \frac{(-1)\cdot{}(3-a)}{3-a} = \frac{-1}1 = -1$$

Aus jeder Summe (bzw. Differenz)
kann minus Eins $(-1)$ ausgeklammert werden, indem alle Vorzeichen der
Summanden umgedreht werden:

$$ -4 + x - 5\cdot(a-b) +c  =\LoesungsRaumLang{ {\color{ForestGreen} (-1)}\cdot{} ({\color{ForestGreen}+}4 {\color{ForestGreen}-}x {\color{ForestGreen}+}
5\cdot{(a{\color{red}-}b)} {\color{ForestGreen}-}c)}$$

Achtung oben: Das Vorzeichen bei $a-b$ ändert sich nicht, denn dies
ist nicht ein Vorzeichen der globalen Summanden!

\GESOAadB{39}{39. h) i)}

\GESOAadB{39}{40. a)}
\newpage


\subsubsection{Klammerausdrücke}\index{Klammerausdruck ausklammern}
Anstelle einfacher Monome können auch ganze Klammerausdrücke
ausgeklammert werden:
$$7x{\color{green}(3+b)} + 5z{\color{green}(b+3)}$$
$$=(7x+5z){\color{green}(3+b)}$$

Tipp: Geben Sie dem Klammerausdruck einen Namen\footnote{Dieser
chinesische Rechentrick löst manches {\color{green}A} bzw. {\color{green}Aha}-Erlebnis aus!}: ${\color{green}A} = {\color{green}(3+b)}$, dann liest sich der
Term wie folgt:
$$7x{\color{green}(3+b)} + 5z{\color{green}(b+3)} = 7x\cdot{\color{green}A}
+ 5z\cdot{\color{green}A} {\stackrel{\textrm{a.}}{=}}
(7x+5z)\cdot{\color{green}A} = (7x+5z){\color{green}(3+b)}$$

\begin{beispiel}{}{}
$$6b(5+x) - x- 5$$
\TNT{5.2}{
$$6b(5+x) -(x+5)$$
$$6b(5+x) -1(x+5)$$
$$6b(x+5) -1(x+5)$$
$$(x+5) (6b-1)$$
}

\end{beispiel}

\newpage


\subsubsection{Minus Eins ausklammern II}\index{Minus Eins ausklammern}
Unterscheiden sich Teilklammern nur durch Vorzeichen, so bietet es
sich an, Minus Eins (-1) auszuklammern
$$7x{\color{green}(b-3)} + 5z{\color{green}(3-b)}$$%%

\TRAINER{%%

$7x{\color{green}(b-3)} - 5z{\color{green}(-3+b)}$
\vspace{10mm}
}%%

\TRAINER{%%

$7x{\color{green}(b-3)} - 5z{\color{green}(b-3)}$
\vspace{10mm}
}%%


\noTRAINER{\mmPapier{3.6}}

\TALS{\aufgabenfarbe{S. 18: Aufg. 29. b) 30. a) c) d) g) 31. a) c) f)
i) 32. b) f)}}
\GESO{\aufgabenfarbe{S. 39: 41. a) e) d) 42. b) c) 43. b)
c) 44. b) e) }}
\newpage



\subsubsection{Mehrmaliges Ausklammern:}\index{ausklammern!mehrmaliges}
 Um Klammerausdrücke zu finden, bietet sich die Methode des mehrmaligen Ausklammerns an.
 Dabei wird die Summe (bzw. Differenz) in gleiche Anzahl \textbf{Teilsummen}\index{Teilsummen} aufgeteilt und Stückweise ausgeklammert.
 Im folgenden Beispiel werden die beiden ersten Summanden und die beiden letzten «Summanden» zunächst unabhängig voneinander betrachtet.

$$3mk+6nk-5m-10n $$
$$= 3k{\color{green}(m+2n)}-5{\color{green}(m+2n)} $$
$$= (3k-5){\color{green}(m+2n)}$$


\begin{beispiel}{}{}
  $$5a-30+ax-6x = ...$$
  Wie gehen wir vor? Beispiel: Aus den ersten beiden Summanden 5 und
  aus den beiden hinteren Summanden $x$ ausklammern:

\TNT{2.4}{$$... = 5\cdot(a-6) + x\cdot(a-6) = ...$$}

  Nun aus beiden Summanden den Term ......... \TRAINER{$(a-6)$}
  ausklammern:

\TNT{2.4}{$$... = (5+x)\cdot(a-6)$$}
\end{beispiel}
\newpage


\subsection{Binomische Formeln}\index{Faktorisieren!mit binomischen Formeln}%%
\index{Binomische Formeln!zum Ausklammern}

\GESO{S. Kap. 2.2.3 S. 32 \cite{marthaler17}}%%
\TALS{S. 18 Kap. 1.3.2 \cite{frommenwiler17alg}}


Zerlegen wir die folgenden Terme in einzelne Faktoren:
$$ a^2 - 16 = \LoesungsRaum{(a+4)(a-4)} $$

$$x^2 -18x + 81 = \LoesungsRaum{(x-9)(x-9)}$$

$$64y^2 - 49z^6 = \LoesungsRaum{(8y+7z^3)(8y-7z^3)}$$

$$c^4 - 1 = \LoesungsRaum{\mathbf{(c+1)(c-1)}(c^2+1)}$$

\TALS{$$ b^3 - 3b^2a + 3ba^2 - a^3 = \LoesungsRaum{(b-a)^3}$$}

\TALS{\aufgabenfarbe{S. 19: Aufg. 33. a) b) c) h) l) 34. a)}}
\GESO{\aufgabenfarbe{S. 39/40: Aufg. 41. c), 45. a) b) c) d) e), 46. a) c) e)
g) und  47. a) b) e) c)}}
\newpage



\subsection{Zweiklammeransatz}\index{Zweiklammeransatz}
Beispiel $$a^2-4a-5$$
$$=(a-\Box{})(a+\Box{}) = (a-5)(a+1)$$


\GESO{\noTRAINER{\mmPapier{5.2}}\TRAINER{Optional: Taschenrechner poly-solv, danach Vorzeichen
drehen!

Beispiel $x^2 - 2x -48$;

lösen wir mit $a=1$, $b=-2$ und
$c=-48$. Taschenrechner Lösungen 8 und -6. Somit lautet die
faktorisierte Form (nach dem Tauschen der Vorzeichen):
$$x^2-2x-48 = (x-8)(x+6)$$}}

\TALS{\aufgabenfarbe{S. 19: Aufg. 35. a) b) k)}}
\GESO{\aufgabenfarbe{S. 40: Aufg. 48. a) d) g) h) i) 49. a) b) c) d)}}


\newpage
%\noTRAINER{\blankpage{}}
\subsection{Gemischte Anwendung}
Im folgenden Beispiel kommen alle obigen Vorgehensweisen als
Teilschritte vor:

\begin{center}{\fbox{$ a^2px^2 - 9pa^2 + 36pa -4xpxa -5px^2 + 45p$}}\end{center}

%%\begin{center}{\fbox{$$ a^2px^2 - 9pa^2 + 36pa -4xpxa -5px^2 + 45p$$}}\end{center}

1. Gemeinsame Faktoren (hier $p$) ausklammern:
$$p[a^2x^2 - 9a^2 + 36a -4x^2a -5x^2 + 45]$$
2. Teilsummen ausklammern, um gemeinsame Klammerausdrücke zu finden\footnote{
Analog könnte auch wie folgt ausgeklammert werden:
$$p[{\color{green}a^2x^2 -9a^2} {\color{red}+ 36a -4x^2a} {\color{blue}- 5x^2 + 45}]$$
$$p[{\color{green}a^2}(x^2 - 9) + {\color{red}4a}(9 - x^2) + {\color{blue}5}(-x^2 + 9)]$$
}
:
$$p[{\color{green}a^2x^2 } {\color{red}\, -\, 9a^2} {\color{red}\, +\, 36a} {\color{green}\, -\, 4x^2a} {\color{green} {\color{green}\,-\,5x^2} } + {\color{red}45}]$$
$$p[{\color{green}x^2} (a^2 -4a -5) + {\color{red}9}(-a^2 + 4a +5)]$$

3. Minus Eins (-1) ausklammern:
$$p[x^2 (a^2 -4a -5) {\color{green}-} 9({\color{green}+}a^2 {\color{green}-} 4a {\color{green}-} 5)]$$
4. Klammerausdrücke ausklammern:
$$p[x^2 {\color{blue}(a^2 -4a -5)} - 9{\color{blue}(+a^2 - 4a - 5)}]$$
$$p[(x^2-9) {\color{blue}(a^2 -4a -5)}]$$
$$p(x^2-9) (a^2 -4a -5)$$
5. Binomische Formel:
$$p{\color{green}(x^2-9)}(a^2 -4a -5)$$
$$p{\color{green}(x+3)(x-3)}(a^2 -4a -5)$$

6. Zweiklammeransatz:
$$p(x+3)(x-3) (a-\Box{})(a+\Box{})$$

Welche Zahlen ergeben multipliziert $-5$ und addiert $-4$?
$$p(x+3)(x-3)(a+5)(a-1) ???$$
\begin{center}{\fbox{$p(x+3)(x-3)(a-5)(a+1)$}}\end{center}


\subsection*{Aufgaben}
\TALSAadB{18}{29. b), 30. a) c) d) g), 31 .a) c) f) i)}

\TALSAadB{19}{32. b) f), 33. a) b) c) h) l), 34. a) b), 35. a) b) d) k)}

\TALSAadB{20}{41-44}
\GESOAadB{40}{50. a) b) d) und 51. a) b) c)}
