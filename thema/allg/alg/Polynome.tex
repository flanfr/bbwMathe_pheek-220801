%%
%% 2019 07 04 Ph. G. Freimann
%%

\section{Polynome}\index{Polynom}
\sectuntertitel{Poly = viel; Nom = Name}

\theorieTALS{12}{1.1.4}
\theorieGESO{18}{1.4}
%%%%%%%%%%%%%%%%%%%%%%%%%%%%%%%%%%%%%%%%%%%%%%%%%%%%%%%%%%%%%%%%%%%%%%%%%%%%%%%%%
\subsection*{Lernziele}

\begin{itemize}
\item Grad, Grundform und Koeffizienten
\end{itemize}


Eine spezielle Form von Termen sind die
\textbf{Polynome}.\footnote{Siehe dazu auch im Anhang das
  Summenzeichen auf Seite \pageref{Summenzeichen}.}

\begin{definition}{Polynom}{definition_polynom}
  Unter einem \textbf{Polynom} verstehen wir einen Term in einer Variable
  $T(x)$ in der Gestalt:

  \begin{tabular}{rrlllll}\index{$\sum{}$ Summe} 
   $T({\color{red}x}) = \sum\limits_{{\color{blue}i}=0}^{n}{{\color{green}a}_{\color{blue}i}{\color{red}x}^{\color{blue}i}}$ &=& ${\color{green}a}_{\color{blue}0}$ &+ ${\color{green}a}_{\color{blue}1}{\color{red}x}$ &+ ${\color{green}a}_{\color{blue}2}{\color{red}x}^{\color{blue}2}$ &+ $...$ &+ ${\color{green}a}_{\color{blue}n}{\color{red}x}^{\color{blue}n}$\\
    &(=& ${\color{green}a}_{\color{blue}0}{\color{red}x}^{\color{blue}0}$ &+ ${\color{green}a}_{\color{blue}1}{\color{red}x}^{\color{blue}1}$ &+ ${\color{green}a}_{\color{blue}2}{\color{red}x}^{\color{blue}2}$ &+ $...$ &+ ${\color{green}a}_{\color{blue}n}{\color{red}x}^{\color{blue}n}$)
\end{tabular}

    \end{definition}
Dabei bezeichnet n ($\in \mathbb{N}$) den \textbf{Grad} des Polynoms,
während die $a_i$ Koeffizienten in $\mathbb{R}$ sind.

\begin{bemerkung}{}{}Bei Polynomen kommt die Variable weder im Nenner, noch im
  Exponenten, noch unter einer Wurzel vor.\end{bemerkung}

\subsection{Nomenklatur}
Polynome ersten Grades (\zB $ 3x-1$) nennen wir \textbf{lineare}
Polynome\index{Polynom!lineares}.

\GESO{Weitere Definitionen zu \textit{quadratischen} und
  \textit{kubischen} Polynomen finden wir im Buch \cite{marthaler21}
  auf Seite 19.}

\textbf{Achtung} Kein Polynom ist $x^2 - \sqrt{x}$, denn $x$ kommt in
der Wurzel vor. Ebensowenig ist $x - \frac{5}{x^2}$ ein Polynom, denn
es lässt sich nicht in die Grundform verwandeln.

\subsection*{Aufgaben}
\TALSAadB{13}{10 und 11}
\GESOAadB{25}{30-33}
