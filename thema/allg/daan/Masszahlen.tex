%%
%% 2019 07 04 Ph. G. Freimann
%%

\section{Maßzahlen/Kennzahlen}\index{Maßzahlen}\index{Kennzahlen}
\sectuntertitel{Im Vatikan leben pro Quadratkilometer 2.27 Päpste.}

\theorieGESO{389}{25}
\theorieTALS{253}{4.3}
%%%%%%%%%%%%%%%%%%%%%%%%%%%%%%%%%%%%%%%%%%%%%%%%%%%%%%%%%%%%%%%%%%%%%%%%%%%%%%%%%
\subsection*{Lernziele}

\begin{itemize}
\item Lagemaße\index{Lagemaß}
\item Streumaße\index{Streumaß}
\TALS{\item bivariate Daten / Streudiagramm\index{Streudiagramm}}
\TALS{\item Univariate vs. bivariate Daten}
\end{itemize}


\subsection{Lagekennzahlen/Lagemaße}\index{Lagemaß}\index{Lagekennzahl}
\begin{itemize}
\item Mittelwert\index{Mittelwert} ($\mittelwert{x}$) = arithmetisches Mittel
\item Median\index{Median} ($\mediantilde{x}$ oder $X_{\textrm{MED}}$) = mittlerer Wert des geordneten Datensatzes (auch Zentralwert\index{Zentralwert} genannt)
\item Quartil\index{Quartil}: Q1 = Median der unteren Hälfte, Q3 =
  Median der oberen Hälfte
\item Modus\index{Modus} ($x_{\textrm{mod}}$) = häufigster auftretender Wert. Hat es in einer Verteilung mehrere ``häufigste'' Werte, so sprechen wir von einer multimodalen (bzw. bimodalen) Verteilung.
\item Maximum
\item Minimum
\end{itemize}
\newpage


\subsection{Streumaße}\index{Streumaß}
\sectuntertitel{Student zum zerstreuten Professor: ``Sie haben aber
  ein großes $\sigma$.''}
Auftrag: Münzwurf. Eine Münze wird 20 Mal geworfen. Wie oft zeigt sie Kopf? Wiederholen Sie dieses Experiment mit der Klasse so oft, bis eine klare Grafik (Säulendiagramm) entsteht.


\begin{itemize}
\item Standardabweichung\index{Standardabweichung}\GESO{\cite{marthaler21} S. 396 Kap. 25.2}

\item Interquartilsabstand (IQR\footnote{IQR=Inter Quartil
    Range})\index{IQR} = Q3 - Q1

\item Spannweite\index{Spannweite} = Maximum - Minimum

\end{itemize}

\subsubsection{Standardabweichung}\index{Standardabweichung}
Die Standardabweichung gibt ein Maß an, wie weit die einzelnen Werte
vom Mittelwert $\mittelwert{x}$ abweichen\GESO{ (S. \cite{marthaler21}
  S. 396 Kap. 25.2)}\footnote{Dabei wird ein spezieller «Mittelwert»
  eingeführt.}.
Die Standardabweichung einer Stichprobe wird mit $\textrm{SD}$ oder
mit $s$ abgekürzt. Es gilt für die Standardabweichung einer
\textbf{Stichprobe}:
$$s = \sqrt{\frac{\sum\limits_{i=1}^n(x_i-\mittelwert{x})^2}{n-1}}$$

Die Einführung in das Summenzeichen ist im Anhang zu finden \totalrefanhang{Summenzeichen}.

Bemerkung am Rande: Mit $\sigma$ (kleines Sigma) wird die
Standardabweichung der \textbf{Grundgesamtheit} ermittelt (falls diese
bekannt ist):
$$\sigma = \sqrt{\frac{\sum\limits_{i=1}^n(x_i-\mittelwert{x})^2}{n}}$$
Dabei wird lediglich $n$ statt $n-1$ im Nenner der Summe verwendet. Da
die Standardabweichung aber sowieso kein robustes Streumaß ist, kann
dieser Unterschied meist vernachlässigt werden. Die Standardabweichung ist hingegen ein gutes Maß, um normalverteilte Stichproben oder Erhebungen miteinander zu vergleichen. 

\newpage
\subsubsection{Robustheit}\index{Robustheit}
Zusammenfassend können wir unsere Kennzahlen wie folgt klassifizieren:

\begin{tabular}{|c|c|c|}
  \hline
   & Lagemaß & Streumaß \\
  \hline
  \textbf{robust} &
  \begin{minipage} [t] {0.4\textwidth}
    
\vphantom{Ög} 
    \begin{itemize}
    \item  \TNDF{Median $\mediantilde{x}$}
    \item \TNDF{Modus $x_{\textrm{mod}}$}
    \item \TNDF{Quartile Q1 bzw. Q3}
    \item \TNDF{(Ausreißerschwellen)}\\
  \end{itemize} \end{minipage}
  & \begin{minipage} [t] {0.4\textwidth}
      
\vphantom{Ög} 
      \begin{itemize}
    \item \TNDF{Interquartilsabstand (IQR)}
  \end{itemize} \end{minipage} \\
  \hline
  nicht robust & \begin{minipage} [t] {0.4\textwidth}
    
\vphantom{Ög}
    \begin{itemize}
    \item \TNDF{Mittelwert (Durchschnitt) $\overline{x}$}
    \item \TNDF{Minimum (min) und}
    \item \TNDF{Maximum (max)}\\
  \end{itemize} \end{minipage} & \begin{minipage} [t] {0.4\textwidth}
    
\vphantom{Ög}
    \begin{itemize}
    \item \TNDF{Standardabweichung $\sigma$}
    \item \TNDF{Spannweite (max - min)}\\
  \end{itemize} \end{minipage} \\
  \hline
  \end{tabular} 

\begin{definition}{Robustheit}{}\index{Robust}
Verändert sich der Wert eines statistischen Maßes nicht, wenn sich ein
Ausreißer weiter ins Extreme bewegt, so sprechen wir von einem \textbf{robusten} Maß.
\end{definition}
\newpage



\subsection{Bivariate Daten/Streudiagramm\GESO{ (optional)}}\index{bivariate
  Daten}\index{Daten!bivariate}\index{Streudiagramm}
\theorieGESO{386}{24.6}
Bisher hatten wir ausschließlich eine einzelne Datenreihe
betrachtet. 
Bei \textbf{bi}variaten Daten, werden zwei Merkmale der selben Stichprobe miteinander verglichen (\zB beim Bierfest die Haarlänge mit dem Bierkonsum oder die Schuhgröße mit dem Einkommen). Bivariate Daten werden meist in Streudiagrammen\index{Streudiagramm} dargestellt:


\bbwCenterGraphic{5cm}{allg/daan/img/streudia.jpg}

Abbildung aus \cite{marthaler21}.


\subsection*{Aufgaben}
\GESOAadB{408ff}{38. 39. 40. 41. 45. 47. 48. 49.}
\TALSAadB{255}{952. 953. 954.}
\GESO{\aufgabenfarbe{Kompendium S. 40 (Kap. 4.4) Aufg. 11.}}

\GESO{\aufgabenfarbe{Kompendium S. 41 (Kap. 4.5) Aufg 12. und 13.}}



%\TALSAadB{???}{???}%% Keine bivariate Daten?

