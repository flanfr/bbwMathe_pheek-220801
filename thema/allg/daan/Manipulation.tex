%%
%% 2020 02 03 ph. freimann
%%

\section{Manipulation}\index{Manipulation}
\sectuntertitel{Ungefähr 97.223\% aller Statistiken sind frei
  erfunden, wie \zB diese.}

\theorieGESO{389}{25}
\theorieTALS{267}{4.6}
%%%%%%%%%%%%%%%%%%%%%%%%%%%%%%%%%%%%%%%%%%%%%%%%%%%%%%%%%%%%%%%%%%%%%%%%%%%%%%%%%
\subsection*{Lernziele}

\begin{itemize}
\item Bias\index{Bias} \theorieTALS{267}{4.6 (Bias)}
\item Festlegen der «Mitte» (Durchschnitt statt Median) \theorieTALS{268}{4.6.1}
\item Weglassen\index{Weglassen!Manipulation} (Schubladisieren/Yang ohne Yin)
\item Drittvariableneffekt\index{Drittvariableneffekt} (Korrelation\index{Korrelation} ohne Kausalität: «Scheinkorrelation\index{Scheinkorrelation}»)
\item Illusion der Präzision\index{Illusion}
  \theorieTALS{268}{4.6.2}\index{Präzision!Illusion der}\index{Illusion der Präzision}
\item Graphische Darstellung \theorieTALS{269}{4.6.3}
  \begin{itemize}
  \item Skala verschweigen\index{Skala}
  \item $y$-Achse spiegeln
  \item Verschiedene Abstände (Säulenbreiten) bei Histogrammen
  \item Farben\index{Farben}
  \item Optische Verzerrungen:
    \begin{itemize}
    \item Nullpunktverschiebung\index{Nullpunktverschiebung}
    \item Allgemeine Achsenverschiebung\index{Achsenverschiebung} bzw. -stauchung\index{Achsenstauchung}
    \item logarithmische Achse
    \item Fläche (quadratische Vergrößerung)\index{Fläche!Manipulation in der Datenverarbeitung}
    \item Kuchendiagramm ist nicht 100\%
    \item In die Zukunft projizieren mit Pfeilen
    \end{itemize}
  \end{itemize}
\end{itemize}

Bei den meisten obigen «Lügen» kann der sog. Lügenfaktor\index{Lügenfaktor} berechnet werden, indem die gemessene Fläche ins Verhältnis zu echten Größe gesetzt wird.
\newpage
\subsection*{Lügenfaktor (optional)}\index{Lügenfaktor}
Stellen Sie sich vor, Sie haben zwei Messungen vor sich mit einmal «vor» und einmal «nach» der Veränderung:
\begin{itemize}
\item Vor Veränderung: Wert 3
\item Nach Veränderung: Wert 4
\end{itemize}

Zeichnen Sie links zwei Säulen mit der korrekten Nulllinie. Zeichnen Sie rechts zwei Säulen mit der Nulllinie verschoben auf 2.

\TNT{6.4}{
\bbwCenterGraphic{12cm}{allg/daan/img/luegenfaktor.jpg}
}

Wie nimmt das Auge das Verhältnis, bzw. die Veränderung bei korrekter Darstellung und bei manipulierter Darstellung wahr?

\TNT{3.2}{
  Das \textbf{Verhältnis} von 4:3 wird wie 2:1 wahrgenommen, was einem Lügenfaktor von 1.5 entspricht.

  Die \textbf{Zunahme} von $\frac13$ wird wie eine Zunahme von 100\% wahrgenommen, was einem Lügenfaktor von 3 entspricht.
}
\newpage

\subsection*{Aufgaben}
\TALSAadB{267}{992 bis 998}
\GESOAadB{403ff}{11, 34}

Notizen:

\mmPapier{5.2}


\newpage
