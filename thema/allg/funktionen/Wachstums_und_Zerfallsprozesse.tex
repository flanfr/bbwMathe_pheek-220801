%%
%% 2019 07 04 Ph. G. Freimann
%%

\section{Wachstum und Zerfall}\index{Wachstum}\index{Zerfall}
\sectuntertitel{Sagt ein großer Stift zum kleinen Stift: ``Wachsmalstift!''}
%%%%%%%%%%%%%%%%%%%%%%%%%%%%%%%%%%%%%%%%%%%%%%%%%%%%%%%%%%%%%%%%%%%%%%%%%%%%%%%%%
\TRAINER{Video Mathe Mann}%%
\subsection*{Lernziele}

\begin{itemize}
\item Zinseszins
\item Wachstums-, Zerfallsprozesse
\item Verdoppelungs- und Halbwertszeiten
\item Basiswechsel
\end{itemize}

\TALS{(\cite{frommenwiler17alg} S.221 (Kap. Exponentielles Wachstum))}
\TALS{(\cite{frommenwiler17alg} S.223 (Kap. Exponentielle Abnahme))}
\TALS{(\cite{frommenwiler17alg} S.225 (Kap. Zinseszins))}
\GESO{(\cite{marthaler17}       S.342 (Kap. 20))}


\subsection{Exponentielles Wachstum}
Die Funktion $f(t): t \mapsto y = a^t$ ist eine
Exponentialfunktion die rasch gegen «unendlich» ($\infty$) ansteigt:

\bbwFunction{-4}{4}{-1}{8}{exp(\x)}{-4:2}

\subsubsection{Beispiele}
Bei ungebremsten Prozessen sprechen wir dann von einer exponentieller
Zunahme, wenn die Zunahme pro Zeiteinheit jeweils proportional zur aktuellen Anzahl ist.

\begin{itemize}
\item \Lueckentext{Zinseszins}
\item \Lueckentext{Frequenzen in der temperierten Stimmung (Musik). Zunahme der Frequenz pro Halbtonschritt.}
\item \Lueckentext{Keime in der Kuhmilch; Ansteckungsbedingte Krankheitsfälle (\zB viral)}
\item \Lueckentext{Algenbefall in Teichen}
\item \Lueckentext{(ungebremstes) Bevölkerungswachstum / bzw. Tierpopulation}
\item \Lueckentext{\dotfill}
\end{itemize}
\newpage


\subsection{Einstiegsbeispiel}
Der Türlersee ist ein kleiner See im Repischtal. Seine Oberfläche
begann sich vor einigen Jahrzehnten stark mit Algen\footnote{Genau
  genommen handelte es sich um Zooplanktonbiomasse zwischen 1982 und 1994, doch als
  Idee zur Exponentialfunktion soll das Flächenmodell reichen.} zu bedecken.

Anfänglich (zum Zeitpunkt $t=0$) waren gerade mal 20$m^2$ bedeckt. Doch nach fünf Tagen hatte sich diese Fläche verdoppelt und nach weiteren fünf Tagen nochmals verdoppelt (also insgesamt vervierfacht).

Füllen Sie die (prognostizierte) Wertetabelle für 30 Tage ein:

\def\spaceX{\,\,\,\,\,\,\,\,\,\,}
\newcommand\tuerlerB[1]{\noTRAINER{\spaceX}\TRAINER{#1}}
\begin{tabular}{l|c|c|c|c|c|c|c}
  $t$:  & 0 & 5 & 10 & 15 & 20 & 25 & 30 \\
  \hline
  $m^2$ & \tuerlerB{20}  & \tuerlerB{40}  &   \tuerlerB{80}  &  \tuerlerB{160}  &  \tuerlerB{320}  &  \tuerlerB{640}  &  \tuerlerB{1280} \\
\end{tabular}


Zeichnen Sie die Algenpopulation als Graph in eine Koordinatensystem
(beginnen Sie mit dem Ursprung ganz links unten. Eine $x$-Einheit,
also ein Tag,  nach rechts entspricht einem Häuschen und in $y$-Richtung nehmen Sie 2 Häuschen für 100$m^2$ Algenfläche:

\TNT{11.2}{\bbwGraphic{10cm}{allg/funktionen/img/tuerlerAlgen.png}}

Geben Sie eine Funktionsgleichung an, welche das Wachstum der Algenpopulation beschreibt:

\begin{center}
  $f:\,\,\, y=\LoesungsRaumLang{20\cdot{} 2 ^{\frac{t}{5}}}$
  \end{center}
\TRAINER{
Dabei bezeichnet 20 den Startwert, 2 den Zunahmefaktor und 5 die Beobachtungszeitspanne.}%%

\newpage



\subsubsection{Grundform}

Die allgemeine Formel für exponentielle Prozesse lautet:

\begin{definition}{Wachstumsprozess}{}
  $$f(t) = y = b\cdot{}a^{\frac{t}{\tau}}$$
\end{definition}

\subsubsection{Graphische Erläuterung}

\bbwCenterGraphic{11cm}{allg/funktionen/img/exp/exponentielles_wachstum.png}


Dabei sind
\begin{itemize}
\item $b=f(0)$ der Anfangsbestand zum Zeitpunkt $t$ = 0.
\item $\tau$ ist die typische Zeitspanne zwischen zwei Beobachtungszeitpunkten (zum Beispiel zwischen Wert und dessen Verdopplung). Beispiele:
  \begin{itemize}
  \item Verdoppeln in 3 Stunden: $a=2$ und $\tau = 3 [\textrm{h}]$
  \item Verfünffachen einer Viertelstunde (= 15 Minuten): $a=5$ und
    $\tau=\frac{1}{4} [\textrm{h}]$
  \end{itemize}
\item $a$ ist der Zunahmefaktor zwischen zwei Beobachtungszeitpunkten (Beispiel $a=2$ bei Verdoppelungsprozessen).
  $a$ berechnet sich durch den Quotienten zwischen zwei Beobachtungswerten $b$ und später $m$. Also $a=\frac{m}b=\frac{f(\tau)}{f(0)}$.
\item $f(t)=b\cdot{}a^{\frac{t}{\tau}}$ ist der Wert (Anzahl, Fläche,
  Bestand, ...) zum Zeitpunkt
  $t$. 
\end{itemize}

Bemerkung: 
\TNT{2.4}{
$\frac{f(\tau)}{f(0)}=\frac{b\cdot{}a^{\frac{t}{\tau}}}{b\cdot{}a^{\frac0{\tau}}}
  = a^{\frac{t}{\tau} -\frac{0}{\tau}} = a^{\frac{t-0}{\tau}}
  = a^{\frac{\tau}{\tau}} = a$}%% END TNT

\newpage

\subsection{Referenzaufgabe}\index{Irland!Bevölkerungswachstum}
Irland hatte 1990 3.5 Mio. Einwohner. Im Jahr 2019 waren es bereits 4.9 Mio.

\textbf{Frage 1}: Was prognostizieren Sie für das Jahr 2025, wenn Sie von einem exponentiellen Wachstum ausgehen?

\TNT{12}{
  1. Skizze:Ersichtlich: $\tau$, 1990, 2019, 3.5 und 4.9 (Rest irrelevant)
  
  Ansatz: Zeitpunkt $t_1 = 1990$, somit $m_1 = 3.5$ (Mio.). Analog Zeitpunkt $t_2 = 2019$ somit $m_2 = 4.9$ (Mio).
  Das $\tau= 29$ Jahre und das zugehörige $a$ ist $\frac{4.9}{3.5}$. $a$ abspeichern!

  Uns fehlt noch das $c$.

  Es gilt nun im Jahr 1990 ($=t_1$): $3.5 = f(1990) = b\cdot{} \left(a\right)^\frac{t_1}{\tau} = b\cdot{} a^\frac{1990}{26}$

  Somit ist $$b = \frac{3.5}{  a^\frac{1990}{29}  } \approx 3.3\cdot{}10^{-10}$$

  Dieses Resultat $b$ abspeichern!

  Nun können wir die Prognose für 2025 einfach berechnen: $f(2025) = b\cdot{}a^\frac{2025}{29}$, denn wir haben ja $a$ und $b$ bereits im Rechner gespeichert.
}%% ENT TNT
\newpage

\textbf{Frage 2}: In wie vielen Jahren hat sich die Bevölkerung verdoppelt?

\TNT{12}{
  Halbwertszeit sei $T$. Ansatz: $a^\frac{t}{29} = 2^\frac{t}{T}$ und somit (auf beiden Seiten die $t$-te Wurzel:

  $a^\frac{1}{29} = 2^\frac{1}{T}$ (Nun beidseitig $\log_a$.)

  $\frac{1}{29} = \frac1T \cdot{}\log_a(2)$ (Nun den Kehrwert)

  $29 = T \cdot{} \frac{1}{\log_a(2)}$ (und noch mal $\log_a(2)$)

  $T = 29 \cdot{} \log_a(2) \approx 59$ Jahre
}%% END Trainer
\newpage

\subsection*{Aufgaben}
\TALSAadB{221ff}{831 - 839}
\GESOAadB{338}{28. Bakterien}
\GESOAadB{352ff}{2., 7., 1. (optional)}
\GESO{\aufgabenfarbe{Kompendium: S. 27ff: 32., 34., 37., 38., 40., 44., 45. und 46.}}
\GESO{\aufgabenfarbe{Nullserie Aufgabe 8}}
\GESO{\aufgabenfarbe{Maturaprüfung 2017, Aufg. 12 (Raupen)\\
Maturaprüfung 2018 (Serie 3), Aufg. 11 (Müll)
}}

\newpage

\subsection{Basiswechsel}
Die Algen (im Türlersee) verdoppeln sich alle fünf Tage.

$$\tau = 5; a= 2; f(t) = b\cdot{}a^\frac{t}{\tau} = b\cdot{}2^\frac{t}5$$

Um wie viel nehmen sie ...

\begin{enumerate}
\item ... alle 15 Tage zu?
\item ... alle 7 Tage zu?
\item In wie vielen Tagen verdreifachen sie sich?
\item Wie sieht es mit der Basis $e$ aus? Mit welchem $m$ kann man die
  Zunahme als $f(t) = b \cdot{} e^{m\cdot{}t}$ angeben?
\end{enumerate}

\begin{center}\fbox{
    \textbf{ $\tau$ und $a$ sind voneinander abhängig!}}
  \end{center}

Es gilt:
\begin{center}\fbox{$b\cdot{}a^{\frac{t}{\tau}} = b\cdot{}a_2^{\frac{t}{\tau_2}} =
    e^{m\cdot{}t}$}
\end{center}

... oder nach ziehen der $t$-ten Wurzel:

\begin{gesetz}{Basiswechsel}{}

  \begin{center}\fbox{\fbox{$a^{\frac{1}{\tau}} =     a_2^{\frac{1}{\tau_2}} = e^{m}$}}\end{center}

 
  $a=$ Wachstumsfaktor\\
  $\tau=$ Beobachtungszeitraum\\
  $e=$ Eulersche Konstante (2.718...)
  $m=\frac1\tau\cdot{}\ln(a) = \frac1{\tau_2}\cdot{}\ln(a_2)$
\end{gesetz}
\newpage


\textbf{Lösungen}

ad. 1.: $\tau_2=15$:
\TNT{3.6}{$2^\frac15 = a_2^\frac1{15}$ Nun auf beiden Seiten ``hoch
  15'' und somit

  $a_2 = (2^{\frac15})^{15} = 2^3 = 8$.}

ad. 2.: $\tau_2=7$:
\TNT{3.6}{$2^\frac15 = a_2^\frac1{7}$ Nun auf beiden Seiten ``hoch
  7'' und somit

  $a_2 = (2^{\frac15})^{7} = 2^{\frac75} \approx 2.639$.}

ad. 3.: $a_2 = 3$ (verdreifachen):
\TNT{4.4}{$2^\frac15 = 3^\frac{1}{\tau_2}$ (beidseitig logarithmieren:)

  $\frac15 \ln(2) = \frac1{\tau_2} \ln(3)$ (Kehrwert:)

  $\frac{5}{\ln(2)} = \frac{\tau_2}{\ln(3)}$ ($\cdot \ln(3)$:)

  $\tau_2 = \frac{5\cdot{}\ln(3)}{\ln(2)}\approx 7.925$ Tage. Sie
  verdreifachen sich ca. alle 8 Tage.
}%% end TNT

ad. 4.: $a^\frac{t}5 = e^{m\cdot{}t}$ (Basis $e$):
\TNT{5.2}{
  $2^\frac15 = e^{m}$ (beidseitig $\ln()$):

  $\ln(2^\frac15) = m\cdot \ln(e) = m\cdot 1 = m \approx 0.1386$ und somit:
  $$f(t) \approx b\cdot{} e^{0.1386\cdot{} t}$$
}%% END TNT

\newpage


\subsubsection{Beispiel: Wissenschaftliche Publikation (optional)}
In einem wissenschaftlichen Journal ist angegeben, dass eine Population annähernd folgendes Verhalten an den Tag lege [$t$ = Tage]:
$$y = 5.6\cdot{} e^{\frac{t}{4.1}}$$

Berechnen Sie die tägliche Zunahme $a$ und geben Sie somit die Funktionsgleichung ohne Bruch im Exponenten an:

$$y = 5.6\cdot{} e^{\frac{t}{4.1}} = b\cdot{}a^t \approx \LoesungsRaum{5.6} \cdot{} \LoesungsRaum{1.2762}^t$$

\TNT{6.4}{Tipp: Setzen Sie $t=0$ und $t=1$ ein. $$a = e^{\frac{1}{4.1}}$$
\vspace{3cm}}
\newpage

\subsubsection{Bedeutung der Zahl $e$ (optional)}

\paragraph{$45\degre$ auf der $y$-Achse:}
Die E-Funktion $y=e^x$ hat die $45\degre$-Steigung genau auf der $y$-Achse; sie beginnt im Gegensatz zu anderen Exponentialfunktionen genau bei $x=0$ so «richtig zu wachsen»\TALS{\footnote{Wenn wir in einem Punkt $x$ den Wert der $e$-Funktion ($e^x$) betrachten, so ist ihr Funktionswert exakt gleich der Steigung einer Tangente bei diesem $x$-Wert.}}.

\bbwGraph{-4}{3}{-1}{5}{
  \bbwFunc{pow(2.71828,\x)}{-3.5:1.5}
  \bbwFuncC{\x+1}{-2:2}{green}
}%% end graph

Dies ist ein weiterer Grund, warum die Zahl $e$ eine solche Bedeutung bei
exponentiellen Prozessen einnimmt.


\TALS{
\paragraph{Tangente an die E-Funktion}\index{E-Funktion}
Als \textbf{E-Funktion} bezeichnen wir die Exponentialfunktion zur
Basis $e \approx 2.71828172846$; also:
$$f(x) = e^x$$
Diese Funktion ist von allen Exponentialfunktionen insofern speziell,
als dass für jedes $x$ der Funktionswert $y=e^x$ genau die Steigung
der Tangente an $e^x$ im Punkt $P_x=\left(x | e^x\right)$ angibt.

So ist die Tangentensteigung im Punkt $(0|e^0) = (0 | 1)$ genau 1 und die Steigung im
Punkt $(1|e^1) = (1|e)$ genau $e$. 

}%% END TALS


\newpage


\subsection{Exponentieller Zerfall}\label{zerfallsfunktion}
Die Funktion $f(t): t \mapsto y = d^{-t}$ ist eine
Exponentialfunktion, die gegen Null geht.
\bbwFunction{-4}{4}{-1}{8}{exp(-\x)}{-2:4}

\newpage

\subsubsection{Beispiele}
\begin{itemize}
	\item \Lueckentext{Zinsliche Abschreibungen (\zB Wert eines Autos)}
	\item \Lueckentext{Radioaktiver Zerfall}
	\item \Lueckentext{Lichtintensität in Medium (Gas / Flüssigkeit / Glasfaser), dies gilt vertikal, wie auch horizontal}
	\item \Lueckentext{Atmosphärischer Luftdruck in Metern über Meer}
  \item \Lueckentext{Entladen einer Batterie bzw. eines Kondensators}
  \item \Lueckentext{Sauerstoffkonzentration in Seen (\zB Herbst bei kontinuierlicher Abnahme)}
  \item \Lueckentext{Abnahme des Bierschaums im Glas}
  \item \Lueckentext{Mischen, wie im Sirup-Beispiel\totalref{sirup_beispiel}}
  \item \Lueckentext{«Halbwertszeit des Wissens» ;-)}
  \item \Lueckentext{\dotfill}
\end{itemize}

\newpage



Zeichnen Sie die Funktionen $f: y=2^x$, $g: y=2^{-x}$ und $h: y=\left(\frac12\right)^x$ in dasselbe Koordinatensystem:

\bbwGraph{-6}{6}{-1}{5}{
  \TRAINER{\bbwFuncC{exp(0.69314718*\x)}{-6:2}{blue}}
  \TRAINER{\bbwFuncC{exp(-0.69314718*\x)}{-2:6}{red} }
  \TRAINER{\bbwLetter{1.5,4}{2^x}{blue}}
  \TRAINER{\bbwLetter{-3,4}{2^{-x}=\left(\frac{1}{2}\right)^x}{red}}
}
\newpage
Die allgemeine Formel für Zerfallsprozesse lautet:

\begin{definition}{Zerfall}{}
$$f(t) = y = b\cdot{}a^{\frac{t}{\tau}}$$
\end{definition}

\bbwCenterGraphic{10cm}{allg/funktionen/img/exp/exponentieller_zerfall.png}

Dabei sind
\begin{itemize}
\item $c$ der Anfangsbestand zum Zeitpunkt $t$ = 0.
\item $\tau$ ist die Zeitspanne zwischen zwei typischen Beobachtungszeitpunkten (Zum Beispiel zwischen Wert und dessen Halbierung).
\item $a=\frac{f(\tau)}{f(0)}$ ist der Abnahmefaktor ($0<a<1$) zwischen zwei Beobachtungszeitpunkten (im Abstand von $\tau$). Bei Halbierungsprozessen \zB wäre $a=0.5$.
\end{itemize}

\begin{bemerkung}{Wachstum vs. Zerfall}{}

  Der einzige Unterschied bei Wachstums- bzw Zerfallsprozessen ist der
  Faktor $a$:

  \begin{itemize}
  \item $a>1$: Wachstum
  \item $0<a<1$: Zerfall
    \end{itemize}
  
  \end{bemerkung}
\newpage

\subsubsection{Umkehrung (Optional)}
Anstelle eines  positiven Exponenten ($\frac{+t}{\tau}$) kann genauso
gut der Kehrwert des Faktors $a$ genommen werden. Dann wird der
Exponent negativ, dafür wird die Basis $> 1$. Seien $m_1$ bzw. $m_2$
zwei Beobachtungswerte.

Mit $a = \frac{m_2}{m_1}$ und $d := \frac{m_1}{m_2} = \frac{1}{a}$, gilt

\begin{center}
  \fbox{$b\cdot{} a^{\frac{+t}{\tau}} = b\cdot{}d^{\frac{-t}{\tau}} $}
\end{center}

\TALS{Beweis:}\GESO{Begründung: So gilt \zB bei Halbierungsprozessen, dass $a=0.5$ und $d=2$:}

\TALS{ $b\cdot{}a^{\frac{t}{\tau}} = b\cdot{} \left(\frac{m_2}{m_1}\right)^{\frac{t}{\tau}} = b\cdot{} \left(\frac{m_1}{m_2}\right)^{\frac{-t}{\tau}} =  b\cdot{}d^{\frac{-t}{\tau}} $}
\GESO{ $c\cdot{}0.5^{\frac{t}{\tau}} = b\cdot{} \left(\frac{1}{2}\right)^{\frac{t}{\tau}} = b\cdot{} \left(\frac{2}{1}\right)^{\frac{-t}{\tau}} =  b\cdot{}2^{\frac{-t}{\tau}} $}
\newpage

%% Sirup-Beispiel
\subsubsection{Mischtank}\index{Mischtank}\index{Sirup}\label{sirup_beispiel}
Wird ein Glas Wasser in ein Glas Sirup geschüttet, so

\TRAINER{\bbwCenterGraphic{5cm}{allg/alg/potenzen_wurzeln/img/Schwapp.png}}%%
\noTRAINER{\bbwCenterGraphic{5cm}{allg/alg/potenzen_wurzeln/img/SchwappOhneFormel.png}}

geschieht erst mal etwas eher klebriges:
\begin{itemize}
  \item Das Wasser verdrängt den Sirup und
  \item das Sirupglas schwappt über.
\end{itemize}

Wenn man nun gleichzeitig im Sirupglas
umrührt, so mischt sich das Wasser mit dem Sirup und je länger man
Wasser einschüttet, umso verdünnter wird der Sirup.


Wie viel Sirup bleibt im Glas?

\TNT{2.4}{
Am Ende bleibt ein
Verhältnis von Wasser : Sirup = $\left(1-\frac{1}{e}\right) : \left(\frac{1}{e}\right)$
\vspace{1.5cm}
}

Diese Konstante wird oft in großen chemischen Mischtanks verwendet,
gibt aber auch ein Maß an, wenn \zB in einer Minergie-Wohnung die Luft
ausgetauscht wird. Wenn nämlich das Volumen der Wohnung einmal neu hineingepumpt (bzw. weggeblasen) wurde während sich alte die Luft im Haus permanent mit der neuen vermischt, so ist noch ein Anteil von \TRAINER{$\frac{1}{e}$}\noTRAINER{ ..... } der alten Luft im Haus.
\newpage


\textbf{Begründung:}\\
1. Gedanke: Jedes eingefüllte Glas, vermindert die vorhandene
Sirupkonzentration um den selben Faktor. Ergo handelt es sich um
einen exponentiellen Zerfall.

\leserluft

2. Gedanke: Wir tauschen drei Mal $\frac13$ aus. Nehmen also im
\begin{itemize}
\item \textbf{ersten Schritt} $\frac13$ des Sirups weg (und ersetzen diesen mit Wasser).
  Es bleiben $\frac23$ Sirup. Den Rest füllen wir mit Wasser auf.
\item Im \textbf{zweiten Schritt} nehmen wir $\frac13$ des Gemisches
weg; es verbleiben also $\frac23$ von $\frac23$ an
Sirup-Konzentrat. Der Rest wird immer wieder mit Wasser aufgefüllt. Mit
anderen Worten: Es bleiben $\frac23 \cdot \frac23
= \left(\frac23\right)^2$ an Sirup\footnote{Man könnte hier auch argumentieren mit: «Wir nehmen von den $\frac23$ einen Drittel weg»: $\frac23 - (\frac13$ von $\frac23)$ = $\frac23 - (\frac13 \cdot\frac23) = \frac23 \cdot(1-\frac13)=\frac23\cdot\frac23$}.
\item Im \textbf{dritten Schritt} entnehmen wir wieder $\frac13$ des
Gemisches; es verbleiben wieder $\frac23$ vom bisherigen Sirup, also
$\frac23$ von $(\frac23)^2$ also $\left(\frac23\right)^3$.

Beim dreistufigen Gedankenexperiment verbleiben
$\left(\frac23\right)^3 = \left(1-\frac13\right)^3$ der ursprünglichen Konzentration.
\end{itemize}
\leserluft

3. Gedanke: Das Experiment vom vorherigen Gedanken können wir natürlich auch mit immer kleineren\TALS{, sogenannten infinitesimalen,} Schritten durchführen.
Mit Centilitern \zB im dl-Glas ersetzen wir 10 Mal je $\frac1{10}$. 
So verbleibt am Schluss $\left(1-\frac{1}{10}\right)^{10}\approx 0.35$ Sirup.

\GESO{Wenn wir (\zB mit dem Taschenrechner) die Schrittanzahl immer weiter vergrößern (und somit die pro Schritt ausgetauschte Menge immer verkleinern), so ergibt sich für 1000 Schritte ein Verhältnis von $\left(1-\frac{1}{1000}\right)^{1000}\approx 0.3677 \approx \frac1e$. }
\TALS{Wenn wir die Schritte permanent erhöhen (und gegen Unendlich gehen lassen), so erhalten wir den Grenzwert (lat. Limes) von

$$\lim_{n\rightarrow\infty} \left(1-\frac{1}{n}\right)^n = \frac1e$$
}
\newpage

\textbf{Aufgabe 1: Sirup}\\
Wie viel Wasser muss eingeschüttet werden, damit das auf der Flasche
angegebene Verhältnis von 1:6 (1 Teil Sirup, 6 Teile Wasser) zustande
kommt?

\TNT{8}{
Bei 1x Schütten, erhalten wir $\left(\frac{1}{e}\right)^1$ Anteil Sirup.

Bei 2x Schütten, erhalten wir $\left(\frac{1}{e}\right)^2$ Anteil Sirup.

Somit erhalten wir den Siebtel (1:6 = $\frac17$-Anteil) indem wir die
folgende Exponentialgleichung lösen:

$$\frac17 = \left(\frac{1}{e}\right)^n$$
Diese Gleichung lösen wir, indem wir beidseitig logarithmieren und so
erhalten wir den einzuschüttenden Teil $$n=\ln(7)\approx{1.946}.$$
}%% END TNT

\textbf{Aufgabe 2: Minerige-Haus}\\
Wenn wir also wissen wollen, wie viel Luft in ein Minergiehaus
eingepumpt werden muss, damit nur noch 1 Promille der alten Luft
vorhanden ist, so erhalten wir
\TNT{2.4}{
$$\textrm{Volumen Neuluft} = \textrm{Wohnungsvolumen}\cdot{}\ln(1000)$$
} %% end TNT


\newpage



\subsection*{Aufgaben}
\TALSAadB{223ff}{840-847}
\GESOAadB{338}{29. (Bierschaum)}
\GESOAadB{352ff}{3, 4, 5, 6}
\GESOAadB{354ff}{9. (Bauchspeicheldrüse)}
\GESO{\aufgabenfarbe{Kompendium S. 27ff Kap. 3.4.1 33., 35., 36., 39., 41. 42., 43. und 47}}
\GESO{\aufgabenfarbe{
    Maturaprüfung 2018 (Serie 4), Aufg 10 (Cäsium 137)\\
    Maturaprüfung 2018 (Serie 2), Aufg 11 (Plutonium)\\
    Maturaprüfung 2018 (Serie 1), Aufg 11 (radioaktive Substanz)\\
    Maturaprüfung 2016, Aufg. 9 (Jod-131)
}}

\newpage

\subsection{Halbwertszeit, Verdopplungszeit}\index{Halbwertszeit}\index{Verdopplungszeit}

Die Zeitspanne, in der sich eine Menge halbiert, nenne wir
\textbf{Halbwertszeit}\footnote{Die Halbwertszeit wird inbesondere
  beim radioaktiven Zerfall verwendet: nach wie vielen Jahren strahlt
  ein Stoff nur noch die Hälfte.}.

Aus $\frac12b = f(t_{0.5}) = b\cdot{}a^{\frac{t_{0.5}}{\tau}}$
folgt\TRAINER{ beidseitig durch $b$ teilen und dann $log_a$}:

\begin{gesetz}{Halbwertzszeit}{}
  $$t_{0.5} = \tau\cdot{}\log_a\left(\frac12\right)$$
\end{gesetz}

Beispiel: Ein Stoff nimmt innerhalb von sieben Tagen auf 80\% ab. Wie
groß ist seine Halbwertszeit $t_{0.5}$?

\TNT{5.2}{$\tau=7$ und $a = 0.8$. Somit
$$t_{0.5} = 7\cdot{} \log_{0.8}\left(\frac12\right)\approx 21.74
  [\textrm{Tage}]$$
\vspace{2cm}}


Analog gilt das Gesetz zur Verdopplung (s. obiges Beispiel Bevölkerung Irlands):
\begin{gesetz}{Halbwertzszeit}{}
  $$t_2 = \tau\cdot{}\log_a(2)$$
\end{gesetz}

\newpage
