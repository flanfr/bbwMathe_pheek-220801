%%
%% 2019 07 04 Ph. G. Freimann
%%

\section{Exponentialfunktionen}\index{Funktion!Exponentialfunktion}\index{Exponentialfunktionen}
\sectuntertitel{Go viral!}
%%%%%%%%%%%%%%%%%%%%%%%%%%%%%%%%%%%%%%%%%%%%%%%%%%%%%%%%%%%%%%%%%%%%%%%%%%%%%%%%%
\subsection*{Lernziele}

\begin{itemize}
\item Definition Exponentialfunktion
\item Koeffizienten interpretieren
\item Graph: Symmetrien, Polstellen, Asymptoten, Schnittpunkte mit
  Achsen
  \item Basiswechsel
\end{itemize}

\TALS{(\cite{frommenwiler17alg} S.215 (Kap. 3.10))}
\GESO{(\cite{marthaler17}       S.322 (Kap. 19))}
\newpage


\subsection{Verschiebung und Streckung}

Eine Verschiebung der Exponentialfunktion $y=b\cdot{}a^x$ in der Zeit ($x$-Richtung) kann auch in Form einer Veränderung der Startfaktors $b$ umgeschrieben werden.

Verschieben wir \zB $$y=2^x$$ um fünf Einheiten nach rechts, so liest sich die neue Funktionsgleichung wie folgt:
$$y=2^{x-5}.$$

(Zeichnen Sie in \texttt{geogebra.org} a) $y=2^x$ und b) $y=2^{x-5}$.)

Dies kann jedoch auch umgeschrieben werden:

$$2^{x-5} = 2^x \cdot{} 2^{-5} = 2^{-5} \cdot{} 2^x = \frac{1}{2^5} \cdot{} 2^x =
\frac{1}{32}\cdot{}2^x$$

\bbwCenterGraphic{8cm}{allg/funktionen/img/exp/verschiebung_gleich_streckung.png}
Bildlegende: Eine Verschiebung ($x$-Richtung) der Exponentialfunktion entspricht einer Stauchung ($y$-Richtung) der selben Exponentialfunktion.

\TALS{Es gilt hier $$a^{x-b}=\frac{a^x}{k}$$ mit
$k=a^b$ und mit $b=\log_a(k)$.}


\subsection*{Aufgaben}
\TALSAadB{215ff}{809-830}
\GESOAadB{335ff}{11. a) b), 15. a) b) [Zeichnen mit \texttt{geogebra.org}], 17. a) b) e), 22. a) b), 23. d), 25. a) b)}
\newpage




\subsection{Basiswechsel}
Die Algen (im Türlersee) verdoppeln sich alle fünf Tage.

$$\tau = 5; a= 2; f(t) = b\cdot{}a^\frac{t}{\tau} = b\cdot{}2^\frac{t}5$$

Um wie viel nehmen sie ...

\begin{enumerate}
\item ... alle 15 Tage zu?
\item ... alle 7 Tage zu?
\item In wie vielen Tagen verdreifachen sie sich?
\item Wie sieht es mit der Basis $e$ aus? Mit welchem $m$ kann man die
  Zunahme als $f(t) = b \cdot{} e^{m\cdot{}t}$ angeben?
\end{enumerate}
\newpage


\textbf{Lösungen}

@ 1.: $\tau_2=15$:
\TNT{3.6}{$2^\frac15 = a_2^\frac1{15}$ Nun auf beiden Seiten ``hoch
  15'' und somit

  $a_2 = (2^{\frac15})^{15} = 2^3 = 8$.

Ergo $b\cdot{}2^\frac{t}5 = b\cdot{} 8^\frac{t}{15}$}

@ 2.: $\tau_2=7$:
\TNT{3.6}{$2^\frac15 = a_2^\frac1{7}$ Nun auf beiden Seiten ``hoch
  7'' und somit

  $a_2 = (2^{\frac15})^{7} = 2^{\frac75} \approx 2.639$.

Ergo $b\cdot{}2^\frac{t}5 \approx b\cdot{} 2.639^\frac{t}{7}$}

@ 3.: $a_2 = 3$ (verdreifachen):
\TNT{4.4}{$2^\frac15 = 3^\frac{1}{\tau_2}$ (Definition Logarithmus):

  $\frac1{\tau_2} = \log_3\left(2^\frac15\right)$

  $\tau_2 = \frac1{\log_3\left(2^\frac15\right)}\approx 7.925$ Tage.

Also: Sie verdreifachen sich ca. alle 8 Tage.

  Ergo: $b\cdot{}2^\frac{t}5 \approx b\cdot{}3^\frac{t}{7.925}$
}%% end TNT

@ 4.: $a^\frac{t}5 = e^{m\cdot{}t}$ (Basis $e$):
\TNT{5.2}{
  $2^\frac15 = e^{m}$ (beidseitig $\ln()$):

  $\ln(2^\frac15) = m\cdot \ln(e) = m\cdot 1 = m \approx 0.1386$ und somit:
  $$f(t) \approx b\cdot{} e^{0.1386\cdot{} t} \approx b\cdot{}2^\frac{t}5$$
}%% END TNT

\newpage

\begin{center}\fbox{
    \textbf{ $\tau$ und $a$ sind voneinander abhängig!}}
  \end{center}

Es gilt:
\begin{center}\fbox{$b\cdot{}a^{\frac{t}{\tau}} = b\cdot{}a_2^{\frac{t}{\tau_2}} =
    b\cdot{}e^{m\cdot{}t}$}
\end{center}

... oder nach ziehen der $t$-ten Wurzel:

\begin{gesetz}{Basiswechsel}{}

  \begin{center}\fbox{\fbox{$a^{\frac{1}{\tau}} =     a_2^{\frac{1}{\tau_2}} = e^{m}$}}\end{center}

 
  $a=$ Wachstumsfaktor\\
  $\tau=$ Beobachtungszeitraum\\
  $e=$ Eulersche Konstante (2.718...)
  $m=\frac1\tau\cdot{}\ln(a) = \frac1{\tau_2}\cdot{}\ln(a_2)$
\end{gesetz}
\newpage

\subsubsection{Beispiel: Wissenschaftliche Publikation (optional)}
In einem wissenschaftlichen Journal ist angegeben, dass eine Population annähernd folgendes Verhalten an den Tag lege [$t$ = Tage]:
$$y = 5.6\cdot{} e^{\frac{t}{4.1}}$$

Berechnen Sie die tägliche Zunahme $a$ und geben Sie somit die Funktionsgleichung ohne Bruch im Exponenten an:

$$y = 5.6\cdot{} e^{\frac{t}{4.1}} = b\cdot{}a^t \approx \LoesungsRaum{5.6} \cdot{} \LoesungsRaum{1.2762}^t$$

\TNT{6.4}{Tipp: Setzen Sie $t=0$ und $t=1$ ein. $$a = e^{\frac{1}{4.1}}$$
\vspace{3cm}}
\newpage

\subsubsection{Bedeutung der Zahl $e$ (optional)}

\paragraph{$45\degre$ auf der $y$-Achse:}
Die E-Funktion $y=e^x$ hat die $45\degre$-Steigung genau auf der $y$-Achse; sie beginnt im Gegensatz zu anderen Exponentialfunktionen genau bei $x=0$ so «richtig zu wachsen»\TALS{\footnote{Wenn wir in einem Punkt $x$ den Wert der $e$-Funktion ($e^x$) betrachten, so ist ihr Funktionswert exakt gleich der Steigung einer Tangente bei diesem $x$-Wert.}}.

\bbwGraph{-4}{3}{-1}{5}{
  \bbwFunc{pow(2.71828,\x)}{-3.5:1.5}
  \bbwFuncC{\x+1}{-2:2}{green}
}%% end graph

Dies ist ein weiterer Grund, warum die Zahl $e$ eine solche Bedeutung bei
exponentiellen Prozessen einnimmt.


\TALS{
\paragraph{Tangente an die E-Funktion}\index{E-Funktion}
Als \textbf{E-Funktion} bezeichnen wir die Exponentialfunktion zur
Basis $e \approx 2.71828172846$; also:
$$f(x) = e^x$$
Diese Funktion ist von allen Exponentialfunktionen insofern speziell,
als dass für jedes $x$ der Funktionswert $y=e^x$ genau die Steigung
der Tangente an $e^x$ im Punkt $P_x=\left(x | e^x\right)$ angibt.

So ist die Tangentensteigung im Punkt $(0|e^0) = (0 | 1)$ genau 1 und die Steigung im
Punkt $(1|e^1) = (1|e)$ genau $e$. 

}%% END TALS


\newpage
