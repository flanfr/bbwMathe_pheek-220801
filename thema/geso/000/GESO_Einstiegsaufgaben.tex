%%
%% 2019 07 04 Ph. G. Freimann
%%

\newpage
\section*{Einstiegsaufgaben}
\sectuntertitel{Der Anfang ist die Hälfte des Ganzen (Aristoteles)}
%%%%%%%%%%%%%%%%%%%%%%%%%%%%%%%%%%%%%%%%%%%%%%%%%%%%%%%%%%%%%%%%%%%%%%%%%%%%%%%%%
\GESO{\cite{marthaler17} ab. S. 22}


\subsection*{Abholen des Bekannten und Geübten}

\GESO{
  
  Vereinfachen Sie den Term so weit wie möglich\footnote{Aufgaben aus
    BMS Aufnahmeprüfungen}:
  $$\sqrt{(7x)^2 + 17x^2 - 2x^2}$$

\TNT{3.2}{
  $$\sqrt{49x^2 + 17x^2 - 2x^2}$$
  $$\sqrt{64x^2} = 8x$$
  \vspace{3cm}} %% END TNT

  %%%%%%%%%%%%%%%%%%%%%%%%%%%%%%%%%%55

  Berechnen Sie die Lösung der Gleichung:

  $$x^2 + 11 = (x+3)^2$$

\TNT{7.2}{$$x = \frac{1}{3}$$
\vspace{5cm}
  }


    %%%%%%%%%%%%%%%%%%%%%%%%%%%%%%%%%%55
  \newpage
  Vereinfachen Sie den Term so weit wie möglich:

  $$\frac{4b^2}{2a}:\frac{b^2}{3a^2} - \frac{a}{5}$$

\TNT{7.2}{$$\frac{4b^2}{2a}\cdot{}\frac{3a^2}{b^2} - \frac{a}{5}$$
    $$\frac{4\cdot{}3a}{2} - \frac{a}{5}$$
    $$\frac{60\cdot{}a}{10} - \frac{2a}{10}$$
    $$\frac{58a}{10} = 5.8a$$
\vspace{5cm}
  
}
  \newpage


In einer Schachtel liegen vier grüne und fünf rote Kugeln.
Sie ziehen nacheinander zwei Kugeln, ohne sie wieder zurückzulegen.

a) Zeichnen Sie einen entsprechenden Wahrscheinlichkeitsbaum und tragen Sie die
Wahrscheinlichkeiten bei den Ästen ein.

\TNT{4.8}{\vspace{48mm}}


b) Berechnen Sie die Wahrscheinlichkeit, zwei grüne Kugeln zu ziehen.

\TNT{4.8}{\vspace{48mm}}


%%%%%%%%%%%%%%%%%%%%%%%%%%%%%%%%%%%%%%%%%%%%%%%5
%  Ein Radfahrer fährt von zu Hause zum Arbeitsplatz. Am Anfang fährt
%  er während 15 Minuten mit einer Geschwindigkeit von 30 km/h. An
%  einer Ampel muss er für 3 Minuten anhalten. Anschließend fährt er
%  während 30 Minuten mit einer Geschwindigkeit von 20 km/h bis zum
%  Arbeitsplatz.

%  Was ist die durchschnittliche Geschwindigkeit des Radfahrers von
%  seinem Wohnort bis zur Arbeit in km/h?

%\noTRAINER{  \mmPapierTheorie{8.0}}
%  \TRAINER{1. Totale Strecke rechnen (17.5 km). 2. Totale Zeit
%    rechnen: 48 Minuten = 0.8 h. Danach: v = s/t: 21.875km/h.
%    \vspace{5cm}}



}%% END GESO
