%%
%% Stochastik Grundlagen
%% 2020 - 08 - 03 φ
%%

\subsection{Zufallsexperimente}\index{Experiment}\index{Zufallsexperimente}

\subsubsection{Laplace-Experiment}\index{Experiment!Laplace}\index{Laplace-Experiment}
Nach Pierre-Simon Laplace (1749-1827).

Ein fairer Spielwürfel wird geworfen. Jede Seite hat genau die selbe
Wahrscheinlichkeit. Wie groß ist nun die Wahrscheinlichkeit, mit einem
Wurf eine größere Zahl als eine 4 zu werfen?

\TNT{2.4}{Günstige Ereignisse: 5 und 6 (\textbf{zwei} Stück). Alle möglichen Ergebnisse: 1 bis 6 (\textbf{sechs} Stück). Wahrscheinlichkeit = $\frac{2}{6}=\frac13$}

Dabei handelt es sich um ein Laplace-Experiment:
\begin{itemize}
\item Jeder einzelne Ausgang (Ergebnis) hat die selbe Wahrscheinlichkeit
      (hier $p = \frac{1}{6}$).
\item Die Wahrscheinlichkeit, dass ein Ereignis (hier $E = \epsdice{5}$
oder $\epsdice{6}$) eintritt, wird berechnet mit $P(E)
= \frac{\textrm{Anzahl gewünschte Ergebnisse}}{\textrm{Anzahl
mögliche Ergebnisse}}$.

Hier $P(\{\epsdice{5}\}\cup\{\epsdice{6}\}) = \frac{\left|\left\{\epsdice{5}, \epsdice{6}\right\}\right|}{|\Omega|}= \frac{|E|}{|\Omega|}=\LoesungsRaum{\frac{2}{6}}$.
\end{itemize}

\begin{definition}{Laplace Experiment}{}
  Zufallsexperimente, deren Elementarereignisse alle gleich wahrscheinlich sind, werden als
  \textbf{Laplace-Experimente} bezeichnet. Die Wahrscheinlichkeit berechnet sich wie folgt:
  $$P(E) = \frac{|E|}{|\Omega|}$$
\end{definition}


Kein Laplace-Experiment ist \zB das Werfen eines gezinkten Würfels,
bei dem die Augenzahl 6 häufiger auftritt als die anderen Augenzahlen.

\subsection*{Aufgaben}
\aufgabenfarbe{Kompendium: S. 42 Kap. 5.2 Aufg. 1. - 3.}

\newpage


%%%%%%%%%%%%%%%%%%%%%%%%%%%%%%%%%%%%%%%%%%%%%%%%%%%%%%%%%%
\subsection{Baumdiagramme}\index{Baumdiagramm!Stochastik}
(Einstufige und mehrstufige Experimente)

\TRAINER{(S. Youtube Videos dazu: Einfach-Mathe!)}

\subsubsection{Einstiegsbeispiel: Glücksrad}
\bbwCenterGraphic{4cm}{geso/stoch/img/gluecksrad75deg}
An einem Glücksrad wird gedreht. $75\degre$ erzielen sofort einen Gewinn. Die anderen $285\degre$ geben einen Verlust an. Zum Glück darf ich am Rad drei mal drehen. Das Spiel endet entweder beim ersten Gewinn oder aber spätestens nach dreimaligem Drehen.

Wie groß ist die Wahrscheinlichkeit, bei diesem Spiel zu gewinnen?

Zeichnen Sie dazu das Baumdiagramm, das bei jedem ``Sieg'' sofort endet und
maximal drei Stufen tief geht.

\TNT{8}{Einseitiger entarteter Baum mit Sieg = 75/360 und Verlust = 285/360. Drei Mal Verlust bei $\left(\frac{285}{360}\right)^3 \approx 49.6\%$; somit Gewinn beim Gegenereignis $1-\left(\frac{285}{360}\right)^3\approx{} 50.38\%$.}

\TRAINER{\bbwCenterGraphic{8cm}{geso/stoch/img/Rad75Baum.png}}

\newpage

\subsection{Pfad- und Summenregel}
Um Wahrscheinlichkeiten aufzuzeichnen, bietet sich das Baumdiagramm an.
Dabei zeichnet man einen «Baum» von links-nach-rechts oder wie in der Informatik üblich von oben nach unten.

Zu den Ästen schreibt man deren Wahrscheinlichkeiten.

Man beginnt bei der Wurzel und für jeden möglichen Ausgang zeichnet man einen Ast. Dabei gelten die folgenden Gesetze:



\begin{gesetz}{Produktregel/Pfadregel}{}\index{Produktregel!Baumdiagramm}\index{Pfadregel!Baumdiagramm}
Alle Wahrscheinlichkeiten von der Wurzel bis zum Endknoten werden aufmultipliziert, um die Endwahrscheinlichkeit (am Endknoten) zu erhalten.
\end{gesetz}

\begin{gesetz}{Summenregel I}{}\index{Summenregel!Baumdiagramm}
Alle von einem Knoten ausgehenden Äste haben in der Summe die Wahrscheinlichkeit 1 (= 100\%).
\end{gesetz}

\begin{gesetz}{Summenregel II}{}
Alle Endwahrscheinlichkeiten (bei den Endknoten\footnote{Endknoten werden auch als Blätter des Baumes bezeichnet.}) zusammen addiert, ergeben in der Summe die Wahrscheinlichkeit 1 (= 100\%).
\end{gesetz}

\begin{gesetz}{Summenregel III}{}
  Um die Wahrscheinlichkeit eines Ereignisses zu berechnen, werden die
  Wahrscheinlichkeiten all derjenigen Endknoten zusammengezählt, deren Ergebnis zum gewünschten Ereignis gehören.
  \end{gesetz}

\newpage


\subsubsection{Referenzaufgaben}

\paragraph{Urne ohne Zurücklegen} In einer Urne liegen drei grüne, zwei blaue und eine rote Kugel. Ich ziehe blind (Urne) zwei Kugeln hintereinander, ohne diese wieder zurückzulegen.

Wie groß ist die Wahrscheinlichkeit, dass genau eine blaue Kugel dabei ist?

Zeichnen Sie das zugehörige Baumdiagramm:

\TNT{14}{
\bbwCenterGraphic{15cm}{geso/stoch/img/UrneBaum.png}
  
  Lösung: $P(E)=\frac8{15}=0.5333...$\vspace{13cm}}
  
\newpage

\paragraph{Ziege}\index{Ziege} Heidi und Peter spielen mit einem Würfel um eine Ziege.

Sie vereinbaren folgende Regeln:

\begin{enumerate}
\item Die Ziege erhält sofort, wer eine 5 oder eine 6 würfelt.
\item Zuerst würfelt Heidi, dann (falls Heidi noch nicht gewonnen hat) Peter, dann allenfalls noch einmal Heidi.
\item Ist nach drei Würfen noch nichts entschieden, dann erhält Peter die Ziege.
\end{enumerate}


Zeichnen Sie das zugehörige Baumdiagramm und entscheiden Sie, ob es sich um ein faires Spiel handelt:

\TNT{14}{\bbwCenterGraphic{15cm}{geso/stoch/img/HeidiPeter.png}

Peter gewinnt also mit $\frac29 + \frac8{28} = \frac{14}{27}$, was etwas mehr als 50\% ausmacht. Das Spiel ist nicht fair, denn wer als zweiter würfeln darf, hat eine leicht höhere Gewinnchance.}
\newpage




\subsection*{Aufgaben}

Mehrstufige Zufallsexperimente:

\aufgabenfarbe{Kompendium: S.42 Kap. 5.2 Aufg. 4. und S. 47 Kap. 5.5.2 Aufg 17., 19., 20. und 21.}
\newpage



%%%%%%%%%%%%%%%%%%%%%%%%%%%%%%%%%%%%%%%%%%%%%%%%%%%%%%%%%%%%%%%%%%%%%%%%

\subsection{Binomialverteilung}


\subsubsection{Einstiegsaufgabe}

\paragraph{Genau zwei Sechser} Ein Würfel wird \textbf{dreimal} hintereinander geworfen.

Wie groß ist die Wahrscheinlichkeit, dass genau zwei Sechser dabei sind?


Zeichnen Sie das zugehörige Baumdiagramm:

\TNT{18}{
\bbwCenterGraphic{15cm}{geso/stoch/img/genau2sechser.png}


  Lösung mit Baum oder  Binomialverteilung.

  Genau zwei Sechser: $\left(\frac16\right)^2\cdot{}\frac56 +
  \left(\frac16\right)^2\cdot{}\frac56 +
  \left(\frac16\right)^2\cdot{}\frac56 =3\cdot{}
  \left(\frac16\right)^2\cdot{}\frac56= \frac{5}{72}\approx 0.06944$%%
%%
  \vspace{10cm}%%
}%% END TNT
\newpage

\TRAINER{Einstiegsvideo: ``EinfachMathe!'' (S. Wiki-Lernvideos)}


\subsubsection{Bernoulli-Experiment}\index{Experiment!Bernoulli}\index{Bernoulli-Experiment}
Ein mehrfacher Münzwurf ist ein typisches
Bernoulli-Experiment\footnote{Jakob I Bernoulli, Schweizer Mathematiker,
  1654-1705}, d.\,h.:

\begin{definition}{Bernoulli-Experiment}{}
\begin{itemize}
\item Das Experiment wird $n$-mal durchgeführt\footnote{Daher wird dieser
Bernoulli-Prozess manchmal auch Bernoulli-Kette genannt. Quelle
\texttt{www.wikipedia.org} 2021-04-08}.
\item Das Einzelexperiment hat genau zwei Ausgänge: Erfolg/Misserfolg.
\item Jedes Einzelexperiment hat für die erfolgreichen Ausgänge die gleiche
      Wahrscheinlichkeit $p$ (somit ist die Wahrscheinlichkeit $q$ für den
      Misserfolg gleich $1-p$).
\item Die Einzelexperimente sind voneinander unabhängig.
\end{itemize}
\end{definition}

\begin{bemerkung}{Bernoulli-Experiment}{}
Typischerweise sind wir bei Bernoulli-Experimenten an der
Zufallsvariable $X$ interessiert, welche angibt, wie oft ein Erfolg
eingetreten ist: $X$ hat also die Werte 0, 1, 2, 3, 4, ..., $n$.
\end{bemerkung}

\textbf{Kein} Bernoulli-Experiment ist beispielsweise das Ziehen von zwei Bällen aus
einer Urne mit drei gelben und sechs orangefarbenen Bällen ohne diese
jeweils zurückzulegen; denn dabei verändern sich die
Wahrscheinlichkeiten der verbleibenden Bälle nach jedem Herausnehmen.
\newpage


\paragraph{Beispiele von Bernoulli-Experimenten}

\begin{itemize}
\item
In einer Urne liegen zehn Kugeln: vier blaue und sechs grüne. Ein
Einzelexperiment besteht darin, genau eine Kugel zu ziehen, die Farbe
zu notieren und die Kugel sogleich wieder zurückzulegen.
Das Gesamtexperiment besteht darin, fünf Einzelexperimente
durchzuführen. Jedesmal ist die Wahrscheinlichkeit, eine blaue Kugel
zu erwischen, gleich groß und wir sprechen von einem Bernoulli-Experiment.

\item Mehrfaches Drehen am Glücksrad.

\item Mehrfaches Würfeln mit einem Spielwürfel, bei dem wir
  interessiert sind, wie oft eine Augenzahl kleiner als 3 geworfen wird.
  
\item
Es müssen aber nicht immer Würfel, Münzen oder Kugeln in Urnen sein:
Der Torwart Alex Calderoni\footnote{Alex Calderoni * 1976: Bekannter
  italienischer Torwart beim Fußbllspiel (Quelle Wikipedia 2022).}
fängt womöglich einen Ball mit einer
Warhscheinlichkeit von 38\%.
Wie groß ist die Wahrscheinlichkeit, dass er sechs von zehn Torschüssen
abfangen kann?
\end{itemize}
\newpage


\subsubsection{Formel zur Binomialverteilung}
\TRAINER{Einstiegs-Viedo: \texttt{https://www.youtube.com/watch?v=WC5O317JzW0}}


\begin{beispiel}{Sieben mal Würfeln}{}
Wie groß ist die Wahrscheinlichkeit, mit einem Würfel bei
siebenmaligem Werfen genau fünf mal eine der Zahlen \epsdice{1} oder
\epsdice{2} zu erreichen?
\end{beispiel}

Bei der Binomialverteilung handelt es sich um ein
Bernoulli-Experiment
in einem mehrstufigen Baum ($n$ Versuche). Dabei interessiert
uns, wie groß die Wahrscheinlichkeit ist, dass von den beiden
möglichen Ausgängen (Treffer / Niete) einer davon $k$-mal auftritt.

Dabei legen wir fest (vgl. Beispiel oben: Siebenmaliges Würfeln):
\begin{itemize}
\item
  $n$ = Anzahl mögliche Durchführungen; hier 7
\item
  $k$ = Anzahl der gewünschten ``Treffer''; hier 5, denn wir wollen fünf
Mal eine \epsdice{1} oder eine \epsdice{2} erzielen.
\item
  $p$ = Wahrscheinlichkeit, dass eine \epsdice{1} oder \epsdice{2}
  geworfen wird. Hier $p = \frac26$.\\
  Somit ist die Misserfolgsquote $q = (1-p)$; hier $1-\frac26=\frac46$.
\end{itemize}

Nun gilt

\begin{gesetz}{Bernoulli-Formel (Binomialverteilung)}{}
  $$P(X=k) = {{n}\choose {k}}\cdot{}p^k\cdot{}(1-p)^{n-k}$$
\end{gesetz}

In obigem Beispiel ergibt sich:
$$P(X=5) = \LoesungsRaumLang{{7\choose 5} \cdot{} \frac26^5 \cdot{} (1-\frac26)^{7-5}} = \LoesungsRaumLang{{7\choose 5} \cdot{} \frac26^5 \cdot{} \frac46^2}$$

Dies kann mit dem Taschenrechner gelöst werden:
Entweder direkt ...

$$\LoesungsRaumLang{( 7 \textrm{\,\,nCr\,\,} 5 ) *\frac26^5 *
\left(1-\frac26\right)^{7-5}} \approx \LoesungsRaum{0.3841 = 3.841\%}$$

... oder aber mit der eingespeicherten Formel bei
\tiprobutton{data_stat-reg-distr} und dort unter \texttt{DISTR: 4
  Binomialpdf: SINGLE} dann $n=7$ bei $n$, $p=2/6$ bei $p$ und
$k=5$ bei $x$ eintragen ($\approx \LoesungsRaum{0.03841}$).
\newpage

\subsection*{Aufgaben}

\aufgabenfarbe{Kompendium: S. 50 Kap. 5.5.4 Aufg. 28. und 29. und Maturaprüfungen: 2016 (GESO) Aufgabe 11 (Geburten), 2017 (GESO) Aufgabe 17. (borkenkäferresistent), 2018 (GESO) Serie 1 Aufg. 16. (Papayasendung), 2018 (GESO) Serie 2 Aufg. 15. (Sprössling), 2018 (GESO) Serie 3 Aufg. 14. (Ikosaeder)}


\newpage
\subsubsection{Kumulierte Wahrscheinlichkeit}

Basketballspieler «Basil Ballisti» trifft mit einer
Wahrscheinlichkeit von 85\%.

Die Wahrscheinlichkeit, dass er von 20 Würfen genau 17 Mal trifft, kennen wir
bereits von der Binomialverteilung (Bernoulli-Experiment):

$$P(E=17) = \LoesungsRaumLang{{20 \choose 17}\cdot 0.85^{17}\cdot(1-0.85)^{20-17}}=$$
$$\LoesungsRaumLang{{20 \choose 17}\cdot 0.85^{17}\cdot (0.15)^{3}}\approx{}\LoesungsRaum{24.28\%}$$
Dabei bezeichnet $E$ das Ereignis ``siebzehn Treffer in den 20 Würfen''.

Wie groß ist nun aber die Wahrscheinlichkeit, dass er bei 20 Würfen
\textbf{maximal} 17 mal trifft?

Lösung: Wir summieren alle Wahrscheinlichkeiten auf, bei denen er
während seinen 20 Würfen
keinen Treffer, einen Treffer, zwei Treffer, ... bis zu 17 Treffer
erzielt. Dies nennen wir die kumulierte\footnote{«Kumulieren» kommt von
  lat. \texttt{cumulus} = Anhäufung.} Wahrscheinlichkeit\index{Wahrscheinlichkeit!kumulierte}:

$$P(E\le 17) = $$
\TNT{1.6}{$$P(E=0) + P(E=1) + P(E=2) + ... + P(E=16) + P(E=17)$$}%% END TNT

Diese Summe
$$P(X\le 17) = \LoesungsRaumLang{\sum_{k=0}^{17} {20 \choose k} \cdot 0.85^k \cdot 0.15^{20-k}}$$
können wir anders schreiben und sofort mit dem
Taschenrechner lösen:

$$\LoesungsRaumLang{\sum_{x=0}^{17}\left(( 20 \textrm{\,\,nCr\,\,} x ) * 0.85^x *
0.15^{20-x}\right)} \approx \LoesungsRaum{0.5951}$$


Doch auch hierzu gibt es eine eingespeicherte Formel
bei
\tiprobutton{data_stat-reg-distr} und dort unter \texttt{DISTR: 5
  Binomialcdf: SINGLE} dann $n=20$ bei $n$, $p=0.85$ bei $p$ und
$k=17$ bei $x$ eintragen ($\approx 0.5951$).

\begin{gesetz}{Kumulierte Wahrscheinlichkeit}{}
  $$P(X < k) = \sum_{i=0}^k {n \choose i} \cdot{} p^i \cdot{} (1-p)^{n-i}$$
  \end{gesetz}

\newpage
\subsection*{Aufgaben}

\aufgabenfarbe{Kompendium: S. 53ff Kap. 5.7. Aufg. 36. a) b) und 37.}
\newpage

\subsection{Wahrscheinlichkeitsverteilungen ...}\index{Wahrscheinlichkeitsverteilung}

Für ein Bernoulli-Experiment können wir uns auch ein Histogramm\index{Histogramm} aufzeichnen.

Nehmen wir wieder unseren Basketballspieler «Basil», der mit Wahrscheinlichkeit von 85\% jeweils trifft. Vorhin hatten wir uns gefragt, wie groß die Wahrscheinlichkeit ist, dass er in 20 Würfen genau 17 Mal [bzw. maximal 17 Mal] trifft.

Dies können wir in einer Wertetabelle und somit auch in einem Histogramm darstellen.
\newpage


\subsubsection{... genaue Anzahl Treffer}
Notieren wir uns, wie groß die Wahrscheinlichkeit ist, dass Basil genau $n$ Mal trifft innerhalb von 20 Würfen:

\begin{tabular}{c|cccccccccccccccc}
  n & 0 & 1 & ... & 8 & 9 & 10   & 11   & 12   & 13  & 14  & 15 & 16 & 17 & 18 & 19 &  20\\
  \%& 0 & 0 & ... & 0 & 0 & 0.02 & 0.11 & 0.46 & 1.6 & 4.5 & 10 & 18 & 24 & 23 & 14 & 3.9
\end{tabular}

Wenn wir diese Zahlen in ein Histogramm eintragen erhalten wir eine
Wahrscheinlichkeitsverteilung:


\bbwCenterGraphic{10cm}{geso/stoch/img/BernoulliSingleBasil.png}


\subsubsection{... kumulierte Anzahl Treffer}
Wie im vorangehenden Kapitel können wir uns die Wahrscheinlichkeiten
auch zusammenzählen und \zB fragen, wie groß ist die
Wahrscheinlichkeit, dass Basil bei 20 Würfen \textbf{maximal} 17
Treffer landet. Dabei darf er für maximal 17 Treffer natürlich auch
weniger Treffer landen. So werden für ein Anzahl Treffer alle
Wahrscheinlichkeiten der weniger erzielten Treffer hinzugezählt. So
erklären sich auch die 100\% um maximal 20 Treffer in 20 Würfen zu
landen.

\begin{tabular}{c|cccccccccccccccc}
  n  & 0 & 1 & ... & 8 & 9 & 10  & 11   & 12   & 13  & 14  & 15 & 16 &  17 & 18 & 19 &  20\\
  \% & 0 & 0 & ... & 0 & 0 &  0.02  &0.13   & 0.59 & 2.2 & 6.7 & 17 & 35 &  60 & 82 & 96 & 100
\end{tabular}

Das entsprechende Diagramm sieht wie folgt aus:

\bbwCenterGraphic{10cm}{geso/stoch/img/BernoulliCumulativeBasil.png}
\newpage

\subsection*{Aufgaben}

Zeichnen Sie in folgender Aufgabe beide Diagramme: Die
Wahrscheinlichkeiten für die genaue Anzahl Treffen, aber auch die
Wahrscheinlichkeiten für die kumulativen Wahrscheinlichkeiten:

\aufgabenfarbe{Kompendium: S. 53 Kap. 5.7. Aufg. 36. c)}

\newpage

\subsection{Hypergeometrische Verteilung}\index{Verteilung!Hypergeometrische}\index{Hypergeometrische Verteilung}
In einer Urne liegen sieben Kugeln. Drei davon sind Treffer (grün) und vier davon sind Nieten (schwarz).
Wir dürfen zwei Kugeln blind herausnehmen (ohne diese wieder zurückzulegen).

\bbwCenterGraphic{5cm}{geso/stoch/img/Urne3T4N.png}

Wie groß ist nun die Wahrscheinlichkeit
\begin{itemize}
\item beide Treffer zu erzielen ($t=2$),
\item genau einen Treffer zu erzielen ($t=1$) oder
\item gar keinen Treffer zu erzielen ($t=0$)?
\end{itemize}

Diese Wahrscheinlichkeiten können wir nun einerseits durch ein Baumdiagramm lösen, indem wir die Treffer mit \textbf{\color{green}T} und die Nieten mit \textbf{\color{red}N} bezeichnen:

\noTRAINER{\mmPapier{7.2}}
\TRAINER{
\bbwCenterGraphic{7cm}{geso/stoch/img/Baum2aus3T4N.png}
}

Somit erhalten wir für zwei Treffer ($t=2$) eine Wahrscheinlichkeit von $\frac{6}{42}$; für einen Treffer ($t=1$) die Summe aus $\frac{12}{42}$ und nochmals $\frac{12}{42}$ also total $\frac{24}{42}$ und schlussendlich für keinen Treffer ($t=0$) die Wahrscheinlichkeit $\frac{12}{42}$.
Die Summe aller Wahrscheinlichkeiten muss dann immer gleich 1.0 (100\%) sein.
\newpage

Andererseits gibt es dazu auch eine Formel, die \textbf{hypergeometrische Verteilung}\footnote{Die Folgeglieder der \textit{geometrischen Verteilung} $P(X=t) = p\cdot{}q^{t-1}$ verhalten sich wie eine geometrische Reihe (die Quotienten von zwei Folgengliedern $P(X=t) : P(X=t+1)$ sind jeweils konstant). Da die hier betrachtete Verteilung noch rascher ``abfällt'', wie die \textit{geometrische Verteilung}, so sprechen wir hier von \textbf{hyper}geometrisch.}.

Es seien:
\begin{itemize}
\item $T$ die Anzahl der Treffer in der Urne. Hier die Anzahl der grünen Kugeln (also $T = 3$)
\item $N$ die Anzahl der Nieten in der Urne. Hier die Anzahl schwarze Kugeln ($N=4$)
\item $T+N$ die Anzahl Elemente (Treffer + Nieten) in der Urne (hier $T+N = 7$)

\item $t$ die Anzahl der gewünschten bzw. zu erreichenden Treffer. Hier \zB $t=2$.
  
\item $n$ die Anzahl «gewünschten» Nieten (hier z. B. $n = 0$)
\item $t+n$ die Anzahl total herausgezogener Kugeln (hier $t+n=2+0=2$)
\end{itemize}
\begin{gesetz}{}{}
$$P(X=t) = \frac{ {T\choose t} \cdot{} {{N}\choose {n}}}{{{T+N} \choose {t+n}}}$$
\end{gesetz}
\textbf{Beispiele} mit obigen sieben Kugeln (drei Treffer und vier
Nieten)
$$T=3, N=4, T+N=7$$

\textbf{Beispiel 1}: Ich will bei zwei gezogenen Kugeln je einen Treffer:

($t=2$) mit zwei Ziehungen ($t+n=2$), somit keine Niete ($n=0$).
Wir erhalten:

$$P(t=2) = \frac{ {T\choose t} \cdot{} {{N}\choose {n}}}{{{T+N}
    \choose {t+n}}} = \LoesungsRaumLang{\frac{ {3\choose 2} \cdot{} {{4}\choose {0}}}{{7
    \choose 2}} = \frac{3\cdot{}1}{21} = \frac17 \approx 0.14}$$

\textbf{Beispiel 2}: Ich will bei zwei gezogenen Kugeln genau einen Treffer:

($t=1$) mit zwei Ziehungen ($t+n=2$), somit eine Niete ($n=1$).
Wir erhalten:

$$P(t=1) = \frac{ {T\choose t} \cdot{} {{N}\choose {n}}}{{{T+N}
    \choose {t+n}}} = \LoesungsRaumLang{\frac{ {3\choose 1} \cdot{} {{4}\choose {1}}}{{7
    \choose 2}} = \frac{3\cdot{4}}{21} = \frac{12}{21} \approx 0.57}$$


\newpage

\subsection*{Aufgaben}

\aufgabenfarbe{Kompendium: S. 49 Kap. 5.5.3 Aufgaben 25., 26., 24. und optional 27.}
\newpage

