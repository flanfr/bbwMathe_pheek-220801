\section{Dreidimensionale Objekte}\index{Objekte!dreidimensionale}\index{dreidimensionale Objekte}
\sectuntertitel{Endlich technisch möglich: 3D in Farbe: {\color{red}D}{\color{blue}D}{\color{green}D}} 

\theorieTALSGeom{133}{3}

\subsection*{Lernziele}
\begin{itemize}
\item allgemeiner Zylinder
  \begin{itemize}
  \item Prisma
  \item Kreiszylinder
    \end{itemize}
\item Konus (verallgemeinerter Kegel)
  \begin{itemize}
  \item Pyramide
  \item Kreiskegel
  \end{itemize}
  \item stumpfe Körper
  \item Kugel
\end{itemize}
\newpage
\subsection{Überblick über die betrachteten Körper}\index{Prisma}\index{Kreiszylinder}\index{Pyramide}\index{Kreiskegel}\index{Kugel}\index{Kouns}

\bbwCenterGraphic{16cm}{tals/stereo/img/Ueberblick.png}

Bemerkungen: Auch wenn es beliebige allgemeine Zylinder oder Kegel
gibt, betrachten wir nur solche, welche als Grundfläche ein Polygon
(Vieleck) oder einen Kreis aufweisen.\footnote{Der allgemeine Kegel wird auch als Konus oder «echter Kegel in $\mathbb{R}^3$» bezeichnet.}

Die folgenden Begriffe werden für die Berechnungen benötigt:

\begin{tabular}{|l|p{6cm}|p{9cm}|}
  \hline
  Kürzel & Begriff& Bemerkung\\\hline
  $r$  & Radius & (bei Kreiszylinder, Kreiskegel und Kugel)\\\hline
  $G$  & Grundfläche & (nicht vorhhanden bei der Kugel)\\\hline
  $h$  & Höhe & Die Höhe steht senkrecht zur Grundfläche und reicht bis zur Deckfläche oder bis zur Spitze\\\hline
  $D$  & Deckfläche & verschwindet beim Konus (= allgemeiner Kegel)\\\hline
  $M$  & Mantelfläche & Summe aller Seitenflächen ohne Grund- und Deckfläche\\\hline
  $S$  & Oberfläche (Surface)& Summe von Grund-, Deck- und Mantelfläche
  (wird manchmal als $O$ bezeichnet)\\\hline
  $V$  & Volumen & \\\hline
  \end{tabular} 
\newpage
