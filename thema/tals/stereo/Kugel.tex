\section{Kugel}\index{Kugel}

\theorieTALSGeom{166}{3.3.3}

\subsection*{Lernziele}
\begin{itemize}
\item Volumen der Kugel
\item Oberfläche der Kugel
\item Einbeschriebene und umschriebene Körper
\item Kugelsektor (optional)
\item Kuglesegment (optional)
\item Kugelhaube (Klaotte)
\item Kugelschicht
\end{itemize}
\newpage

\subsection{Kugelkörper}

\bbwCenterGraphic{5cm}{tals/stereo/img/Sphaere.png}

\begin{definition}{Kugel}{}\index{Kugel}\index{Sphäre}\index{Vollkugel}
  Die Menge aller Punkte, welche von einem Punkt $M$ (dem \textbf{Mittelpunkt}) den selben Abstand ($r$ = Radius) haben, nennen wir die \textbf{Kugelfläche}, \textbf{Kugeloberfläche} oder \textbf{Sphäre}.
  Ist das Innere der Kugel mitgemeint, so sprechen wir vom \textbf{Kugelkörper} oder von der \textbf{Vollkugel}.
\end{definition}
\newpage

\subsection{Kugelvolumen}\index{Volumen!Kugel}\index{Kugel!Volumen}

\TRAINER{Video Volumen/Oberflächenformel:\\\texttt{https://www.youtube.com/watch?v=jQoHp\_P8Y0A}}

\begin{gesetz}{Volumen}{}

  $V$ = Volumen der Kugel

  $r$ = Radius der Kugel

  $$V = \frac43\cdot{}\pi\cdot{}r^3$$
  \end{gesetz}

Herleitung:\\
\TNT{16}{

  \bbwCenterGraphic{15cm}{tals/stereo/img/Kugelvolumen.png}
  Wähle beliebige Höhe $h$ in beiden Figuren (links Halbkugel, rechts Vergleichskörper).

  Berechne die Schnittflächen der beiden Körper in Höhe $h$:
  
  \begin{tabular}{p{8cm}|p{8cm}}\hline
    $A = R^2\pi = (r^2-h^2)\pi$, wegen Pythagoras ($R^2$ = $r^2-h^2$)&
    Kreisring = Außenkreis ($r^2\pi$) - Innenkreis ($h^2\pi$), denn
    Kegel ist 45\degre{} ausgeschnitten. \\
    \hline
    \end{tabular} 

  Volumen Halbkugel = (dank Cavalieri) Volumen Vergeichskörper =
  Volumen Kreisszylinder - Volumen Kreiskegel = $r^2\pi r - \frac13
  r^2\pi r = \frac23 r^3 \pi$
  
  Somit ist das Volumen der ganzen Kugel = 2 $\cdot$ Volumen Halbkugel = $V = \frac43\pi r^3$.  
}%% END TNT
\newpage


\subsection{Kugeloberfläche}
\begin{gesetz}{Oberfläche}{}

  $S$ = Oberfläche der Sphäre (Surface)

  $r$ = Radius der Kugel

  $$S = 4\cdot{}\pi\cdot{}r^2$$
  \end{gesetz}

\subsection*{Aufgaben}
\TALSGeomAadB{167}{177., 179., 180. (Pyramide in Kugel), 181. (Zylinder in Kugel), 182. (Kugel in Kegel) und 200. (Kugelschale)}
\newpage
