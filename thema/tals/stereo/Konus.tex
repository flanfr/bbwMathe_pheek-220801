\section{Konus (Pyramide und Kreiskegel)}\index{Kegel}\index{Kreiskegel}\index{Pyramide}\index{Konus}\index{allgemeiner Kegel}

(Konus steht hier für \textit{allgemeiner Kegel}.)

\TRAINER{Einstiegsvideo: Beweis der 1/3 Formel auf Youtube (s. Wiki).}

\bbwCenterGraphic{7cm}{tals/stereo/img/AllgemeinerKonus.png}

\begin{gesetz}{Volumen Konus (Pyramide und Kreiskegel)}{}
  
  $V$ = Volumen
  
  $G$ = Grundfläche

  $h$ = Höhe
  $$V = \frac13\cdot{}G\cdot{}h$$
\end{gesetz}

\begin{gesetz}{Oberfläche}{}

  $S$ = Oberfläche

  $G$ = Grundfläche

  $M$ = Mantelfläche

  $$S = G + M$$
  \end{gesetz}
\newpage

\subsection{Spezialfall: Pyramide}\index{Pyramide}

\bbwCenterGraphic{11cm}{tals/stereo/img/Pharao.png}

\begin{center}
{\textit{«Keine Fenster? Keine sanitären Anlagen? Kein W-LAN? Ist ja schon
jetzt eine Ruine. Da kann ich mich gleich darin begraben
lassen.»}}\footnote{Die rund 100m hohe \textbf{rote Pyramide} ist das
  erste in echter Pyramidenform erbaute Wahrzeichen im «Alten
  Ägyptischen Reich». Sie wurde um 2500 v.\,u.\,Z. von König Snofru in
Dahschur erbaut.}
\end{center}

\bbwCenterGraphic{7cm}{tals/stereo/img/Pyramide.png}

\begin{bemerkung}{Höhe}{}
  Wir unterscheiden

  $h$ = Höhe der Pyramide

  $h_s$ = Höhe einer Seitenfläche (Dreieckshöhe)
\end{bemerkung}
\begin{gesetz}{}{}
  Es gilt
  $$h_S \ge h$$
  \end{gesetz}
\newpage
\subsection*{Aufgaben}
\TALSAadB{146ff}{86. (Trigonometrie)}
\newpage

\subsection{Spezialfall: Kreiskegel}\index{Kreiskegel}

\bbwCenterGraphic{5cm}{tals/stereo/img/Kreiskegel.png}

\begin{gesetz}{Oberfläche eines Kreiskegels}{}

  $S$ = Oberfläche (Surface)

  $r$ = Radius der Grundfläche
  
  $G$ = Grundfläche = $r^2\pi$

  $h$ = Höhe

  $m$ = Mantellinie = $\sqrt{h^2 + r^2}$
  
  $M$ = Mantelfläche = $\frac{2r\pi \cdot{} m}2 = r\pi\cdot{}m$

  
  $$S = G + M = r^2\pi + r\pi m = r\pi\cdot{}(r+m)$$
\end{gesetz}


\begin{gesetz}{Volumen eines Kreiskegels}{}

  $V$ = Volumen

  $r$ = Radius der Grundfläche
  
  $G$ = Grundfläche = $r^2\pi$

  $h$ = Höhe

  $$V = \frac13\cdot{} G\cdot{}h = \frac13 \cdot{} r^2\pi \cdot{} h$$
\end{gesetz}


\begin{bemerkung}{Mantellinie}{}
  Die Mantellinie $m$ ist immer größer oder gleich der Höhe $h$:
  $$m \ge{} h$$
\end{bemerkung}

\newpage


\begin{bemerkung}{}{}
  Wir unterscheiden beim Kreiskegel die Winkel

  $\alpha$ = Öffnungswinkel des Kegels an der Spitze

  $\beta$ = Böschungswinkel (= Neigung der Mantellinie = $90\degre - \frac{\alpha}{2}$)

  Für den Sektorwinkel $\stackrel{\frown}{\varphi}$ gilt im Bogenmaß: $m\cdot\stackrel{\frown}{\varphi} = 2r\pi$ (S. Abwicklung)\\
  und somit im Gradmaß:\\ $\varphi = \frac{180\degre}\pi \cdot{} \stackrel{\frown}{\varphi} = \frac{180\degre}\pi \cdot \frac{2r\pi}m = 360\degre\cdot{}\frac{r}m$ 
  
  $\varphi$ = Sektorwinkel des abgewickelten Mantels = $360\degre \cdot \sin\left(\frac{\alpha}2\right)$

\end{bemerkung}

Beweis der $\sin()$-Formel:

\TNT{16}{
  Skizze Kegel mit a) $m$ - Mantellinie und Radius $r$ und b) Abwicklung mit $\varphi$, $m$ und $2r\pi$ als Bogenlänge.

  \bbwCenterGraphic{16cm}{tals/stereo/img/KegelSinus.jpg}
  
  Gegeben: Öffnungswinkel des Kegels: $\alpha$

  Gesucht: Sektorwinkel $\varphi$ der Abwicklung (Mantel)

  1) Seitenriss: Es gilt: $\frac{r}m = \sin\left(\frac\alpha2\right)$

  2) $\varphi = 360\degre\cdot{}\frac{r}m$ und somit (1 Einsetzen in 2):

  3) $\varphi = 360\degre \cdot{} \sin\left(\frac\alpha2\right)$
}%% END TNT

\newpage

\subsection*{Aufgaben}


\TALSAadB{158ff}{150., 151., 157. (Netzabw.), 158. (Pyramide in Kegel), 163. (zwei Kegel im Zylinder)}
\newpage

