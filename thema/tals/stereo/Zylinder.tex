\section{Allgemeiner Zylinder (Prisma und Kreiszylinder)}\index{Zylinder}\index{Prisma}

\bbwCenterGraphic{5cm}{tals/stereo/img/AllgemeinerZylinder.png}

\begin{gesetz}{Volumen allgemeiner Zylinder (inkl. Prisma)}{}\index{Volumen!Zylinder ink. Prisma}

  $V$ = Volumen\\
  $G$ = Grundfläche\\
  $h$ = Höhe\\
  \begin{center}\fbox{$V = G\cdot{}h$}\end{center}
\end{gesetz}

\begin{gesetz}{Oberfläche allgemeiner Zylinder (inkl. Prisma)}{}\index{Oberfläche!Zylinder inkl. Prisma}

  $S$ = Oberfläche (manchmal als $O$ bezeichnet)\\
  $G$ = Grundfläche\\
  $D$ = Deckfläche (Es gilt: $D=G$)\\
  $h$ = Höhe\\
  $M$ = Mantelfläche = $h \cdot{} \textrm{Umfang der Grundfläche}$\\
  \begin{center}\fbox{$S = G + D + M  = 2\cdot{}G + M$}\end{center}
\end{gesetz}



\newpage
\subsection{Quader und Würfel}\label{QuaderUndWuerfel}
Die beiden speziellsten Prismen sind der Quader und der Würfel.

Für ein Quader mit den Seitenlängen $a$, $b$ und $c$ gilt:


Im Spezialfall «Würfel» gilt:
\begin{gesetz}{}{}\\
  Volumen $V = G\cdot{} h = a^3$\\
  Oberfläche $S=2\cdot{}G + M = 6\cdot{}a^2$\\
  Raumdiagonale $d = \sqrt{3\cdot{}a^2} = a\cdot{}\sqrt{3}$
\end{gesetz}

\bbwCenterGraphic{12cm}{tals/stereo/img/WuerfelUndQuader.png}

Der beliebige Quader hat die Seiten $a$, $b$ und $c$:
\begin{gesetz}{}{}\\
  Volumen $V = G\cdot{} h = (a\cdot{}b)\cdot{}h = a\cdot{}b\cdot{}c$\\
  Oberfläche $S=2\cdot{}G + M = 2\cdot{}ab + ac + bc + ac + bc = 2\cdot{}(ab + ac + bc)$\\
  Raumdiagonale $d = \sqrt{a^2 + b^2 + c^2}$
\end{gesetz}

\subsection*{Aufgaben}
\TALSGeomAadB{139}{21. d)}
\newpage



\subsection{Prisma}\index{Prisma}
Prisma: \theorieTALS{141}{3.2.1}\\

\bbwCenterGraphic{18cm}{tals/stereo/img/toblerbms.png}
\begin{center}Bildherkunft offizielle Webseite: \texttt{https://toblerone.fr}\end{center}

\bbwCenterGraphic{5cm}{tals/stereo/img/AllgemeinesPrisma.png}

\begin{bemerkung}{}{}
  Alle Seitenflächen eines geraden Prismas sind Rechtecke mit derselben Höhe $h$.
  \end{bemerkung}

\subsection*{Aufgaben}

\TALSAadB{142ff}{32., 33., 57. und 60.}
\newpage


\subsection{Kreiszylinder}\index{Zylinder!Kreiszylinder}\index{Kreiszylinder}

Kreiszylinder: \theorieTALS{158}{3.3.1}

\bbwCenterGraphic{8cm}{tals/stereo/img/KreiszylinderVolumen.png}

\subsubsection{Volumen}

\begin{bemerkung}{Grundfläche}{}
  Die \textbf{Grundfläche} im Kreisyzlinder ist eine Kreisfläche:

  $G$ = Grundfläche

  $r$ = Zylinderradius
  
  $$G = r^2\pi$$
\end{bemerkung}

\begin{gesetz}{Volumen}{}
  Das \textbf{Volumen} des Zylinders ist gegeben durch

  $V$ = Volumen

  $r$ = Zylinderradius

  $h$ = Zylinderhöhe

  $G$ = Grundfläche (= $r^2\pi$)

  $$V = G\cdot{}h = r^2\pi\cdot{} h$$
\end{gesetz}
\newpage

\subsubsection{Oberfläche}
\bbwCenterGraphic{12cm}{tals/stereo/img/Kreiszylinder.png}


\begin{bemerkung}{Mantelfläche}{}
  Die \textbf{Mantelfläche} im Kreisyzlinder kann am einfachsten durch eine «Abwicklung» erklärt werden. Dabei entsteht ein Rechteck:

  $M$ = Mantelfläche

  $r$ = Zylinderradius

  $h$ = Zylinderhöhe

  $U$ = Umfang der Grundfläche $= 2r\pi$
  
  $$M = U\cdot{}h = 2r\pi\cdot{}h$$
\end{bemerkung}

\begin{gesetz}{Oberfläche}{}
  Die \textbf{Oberfläche} $S$ ist hier im speziellen:

  $S$ = Oberfläche

  $r$ = Zylinderradius

  $h$ = Zylinderhöhe

  $G$ = Grundfläche $=r^2\pi$ = Deckfläche $D$

  $M$ = Mantelfläche $=2r\pi\cdot{}h$
  
  $$S = G + D + M = 2\cdot{}r^2\pi + 2r\pi\cdot{}h = 2r\pi\cdot{}(r + h)$$
\end{gesetz}
\newpage

\subsection*{Aufgaben}
\TALSGeomAadB{158ff}{136}

\newpage
