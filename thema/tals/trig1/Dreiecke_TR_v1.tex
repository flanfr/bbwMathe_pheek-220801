%%
%% Trigonometrische Funktionen, die mit dem Taschenrechner gleöst
%% werden.
%% Es spielen mit: Sinus, Cosinus, Tangens und die arc-Funktionen.
%%


\begin{frage}[1]
 Berechnen Sie den Winkel $\alpha$ (Alpha) mit Hilfe des Taschenrechners
 und geben Sie das Resultat auf exakt \textbf{drei} signifikante Ziffern an:
\begin{center}
\raisebox{-1cm}{\includegraphics[width=5cm]{P_TRIG/img/dreiecke/aufgabe5cm3cm.png}}
\end{center}

$$\alpha \approx \LoesungsRaum{53.1 Grad}$$
  
\platzFuerBerechnungen{5.2}
\end{frage}


\begin{frage}[1]
 Berechnen Sie die Länge der Strecke $x$ mit Hilfe des Taschenrechners
 und geben Sie das Resultat auf exakt \textbf{drei} signifikante Ziffern an:
\begin{center}
\raisebox{-1cm}{\includegraphics[width=5cm]{P_TRIG/img/dreiecke/aufgabe35grad13mm.png}}
\end{center}

$$x \approx \LoesungsRaum{ 9.10 mm}$$

\platzFuerBerechnungen{5.2}
\end{frage}
  
  

%\begin{frage}[1]
% Berechnen Sie die Länge der Strecke $x$ mit Hilfe des Taschenrechners
% und geben Sie das Resultat auf drei signifikante Ziffern an:
%\begin{center}
%\raisebox{-1cm}{\includegraphics[width=3cm]{p_img/trigo/aufgabe70grad3cm.png}}
%\end{center}
%
%$$x \approx ..................\TRAINER{8.77 cm}$$
%  
%\platzFuerBerechnungen5}
%\end{frage}
  

\begin{frage}[1]
 Berechnen Sie den Winkel $\beta$ (Beta) mit Hilfe des Taschenrechners
 und geben Sie das Resultat auf exakt \textbf{drei} signifikante Ziffern an:
\begin{center}
\raisebox{-1cm}{\includegraphics[width=5cm]{P_TRIG/img/dreiecke/aufgabe9cm8cm.png}}
\end{center}

$$\beta \approx \LoesungsRaum{41.6 Grad}$$
  
\platzFuerBerechnungen{5}
\end{frage}
  
