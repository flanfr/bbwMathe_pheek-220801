%% Einbinden in Trigo: Rechtwinkliges Dreieck, aber auch
%% in Planimetrie: Satz des Pythagoras
%%
%% Hier:
%%   * Allgemeine Form
%%   * Höhensatz
%%   Nicht dabei:
%5     -Sinus/Cosinus/Tangens (dies ist nur in Trigo)
%5     -Höhensatz (der kommt nur bei der Planimetrie)



%% Load this only once (the first occurence)!
%% see here: https://tex.stackexchange.com/questions/195157/is-there-any-analog-to-pragma-once-in-latex
\ifcsname XX_Pythagoras.tex\endcsname
  \expandafter\endinput
\fi
\expandafter\gdef\csname XX_Pythagoras.tex\endcsname{loaded}



%%%%%%%%%%%%%%%%%%%%%%%%%%%%%%%%%%%%%%%%%%%%%%%%%%

\subsection{Satz des Pythagoras (Repetition)}\index{Pythagoras!Satz des|textbf}\index{Satz des Pythagoras|textbf}

\fbox{\parbox{\textwidth}{
$$a^2 + b^2 = c^2$$
}}

\TNT{7.6}{Platz für graphischen Beweis\vspace{3cm}}

\ifisTRAINER{\newpage}\fi%%
\subsection{Höhensatz}\index{Höhensatz}

Im rechtwinkligen Dreieck gilt:
$$h^2=p\cdot{}q$$

Beweise:

\TNT{6.4}{Beweis mit Pythagoras: $h^2 = a^2 - p^2$ und $h^2 = b^2 -
  q^2$.\\
Somit gilt
$2h^2 = a^2 + b^2 - p^2 - q^2 = c^2 - p^2 - q^2 = (p+q)^2 -p^2 - q^2 =
  p^2 +2pq + q^2 - p^2 - q^2 = 2pq$\\
Daraus folgt:$$h^2=pq$$ Nun brauchen wir nur noch auf beiden Seiten die
Wurzel zu ziehen.

Beweis mit Ähnlichkeit: $p:h = h:q$, somit folgt der Satz automatisch.}
