%%
%% 2019 07 04 Ph. G. Freimann
%%

\section{Elemente und Zusammenhänge}
\sectuntertitel{$H_2O$ ist kein Element}
%%%%%%%%%%%%%%%%%%%%%%%%%%%%%%%%%%%%%%%%%%%%%%%%%%%%%%%%%%%%%%%%%%%%%%%%%%%%%%%%%

\TALSTadBFWG{8}{1.1; 1.1.1}
\TALSTadBFWG{1.1.2}{1.1.2}


\subsection*{Lernziele}

\begin{itemize}
\item Elemente:
  \begin{itemize}
    \item Höhen, Seiten- und Winkelhalbierende und
      Mittelsenkrechte im Dreieck,
    \item Mittellinie im Trapez,
    \item Kreis: Tangente
      \end{itemize}
  \item Zusammenhänge: Umfang, Flächeninhalt und Abstand
\end{itemize}

\subsection{Tipps}
Um geometrische Aufgaben zu lösen, hat bei mir folgendes meist geholfen:

\begin{enumerate}
\item Machen Sie eine Skizze
\item Machen Sie eine möglichst genaue Konstruktion
\item Geben Sie Gegebenem und Gesuchtem Namen
\item Verwenden Sie Farben für Gegebenes
\item Verwenden Sie die selben Farben (od. Symbole) für die selben Streckenlängen, Winkel, Flächen
\item Bei Aufgaben mit Kreisen: Verbinden Sie die Mittelpunkte 
\item Suchen Sie rechtwinklige Dreiecke (Pythagoras)

\end{enumerate}

\subsection*{Aufgaben}
%%\TALSAadBFWG{???}{???}
Das Buch hat hierzu (mit Ausnahme der Tangente \TALSTadBFWG{33}{1.2.5}) keine zusammengefasste Theorie: Die Aufgaben dazu sind verstreut in den übrigen Kapiteln zur Planimetrie zu finden (Seite 10 - 82 \cite{frommenwiler18geom}).
\TALSAadBFWG{10ff}{2. (1) (2) (3) 4. }
\TALSAadBFWG{16ff (Winkel am Kreis)}{23. a) c) e) }
\TALSAadBFWG{27ff (Kreisberührung)}{81. 82. 83. 86.}
\TALSAadBFWG{33ff (Tangenten)}{106. 107.}
\TALSAadBFWG{34ff (vermischte Aufgaben)}{113. }
\GESOAadBMTA{???}{???}
\newpage
