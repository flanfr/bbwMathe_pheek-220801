%%
%% 2020 03 27 Ph. G. Freimann
%%
\subsection{Spezialfälle}\index{Spezialfälle!quadratische Gleichungen}

%%\TALSTadBFWA{95 ff}{2.3.3.1}

\subsubsection{$c=0$}
Wenn eine quadratische Gleichung der Form
$$ax^2 +bx = 0$$
gegeben ist, so kann man einfach ein $x$ ausklammern:

$$x(ax+b)=0$$
 Die Gleichung ist erfüllt, wenn nun entweder $x$ selbst oder aber
 der Klammerausdruck $(ax+b)$ Null werden. Somit haben wir sofort zwei
 Lösungen gefunden:
 $$\lx=\left\{0, \frac{-b}{a}\right\}$$

 %%\TALSAadBMTA{95ff}{265. a) b) c) e) g)}
 \newpage

 
 \subsubsection{$b=0$ (reinquadratische Gleichungen)}
\TadBMTA{166}{10.2.1}
 Ist die quadratische Gleichung in der Form
 $$ax^2 + c = 0$$
 gegeben, so gibt es ebenfalls eine einfache Lösungsformel. Es folgt:
 $$ax^2 = -c$$
 und daraus:
 $$x^2 = \frac{-c}{a}$$

 Die Lösungsmenge ist also schlicht
 $$\lx=\left\{ + \sqrt{\frac{-c}{a}}, -\sqrt{\frac{-c}{a}} \right\}.$$

%%\TALSAadBMTA{95ff}{266. b) c) d)}
\newpage
