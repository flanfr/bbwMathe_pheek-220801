\section{Vektorbegriff}\index{Vektor}

\theorieTALSGeom{175}{4}
Weitere Theorie: S. 176-178, S. 181, S. 187 und S. 189


\subsection*{Lernziele}
\begin{itemize}
\item Vektor als Repräsentant
\item Betrag
\item Richtung
\item Skalar vs. Vetkor
\end{itemize}

\begin{bemerkung}{Vektor}{}
  Der Begriff «Vektor» kommt aus dem Lateinischen und bedeutet so viel
  wie «Träger» oder «Fahrer».
\end{bemerkung}
\newpage
\subsection{Motivation}
\bbwCenterGraphic{18cm}{tals/vecg1/img/DreiSchwimmer.png}

Wer erreicht das andere Ufer am raschesten?
\TNT{7.6}{

  Für die Geschwindigkeiten ist lediglich die «$y$»-Koordinate
  zuständig. Daneben rechnen wir erst alles in km / h.

Schwimmer A: $y = 2\cdot{}\sqrt2 $ (km/h) und somit erreicht er
das andere Ufer nach
$t=\frac{s}{v}=0.2 : (2\cdot{}\sqrt2) \approx 0.07071 [\textrm{h}] = 4.24 \textrm{ min.}$

Schwimmer B: $y = 3 $ (km/h) und somit erreicht er
das andere Ufer nach
$t=\frac{s}{v}=0.2 : 3 = 4 \textrm{ min.}$

Schwimmer(in) C: $y = 3.5\cdot{}\frac{\sqrt{3}}{2}$ (km/h) und somit erreicht sie
das andere Ufer nach
$t=\frac{s}{v}=0.2 : (1.75\cdot{}\sqrt{3}) \approx 3.96 \textrm{ min.}$

}%% END TNT
\newpage


Wo landen die drei Schwimmer am anderen Ufer?

\TNT{12}{
  Bei allen dreien «addieren» wir die Pfeile. Das entstehende Dreieck
  ist ähnlich wie ein Dreieck über den ganzen Fluss. $x$ sei der
  «Drift» nach rechts.
  
  Schwimmer A: $\frac{6-2\cdot{}\sqrt{2}}{2\cdot{}\sqrt2}
  =\frac{x}{0.2} \Longrightarrow x\approx 224 \textrm{ m}$
  
  Schwimmer B: $\frac63 =\frac{x}{0.2} \Longrightarrow x = 400 \textrm{ m}$
  
  Schwimmer(in) C: $\frac{1.75 + 6}{1.75\cdot{}\sqrt3}=\frac{x}{0.2} \Longrightarrow x\approx 511 \textrm{ m}$
  
  }

\newpage
\subsection{Repräsentant}\index{Repräsentant!Vektorgeometrie}

\subsubsection{Notation}\index{Notation!Vektor}\index{Vektor!Notation}

Ein Vektor wird entweder mit einem Pfeil über dem Buchstaben
${\color{blue}\vec{a}}$ oder aber mit einem Pfeil über der zwei
Punkebezeichnungen ${\color{red}\overrightarrow{{PQ}}}$ angegeben:

\bbwGraph{-3}{6}{-1}{3}{
\bbwLetter{2,2}{\vec{a}}{blue}
\draw [->,blue] (1,1) -- (4,2);

\bbwLetter{-1.5,1}{P}{red}
\bbwLetter{-2.5,3}{Q}{red}
\draw [->,red] (-1,1) --(-2,3);
}%% END bbwGraph
\newpage



\subsubsection{freie Vektoren}
Betrachten Sie die beiden folgenden (freien) Vektoren ${\color{blue} \vec{a}}$ und
${\color{red}\vec{b}}$, welche beide durch mehrere Repräsentanten\index{Repräsentant!Vektor}
(Vertreter\index{Vertreter!Vektor}) eingezeichnet sind:

\bbwGraph{-4}{7}{-3}{3}{
\bbwLetter{3.5,3}{\vec{a}}{blue}
\draw [->,blue] (1,1) -- (4,2);
\draw [->,blue] (2,2) -- (5,3);
\draw [->,blue] (-3.5,-1) -- (-0.5,0);
\draw [->,blue] (2,0.5) -- (5,1.5);
\bbwLetter{-1,3}{\vec{b}}{red}
\draw [->,red] (-1,1) --(-2,3);
\draw [->,red] (-1,-2) --(-2,0);
\draw [->,red] (2.5,-0.5) --(1.5,1.5);
\draw [->,red] (5,-3) --(4,-1);
\draw [->,red] (7,0.5) --(6,2.5);
}%% END bbwGraph

Tragen Sie die fehlenden Werte in die Tabelle ein\footnote{Der
  mathematisch positive Winkel wird ab der $x$-Achse im
  Gegenuhrzeigersinn gemessen.}:

\begin{tabular}{|c|c|c|}\hline
                 & ${\color{blue}\vec{a}}$   & ${\color{red}\vec{b}}$   \\\hline
  $x$-Komponente & \LoesungsRaumLang{3}      & \LoesungsRaumLang{-1}    \\\hline
  $y$-Komponente & \TRAINER{1}               & \TRAINER{2}              \\\hline
  Betrag\index{Betrag!eines Vektors} (=Länge) & \TRAINER{$\sqrt{10}$}     & \TRAINER{$\sqrt{5}$}     \\\hline
  math. pos. Winkel  & \TRAINER{$\arctan{}\left(\frac13\right)\approx
    18.43\degre$} & \TRAINER{$90\degre +
    \arctan{}\left(\frac12\right)\approx 116.6\degre$}               \\\hline
\end{tabular}
\platzFuerBerechnungen{3.2}

\subsection*{Aufgaben}\
\TALSGeomAadB{178ff}{4.}

\newpage


\subsection{Betrag}\index{Betrag!eines Vektors}

\begin{definition}{Vektor}{}
  Ein \textbf{Vektor} besteht aus einer Länge und einer Richtung.
\end{definition}





\subsubsection{Betrag (Länge) von Vektoren}
\theorieTALSGeom{176}{4.1}
Die Länge der Vektoren wird mittels «Pythagoras» berechnet

Sei $\vec{a}  = \begin{pmatrix}x_a\\y_a\end{pmatrix}$. Somit ist die Länge von
    $\vec{a}$ wie folgt zu berechnen:

    \begin{gesetz}{Betrag, Länge}{}
      Betrag von $\vec{a}$ := Länge von $\vec{a}$

      $$a = |\vec{a}| = \sqrt{x_a^2 + y_a^2}$$
      \end{gesetz}
    Notationen:

    \begin{beispiel}{}{}
      $$ \vec{a}= \begin{pmatrix} 3\\ 1\end{pmatrix}$$
        $$|\vec{a}| = \sqrt{3^2+1^2} = \sqrt{10}\approx 3.162$$
      \end{beispiel}
    
    \begin{definition}{Länge}{}
      Ein Vektor $\vec{a}$ von Punkt $A$ nach $B$ hat den Betrag (= Länge)

      $$a = |\vec{a}| = \overline{AB} = \left|\overrightarrow{AB}\right|$$
    \end{definition}

    \begin{bemerkung}{}{}
      Im Taschenrechner werden die Vektoren mit eckigen Klammern
      definiert:

      \texttt{a := [3; 1]}

      Die Länge (Betrag) wird mit dem Befehl \texttt{norm} ermittelt:

      \texttt{norm(a)}
    \end{bemerkung}
        
\subsection*{Aufgaben}    
    \TALSGeomAadB{178}{2., 5., 46. a) c) und d)}
\newpage

\subsection{Richtung}\index{Richtung!eines Vektors}
Die \textbf{Richtung} eines Vektors wird nicht durch die Pfeillänge,
sondern durch den Strahl ab dem Anfangspunkt durch den Endpunkt des
Vektors festgelegt.


\subsection{Skalar}\index{Skalar}
Ein \textbf{Skalar} ist eine ungerichtete Größe, wie \zB die 2.5 in
``2.5 kg''.
Ein \textbf{Vektor} hingegen hat (mit Ausnahme des Nullvektors) immer
eine Richtung.

\subsection*{Aufgaben}
\TRAINER{(TODO: Hier noch schauen, welche Aufgaben zu den
  Polarkoordinaten gehören und diese ins entsprechende Kapitel «zügeln».)}
\TALSAadB{178}{1.-10., 12.-17., 47., 48.}
