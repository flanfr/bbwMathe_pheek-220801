%% Vektoren in Koordinatensystemen
%% 2020 - 12 - 25 ph. g. freimann @bbw.ch

\section{Koordinatensysteme}



\subsection*{Lernziele}
\begin{itemize}
  \item Vektoren in kartesischen Koordinaten
\item Vektoren in der Polarform
  zeichnen
\item Umrechnen von Kartesischen Koordinaten zu Polarkoordinaten und
  umgekehrt
\item Vektoren in Kartesischen Koordinaten sowie in Polarkoordinaten
  im Taschenrechner 
\end{itemize}
\newpage

\subsection{Kartesische Koordinaten}


\begin{definition}{Kartesische Koordinaten}{}
  Vektoren im Kartesischen Koordinatensystem werden durch ihre $x$-
  bzw. $y$-Komponente angegeben:

  $$\vec{v} = \begin{pmatrix} x_v  \\ y_v \end{pmatrix}$$
\end{definition}

Betrachten Sie nochmals die beiden folgenden (freien) Vektoren ${\color{blue} \vec{a}}$ und
${\color{red}\vec{b}}$:

\bbwGraph{-4}{7}{-3}{3}{
\bbwLetter{3.5,3}{\vec{a}}{blue}
\draw [->,blue] (1,1) -- (4,2);
\bbwLetter{-1,3}{\vec{b}}{red}
\draw [->,red] (-1,1) --(-2,3);
}%% END bbwGraph

Tragen Sie die fehlenden Werte in die Tabelle ein\footnote{Der
  mathematisch positive Winkel wird ab der $x$-Achse im
  Gegenuhrzeigersinn gemessen.}:

\renewcommand{\arraystretch}{2}
\begin{tabular}{|c|c|c|}\hline
                 & ${\color{blue}\vec{a}}$   & ${\color{red}\vec{b}}$   \\\hline
  $x$-Komponente & \TRAINER{3}\noTRAINER{\hspace{10em}}      & \TRAINER{-1}\noTRAINER{\hspace{10em}}   \\\hline
  $y$-Komponente & \TRAINER{1}      & \TRAINER{2}     \\\hline
  Betrag\index{Betrag!eines Vektors} (=Länge) & \TRAINER{$\sqrt{10}$}     & \TRAINER{$\sqrt{5}$}     \\\hline
  math. pos. Winkel  & \TRAINER{$\arctan{}\left(\frac13\right)\approx
    18.43\degre$} & \TRAINER{$90\degre +
    \arctan{}\left(\frac12\right)\approx 116.6\degre$}               \\\hline
\end{tabular}
\renewcommand{\arraystretch}{1}

Koordinatenschreibweise von $\color{blue}\vec{a}$ und $\color{red}\vec{b}$:\,\,
$\vec{a} = \left( \TRAINER{3}\noTRAINER{\,\,\,\,} \atop \TRAINER{1} \right)$
$\vec{b} = \left( \TRAINER{-1}\noTRAINER{\,\,\,\,\,} \atop \TRAINER{2} \right)$
\newpage
\subsubsection{Addition in kartesischen Koordinaten}
Vektoren im kartesischen Koordinatensystem werden addiert, indem ihre
$x$- bzw $y$-Komponenten separat addiert werden:

\begin{gesetz}{Addition}{}\index{Addition!Vektoren}\index{Vektoraddition}
$$\vec{a} + \vec{b} =   \left(x_a \atop x_b \right)  + \left( x_b \atop y_b \right) =
  \left( x_a + x_b \atop y_a + y_b \right)$$
  \end{gesetz}

\begin{beispiel}{Addition}{}
  $$\vec{a} = \left(3\atop 1\right) \textrm{ und } \vec{b} = \left(-1
  \atop 2\right)$$
  $$\Longrightarrow \vec{a} + \vec{b} = \left(3 + (-1) \atop 1 +
  2\right) = \left(2 \atop 3\right)$$
  \end{beispiel}

\platzFuerBerechnungen{6}
\TRAINER{\vspace{6cm}}

\subsection*{Aufgaben}    
    \TALSGeomAadB{188}{46. a) c) und d), 47. und 48.}

\newpage



\subsection{Vektoren im Taschenrechner}\index{Taschenrechner!Vektorgeometrie}

\subsubsection{Eingabe}
Vektoren im Taschenrechner werden entweder mit eckigen Klammern und
einem Strichpunkt (Semikolon) wie folgt eingegeben:
$$a := [3; 1]$$
oder direkt mit dem Vektor-Befehl bei den mathematischen Symbolen:

\bbwCenterGraphic{10cm}{tals/vecg1/img/TR_eingabe.png}

\subsubsection{Vektoraddition, Vektorsubtraktion}
Vektoren können mit $+$ bzw. $-$ addiert bzw. subtrahiert
werden:
\bbwCenterGraphic{3cm}{tals/vecg1/img/TR_plusminus.png}

\subsubsection{Gegenvektor}
Der Gegenvektor wird mit einem Minuszeichen erzeugt:
$$a:=\left(3\atop 1\right) \Longrightarrow -a = \left( -3 \atop
-1\right)$$
\newpage

\subsubsection{Skalarmultiplikation}\index{Skalarmultiplikation!Taschenrechner}
Wird ein Vektor um ein Vielfaches gestreckt, so sprechen wir von einer
Skalarmultiplikation. Diese wird wie eine normale Multiplikation
eingegeben:
\bbwCenterGraphic{3cm}{tals/vecg1/img/TR_skalarmultiplikation.png}


\subsubsection{Linearkombination finden}\index{Linearkombination!Taschenrechner}
Will ich einen Vektor in eine Linearkombination
\totalref{linearkombination} zerlegen, so geschieht das einfch mit dem
«solver»:
\bbwCenterGraphic{10cm}{tals/vecg1/img/TR_linearkombination.png}
\newpage


Algebraisch kann man auch zeigen, dass eine Zerlegung in zwei Vektoren
nicht immer möglich ist:
\TNT{10}{
Begründung Gleichungssystem:
  
  $$\vec{c} = t\cdot{}\vec{a\vphantom{b}} + s\cdot{}\vec{b}$$ heißt:
  \gleichungZZ{c_x}{t\cdot{}a_x + s\cdot{}b_x}{c_y}{t\cdot{}a_y +
    s\cdot{}b_y}

  Die Lösung (\zB mittels Einsetz-Verfahren) ist

%%  $$\det{} = a_xb_y - a_yb_x$$

  $$s = \frac{a_xc_y - a_yc_x}{a_xb_Y-a_yb_x}$$
  und
  $$t = \frac{b_yc_x - b_xc_y}{a_xb_y-a_yb_x}$$

  Ist der Nenner Null, so gibt es keine (oder keine
  eindeutige) Zerlegung.
}%% END TNT

\newpage

\newpage

\subsection{Länge im kartesischen Koordinatensystem}

Die Länge der Vektoren wird mittels «Pythagoras» berechnet.

Sei $\vec{a}  = \begin{pmatrix}x_a\\y_a\end{pmatrix}$. Somit ist die Länge von
    $\vec{a}$ wie folgt zu berechnen:

    \begin{gesetz}{Betrag, Länge}{}
      Betrag von $\vec{a}$ := Länge von $\vec{a}$

      $$a = |\vec{a}| = \sqrt{x_a^2 + y_a^2}$$
      \end{gesetz}
    Notationen:

    \begin{beispiel}{}{}
      $$ \vec{a}= \begin{pmatrix} 3\\ 1\end{pmatrix}$$
        $$|\vec{a}| = \sqrt{3^2+1^2} = \sqrt{10}\approx 3.162$$
      \end{beispiel}
    

    \begin{bemerkung}{}{}
      Im Taschenrechner werden die Vektoren mit eckigen Klammern
      definiert:

      \texttt{a := [3; 1]}

      Die Länge (Betrag) wird mit dem Befehl \texttt{norm} ermittelt:

      \texttt{norm(a)}
    \end{bemerkung}

\newpage

\subsection{Polarkoordinaten}\index{Polarkoordinaten}
\sectuntertitel{Opposite of a polarbear? A cartesian bear!}

\bbwCenterGraphic{16cm}{tals/vecg1/img/polen.jpg}

\newpage

Anstelle der $x$- und der $y$-Komponente können wir mit dem selben
Informationsgehalt auch den Winkel ($\varphi$)
(in mathematisch positiver Richtung) und die Länge ($r$) eines Vektors
angeben.

\bbwCenterGraphic{10cm}{tals/vecg1/img/polar.png}

Dieser Winkel wird im mathematisch positiven Sinne ab der $x$-Achse
angegeben. Der Vektor
$\begin{pmatrix}0\\1\end{pmatrix}$ hat somit den Winkel $90\degre$.
  \begin{definition}{Polarkoordinaten}{}
    Vektoren in Polarkoordinaten werden in der Reihenfolge (Länge |
    Winkel) angegeben:
    $$\vec{a} = (2 | 60\degre) = \left(1\atop \sqrt{3}\right)$$
    \end{definition}
  \begin{bemerkung}{Nullvektor}{}\index{Nullvektor}
    Der Nullvektor hat keine Richtung.
    \end{bemerkung}
  \newpage
\subsection{Transformation}\index{Transformation!Polar-
    vs. Kartesische Koordinaten}
  Die Umrechnung von Polar- zu kartesischen
  Koordinaten und umgekehrt wird Transformation genannt.
  
  Dank unseren Freunden \textit{Sinus} und \textit{Cosinus} ist die
  Transformation aus Polarkoordinaten relativ einfach:
  \begin{rezept}{Transformation «polar» nach «kartesisch»}{}

    Gegeben:  Winkel $\varphi$ und Länge $r$

    Gesucht: $x$ und $y$ in kartesischen Koordinaten

    $$x = r\cdot{}\cos(\varphi)$$
    $$y = r\cdot{}\sin(\varphi)$$
  \end{rezept}

  Die Umkehrung ist etwas komplizierter, denn dabei müssen wir
  beachten, in welchen Quadranten die Pfeilspitze des Ortsvektors
  zeigt.
  Zum Glück nimmt uns das der Taschenrechner ab.

  \subsubsection{Kartesische Koordinaten aus Polarkoordinaten}
  Der Taschenrechner verwandelt Polarkoordinaten automatisch immer in
  kartesische Koordinaten um:

    \bbwCenterGraphic{5cm}{tals/vecg1/img/Polar2Kartesisch.png}
    \newpage

    
  \subsubsection{Polarkoordinaten aus kartesischen}
  \begin{beispiel}{kartesisch zu polar}{}
    Gegeben der Vektor $\vec{a} = \left(-\sqrt{3} \atop 1\right)$:

    Eine Skizze zeigt uns rasch, dass es sich um $150\degre$ und eine
    Länge von $2$ handeln muss.

    \bbwCenterGraphic{5cm}{tals/vecg1/img/Kartesisch2Polar.png}
    \end{beispiel}

Rechnen Sie in kartesische Koordinaten um:

\begin{tabular}{rcl}
  $(2   | 30\degre)$ & $=$ & \LoesungsRaum{$\left(\sqrt3\atop 1\right)$}\\
  $(321 | 270\degre)$ & $=$ & \LoesungsRaum{$\left(0\atop -321\right)$}\\
  $(0   | 103.83346\degre)$ & $=$ & \LoesungsRaum{$\left(0\atop 0\right)$}
\end{tabular}

Rechnen Sie in Polarkoordinaten um:

\begin{tabular}{rcl}\vspace{2mm}
  $\left(2\atop 2 \right)$ & $=$ & \LoesungsRaum{$(2\cdot{}\sqrt2|45\degre)$}\\\vspace{2mm}
  $\left(0\atop -13 \right)$ & $=$ & \LoesungsRaum{$(13|270\degre)$}\\\vspace{2mm}
  $\left(-5\atop 1 \right)$ & $=$ & \LoesungsRaum{$(\sqrt{26}\cdot{} | {180\degre - \arctan(\frac15)}) \approx (5.099 | 168.69\degre)$}\\\vspace{2mm}
  $\left(0\atop 0 \right)$ & $=$ & \LoesungsRaum{$(0|18.35\degre) = (0|299.68\degre)= ...$} \\\vspace{2mm}
  
\end{tabular}

\TNT{5.2}{\vspace{5.2cm}}

\newpage
\subsubsection{Drei Schwimmer im Taschenrechner}
Mit dem Taschenrechner wird unsere Einstiegs-Aufgabe mit den drei
Schwimmern stark vereinfacht. Wir geben die Geschwindigkeit und
Richtung des Schwimmers «A» in Polarkoordinaten an
(\textbf{\texttt{schwimmera}}).

Die Geschwindigkeit und Richtung des Flusses können wir auch in kartesischen
Koordinaten angeben (\texttt{\textbf{fluss}}).

Der resultierende Vektor aus Schwimmer und Fluss ist einfach
die Summe der beiden Vektoren.

Wir suchen nun also ein Vielfaches ($t$) der resultierenden
Geschwindigkeit, sodass in $y$-Richtung gerade die 0.2 km erreicht
wird. Die «Verschiebung» am anderen Ufer in $x$-Richtung bezeichnen
wir einfach mit $x$.

\bbwCenterGraphic{12cm}{tals/vecg1/img/TR_DreiSchwimmer.png}

\begin{bemerkung}{}{}
  Auch wenn es nur eine Gleichung mit zwei Unbekannten ist, dürfen wir
  nicht vergessen, das Vektoren in der Ebene aus zwei Komponenten
  bestehen, somit haben wir genau genommen auch ein lineares
  Gleichungssystem mit zwei Gleichungen und zwei Unbekannten zu lösen.
  \end{bemerkung}
\newpage
  
\subsection*{Aufgaben}
\TALSAadB{181ff}{18., 21. und 24.}

Mit Taschenrechner:

\TALSAadB{181ff}{23.}
\newpage
