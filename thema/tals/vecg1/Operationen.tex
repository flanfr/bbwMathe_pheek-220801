\section{Vektor-Operationen}

\begin{itemize}
\item Addition von Vektoren
\item Gegenvektoer
\item Subtraktion von Vektoren
\item Multiplikation mit Skalar
\item Nullvektor
\item Zerlegung eines Vektors entlang vorgegebener Richtungen
\end{itemize}

\theorieTALSGeom{178}{4.2}


\subsection{Addition}\index{Addition!von Vektoren}

Zwei Vektoren werden addiert, indem sie hintereinander gelegt
werden. Im Bild: $\vec{c} = \vec{a} + \vec{b}$

\bbwGraph{-1}{7}{-2}{3.5}{
\bbwLetter{2,2}{\vec{a}}{blue}
\draw [->,blue] (1,1) -- (4,2);
\bbwLetter{5,1}{\vec{b}}{red}
\draw [->,red] (4,2) --(5,-1);
\bbwLetter{3,0.5}{\vec{c}}{ForestGreen}
\draw [->,green] (1,1) --(5,-1);
}%% END bbwGraph
\newpage


\subsection{Gegenvektor}\index{Gegenvektor}


\subsubsection{Gegenvektor}\index{Gegenvektor}
\begin{definition}{Gegenvektor}{}
  Ein Vektor mit der selben Länge, aber in entgegengesetzter Richtung
  heißt \textbf{Gegenvektor}.

  Der Gegenvektor zu $\vec{v}$ wird mit $-\vec{v}$ bezeichnet.
\end{definition}


\subsection{Subtraktion}\index{Subtraktion!von Vektoren}


\subsubsection{Subtraktion}\index{Subtraktion!von Vektoren}
Die Subtraktion ist die Addition des Gegenvektors:

\begin{definition}{Subtraktion von Vektoren}{}
  $$\vec{a} - \vec{b} := \vec{a} + (-\vec{b})$$
\end{definition}

\begin{bemerkung}{}{}
  $$\vec{a} + \vec{b} = \vec{c} \Longleftrightarrow \vec{a} = \vec{c} - \vec{b}$$
  \end{bemerkung}

\subsection*{Aufgaben}
\TALSGeomAadB{179ff}{6., 9., 10. a) c) i)., 12., 47., 48.}
\newpage


\subsubsection{Multiplikation mit Skalar}
\theorieTALSGeom{178}{4.2}
\begin{definition}{Skalar}{}
  Vektoren haben eine Länge und eine Richtung. Eine Größe ohne
  Richtung aber mit einem Betrag nennen wir \textbf{Skalar}. 
\end{definition}
\newpage
