\section{Vektor-Operationen}

\begin{itemize}
\item Addition von Vektoren
\item Gegenvektoer
\item Subtraktion von Vektoren
\item Multiplikation mit Skalar
\item Nullvektor
\item Zerlegung eines Vektors entlang vorgegebener Richtungen
\end{itemize}

\theorieTALSGeom{178}{4.2}
\newpage

\subsection{Addition}\index{Addition!von Vektoren}

Zwei Vektoren werden addiert, indem sie hintereinander gelegt
werden. Im Bild: $\vec{c} = \vec{a} + \vec{b}$

\bbwGraph{-1}{7}{-2}{3.5}{
\bbwLetter{2,2}{\vec{a}}{blue}
\draw [->,blue] (1,1) -- (4,2);
\bbwLetter{5,1}{\vec{b}}{red}
\draw [->,red] (4,2) --(5,-1);
\bbwLetter{3,0.5}{\vec{c}}{ForestGreen}
\draw [->,ForestGreen] (1,1) --(5,-1);
}%% END bbwGraph

\subsection*{Aufgaben}
\TALSGeomAadB{179}{7.}

\newpage


\subsection{Gegenvektor}\index{Gegenvektor}\index{inverser Vektor}

\begin{definition}{Gegenvektor}{}
  Ein Vektor mit der selben Länge, aber in entgegengesetzter Richtung
  heißt \textbf{Gegenvektor} oder \textbf{inverser Vektor}.

  Der Gegenvektor zu $\vec{v}$ wird mit $-\vec{v}$ bezeichnet.
\end{definition}

\begin{bemerkung}{Gegenvektor}{}
  Für den Gegenvektor $(-\vec{a})$ zum Vektor $\vec{a}$ gilt:
  $$\vec{a} + (-\vec{a}) = \vec{0}$$
Wobei $\vec{0}$ den Nullvektor bezeichnet, der keine Länge aufweist.
  
\end{bemerkung}

\bbwGraph{-1}{7}{-2}{3.5}{
\bbwLetter{2,2}{\vec{a}}{blue}
\draw [->,blue] (1,1) -- (4,2);
\bbwLetter{3,-1}{(\vec{-a})}{ForestGreen}
\draw [->,ForestGreen] (5,-1) --(2,-2);
}%% END bbwGraph


\newpage


%%%%%%%%%%%%%%%%%%%%%%%%%%%%%%%%
\subsection{Subtraktion}\index{Subtraktion!von Vektoren}

Die Subtraktion ist die Addition des Gegenvektors:

\begin{definition}{Subtraktion von Vektoren}{}
  $$\vec{a\vphantom{b}} - \vec{b} := \vec{a\vphantom{b}} + (-\vec{b})$$
\end{definition}

\begin{bemerkung}{}{}
  $$\vec{a\vphantom{b}} + \vec{b} = \vec{c\vphantom{b}} \Longleftrightarrow \vec{a\vphantom{b}} = \vec{c\vphantom{b}} - \vec{b}$$
\end{bemerkung}

Zeigen Sie geometrisch, dass gilt $\vec{c\vphantom{b}} - \vec{b} = \vec{c\vphantom{b}} +
(-\vec{b})$.

\TNT{7.2}{
\bbwGraph{-1}{7}{-2}{3.5}{
\bbwLetter{2,2}{\vec{a}}{blue}
\draw [->,blue] (1,1) -- (4,2);
\bbwLetter{5,1}{\vec{b}}{red}
\draw [->,red] (4,2) --(5,-1);
\bbwLetter{3,0.5}{\vec{c}}{ForestGreen}
\draw [->,green] (1,1) --(5,-1);
}%% END bbwGraph

Grafik: Es gilt:e $\vec{a} = \vec{c} - \vec{b} = \vec{c} + (-\vec{b})$.
}%% END TNT

\subsection*{Aufgaben}
\TALSGeomAadB{179ff}{9., 10. a) c) i)., 47. und 48.}
\newpage


\subsubsection{Multiplikation mit einem Skalar}\index{Skalarmultiplikation}
\theorieTALSGeom{178}{4.2}
\begin{definition}{Skalar}{}
  Vektoren haben eine Länge und eine Richtung. Eine Größe ohne
  Richtung aber mit einem Betrag nennen wir \textbf{Skalar}. 
\end{definition}

\begin{beispiel}{Skalarmultiplikation}{}
  Anstelle von
  $$\vec{a}+\vec{a}+\vec{a}+\vec{a}+\vec{a}$$
  schreiben wir
  $$5\cdot{}\vec{a}$$
\end{beispiel}

\begin{definition}{Skalarmultiplikation}{}
  $$n\cdot{}\vec{a} = \underbrace{\vec{a} + \vec{a} + ... + \vec{a}}_{n \textrm{ Summanden}}$$
\end{definition}

\begin{definition}{Skalarmultiplikation}{}
  Für $k$ in $\mathbb{R}$ ist $k\cdot{}\vec{a}$ ein Vektor mit
  folgenden Eigenschaften.

  \begin{itemize}
  \item $k>0$, so zeigen $k\cdot{}\vec{a}$ und $\vec{a}$ in die selbe Richtung.
  \item $k<0$, so zeigen $k\cdot{}\vec{a}$ und $\vec{a}$ in die entgegengesetzte Richtung.
  \item Die Länge von $k\cdot{}\vec{a}$ ist das $k$-Fache der Länge
    von $\vec{a}$: $$|k\cdot{}\vec{a}| = |k|\cdot{}|\vec{a}|$$
   \end{itemize}
\end{definition}

\begin{definition}{kollinear}{}\index{kollinear}

  Zwei Vektoren $\vec{a\vphantom{b}}$ und $\vec{b}$ heißen \textbf{kollinear}, wenn es ein
  $k\in\mathbb{R}$ gibt, sodass
  $$\vec{a\vphantom{b}} = k\cdot{}\vec{b}$$
  \end{definition}

\newpage


\subsection{Nullvektor}\index{Nullvektor}
\sectuntertitel{Kommt ein Nullvektor zum Psychiater: ``... ich bin so orientierungslos!''}

Der \textbf{Nullvektor} hat keine Länge und somit auch keine
Richtung. Der Nullvektor beschreibt \zB die Bewegungs (inklusive
Richtung) eines stehenden Fahrzeuges.
Das stehende Fahrzeug kann theoretsich eine Richtung (Ausrichtung des
Fahrzeugs) haben, da es jedoch keine ``Fahrtrichtung'' hat, spielt die
Ausrichtung keine Rolle.

\newpage

\subsection{Zerlegung von Vektoren}\index{Vektoren!zerlegen}\index{Zerlegung von Vektoren}

Sind zwei Vektoren $\vec{a}$ und $\vec{b}$ gegeben, so kann im
allgemeinen Fall ein
dritter Vektor $\vec{c}$ als sog. Linearkombination der beiden
gegebenen Vektoren angegeben werden:

\begin{gesetz}{Linearkombination}{}

  Gegeben $\vec{a}$ und $\vec{b}$.

  Gegeben $\vec{c}$.

  Gesucht $t$ und $s$ mit

  $$\vec{c} = t\cdot{}\vec{a} + s\cdot{}\vec{b}$$
\end{gesetz}

\textbf{Beispiel}

\bbwGraph{-1}{8}{-1}{5}{
\bbwLetter{3,1.5}{\vec{a}}{blue}
\draw [->,blue] (1,2) -- (5,2);
\bbwLetter{1.5,3.5}{\vec{b}}{blue}
\draw [->,blue] (1,2) -> (3,3.999);
\bbwLetter{6.5,3}{\vec{c}}{ForestGreen}
\draw [->,ForestGreen] (6,2) -- (6,4);
}%% END bbwGraph

In obigem Beispiel ist $\vec{c} = -\frac12 \vec{a} + \vec{b}$.

In welchen Fällen ist eine solche Zerlegung nicht möglich nicht möglich?

\subsection*{Aufgaben}


\TALSGeomAadB{180ff}{12., 13., 14., 16. und 17.}
