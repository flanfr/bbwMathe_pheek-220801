%% TALS Polarkoordinaten
%% 2020 - 12 - 25 ph. g. freimann @bbw.ch

\section{Polarkoordinaten}\index{Polarkoordinaten}
\sectuntertitel{Opposite of a polarbear? A rectangular bear!}

\subsection*{Lernziele}
\begin{itemize}
\item Vektoren in der Polarform zu schreiben bzw. aus der Polarform zu
  zeichnen
\item Umrechnen von Kartesischen Koordinaten zu Polarkoordinaten und
  umgekehrt
\item Polarkoordinaten im Taschenrechner
\end{itemize}

Anstelle der $x$- und der $y$-Komponente können wir mit dem selben
Informtaionsgehalt auch den Winkel ($\varphi$)
(in mathematisch positiver Richtung) und die Länge ($r$) eines Vektors
angeben.

\bbwCenterGraphic{10cm}{tals/vecg1/img/polar.png}

Dieser Winkel wird im mathematisch positiven Sinne ab der $x$-Achse
angegeben. Der Vector
$\begin{pmatrix}0\\1\end{pmatrix}$ hat somit den Winkel $90\degre$.
  \begin{definition}{Polarkoordinaten}{}
    Vektoren in Polarkoordinaten werden in der Reihenfolge (Länge |
    Winkel) angegeben:
    $$\vec{a} = (2 | 60\degre) = \left(1\atop \sqrt{3}\right)$$
    \end{definition}
  \begin{bemerkung}{Nullvektor}{}\index{Nullvekttor}
    Der Nullvektor hat keine Richtung.
    \end{bemerkung}
  \newpage
\subsection{Transformation}\index{Transformation!Polar-
    vs. Kartesische Koordinaten}
  Die Umrechnung von Polar- zu kartesischen
  Koordinaten und umgekehrt wird Transformation genannt.
  
  Dank unseren Freunden \textit{Sinus} und \textit{Cosinus} ist die
  Transformation aus Polarkoordinaten relativ einfach:
  \begin{rezept}{Transformation «polar» nach «kartesisch»}{}

    Gegegben:  Winkel $\varphi$ und Länge $r$

    Gesucht: $x$ und $y$ in kartesischen Koordinaten

    $$x = r\cdot{}\cos(\varphi)$$
    $$y = r\cdot{}\sin(\varphi)$$
  \end{rezept}

  Die Umkehrung ist etwas komplizierter, denn dabei müssen wir
  beachten, in welchen Quadranten die Pfeilspitze des Ortsvektors
  zeigt.
  Zum Glück nimmt uns das der Taschenrechner ab.

  \subsubsection{Polarkoordinaten im Taschenrechner}
  \begin{beispiel}{kartesisch zu polar}{}
    Gegeben der Vektor $\vec{a} = \left(-\sqrt{3} \atop 1\right)$:

    Eine Skizze zeigt uns rasch, dass es sich um $150\degre$ und eine
    Länge von $2$ handeln muss.

    \bbwCenterGraphic{5cm}{tals/vecg1/img/Kartesisch2Polar.png}
    \end{beispiel}
  \newpage

Rechnen Sie in kartesische Koordinaten um:

\begin{tabular}{rcl}
  $(2   | 30\degre)$ & $=$ & \LoesungsRaum{$\left(\sqrt3\atop 1\right)$}\\
  $(321 | 270\degre)$ & $=$ & \LoesungsRaum{$\left(0\atop -321\right)$}\\
  $(0   | 103.83346\degre)$ & $=$ & \LoesungsRaum{$\left(0\atop 0\right)$}
\end{tabular}

Rechnen Sie in Polarkoordinaten um:

\begin{tabular}{rcl}\vspace{2mm}
  $\left(2\atop 2 \right)$ & $=$ & \LoesungsRaum{$(2\cdot{}\sqrt2|45\degre)$}\\\vspace{2mm}
  $\left(0\atop -13 \right)$ & $=$ & \LoesungsRaum{$(13|270\degre)$}\\\vspace{2mm}
  $\left(-5\atop 1 \right)$ & $=$ & \LoesungsRaum{$(\sqrt{26}\cdot{} | {180\degre - \arctan(\frac15)}) \approx (5.099 | 167.43\degre)$}\\\vspace{2mm}
  $\left(0\atop 0 \right)$ & $=$ & \LoesungsRaum{$(0|18.35\degre) = (0|299.68\degre)= ...$} \\\vspace{2mm}
  
\end{tabular}

\TNT{5.2}{\vspace{5.2cm}}

\subsection*{Aufgaben}
\TALSAadB{181ff}{18., 24.}

Mit Taschenrechner:

\TALSAadB{181ff}{21., 23.}
\newpage
