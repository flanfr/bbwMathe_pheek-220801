%% TALS Polarkoordinaten
%% 2020 - 12 - 25 ph. g. freimann @bbw.ch

\section{Polarkoordinaten}\index{Polarkoordinaten}
\sectuntertitel{Opposite of a polarbear? A rectangular bear!}

\subsection*{Lernziele}
\begin{itemize}
\item Vektoren in der Polarform zu schreiben bzw. aus der Polarform zu
  zeichnen
\item Umrechnen von Kartesischen Koordinaten zu Polarkoordinaten und
  umgekehrt
\item Polarkoordinaten im Taschenrechner
\end{itemize}

Anstelle der $x$- und der $y$-Komponente können wir mit dem selben
Informtaionsgehalt auch den Winkel ($\varphi$)
(in mathematisch positiver Richtung) und die Länge ($l$) eines Vektors
angeben.

\bbwCenterGraphic{10cm}{tals/vecg1/img/polar.png}

Dieser Winkel wird im mathematisch positiven Sinne ab der $x$-Achse
angegeben. Der Vector
$\begin{pmatrix}0\\1\end{pmatrix}$ hat somit den Winkel $90\degre$.

Rechnen Sie um:

... coming soon ...

\subsection*{Aufgaben}
\TALSAadB{181ff}{18., 24.}

Mit Taschenrechner:

\TALSAadB{181ff}{21., 23.}


\subsection{Polarkoordinaten im Taschenrechner}

... coming soon ...
