%%
%% 2019 07 04 Ph. G. Freimann
%%

\subsection{Grad- und Bogenmaß}\index{Gradmaß}\index{Bogenmaß}

%%\TALSTadBFWG{46}{1.3.2}
\TadBMTG{87}{6.1}

Das Bogenmaß ist eine alternative Einteilung des Kreises zum
klassischen $360\degre$-Gradmaß. Dabei wird der volle Kreis nicht in
$360\degre$ Grad sondern im Verhältnis zum Radius dargestellt. Mit
anderen Worten: Bei gegebenem Winkel $\alpha$ ist das Bogenmaß nichts
anderes als die Länge des Bogens zum Winkel $\alpha$ gemessen im Einheitskreis\index{Einheitskreis}.

\bbwGraphic{5cm}{tals/trig2/img/bogenmass.png}

\begin{definition}{Bogenmaß}{}
$$360\degre \entspricht 2\cdot\pi \textrm{\,\,rad}$$
\end{definition}

\begin{bemerkung}{rad}{}
Der Winkel im Bogenmaß wird nicht in der Maßeinheit Grad (${}\degre$)
sondern in $\textrm{rad}$ angegeben.
\end{bemerkung}

\begin{gesetz}{Bogenmaß}{}
$$1 \textrm{ rad } = \frac{180}{\pi}\degre \approx 57.30\degre$$
$$1\degre = \frac{\pi}{180} \textrm{ rad } \approx 0.01745 \textrm{ rad }$$
$$\stackrel{\frown}{\alpha} \textrm{rad} = \left(\frac{\alpha\cdot{}180}{\pi}\right)^\circ$$
$$\alpha\degre = \left(\frac{\alpha\pi}{180}\right) \textrm{ rad }$$  
\end{gesetz}

\begin{beispiel}{Bogenmaß}{}
 Zum Beispiel entspricht
$\frac{3}{4}\pi\,\textrm{rad}$ unseren bekannten $135\degre$ in der
(bereits sumerischen/babylonischen) $360\degre$-Einteilung.

 $$\frac34\pi \textrm{ rad } = \left(\frac{\frac34\pi\cdot{}180}{\pi}\right)^\circ = 135\degre$$
\end{beispiel}
\newpage



\subsection*{Aufgaben}

\olatLinkArbeitsblatt{Bogenmaß}{https://olat.bbw.ch/auth/RepositoryEntry/572162090/CourseNode/103430829669712}{1.1 bis 1.4}

%%\TALSAadBFWG{46}{175. a) b) d) 176. a) f) i) und das Aufgabenblatt im OLAT}
\AadBMTG{97}{1., 2.}
\newpage
