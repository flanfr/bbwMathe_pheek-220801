%%
%% 2020 12 25 Ph. G. Freimann
%% Polynomgleichungen höheren Grades (TALS SPF)
%%

\section{Polynom(un)gleichungen}\index{Polynomgleichungen}


%%%%%%%%%%%%%%%%%%%%%%%%%%%%%%%%%%%%%%%%%%%%%%%%%%%%%%%%%%%%%%%%%%%%%%%%%%%%%%%%%
\subsection*{Lernziele}

\begin{itemize}
\item Produktform (Typ: Produkt gleich «Null»)
\item Taschenrechner
\end{itemize}

\newpage

%% 2020 12 25 Ph. G. Freimann
%% Tals SPF Gleichungen III 1/2: Wurzelgleichungen durch Quadrieren
%%

\subsubsection{Typ: Produkt = Null}

%%%%%%%%%%%%%%%%%%%%%%%%%%%%%%%%%%%%%%%%%%%%%%%%%%%%%%%%%%%%%%%%%%%%%%%%%%%%%%%%%

\TALS{(\cite{frommenwiler17alg} S.115 (Kap. 2.4.3))}
%%\GESO{(\cite{marthaler21}       S.??? (Kap. ???))}

\subsection{Einstiegsbeispiel}

Finden Sie die Lösungsmenge folgender Gleichnug:

$$x^4 + 7x^2 - 44 = 0$$

\TNT{6}{
  1. Faktorzerlegung
  $$(x^2 + 11)\cdot{}(x^2-4) = 0$$
  ... weiter ...
  $$(x^2 + 11)\cdot{}(x+2)\cdot{}(x-2) = 0$$

    Nun reicht es, wenn einer der Faktoren = 0 ist:

    $$\lx=\{-2; 2\}$$
  }

\subsection*{Aufgaben}
\aufgabenFarbe{Finden Sie die Lösungsmenge $\lx$ für die folgende
  Gleichung: $$(x^2+x-2)(x+3)=0$$}%% END Aufgabenfrabe
\TNT{2.4}{Faktorisiert: $$(x+2)(x-1)(x+3)=0$$
  $$\lx=\{-3, -2, 1\}$$}%% END TNT

\TALSAadBMTA{115}{351. d), 352. f) [Tipp: Substitution] g), 353. a) b) c) h)}
\GESOAadBMTA{???}{???}

\newpage

\subsection{Taschenrechner}
Beispiel

$$x^3 + 70 \ge{} 4x^2 + 31$$

Verschiedene Lösungsansätze



a) Taschenrechner: \LoesungsRaumLang{\texttt{solve($x^3+70 \ge 4x^2 +
    31$, x)}}

\leserluft{}

b) Graphisch auch mit TR (ablesen aus Graph):

\TNT{2}{\texttt{g(x) := $x^3 + 70$} und \texttt{h(x) := $4x^2 + 31$}
  
Lies ab, wo nun $g(x) \ge h(x)$\vspace{6mm}
}%% ENd TNT

  
c) Die harte Tour\footnote{Solche und ähnliche Verfahren wendet auch der
  Taschenrechner an.}:

\TNT{14}{
  \begin{enumerate}
  \item Wie bei einer quadratischen Gleichung alles auf eine Seite
    nehmen: $$x^3 - 4x^2 -31x + 70 \ge{} 0$$

  \item Annähern, pröbeln  ($x=0, x=1, x=2$)
    Hier finden wir, dass $x=2$ die Ungleichung (bzw. die Gleichung) löst.

  \item Ziel: Gleichung vom Typ «Produkt gleich Null»:
    $$(x-2)\cdot(x + ???)\cdot(x + ???) = 0$$

  \item (optional Vorzeigen) Polynomdivison:
    $$(x^3 - 4x^2 - 31x + 70) = (x-2) \cdot (???)$$
    $$(x^3 - 4x^2 - 31x + 70) : (x-2) = (x^2 + 2x - 35)$$

  \item Lösung: $$x^3 - 4x^2 -31x + 70 = (x-2)\cdot(x-5)\cdot(x+7)$$
    Somit gilt $$-7, 2 \textrm{ und } 5 \in \lx $$
\end{enumerate}
}%% END TNT
\newpage

\subsubsection{halbgraphische Methode}
... weiter im Beispiel ...
\TNT{14}{
Halbgraphische Methode mit $(x-2)$, $(x-5)$ und $(x+7)$ zeichnen und
interpretieren:
\vspace{13cm}
}%% END TNT



%%Erstellen Sie zu einer der folgenden Aufgaben eine Musterlösung für OLAT!

\subsection*{Aufgaben}
\TALSAadB{157ff}{555. (TR), 559., 561., 562., 564., 568., 572., 574.,
  575. und Seite 203: 761., 762 und 763. \TRAINER{764. gehört zu den
    Ungleichungen (anderes Skript).}}
\TALSAadB{203}{761-764}
\GESOAadB{???}{???}

