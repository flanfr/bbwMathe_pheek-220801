
%% 2019 07 04 Ph. G. Freimann
%%

\section{Logarithmische Gleichungen}\index{Gleichungen!logarithmische}\index{Logarithmische Gleichungen}

%%%%%%%%%%%%%%%%%%%%%%%%%%%%%%%%%%%%%%%%%%%%%%%%%%%%%%%%%%%%%%%%%%%%%%%%%%%%%%%%%
\subsection*{Lernziele}

\begin{itemize}
\item Lösen logarithmischer Gleichungen
\end{itemize}

\TALS{Theorie: (\cite{frommenwiler17alg} S.121 (Kap. 2.4.5))}

\begin{definition}{logarithmische Gleichung}{}
  Bei einer \textbf{logarithmischen} Gleichung kommt die Unbekannte im Argument eines Logarithmus vor.

\end{definition}

\subsection{Einstiegsbeispiel}

$$3+\log_2(x) = \log_2(40-2x)$$

\TNT{10}{
  \begin{enumerate}
  \item
    Definitionsbereich? a) $x>0$ und b) $40-2x>$, d.\,h.\\ $\mathbb{D} = \{x\in\mathbb{R}| x>0 \textrm{ und } x<20\}$

  \item Ordnen:
    $$3 = \log_2(40-2x) - \log_2(x)$$

  \item Logarithmengesetz:
    $$3 = \log_2\left(\frac{40-2x}{x}\right)$$


  \item Definition Logarithmus ($\log_a(x) = n \Longleftrightarrow a^n=x$):
  $$2^3 = \frac{40-2x}{x}$$

\item Gleichung lösen:
  $$8 = \frac{40-2x}{x}$$
  $$8x = 40-2x$$
  $$10x = 40$$
  $$x = 4$$
\item Probe mit Definitionsbereich
  $$4 \in \mathbb{D} \Longrightarrow \lx=\{4\}$$
    \end{enumerate}
}%% END TNT

\newpage
\begin{rezept}{logarithmische Gleichung}{}
  \begin{itemize}
  \item Definitionsmenge festlegen (Logarithmus-Argument > 0)
  \item Alle Logarithmen separieren (auf eine Seite bringen)
  \item Zusammenfassen (Logarithmen-Gesetze)
  \item beide Seiten potenzieren (Logarithmus wegbringen)
  \item Gleichung auflösen
  \item Definitionsbereich prüfen
 \end{itemize}
\end{rezept}

\subsection*{Aufgaben}
\TALSAadBMTA{122}{371. a) d), 372. a), 374. a) e), 376. a) [Substitution]}
