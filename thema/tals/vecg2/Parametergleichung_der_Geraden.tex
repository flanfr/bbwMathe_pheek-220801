%% 2020 12 25 ph. g. Freimann
%%
\section{Parametergleichung der Geraden}\index{Gerade!Parametergleichung}\index{Parametergleichung!der Geraden}

\subsection*{Lernziele}
\begin{itemize}
\item Parametergleichung
\end{itemize}



\TALSTadBFWG{}{}
\newpage

\subsection{Parametergleichnug}
Jede Gerade $\vec{r}$ in der Ebene oder im Raum kann durch die
folgende Gleichung dargestellt werden:

\begin{definition}{Parametergleichung}{}
  Bei gegebenem Referenzpunkt $A$ auf der Geraden und einer gegebenen
  Richtung $\vec{u}$ kann die Gerade $\vec{r}$ geschrieben werden als.
  
  $$\vec{r} = \vec{r}(t) = \vec{r}_A + t\cdot{} \vec{u}$$

  Dabei ist $t\in\mathbb{R}$ beliebig und $\vec{r}_A$ ist der
  Ortsvektor zum Referenzpunkt $A$.
\end{definition}
