\section{Umkehrfunktion}\index{Umkehrfunktion}\index{Funktion!Umkehrung}

\subsection{Lernziele}

\begin{itemize}
\item Funktionen umkehren
\item Symmetrieeigenschaften
\item Definitions- und Wertebereiche
\end{itemize} 
\newpage

\subsection{Einstigsbeispiel}
\begin{beispiel}{Umkehrfunktion}{}
  Eine Schulnote wird aus einer Prüfung mit max. 17 Punkten errechnet.
  Die Skala sei linear. 0 Punkte ergeben die Note 1, 17 Punkte die
  Note 6. Wie sieht die Zuordnung von «Punkte» (unaabhängige Variable)
  zur «Note» (abhängige Variable) aus?

  \leserluft
  
  $$x \mapsto  \LoesungsRaumLang{\frac{x}{17}\cdot{} 5 + 1}$$

  Was ist der Definitions- bzw. der Wertebreich dieser Funktion?

  \leserluft
  
  $$\mathbb{D} = \LoesungsRaumLang{[0; 17]}$$

  \leserluft
  
  $$\mathbb{W} = \LoesungsRaumLang{[1; 6]}$$

  Doch wie sieht es aus, wenn ich die Fragestellung \textbf{umkehre}?
  Wie viele Punkte brauchue ich für Note 4? Gegeben ist diesmal die
  Unabhängige Variable der Note und berechnet wird die nötige
  Punktezahl.

  \leserluft{}
  
  $$x \mapsto \LoesungsRaumLang{\frac{17}5 x - \frac{17}5}$$

    Wie sehen Definitions- und Wertebereiche jetzt aus?

    $$\mathbb{D} = \LoesungsRaumLang{ [1; 6] }$$


    $$\mathbb{W} = \LoesungsRaumLang{ [0;17] }$$
  
\end{beispiel}
\newpage

Skizzieren Sie die beiden Funktionen:

\bbwGraph{0}{17}{0}{17}{
  \TRAINER{\bbwFunc{5/17*\x+1}{0:17}}
  \TRAINER{\bbwFunc{17/5*(\x-1)}{1:6}}
}
\newpage

\subsection{Umkehrbar}
\begin{definition}{Umkehrbar}{}
  Eine Funktion heißt umkehrbar auf $\mathbb{D}$, wenn es eine inverse
  Funktion

  $$f^{-1} = \bar{f}$$

  gibt, sodass gilt:

  $$x= f^{-1}(f(x))$$
\end{definition}

\begin{gesetz}{Umkehrbar}{}
  Damit eine Funktion auf $\mathbb{D}$ umkehrbar ist, muss für $x_1$ und $x_2$ gelten dass
  $$f(x_1) \ne f(x_2)$$
\end{gesetz}

\begin{bemerkung}{Umkehrung}{}
 Der Graph der Umkehrfunktion (falls diese existiert) ist das
 Spieglebild der Originalfunktion 
 an der Winkelhalbierenden des 1. bzw. 3. Quadranten ($y=x$).
\end{bemerkung}

\newpage


\subsection{Aufgaben}
Skizzieren Sie die folgenden Funktionen.
Kehren Sie diese danach auf ihrem maximal möglichen
Definitionsbereich um und skizzieren Sie auch deren Umkehrfunktion.

$$f(x) = x^3 + 1$$
\TNT{3.6}{
  $f^{-1}(x) = \sqrt[3]{y-1}$ für $y\ge 1$ ansonsten:
  $f^{-1}(x) = -\sqrt[3]{1-y}$ 
}

$$x\mapsto x^2 - 2$$
\TNT{3.6}{
  Die Funktion ist auf $\mathbb{D}=\mathbb{R}$ nicht umkehrbar. Aber
  Bsp. im 1. Quadranten $T=[0;\infty[ \subset \mathbb{D}$:
 
  $f^{-1}(x) = \sqrt[2]{y+2}$ für $y\ge -2$.
}

\newpage

\subsection{Rezept}
\begin{rezept}{Umkehrfunktion}{}
  Um auf die Umkehrfunktion zu kommen, sind folgende Schritte nötig
  \begin{enumerate}
  \item Definitions- und Wertebereiche von $f$ finden: $\mathbb{D} =
    ...$, $\mathbb{W} = ...$
  \item Funktion nach $y$ auf\/lösen (falls möglich)
  \item $x$ und $y$ tauschen
  \item Definitions- und Wertebereiche tauschen:\\
    $\mathbb{D}_{f^{-1}} = \mathbb{W}_f$ und  
    $\mathbb{W}_{f^{-1}} = \mathbb{D}_f$.
    \end{enumerate} 
  \end{rezept}
\newpage%%
