\section{Umkehrfunktion}\index{Umkehrfunktion}\index{Funktion!Umkehrung}

\subsection{Lernziele}

\begin{itemize}
\item Funktionen umkehren
\item Symmetrieeigenschaften
\item Definitions- und Wertebereiche
\end{itemize} 
\newpage

\subsection{Einstiegsbeispiel}
\begin{beispiel}{Umkehrfunktion}{}
  Eine Schulnote wird aus einer Prüfung mit max. 13 Punkten errechnet.
  Die Skala sei linear. 0 Punkte ergeben die Note 1, 13 Punkte die
  Note 6. Wie sieht die Zuordnung von «Punkte» (unaabhängige Variable)
  zur «Note» (abhängige Variable) aus?

  \leserluft
  
  $$x \mapsto  \LoesungsRaumLang{\frac{x}{13}\cdot{} 5 + 1}$$

  Was ist der Definitions- bzw. der Wertebreich dieser Funktion?

  \leserluft
  
  $$\mathbb{D} = \LoesungsRaumLang{[0; 13]}$$

  \leserluft
  
  $$\mathbb{W} = \LoesungsRaumLang{[1; 6]}$$

  Doch wie sieht es aus, wenn ich die Fragestellung \textbf{umkehre}?
  Wie viele Punkte brauchue ich für Note 4? Gegeben ist diesmal die
  Unabhängige Variable der Note und berechnet wird die nötige
  Punktezahl.

  \leserluft{}
  
  Die nötige Punktezahl für Note 4 ist \LoesungsRaumLang{7.8}.

  \leserluft{}
  
  $$x \mapsto \LoesungsRaumLang{\frac{13}5 x - \frac{13}5}$$

    Wie sehen Definitions- und Wertebereiche jetzt aus?

    $$\mathbb{D} = \LoesungsRaumLang{ [1; 6] }$$


    $$\mathbb{W} = \LoesungsRaumLang{ [0;13] }$$
  
\end{beispiel}
\newpage

Skizzieren Sie die beiden Funktionen:

\bbwGraph{0}{13}{0}{13}{
  \TRAINER{\bbwFunc{5/13*\x+1}{0:13}}
  \TRAINER{\bbwFunc{13/5*(\x-1)}{1:6}}
}%% END bbwGraph
\newpage


\subsection{Aufgaben}
Lösen Sie die Aufgaben im OLAT im Aufgabenblatt
\weblink{Finv (Funktionen 3 / Potenzfunktionen/ Umkehrfuntkonen)}{https://olat.bbw.ch/auth/RepositoryEntry/572162090/CourseNode/105951757522662}.

Lösen Sie die Strukturaufgaben im Schwerpunktfach (SPF) Seite 12: VT
1\_1 bis VT 1\_3

\newpage


\subsection{Umkehrbar}
\begin{definition}{Umkehrbar}{}
Die \textbf{Umkehrung} einer Funktion $f$ auf ihrem Definitionsbereich
$\mathbb{D}$ wird mit

  $$f^{-1} \textrm{ bzw. } \bar{f}$$

bezeichnet.
\end{definition}

\begin{gesetz}{Umkehrung der Umkehrung}{}

  Für eine umkehrbare Funktion $f$ gilt:
  
  $$x= f^{-1}(f(x))$$
\end{gesetz}


\begin{gesetz}{Umkehrbar}{}
  Damit eine Funktion auf $\mathbb{D}$ umkehrbar ist, muss für $x_1\ne x_2$ gelten dass
  $$f(x_1) \ne f(x_2)$$
\end{gesetz}

\begin{bemerkung}{Umkehrung}{}
 Der Graph der Umkehrfunktion (falls diese existiert) ist das
 Spieglebild der Originalfunktion 
 an der Winkelhalbierenden des 1. bzw. 3. Quadranten ($y=x$).
\end{bemerkung}
\newpage


\subsection{Rezept}
\begin{rezept}{Umkehrfunktion}{}
  Um auf die Umkehrfunktion zu kommen, sind folgende Schritte nötig
  \begin{enumerate}
  \item Sich klar werden über Definitions- und Wertebereiche von $f$:
    $\mathbb{D} =  ...$, $\mathbb{W} = ...$. Allenfalls Wertebereich aus Grundmenge $\mathbb{G}$ und Definitionsbereich ermitteln.
  \item Funktionsgleichung nach $x$ auf\/lösen (falls möglich)
  \item vertauschen:
    \begin{itemize}
  \item $x$ mit $y$ 
  \item Definitions- mit Wertebereich\\

      $\mathbb{D}_{f^{-1}} = \mathbb{W}_f$ und  \\

      $\mathbb{W}_{f^{-1}} = \mathbb{D}_f$.
      \end{itemize}
    \end{enumerate} 
  \end{rezept}
\newpage%%
