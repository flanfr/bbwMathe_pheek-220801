%% 2020 12 25 Ph. G. Freimann @ BBW.CH
%% TALS Fct3 Polynomfunktionen

\section{Polynomfunktionen}\index{Polynomfunktionen}\index{Funktionen!Polynom-}

\subsection*{Lernziele}

\begin{itemize}
\item Polynomfunktion
\item Linearfaktoren
\item Nullstellen, Nullstellenform
\item Charakteristische Punkte
\item lokale und globale Extremwerte
\item Doppellösungen
\end{itemize}

\theorieTALS{199}{3.6}

\subsection{gerade und ungerade Funktionen}

\subsection{Charakteristische Punkte und Extremwerte}


\subsection{Linearfaktoren und Nullstellenform}


\subsection*{Aufgaben}

\aufgabenfarbe{Strukturaufgaben: SPF V. 4.0 Seite 6ff: Aufg. 6.,
  7. und 10.}

\TALSAadB{200ff}{746. f3) f4) f8) f9) 748.-764.}


\newpage

