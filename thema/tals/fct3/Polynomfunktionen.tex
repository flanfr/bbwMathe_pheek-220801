%% 2020 12 25 Ph. G. Freimann @ BBW.CH
%% TALS Fct3 Polynomfunktionen

\section{Polynomfunktionen}\index{Polynomfunktionen}\index{Funktionen!Polynom-}

\subsection*{Lernziele}

\begin{itemize}
\item Polynomfunktion
\item Linearfaktoren
\item Nullstellen, Nullstellenform
\item Charakteristische Punkte
\item lokale und globale Extremwerte
\item Doppellösungen
\end{itemize}

\theorieTALS{199}{3.6}
\newpage


\subsection{Einstieg}
Vergleichen Sie die Graphen der Funktion
$$f: y= \frac1{10}\cdot{}(x^3-2x^2-11x+12)$$
mit demjenigen der folgenden Funktion:
$$g: y=\frac1{10}\cdot{}(x-4)\cdot{}(x+3)\cdot{}(x-1)$$


\bbwGraph{-4}{5}{-2}{3}{%%
  \TRAINER{%%
   \bbwFunc{(\x*\x*\x - 2*\x*\x - 11*\x + 12)/10}{-3.5:4.5}%% END Function
   \bbwFunc{((\x-4)*(\x+3)*(\x-1))/10}{-3.5:4.5}%% END Function
  }%% END TRAINER
}%% END bbwGraph

\subsection*{Aufgaben}

\aufgabenfarbe{Strukturaufgaben Seite 6 Teil 1 ohne Taschenrechner Aufgabe 7.}

\TALSAadB{200ff}{746. f3) f4) f8) f9)}

\newpage



\subsection{gerade und ungerade Funktionen}


\begin{definition}{gerade/ungerade Funktion}{}\index{Funktion!gerade/ungerade}\index{gerade Funktion}\index{ungerade Funktion}
  Funktionen, welche an der $y$-Achse gespiegelt sind, werden
  \textbf{gerade} Funktionen genannt und haben die Eigenschaft

  $$f(x) = f(-x)$$

  
  Funktionen, welche am Ursprung $O = (0|0)$gespiegelt sind, werden
  \textbf{ungerade} Funktionen genannt und haben die Eigenschaft

  $$f(-x) = -f(x)$$
\end{definition}


\begin{gesetz}{gerade Exonenten}{}
  Polynomfunktionen mit ausschließlich \textbf{geraden} Exponenten sind gerade Funktionen:

  $$g: y= \sum_i a_i\cdot{}x^{2\cdot{}i}$$

  Begründung:

  $$(-x)^{2\cdot{}i} = ((-1)\cdot{}x)^{2\cdot{}i} = (-1)^{2\cdot{}i} \cdot{} x^{2\cdot{}i} = \left((-1)^2\right)^i \cdot{} x^{2\cdot{}i} = 1^i\cdot{}x^{2\cdot{}i} = x^{2\cdot{}i}$$
\end{gesetz}

\begin{beispiel}{gerade Funktion}{}
  $$g(x) = 0.3\cdot{}x^6 + 2\cdot{}x^2 + 4 -6\cdot{}x^{-4}$$
  Bemerkung: Auch $0$ ist ein gerader Exponent:
  $$g(x) = 0.3\cdot{}x^6 + 2\cdot{}x^2 + 4\LoesungsRaum{\cdot{}x^0} -6\cdot{}x^{-4}$$  
\end{beispiel}
\newpage


\begin{gesetz}{\textbf{un}gerade Exponenten}{}
  Polynomfunktionen mit ausschließlich \textbf{un}geraden Exponenten sind \textbf{un}gerade Funktionen:

  $$g: y= \sum_i a_i\cdot{}x^{2\cdot{}i+1}$$

  Begründung:

  $$(-x)^{2\cdot{}i + 1} = ((-1)\cdot{}x)^{2\cdot{}i+1} = (-1)^{2\cdot{}i+1} \cdot{} x^{2\cdot{}i+1} = (-1)\cdot{}x^{2\cdot{}i+1} = -\left(x^{2\cdot{}i+1}\right)$$

\end{gesetz}

\begin{beispiel}{ungerade Funktion}{}
  $$g(x) = 0.3\cdot{}x^5 + 2\cdot{}x  -6\cdot{}x^{-1}$$
  Bemerkung: Auch $x=x^1$ ist ein ungerader Exponent:
  $$g(x) = 0.3\cdot{}x^5 + 2\LoesungsRaum{\cdot{}x^1} -6\cdot{}x^{-1}$$  
\end{beispiel}



\subsection*{Aufgaben}

\aufgabenfarbe{Zeichnen Sie Funktionen obiger beiden Beispiele mit dem Taschenrechner.}

\aufgabenfarbe{Trigonometrische Funktionen: Welche der Funktionen $\sin()$, $\cos()$ und $\tan()$ sind gerade, welche ungerade?}
\TNT{2.8}{
  $\sin()$ ist am Ursprung gespiegelt, somit ungerade.

  $\cos()$ ist an der $y$-Achse gespiegelt, somit gerade.

  $\tan()$ ist am Ursprung gespiegelt, somit ungerade.
}

\TALSAadB{201}{749. a) b), 750. $f_3$) $f_4$), 751. $f_3$) $f_9$)}
\newpage



\subsection{Charakteristische Punkte und Extremwerte}

Zeichnen Sie die charakteristischen Punkte ein:

\bbwGraph{-4}{5}{-2}{3}{%%
   \bbwFunc{((\x-4)*(\x+3)*(\x-1))/10}{-3.5:4.5}%% END Function
}%% END bbwGraph

\begin{tabular}{|c|c|}\hline
  -3, 1 und 4 & Nullstellen \\\hline
  $\approx -1.36$ und $\approx{} 2.69$ & Extrem\textbf{stellen}\\\hline
  $\approx 2.07 $ und $\approx{} -1.26$ & zugehörige Extrem\textbf{werte}\\\hline
  $\left(\frac23\middle|\frac{11}{30}\right)$ & Wendepunkt\\\hline
  \end{tabular} 

\subsection*{Aufgaben}
\aufgabenfarbe{Strukturaufgaben mit TR S. 16 Aufg 52.}

\newpage



\subsection{Linearfaktoren und Nullstellenform}


\subsection*{Aufgaben}


\aufgabenfarbe{Strukturaufgaben: SPF V. 4.0 Seite 6ff: Ohne Taschenrechner: Aufg. 6. und 10.}



\newpage

