\section{Wurzelfunktionen}\index{Wurzelfunktionen}\index{Funktionen!Wurzelfct.}


\theorieTALS{213}{3.9}


Skizzieren Sie $f: y = \sqrt{x}$ und $g: y=\sqrt[5]{x}$ im Bereich $[-1; 9]$ indem Sie für jeden 0.5-er $x$-Wert die zugehörigen $y$-Werte berechnen:

\bbwGraph{-2}{9.5}{-1}{3.5}{%%
  \TRAINER{\bbwFunc{sqrt(\x)}{0:9}}
  \TRAINER{\bbwFunc{pow(\x, 0.2)}{0:9}}
}%%

\begin{bemerkung}{Wurzelfunktion}{}
 Je \textbf{höher} der Wurzelexponent $n$, umso
 \LoesungsRaumLang{flacher} wird der Graph der Wurzelfunktion $$y=\sqrt[n]{x}.$$ 
\end{bemerkung}


\begin{definition}{Wurzelfunktion}{}
  Der Definitionsbereich der Wurzelfunktion beschränkt sich auf die
  positiven reellen Zahlen:
  $$\mathbb{D} = \mathbb{R}_0^{+} = [0;\infty[$$
      Der Wertebereich beschränkt sich somit auch nur auf die positive $y$-Achse:
  $$\mathbb{W} = \mathbb{R}_0^{+} = [0;\infty[$$
      
\end{definition}


\subsection*{Aufgaben}

\aufgabenFarbe{Strukturaufgaben SPF (V4.0) S. 14ff: Aufg. 46. und 50.}

\TALSAadB{213ff}{803. n), 804. h), 805. c), 806. c), 807. a) c) 807. b)}


\newpage

\subsection{Wurzelfunktion als
  Umkehrfunktion}\index{Umkehrfunktion!der Wurzelfunktion}\index{Funktionen!Umkehrfunktion}

\subsection*{Lernziele}

\begin{itemize}
\item Wurzelfunktion als Umkehrfunktion
\item Symmetrieeigenschaften
\end{itemize}

\theorieTALS{209}{3.8}


Skizzieren Sie $f: y = \sqrt[3]{x}$ und $g: y=x^3$ im Bereich $[-1; 9]$ indem Sie für jeden 0.5-er $x$-Wert das jeweils zugehörige $y$ berechnen:

\bbwGraph{-2}{9.5}{-1}{9}{%%
  \TRAINER{\bbwFunc{\x*\x*\x}{0:2.05}}
  \TRAINER{\bbwFunc{pow(\x, 0.33))}{0:9}}
}%%

\begin{bemerkung}{Umkehrfunktion}{}
  Eine \textbf{Umkehrfunktion} ist das Spiegelbild der der Funktion
  gespiegelt an der Geraden \LoesungsRaum{$y=x$}.
\end{bemerkung}  
\newpage
\subsection*{Aufgaben}

\TALSAadB{213ff}{801. für $n=2,3$, 802. mit Geogebra oder TR}

\newpage
