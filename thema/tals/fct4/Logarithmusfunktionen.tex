%%
%% 2019 07 04 Ph. G. Freimann
%%

\section{Logarithmusfunktionen}\index{Funktionen!Logarithmus}\index{Logarithmusfunktionen}
\sectuntertitel{$e^x$}
%%%%%%%%%%%%%%%%%%%%%%%%%%%%%%%%%%%%%%%%%%%%%%%%%%%%%%%%%%%%%%%%%%%%%%%%%%%%%%%%%
\subsection*{Lernziele}

\begin{itemize}
\item Umkehrung der Exponentialfunktion
\item Nullstellen, Definitionsbereich
\end{itemize}

\TALS{(\cite{frommenwiler17alg} S.227 (Kap. 3.11))}
\GESO{(\cite{marthaler21}       S.329 (Kap. 19.2))}


\subsection{Definition}
Die Funktion $f(x): x \mapsto y = log(x)$ ist eine
Logarithmusfunktion.

\subsubsection{Umkehurng der
  Exponentialfunktion}\index{Umkehrfunktion!der Exponentialfunktion}

Skizzieren Sie $f(x) = 2^x$ und $g(x) = \log_2(x)$ ins selbe
Koordinatensystem und markieren Sie die charakteristischen Punkte. Was
ändert sich, wenn Sie bei $f$ und $g$ statt der Zahl $2$ die Zahl $3$
verwenden?

\TNTeop{
  Bei der Exponentialfunktion $f$ ist 1 der $y$-Achsenabschnitt. Bei
  $g$ ist 1 die Nullstelle.

  $f$ hat keinen Funktionswert $\le 0$ während der Definitionsbereich
  von $g$ auch nur Zahlen größer als $0$ sein können.

  Die beiden Funktionen $f$ und $g$ sind sich gegenseitig
  Umkehrfunktionen und somit das Spiegelbild an der Diagonalen $x=y$
  im Koordinatensystem.

  
}
\newpage


\subsection*{Aufgaben}
\TALSAadBMTA{227ff}{856-879}
  \olatLinkTALSStrukturaufgabenSPF{Basiskenntnisse Funktionen Teil
    1}{4}{3.}
  \olatLinkTALSStrukturaufgabenSPF{Basiskenntnisse Funktionen Teil
    2}{15}{53. und 54.}


\GESOAadBMTA{340ff}{42-51}
