%%
%% 2019 07 04 Ph. G. Freimann
%%
\newpage
\section{Quadratwurzeln}\index{Quadratwurzel}\index{$\sqrt{\mathstrut{}\,}$}
\sectuntertitel{Lasst uns im Garten ein paar Wurzeln ziehen}

\theorieTALS{44}{1.6.1}
%%%%%%%%%%%%%%%%%%%%%%%%%%%%%%%%%%%%%%%%%%%%%%%%%%%%%%%%%%%%%%%%%%%%%%%%%%%%%%%%%
\subsection*{Lernziele}

\begin{itemize}
\item Wurzelgesetze \TALS{\cite{frommenwiler17alg}
  S. 44}\GESO{\cite{marthaler17} S. 78}
\TRAINER{\item \textit{Optional: Normalform}}
\end{itemize}

\subsection{Wortherkunft}
Radix\index{Radix} = lat. Wurzel.
Das kleine $r$ erinnert noch vage an das Wurzelzeichen\index{Wurzelzeichen} $\sqrt{\strut{}\ \ }$.

\subsection{Gesetze für positive Zahlen}
Seien $a \in \mathbb{R^+}$, m. a. W.: $a>0$.

Definition:

$\sqrt{a} \cdot \sqrt{a} = a$

$\sqrt{a} \ge 0$

«Wurzel aus $a$ Quadrat» kann auf zwei Arten aufgefasst werden:

$$a = \sqrt{a^2} = \sqrt{a}^2 = a$$

Dieser Unterschied spielt zum Glück für positive Zahlen keine Rolle.


\newpage
\subsection{Multiplizieren und Dividieren}
Es gilt (für positive Zahlen):

$$A^2\cdot B^2 = (A\cdot B)^2$$

$\Rightarrow$

\begin{center}
$\sqrt{\mathstrut{}A^2\cdot B^2}=\sqrt{(AB)^2}=AB=\sqrt{A^2}\cdot\sqrt{B^2}$
\end{center}

$\Rightarrow$

\begin{center}
\fbox{$\sqrt{\mathstrut{}a\cdot b} = \sqrt{\mathstrut{}a}\cdot\sqrt{\mathstrut{}b}$}
\end{center}

Aber auch $\frac{\sqrt{a\mathstrut{}}}{\sqrt{b}}=\sqrt{\frac{a\mathstrut{}}{b}}$.

\textbf{Achtung:}
$\sqrt{a+b} = ?$
$\sqrt{a\mathstrut{}} + \sqrt{b\mathstrut{}} = ?$

\subsection*{Aufgaben}
\GESOAadB{84}{22}
\TALSAadB{44ff}{124. a) d) f)\\
Taschenrechner: 126. a) b) c) f) j)\\
Von Hand: 127. h) 129. a)}
